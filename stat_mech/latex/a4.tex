\documentclass[10pt]{article} % Font size - 10pt, 11pt or 12pt

\usepackage[hmargin=1.25cm, vmargin=1.5cm]{geometry} % Document margins

\usepackage{graphicx}
\usepackage{amsmath}
%% \usepackage{marvosym} % Required for symbols in the colored box
%% \usepackage{ifsym} % Required for symbols in the colored box

\usepackage[usenames,dvipsnames]{xcolor} % Allows the definition of hex colors

% Fonts and tweaks for XeLaTeX
\usepackage{fontspec,xltxtra,xunicode}
\defaultfontfeatures{Mapping=tex-text}
%\setmonofont[Scale=MatchLowercase]{Andale Mono}

% Colors for links, text and headings
\usepackage{hyperref}
\definecolor{linkcolor}{HTML}{506266} % Blue-gray color for links
\definecolor{shade}{HTML}{F5DD9D} % Peach color for the contact information box
\definecolor{text1}{HTML}{2b2b2b} % Main document font color, off-black
\definecolor{headings}{HTML}{701112} % Dark red color for headings
% Other color palettes: shade=B9D7D9 and linkcolor=A40000; shade=D4D7FE and linkcolor=FF0080

\hypersetup{colorlinks,breaklinks, urlcolor=linkcolor, linkcolor=linkcolor} % Set up links and colors

\usepackage{fancyhdr}
\pagestyle{fancy}
\fancyhf{}
% Headers and footers can be added with the \lhead{} \rhead{} \lfoot{} \rfoot{} commands
% Example footer:
%\rfoot{\color{headings} {\sffamily Last update: \today}. Typeset with Xe\LaTeX}

\renewcommand{\headrulewidth}{0pt} % Get rid of the default rule in the header

\usepackage{titlesec} % Allows creating custom \section's

\allowdisplaybreaks

% Format of the section titles
\titleformat{\section}{\color{headings}
\scshape\Large\raggedright}{}{0em}{}[\color{black}\titlerule]

\title{Statistical Mechanics Assignment Four}
\author{Elliott Capek}
\titlespacing{\section}{0pt}{0pt}{5pt} % Spacing around titles

\begin{document}

\maketitle{}

\section{Problem Two: Two-Dimensional Gas}
To tackle this problem, we start by finding the 1-D ideal gas partition function, use it to get the
2-D partition function, use this new partition function to find the energy, then use the energy
to find the heat capacity of a 2-D ideal gas.\\

1-D Partition function:
\begin{align*}
  Z_x &= \int_{-\infty}^{\infty} e^{-mv_x^2/2kT} dv_x\\
  &= \sqrt{\frac{2kT\pi}{m}}\\
\end{align*}

2-D Partition function:
\begin{align*}
  Z &= Z_xZ_y = Z_x^2 = \frac{2kT\pi}{m}\\
\end{align*}

We then find the energy of the gas:

\begin{align*}
  <E> &= N\sum_{i} \epsilon_i p_i\\
  &= \frac{N}{Z}\int_{-\infty}^{\infty}\int_{-\infty}^{\infty}
  \frac{m(v_x^2+v_y^2)}{2}e^{-m(v_x^2+v_y^2)/2kT} dv_xdv_y\\
  &= \frac{Nm}{2kT\pi}
  \int_{0}^{\infty}\int_0^{2\pi}\frac{mv_r^2}{2}e^{-mv_r^2/2kT}vdv_rd\theta\\
  &= \frac{Nm^2}{2kT} \int_{0}^{\infty} v_r^3e^{-mv_r^2/2kT}dv\\
  &= \frac{Nm^2}{2kT}\frac{2k^2T^2}{m^2}\\
  &= NkT\\
\end{align*}

\begin{align*}
  C_v &= \frac{d<E>}{dT} = Nk\\
\end{align*}

This is the number we expect. Because vector decomposition works on squared velocities and the
energy of an ideal gas only depends on the squared velocities of its particles, we can separate
the ideal gas energy into dimensional components and just add them up to get the total energy.
We found that each dimension has energy $\frac{1}{2}NkT$, so we get double that for our final
energy.\\

\end{document}

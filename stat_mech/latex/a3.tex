\documentclass[10pt]{article} % Font size - 10pt, 11pt or 12pt

\usepackage[hmargin=1.25cm, vmargin=1.5cm]{geometry} % Document margins

\usepackage{graphicx}
\usepackage{amsmath}
%% \usepackage{marvosym} % Required for symbols in the colored box
%% \usepackage{ifsym} % Required for symbols in the colored box

\usepackage[usenames,dvipsnames]{xcolor} % Allows the definition of hex colors

% Fonts and tweaks for XeLaTeX
\usepackage{fontspec,xltxtra,xunicode}
\defaultfontfeatures{Mapping=tex-text}
%\setmonofont[Scale=MatchLowercase]{Andale Mono}

% Colors for links, text and headings
\usepackage{hyperref}
\definecolor{linkcolor}{HTML}{506266} % Blue-gray color for links
\definecolor{shade}{HTML}{F5DD9D} % Peach color for the contact information box
\definecolor{text1}{HTML}{2b2b2b} % Main document font color, off-black
\definecolor{headings}{HTML}{701112} % Dark red color for headings
% Other color palettes: shade=B9D7D9 and linkcolor=A40000; shade=D4D7FE and linkcolor=FF0080

\hypersetup{colorlinks,breaklinks, urlcolor=linkcolor, linkcolor=linkcolor} % Set up links and colors

\usepackage{fancyhdr}
\pagestyle{fancy}
\fancyhf{}
% Headers and footers can be added with the \lhead{} \rhead{} \lfoot{} \rfoot{} commands
% Example footer:
%\rfoot{\color{headings} {\sffamily Last update: \today}. Typeset with Xe\LaTeX}

\renewcommand{\headrulewidth}{0pt} % Get rid of the default rule in the header

\usepackage{titlesec} % Allows creating custom \section's

\allowdisplaybreaks

% Format of the section titles
\titleformat{\section}{\color{headings}
\scshape\Large\raggedright}{}{0em}{}[\color{black}\titlerule]

\title{Statistical Mechanics Assignment Three}
\author{Elliott Capek}
\titlespacing{\section}{0pt}{0pt}{5pt} % Spacing around titles

\begin{document}

\maketitle{}

\section{Problem One: Minimizing a 4-edged box}
\textbf{a.) Calculus}

\begin{align*}
  A &= 2xy + 2yz + xz\\
  2y(x+z) + xz &= A\\
  y = \frac{\left(A - xz\right)}{2\left(x+z\right)}\\
  V = xyz = xz\frac{\left(A - xz\right)}{2\left(x+z\right)}\\
  \frac{dV}{dx} &= 0 = z\frac{\left(A - xz\right)}{2\left(x+z\right)}
  - xz\frac{z}{2\left(x+z\right)} + xz\frac{\left(A - xz\right)}{\left(x+z\right)^2}\\
  \frac{2xz^2 - Az}{2\left(x+z\right)} =& xz\frac{\left(A - xz\right)}{\left(x+z\right)^2}\\
\end{align*}

\textbf{b.)}

\begin{align*}
  g(x,y,z) &= 2xy + 2yz + xz\\
  \nabla f &= \lambda \nabla g\\
  yz &= \lambda(2y+z)\\
  xz &= \lambda(2x+2z)\\
  xy &= \lambda(x+2y)\\
\end{align*}

Then...

\begin{align*}
  xyz &= x\lambda(2y+z) = 2\lambda xy + \lambda xz\\
  xyz &= y\lambda(2x+2z)= 2\lambda xy + 2\lambda yz\\
  xyz &= z\lambda(x+2y) = \lambda xz + 2\lambda yz\\
  \lambda xz + 2\lambda yz &= 2\lambda xy + 2\lambda yz\\
  \lambda xz &= 2\lambda xy\\
  z &= y\\
  2\lambda xy + 2\lambda y^2 &= 2\lambda xy + \lambda xy\\
  2\lambda y^2 &= \lambda xy\\
  y^2 &= xy \rightarrow x = y\\
  x = y = z\\
\end{align*}


\section{Problem Two: Two-system entropy}
\textbf{a.)} The probability of two unrelated events occuring is the product of the
two individual probabilities. Because the two systems are weakly interacting,
they are ``unrelated'' and so $P_{rs} = P_rP_s$.\\

\textbf{b.)}
\begin{align*}
  S &= -k_B\sum_r \sum_s P_{rs} \ln\left(P_{rs}\right)\\
  &= -k_B\sum_r \sum_s P_rP_s \ln\left(P_rP_s\right)\\
  &= -k_B\sum_r \sum_s P_rP_s\left(\ln P_r + \ln P_s\right)\\
  &= -k_B\sum_r \sum_s P_rP_s\ln P_r + -k_B\sum_r \sum_s P_rP_s\ln P_s\\
  &= -k_B\sum_r P_r\ln P_r + -k_B\sum_s P_s\ln P_s
  \hspace{2cm}\mbox{since }\sum_rP_r = 1\\
  &= S_1 + S_2\\
\end{align*}

\section{Problem Three: Two-system probability}
\textbf{a.)}
\begin{align*}
  S - S_0 &= -k_B\sum_rP_r^0\ln P_r^0 + k_B\sum_rP_r\ln P_r\\
  &= -k_B\sum_rP_r^0\ln P_r^0 + P_r\ln P_r + P_r\ln P_r^0 - P_r\ln P_r^0
  \hspace{2cm}P_r\ln P_r^0 - P_r\ln P_r^0=0\\
\end{align*}

We need to get rid of $P_r^0\ln P_r^0 - P_r\ln P_r^0$, so we do some fancy math. By taking
the logarithm of our probability definition, we see that:

\begin{align*}
  \ln P_r^0 &= -\beta E_r - \ln Z\\
  \sum_r P_r\ln P_r^0 &= -\sum_r P_r\beta E_r - \sum_r P_r\ln Z
  \hspace{2cm}\sum_r P_r=1\\
  \sum_r P_r\ln P_r^0 &= -\beta E - \ln Z\\
  -\sum_r P_r\ln P_r^0 &= \beta E + \ln Z\\
\end{align*}

\begin{align*}
  \ln P_r^0 &= -\beta E_r - \ln Z\\
  \sum_r P_r^0\ln P_r^0 &= -\sum_r P_r^0\beta E_r - \sum_r P_r^0\ln Z
  \hspace{2cm}\sum_r P_r^0=1\\
  \sum_r P_r^0\ln P_r^0 &= -\beta E - \ln Z\\
\end{align*}

Combining these two final equations, we see that:

\begin{align*}
  \sum_r P_r^0\ln P_r^0 - \sum_r P_r\ln P_r^0 = 0\\
\end{align*}

which is what we wanted to show. We plug this into our original equation:

\begin{align*}
  S - S_0 &= -k_B\sum_rP_r\ln P_r + P_r\ln P_r^0\\
  &= -k_B\sum_r P_r \ln\left(\frac{P_r^0}{P_r}\right)\\
\end{align*}

\textbf{b.)}
\begin{align*}
  S - S_0 &= -k_B\sum_r P_r \ln\left(\frac{P_r^0}{P_r}\right)\\
  S - S_0 &\leq -k_B\sum_r P_r \frac{P_r^0}{P_r} - 1
  \hspace{2cm}\mbox{By the relation in the problem}\\
  \sum_r P_r \frac{P_r^0}{P_r} -1 = 0 \rightarrow S \leq S_0\\
\end{align*}

\end{document}

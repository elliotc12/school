\documentclass[10pt]{article} % Font size - 10pt, 11pt or 12pt

\usepackage[hmargin=1.25cm, vmargin=1.5cm]{geometry} % Document margins

\usepackage{graphicx}
\usepackage{amsmath}
\usepackage{marvosym} % Required for symbols in the colored box
\usepackage{ifsym} % Required for symbols in the colored box

\usepackage[usenames,dvipsnames]{xcolor} % Allows the definition of hex colors

% Fonts and tweaks for XeLaTeX
\usepackage{fontspec,xltxtra,xunicode}
\defaultfontfeatures{Mapping=tex-text}
%\setmonofont[Scale=MatchLowercase]{Andale Mono}

% Colors for links, text and headings
\usepackage{hyperref}
\definecolor{linkcolor}{HTML}{506266} % Blue-gray color for links
\definecolor{shade}{HTML}{F5DD9D} % Peach color for the contact information box
\definecolor{text1}{HTML}{2b2b2b} % Main document font color, off-black
\definecolor{headings}{HTML}{701112} % Dark red color for headings
% Other color palettes: shade=B9D7D9 and linkcolor=A40000; shade=D4D7FE and linkcolor=FF0080

\hypersetup{colorlinks,breaklinks, urlcolor=linkcolor, linkcolor=linkcolor} % Set up links and colors

\usepackage{fancyhdr}
\pagestyle{fancy}
\fancyhf{}
% Headers and footers can be added with the \lhead{} \rhead{} \lfoot{} \rfoot{} commands
% Example footer:
%\rfoot{\color{headings} {\sffamily Last update: \today}. Typeset with Xe\LaTeX}

\renewcommand{\headrulewidth}{0pt} % Get rid of the default rule in the header

\usepackage{titlesec} % Allows creating custom \section's

\allowdisplaybreaks

% Format of the section titles
\titleformat{\section}{\color{headings}
\scshape\Large\raggedright}{}{0em}{}[\color{black}\titlerule]

\title{Statistical Mechanics Assignment One}
\author{Elliott Capek}
\titlespacing{\section}{0pt}{0pt}{5pt} % Spacing around titles

\begin{document}

\maketitle{}

\section{Problem One: Adiabatic Ideal Gas}
\textbf{a.)}
First we calculate the final temperature of the gas, given $V_i, V_f$ and $T_i$:

\begin{align*}
  P_iV_i^\gamma &= nRT_iV_i^{\gamma-1}\\
  nRT_iV_i^{\gamma-1} &= nRT_fV_f^{\gamma-1}\\
  T_f &= T_i\left(\frac{V_i}{V_f}\right)^{\gamma-1}
\end{align*}

\textbf{b.)}
If $dS=0$, then $dQ/T=0$ and so $dU = -pdV$. We use this as another route for finding $T_f$ in
terms of $V_i, V_f$ and $T_i$:

\begin{align*}
  dU &= -pdV = C_vdT\\
  -\frac{RT}{V}dV &= C_vdT \hspace{2cm}\mbox{Molar specific heat, so n=1}\\
  -R\int_{V_i}^{V_f}\frac{1}{V}dV &= C_v\int_{T_i}^{T_f}\frac{1}{T}dT\\
  -R\ln\left(\frac{V_f}{V_i}\right) &= C_v\ln\left(\frac{T_f}{T_i}\right)\\
  -\frac{C_p-C_v}{C_v}\ln\left(\frac{V_f}{V_i}\right) &= \ln\left(\frac{T_f}{T_i}\right)\\
  \ln\left(\frac{V_f}{V_i}\right)^{1-\gamma} &= \ln\left(\frac{T_f}{T_i}\right)\\
  \left(\frac{V_f}{V_i}\right)^{1-\gamma} &= \frac{T_f}{T_i}\\
  T_f &= T_i\left(\frac{V_f}{V_i}\right)^{1-\gamma} = T_i\left(\frac{V_i}{V_f}\right)^{\gamma-1}\\
\end{align*}

\end{document}

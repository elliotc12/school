\documentclass[10pt]{article} % Font size - 10pt, 11pt or 12pt

\usepackage[hmargin=1.25cm, vmargin=1.5cm]{geometry} % Document margins

\usepackage{graphicx}
\usepackage{amsmath}
%% \usepackage{marvosym} % Required for symbols in the colored box
%% \usepackage{ifsym} % Required for symbols in the colored box

\usepackage[usenames,dvipsnames]{xcolor} % Allows the definition of hex colors

% Fonts and tweaks for XeLaTeX
\usepackage{fontspec,xltxtra,xunicode}
\defaultfontfeatures{Mapping=tex-text}
%\setmonofont[Scale=MatchLowercase]{Andale Mono}

% Colors for links, text and headings
\usepackage{hyperref}
\definecolor{linkcolor}{HTML}{506266} % Blue-gray color for links
\definecolor{shade}{HTML}{F5DD9D} % Peach color for the contact information box
\definecolor{text1}{HTML}{2b2b2b} % Main document font color, off-black
\definecolor{headings}{HTML}{701112} % Dark red color for headings
% Other color palettes: shade=B9D7D9 and linkcolor=A40000; shade=D4D7FE and linkcolor=FF0080

\hypersetup{colorlinks,breaklinks, urlcolor=linkcolor, linkcolor=linkcolor} % Set up links and colors

\usepackage{fancyhdr}
\pagestyle{fancy}
\fancyhf{}
% Headers and footers can be added with the \lhead{} \rhead{} \lfoot{} \rfoot{} commands
% Example footer:
%\rfoot{\color{headings} {\sffamily Last update: \today}. Typeset with Xe\LaTeX}

\renewcommand{\headrulewidth}{0pt} % Get rid of the default rule in the header

\usepackage{titlesec} % Allows creating custom \section's

\allowdisplaybreaks

% Format of the section titles
\titleformat{\section}{\color{headings}
\scshape\Large\raggedright}{}{0em}{}[\color{black}\titlerule]

\title{Statistical Mechanics Assignment Five}
\author{Elliott Capek}
\titlespacing{\section}{0pt}{0pt}{5pt} % Spacing around titles

\begin{document}

\maketitle{}

\section{Problem One: Thermodynamics problem}
Here we are solving for the behavior of a fairly simple thermodynamic system.\\

\textbf{a.) Isobaric Coefficient}

\begin{align*}
  p &= \frac{RT}{V-b} - \frac{a}{V^2} \rightarrow T = \left(\frac{V-b}{R}\right)\left(p + \frac{a}{V^2}\right)\\
  \left(\frac{\partial T}{\partial V}\right)_p &=
  \left(\frac{1}{R}\right)\left(p + \frac{a}{V^2}\right)
  -2\left(\frac{V-b}{R}\right)\frac{a}{V^3}\\
  &= \frac{pV^3}{RV^3} + \frac{aV}{RV^3} + \frac{-2aV}{RV^3} + \frac{2ab}{RV^3}
  = \frac{pV^3 - aV + 2ab}{RV^3}\\
  \mbox{Thus...} &\frac{1}{V}\left(\frac{\partial V}{\partial T}\right)_p
  = \frac{RV^2}{pV^3 - aV + 2ab}\\
\end{align*}

\textbf{b.) Heat capacity only depends on T}
To show this, we show that if we express Cv as Cv(V,T), then its volume derivative is zero.

\begin{align*}
  \left(\partial\frac{C_V}{\partial V}\right)_T &=
  \left(\frac{\partial}{\partial T}\right)_T T\left(\frac{\partial V}{\partial T}\right)_V\\
\end{align*}


\end{document}

\documentclass[10pt]{article} % Font size - 10pt, 11pt or 12pt

\usepackage[hmargin=1.25cm, vmargin=1.5cm]{geometry} % Document margins

\usepackage{graphicx}
\usepackage{amsmath}
\usepackage{marvosym} % Required for symbols in the colored box
\usepackage{ifsym} % Required for symbols in the colored box

\usepackage[usenames,dvipsnames]{xcolor} % Allows the definition of hex colors

% Fonts and tweaks for XeLaTeX
\usepackage{fontspec,xltxtra,xunicode}
\defaultfontfeatures{Mapping=tex-text}
%\setmonofont[Scale=MatchLowercase]{Andale Mono}

% Colors for links, text and headings
\usepackage{hyperref}
\definecolor{linkcolor}{HTML}{506266} % Blue-gray color for links
\definecolor{shade}{HTML}{F5DD9D} % Peach color for the contact information box
\definecolor{text1}{HTML}{2b2b2b} % Main document font color, off-black
\definecolor{headings}{HTML}{701112} % Dark red color for headings
% Other color palettes: shade=B9D7D9 and linkcolor=A40000; shade=D4D7FE and linkcolor=FF0080

\hypersetup{colorlinks,breaklinks, urlcolor=linkcolor, linkcolor=linkcolor} % Set up links and colors

\usepackage{fancyhdr}
\pagestyle{fancy}
\fancyhf{}
% Headers and footers can be added with the \lhead{} \rhead{} \lfoot{} \rfoot{} commands
% Example footer:
%\rfoot{\color{headings} {\sffamily Last update: \today}. Typeset with Xe\LaTeX}

\renewcommand{\headrulewidth}{0pt} % Get rid of the default rule in the header

\usepackage{titlesec} % Allows creating custom \section's

\allowdisplaybreaks

% Format of the section titles
\titleformat{\section}{\color{headings}
\scshape\Large\raggedright}{}{0em}{}[\color{black}\titlerule]

\title{Statistical Mechanics Assignment Two}
\author{Elliott Capek}
\titlespacing{\section}{0pt}{0pt}{5pt} % Spacing around titles

\begin{document}

\maketitle{}

\section{Problem One: Harmonic Oscillator Energy}
\textbf{a.)} First we find the ratio of probabilities for the first and
second energies using the Botzmann distribution:\\
\begin{align*}
  \frac{p_1}{p_0} &= \left(\frac{e^{-E_1/kT}}{Z}\right)
  \left(\frac{Z}{e^{-E_0/kT}}\right)\\
  &= e^{-(E_1-E_0)/kT} = e^{-\hbar\omega/kT}\\
\end{align*}

\textbf{b.)} Then we assume only the first two energy states are occupied, use the
Boltzmann distribution to calculate our partition function, and use the energy formula
$E = \sum p_iE_i$ to find the energy:\\

\begin{align*}
  <E> &= p_1E_1 + p_0E_0\\
  &= \frac{E_1e^{-E_1/kT} + E_0e^{-E_0/kT}}{e^{-E_1/kT} + e^{-E_0/kT}}\\
  &= \left(\frac{\hbar\omega}{2}\right)
  \left(\frac{e^{-3\hbar\omega/2kT} + 3e^{-\hbar\omega/2kT}}{e^{-3\hbar\omega/2kT}
    + e^{-\hbar\omega/2kT}}\right)\\
\end{align*}

We then take some limits to see behavior at low and high temperature:

\begin{align*}
  \lim_{T\rightarrow\infty} <E> &= \frac{\hbar\omega}{2}\left(\frac{1+3}{1+2}\right)
  = \hbar\omega\\
  \lim_{T\rightarrow 0} <E> &= \frac{\hbar\omega}{2}
  \left(\frac{e^{-3\hbar\omega/2kT}}{e^{-3\hbar\omega/2kT}}\right)\\
  &= \frac{\hbar\omega}{2}\\
\end{align*}

These are the expected results. At high temperature, either state is equally likely
to be occupied, so the energy is the average of the two possible states. For the low
temperature state, only the ground-state is occupied, so the average energy is just
the energy of the ground state.\\

\section{Problem Two: Two-energy system}

\textbf{a.) and b.)} The energy as a function of time is given by:

\begin{align*}
  <E> &= \frac{\epsilon_1e^{-\epsilon_1/kT} + \epsilon_2e^{-\epsilon_2/kT}}
  {e^{-\epsilon_1/kT} + e^{-\epsilon_2/kT}}
\end{align*}

By this, at low temperatures the average energy will be $\epsilon_1$, since the
1-labeled exponent will go to zero slower. At high temperature, the energy will
be the average of $\epsilon_1$ and $\epsilon_2$. Plot:

\vspace{3cm}

\textbf{c.)}We now explicitly calculate these values:

\begin{align*}
  <E>
  &= \frac{\epsilon_1e^{-\epsilon_1/kT} + \epsilon_2e^{-\epsilon_2/kT}}
  {e^{-\epsilon_1/kT} + e^{-\epsilon_2/kT}}
\end{align*}

\begin{align*}
  C_V &= \frac{d<E>}{dT}_V\\
  &= \left(\frac{d}{dT}\right)_V
  \frac{\epsilon_1e^{-\epsilon_1/kT} + \epsilon_2e^{-\epsilon_2/kT}}
       {e^{-\epsilon_1/kT} + e^{-\epsilon_2/kT}}\\
  &= \frac{1}{kT^2}\frac{\epsilon_1^2e^{-\epsilon_1/kT} + \epsilon_2^2e^{-\epsilon_2/kT}}
  {e^{-\epsilon_1/kT} + e^{-\epsilon_2/kT}}
  + \frac{-1}{kT^2}\frac{\epsilon_1^2e^{-\epsilon_1/kT} + \epsilon_2^2e^{-\epsilon_2/kT}}
  {e^{-2\epsilon_1/kT} + e^{-2\epsilon_2/kT} + 2e^{-(\epsilon_1+\epsilon_2)/kT}}
  \left(\epsilon_1e^{-\epsilon_1/kT} + \epsilon_2e^{-\epsilon_2/kT}\right)\\
  &= \frac{(\epsilon_2-\epsilon_1)^2e^{-(\epsilon_1+\epsilon_2)/kT}}
       {\left(e^{-\epsilon_1/kT} + e^{-\epsilon_2/kT}\right)^2}\\
\end{align*}

\end{document}

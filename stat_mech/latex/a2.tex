\documentclass[10pt]{article} % Font size - 10pt, 11pt or 12pt

\usepackage[hmargin=1.25cm, vmargin=1.5cm]{geometry} % Document margins

\usepackage{graphicx}
\usepackage{amsmath}
\usepackage{marvosym} % Required for symbols in the colored box
\usepackage{ifsym} % Required for symbols in the colored box

\usepackage[usenames,dvipsnames]{xcolor} % Allows the definition of hex colors

% Fonts and tweaks for XeLaTeX
\usepackage{fontspec,xltxtra,xunicode}
\defaultfontfeatures{Mapping=tex-text}
%\setmonofont[Scale=MatchLowercase]{Andale Mono}

% Colors for links, text and headings
\usepackage{hyperref}
\definecolor{linkcolor}{HTML}{506266} % Blue-gray color for links
\definecolor{shade}{HTML}{F5DD9D} % Peach color for the contact information box
\definecolor{text1}{HTML}{2b2b2b} % Main document font color, off-black
\definecolor{headings}{HTML}{701112} % Dark red color for headings
% Other color palettes: shade=B9D7D9 and linkcolor=A40000; shade=D4D7FE and linkcolor=FF0080

\hypersetup{colorlinks,breaklinks, urlcolor=linkcolor, linkcolor=linkcolor} % Set up links and colors

\usepackage{fancyhdr}
\pagestyle{fancy}
\fancyhf{}
% Headers and footers can be added with the \lhead{} \rhead{} \lfoot{} \rfoot{} commands
% Example footer:
%\rfoot{\color{headings} {\sffamily Last update: \today}. Typeset with Xe\LaTeX}

\renewcommand{\headrulewidth}{0pt} % Get rid of the default rule in the header

\usepackage{titlesec} % Allows creating custom \section's

\allowdisplaybreaks

% Format of the section titles
\titleformat{\section}{\color{headings}
\scshape\Large\raggedright}{}{0em}{}[\color{black}\titlerule]

\title{Statistical Mechanics Assignment Two}
\author{Elliott Capek}
\titlespacing{\section}{0pt}{0pt}{5pt} % Spacing around titles

\begin{document}

\maketitle{}

\section{Problem One: Harmonic Oscillator Energy}
\textbf{a.)} First we find the ratio of probabilities for the first and
second energies using the Botzmann distribution:\\
\begin{align*}
  \frac{p_1}{p_0} &= \left(\frac{e^{-E_1/kT}}{Z}\right)
  \left(\frac{Z}{e^{-E_0/kT}}\right)\\
  &= e^{-(E_1-E_0)/kT} = e^{-\hbar\omega/kT}\\
\end{align*}

\textbf{b.)} Then we assume only the first two energy states are occupied, use the
Boltzmann distribution to calculate our partition function, and use the energy formula
$E = \sum p_iE_i$ to find the energy:\\

\begin{align*}
  <E> &= p_1E_1 + p_0E_0\\
  &= \frac{E_1e^{-E_1/kT} + E_0e^{-E_0/kT}}{e^{-E_1/kT} + e^{-E_0/kT}}\\
  &= \left(\frac{\hbar\omega}{2}\right)
  \left(\frac{e^{-3\hbar\omega/2kT} + 3e^{-\hbar\omega/2kT}}{e^{-3\hbar\omega/2kT}
    + e^{-\hbar\omega/2kT}}\right)\\
\end{align*}

We then take some limits to see behavior at low and high temperature:

\begin{align*}
  \lim_{T\rightarrow\infty} <E> &= \frac{\hbar\omega}{2}\left(\frac{1+3}{1+2}\right)
  = \hbar\omega\\
  \lim_{T\rightarrow 0} <E> &= \frac{\hbar\omega}{2}
  \left(\frac{e^{-3\hbar\omega/2kT}}{e^{-3\hbar\omega/2kT}}\right)\\
  &= \frac{\hbar\omega}{2}\\
\end{align*}

These are the expected results. At high temperature, either state is equally likely
to be occupied, so the energy is the average of the two possible states. For the low
temperature state, only the ground-state is occupied, so the average energy is just
the energy of the ground state.\\

\section{Problem Two: Two-energy system}

\textbf{a.) and b.)} The energy as a function of time is given by:

\begin{align*}
  <E> &= \frac{\epsilon_1e^{-\epsilon_1/kT} + \epsilon_2e^{-\epsilon_2/kT}}
  {e^{-\epsilon_1/kT} + e^{-\epsilon_2/kT}}
\end{align*}

By this, at low temperatures the average energy will be $\epsilon_1$, since the
1-labeled exponent will go to zero slower. At high temperature, the energy will
be the average of $\epsilon_1$ and $\epsilon_2$. Plot:

\vspace{3cm}

\textbf{c.)}We now explicitly calculate these values:

\begin{align*}
  <E>
  &= \frac{\epsilon_1e^{-\epsilon_1/kT} + \epsilon_2e^{-\epsilon_2/kT}}
  {e^{-\epsilon_1/kT} + e^{-\epsilon_2/kT}}
\end{align*}

\begin{align*}
  C_V &= \frac{d<E>}{dT}_V\\
  &= \left(\frac{d}{dT}\right)_V
  \frac{\epsilon_1e^{-\epsilon_1/kT} + \epsilon_2e^{-\epsilon_2/kT}}
       {e^{-\epsilon_1/kT} + e^{-\epsilon_2/kT}}\\
  &= \frac{1}{kT^2}\frac{\epsilon_1^2e^{-\epsilon_1/kT} + \epsilon_2^2e^{-\epsilon_2/kT}}
  {e^{-\epsilon_1/kT} + e^{-\epsilon_2/kT}}
  + \frac{-1}{kT^2}\frac{\epsilon_1^2e^{-\epsilon_1/kT} + \epsilon_2^2e^{-\epsilon_2/kT}}
  {e^{-2\epsilon_1/kT} + e^{-2\epsilon_2/kT} + 2e^{-(\epsilon_1+\epsilon_2)/kT}}
  \left(\epsilon_1e^{-\epsilon_1/kT} + \epsilon_2e^{-\epsilon_2/kT}\right)\\
  &= \frac{(\epsilon_2-\epsilon_1)^2e^{-(\epsilon_1+\epsilon_2)/kT}}
       {\left(e^{-\epsilon_1/kT} + e^{-\epsilon_2/kT}\right)^2}\\
\end{align*}

\section{Problem Three: Nuclear Spin Energy}
For this problem we only consider energies due to the spin-1 nuclei. Nonzero spins
$\pm \hbar$ have energy $E=\epsilon$ and the zero-spin has energy $E=0$.\\

\textbf{a.)}We first find the total nuclear energy of the system.
Our first task is to find the single-particle partition function:\\

\begin{align*}
  z &= \sum e^{-\beta E_i} = e^{-\beta E_{-1}} + e^{-\beta E_0} + e^{-\beta E_1}\\
  &= 2e^{-\beta\epsilon}+1\\
\end{align*}

We then plug this into our equation for total system energy, using N as Avogadro's
Number:\\

\begin{align*}
  E &= N\sum p_iE_i\\
  &= N\frac{2\epsilon e^{-\beta\epsilon}}{2e^{-\beta\epsilon} + 1}\\
\end{align*}

\textbf{b.)}Next, we find the system's entropy, using $z^N$ as the total partition
function:\\

\begin{align*}
  S &= -\left(\frac{\partial A}{\partial T}\right)_V =
  \frac{\partial}{\partial T} k_BT\ln(z^N)\\
  &= k_BN\ln(Z) + \frac{Nk_BT}{Z}\frac{\partial Z}{\partial T}\\
  &= k_BN\ln\left(2e^{-\beta\epsilon}+1\right)
  + \frac{k_BNT}{2e^{-\beta\epsilon}+1}
  \left(\frac{2\epsilon e^{-\beta\epsilon}}{k_BT^2}\right)\\
  &= k_BN\ln\left(2e^{-\beta\epsilon}+1\right)
  + \frac{2N\epsilon e^{-\beta\epsilon}}{T\left(2e^{-\beta\epsilon}+1\right)}\\
\end{align*}

We then take the limits of this function at low and high T to find extreme
behavior:

\begin{align*}
  \lim_{T\rightarrow\infty} k_BN\ln\left(2e^{-\beta\epsilon}+1\right)
  + \frac{2N\epsilon e^{-\beta\epsilon}}{T\left(2e^{-\beta\epsilon}+1\right)}
  &= Nk_B\ln\left(3\right) = k_B\ln\left(3^N\right)\\
  \lim_{T\rightarrow0} k_BN\ln\left(2e^{-\beta\epsilon}+1\right)
  + \frac{2N\epsilon e^{-\beta\epsilon}}{T\left(2e^{-\beta\epsilon}+1\right)}
  &= 0\\
\end{align*}

Both of these results make sense when we use the $S=k_B\ln\left(\omega\right)$
equation for entropy. At very high temperatures there are $3^N$ possible
equivalent states the system can take on, and at low temperatures the system will
have only one possible state: all in the spin-0 configuration, and hence zero
entropy.\\

\textbf{c.)} At very high temperatures, the denominator of the exponential
in $p_i = e^{-E_i/kT}/Z$ dwarfs the energy in the numerator, so all probabilities
are the same, ie all states are equally accessible. This means the single-particle
partition function is $3$, and hence the system's molar partition function is
$3^N$, where N is Avogadro's number. Then the entropy is
$S = k_B\ln\left(3^N\right)$, which is what we calculated for the limit in part b.\\

For very low temperatures, the exponentials in $p_i = e^{-E_i/kT}/Z$ decay to zero,
so the only nonzero probability is the zero-energy spin-0 state. Thus the only
reachable state is where all particles are in their ground state, so the entropy
is $S = k_B\ln\left(1\right) = 0$.\\

\textbf{d.)} We then calculate the constant-volume heat capacity for the
system:\\

%% \begin{align*}
%%   C_V = T\left(\frac{\partial S}{\partial T}\right)_T
%%   &= \frac{-2N\epsilon}{T\left(2e^{-\beta\epsilon}+1\right)}
%%   + \frac{-2N\epsilon e^{-\beta\epsilon}}{k_BT^2\left(2e^{-\beta\epsilon}+1\right)}
%%   + \frac{-4N\epsilon^2e^{-2\beta\epsilon}}{k_BT^2\left(2e^{-\beta\epsilon}+1\right)^2}\\
%%   &= \left(\frac{-2N\epsilon}{T\left(2e^{-\beta\epsilon}+1\right)}\right)\left(1
%%   + \frac{e^{-\beta\epsilon}}{k_BT}
%%   + \frac{2\epsilon e^{-2\beta\epsilon}}{k_BT\left(2e^{-\beta\epsilon}+1\right)}\right)
%% \end{align*}

\begin{align*}
  U &= \left(\frac{\partial U}{\partial T}\right)_V\\
  &= \left(\frac{\partial}{\partial T}\right)_V
  N\frac{2\epsilon e^{-\beta\epsilon}}{2e^{-\beta\epsilon} + 1}\\
  &= N\frac{2\epsilon^2 e^{-\beta\epsilon}}{k_BT^2\left(2e^{-\beta\epsilon} + 1\right)}
  + N\frac{4\epsilon^2e^{-2\beta\epsilon}}{k_BT^2(2e^{-\beta\epsilon} + 1)^2}\\
  &= N\frac{2\epsilon^2 e^{-\beta\epsilon}}{k_BT^2\left(2e^{-\beta\epsilon} + 1\right)}
  \left(1 + \frac{2e^{-\beta\epsilon}}{2e^{-\beta\epsilon} + 1}\right)\\
  &= N\frac{2\epsilon^2 e^{-\beta\epsilon}}{k_BT^2\left(2e^{-\beta\epsilon} + 1\right)^2}
\end{align*}

This is proportional to $T^2$ for very high T. For very high T, this number goes to
zero due to the T in the denominator and lack of any growing terms anywhere. This is
what we expect, since eventually heating up a system will not increase its energy
anymore, since the partition function changes less and less as the system reaches a
state where all microstates are equally probable.\\

\end{document}

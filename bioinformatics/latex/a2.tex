\documentclass[10pt]{article} % Font size - 10pt, 11pt or 12pt

\usepackage[hmargin=1.25cm, vmargin=1.5cm]{geometry} % Document margins

\usepackage{graphicx}
\usepackage{amsmath}
\usepackage{listings}

\usepackage[usenames,dvipsnames]{xcolor} % Allows the definition of hex colors

% Fonts and tweaks for XeLaTeX
\usepackage{fontspec,xltxtra,xunicode}
\defaultfontfeatures{Mapping=tex-text}
%\setmonofont[Scale=MatchLowercase]{Andale Mono}

% Colors for links, text and headings
\usepackage{hyperref}
\definecolor{linkcolor}{HTML}{506266} % Blue-gray color for links
\definecolor{shade}{HTML}{F5DD9D} % Peach color for the contact information box
\definecolor{text1}{HTML}{2b2b2b} % Main document font color, off-black
\definecolor{headings}{HTML}{701112} % Dark red color for headings
% Other color palettes: shade=B9D7D9 and linkcolor=A40000; shade=D4D7FE and linkcolor=FF0080

\hypersetup{colorlinks,breaklinks, urlcolor=linkcolor, linkcolor=linkcolor} % Set up links and colors

\usepackage{fancyhdr}
\pagestyle{fancy}
\fancyhf{}
% Headers and footers can be added with the \lhead{} \rhead{} \lfoot{} \rfoot{} commands
% Example footer:
%\rfoot{\color{headings} {\sffamily Last update: \today}. Typeset with Xe\LaTeX}

\renewcommand{\headrulewidth}{0pt} % Get rid of the default rule in the header

\usepackage{titlesec} % Allows creating custom \section's

\allowdisplaybreaks

% Format of the section titles
\titleformat{\section}{\color{headings}
\scshape\Large\raggedright}{}{0em}{}[\color{black}\titlerule]

\title{Bioinformatics Assignment 2}
\author{Elliott Capek}
\titlespacing{\section}{0pt}{0pt}{5pt} % Spacing around titles

\begin{document}

\maketitle{}

\section{Problem One: Scoring Alignments}
Here we are asked to compute the score for different given alignments.\\
\textbf{A.} 3\\
\textbf{B.} 1\\

\section{Problem Two: BLAST}
When BLASTp'ing the \textit{Drosophila melanogaster} hedgehog gene, six results come up when
compared to the human protein database. These all seme like ``valid'' hits just by looking
at their names (they are all ``hedgehogs'' of some type). When looking for valid hits, identity
and query cover are very important. Having high query cover, as do the ``desert'' and ``indian''
hedgehog genes, means the genes occupy roughly the same space, ie one is not a subcomponent of
the other. Having a high identity means the genes are similar. Another important factor is having
a low E-value. All the elements in this list have low E-values, meaning it is unlikely any of
them are just random hits. Total score changes based on the lengths of the sequences, so is
more difficult to interpret.\\

\section{Problem Three: Computing frequency matrix elements}
In this problem we use the following information: (2S, 4N, 2T) to compute various frequency,
normalized frequency and probability elements:\\

\begin{align*}
  f &= \mbox{Number of occurrences}\\
  q &= \mbox{Number of occurences normalized}\\
  p &= \mbox{Number of pairs involving this element, double cound identical pairs}\\
\end{align*}

\begin{align*}
  f_{SS} &= 1\\
  f_{TT} &= 4 * 3 = 12\\
  f_{NN} &= 1\\
  f_{ST} &= 2 * 4 = 8\\
  f_{NS} &= 2 * 2 = 4\\
  f_{NT} &= 2 * 4 = 8\\
  q_{SS} &= 1 / 34\\
  q_{TT} &= 12 / 34 = 6 / 17\\
  q_{NN} &= 1 / 34\\
  q_{ST} &= 8 / 34 = 4 / 17\\
  q_{NS} &= 4 / 34 = 2 / 17\\
  q_{NT} &= 4 / 17\\
  p_S &= 1 + 6 = 7\\
  p_T &= 12 + 1 = 13\\
  p_N &= 1 + 6 = 7\\
\end{align*}

\section{Problem Four: Database size}
We use the definition of E-value to compute database size, using an E-value of $10^{-27}$:\\

\begin{align*}
  m &= \frac{E*2^{S'}}{n} = \frac{10^{-27}*2^{121}}{150} \approx 1.77*10^7\\
\end{align*}

This is about 18 million residues. This seems small for a database, since this would be around
60 million nucleotides, which is two orders of magnitude smaller than the number of nucleotides
in the human genome. However, perhaps it is only a section of the genome, or perhaps it is for
a smaller organism.\\

\section{Problem Five: Dinosaur protein}
Here we begin with the given sequence and use BioPython to transcribe() and translate() it into:\\

MVHWTALITGHWGGKAECGAEADDARVVILYPLAVFASFGNLSDTAILGNPMVRAHKLSFGDAVNLDNIKN
TIIFSWQLLHCDLHENFRLLGDIDAVEWASHEREIVSKDFDPECCQAAWQKLVRVAHAKYH\\

We then run a BLAST nonredundant database protein search on it. We get dozens of hits with 100\%
query cover and roughly 65\% identity. Such a high query cover indicates that the entire sequence
is used in the alignment, meaning the given and matched sequences are likely to be the same gene.
The low identity means that there is a lot of difference between the given and matched sequences.
Most of these sequences have a very low E-value (roughly $10^{-35}$), meaning that it is highly
unlikely that our results are due to chance.\\

This all comes together to mean that this could be a dinosaur Hemoglobin, since the matches are
of high accuracy and there is lots of mutation.\\

\section{Problem Six: Orca homeobox conservation}
After running a BLASTp nonredundant search on the \textit{Orcinus orca} protein sequence, I found
that there was a 99\% identity between the orca and \textit{Sus scrofa} sequences, and only a
96\% similarity between orca and \textit{Bos taurus} sequences. There was also a 100\% sequence
cover with the boar, but only 92\% with the cow. This means that the orca sequence is almost
identical to the boar sequence, meaning orcas and boars are a lot more closely related. The
E-values were very low $10^{-175}$, meaning it is incredibly unlikely that these sequences were
randomly related.\\

\section{Problem Seven: GC Content}
GC content is conserved when taking reverse complement, since Gs and Cs map to
each other.\\

For original probability $p_{N0}$ and reverse complement probability $P_N'$:\\

\begin{align*}
  p_{G0} &= p_C'\\
  p_{C0} &= p_G'\\
  p_G &= \frac{p_{G0} + p_G'}{2} = \frac{p_{G0} + p_{C0}}{2} = \frac{0.63}{2} = 0.315\\
  p_C &= 0.315\\
  p_A &= 0.185\\
  p_T &= 0.185\\
\end{align*}

\section{Problem Eight: MEME}
Here we enter some MEME commands to find motifs:\\
\textbf{A.} meme proteins.fasta -protein -mod zoops -minw 10\\
\textbf{B.} meme sequences.fa -rna -mod anr -minw 5 -maxw 10\\
\textbf{C.} meme cores.fasta -dna -mod oops -nmotifs 5 \# -revcomp not specified,
only search forward strand\\

\end{document}

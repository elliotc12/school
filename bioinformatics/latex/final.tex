\documentclass[10pt]{article} % Font size - 10pt, 11pt or 12pt

\usepackage[hmargin=1.25cm, vmargin=1.5cm]{geometry} % Document margins

\usepackage{graphicx}
\usepackage{amsmath}
\usepackage{listings}

\usepackage[usenames,dvipsnames]{xcolor} % Allows the definition of hex colors

% Fonts and tweaks for XeLaTeX
\usepackage{fontspec,xltxtra,xunicode}
\defaultfontfeatures{Mapping=tex-text}
%\setmonofont[Scale=MatchLowercase]{Andale Mono}

% Colors for links, text and headings
\usepackage{hyperref}
\definecolor{linkcolor}{HTML}{506266} % Blue-gray color for links
\definecolor{shade}{HTML}{F5DD9D} % Peach color for the contact information box
\definecolor{text1}{HTML}{2b2b2b} % Main document font color, off-black
\definecolor{headings}{HTML}{701112} % Dark red color for headings
% Other color palettes: shade=B9D7D9 and linkcolor=A40000; shade=D4D7FE and linkcolor=FF0080

\hypersetup{colorlinks,breaklinks, urlcolor=linkcolor, linkcolor=linkcolor} % Set up links and colors

\usepackage{fancyhdr}
\pagestyle{fancy}
\fancyhf{}
% Headers and footers can be added with the \lhead{} \rhead{} \lfoot{} \rfoot{} commands
% Example footer:
%\rfoot{\color{headings} {\sffamily Last update: \today}. Typeset with Xe\LaTeX}

\renewcommand{\headrulewidth}{0pt} % Get rid of the default rule in the header

\usepackage{titlesec} % Allows creating custom \section's

\allowdisplaybreaks

% Format of the section titles
\titleformat{\section}{\color{headings}
\scshape\Large\raggedright}{}{0em}{}[\color{black}\titlerule]

\title{Bioinformatics Final}
\author{Elliott Capek}
\titlespacing{\section}{0pt}{0pt}{5pt} % Spacing around titles

\begin{document}

\maketitle{}

\section{Problem one: Explaining difference}
\textbf{Smith-Waterman vs Needleman-Wunsch}
Needleman-Wunsch is  global alignment algorithm that allows negative
values in its dynamic programming table and always has path traceback
begin at the bottom-right square of the table.\\

Smith-Waterman is a local alignment algorithm that doesn't allow negative
values in its dynamic programming tbale and allows the path traceback to
begin at the highest-scoring square, not necessarily the bottom-right one.\\

\textbf{Paralogs vs Orthologs}
Paralogs: Two genes which come from the same ancestral gene via gene duplication.
Usually in the same species.\\

Orthologs: Two genes which come from the same ancestral gene via speciation.
Therefore in different species.\\

\textbf{RefSeq vs GenBank}
GenBank is a very large database with redundant entries that seeks to have
the most entries possible. It is not curated, and has tons of information. Contains
most of the world's known DNA and RNA sequences.\\

RefSeq is also a large (but less large) database which is curated. It doesn't have
redundant entries, and the entries have a threshold they must meet. Entries for DNA
and protein link together nicely.\\

\textbf{E vs P values}
These are in reference to the BLAST program.\\

The P-value is the probability of a BLAST alignment randomly having a score equal to
or greater than the given score. The lower the better.\\

The E-value is the expected number of hits at the current score due to chance. It is
related to the P-value via an exponential.\\

\section{Problem two: Alignment Scores}
\textbf{A.}
Number of matching aligned BP: 21\\
Number of nonmatching aligned BP: 2\\
Number of gaps: 6\\

Score: 21 - 2 - 6 = 13\\

\textbf{B.}
Number of matching aligned BP: 12\\
Number of nonmatching aligned BP: 4\\
Number of gaps: 2\\

Score: 12 - 2*4 - 2*2 = 0\\

\section{Problem three: BLOSUM matrix}
Here we calculate BLOSUM matrix elements for an alignment with
\textbf{4Cs, 5Vs, 2Ms}.\\

\begin{align*}
  f_{CC} &= \begin{pmatrix} n_C\\ 2\\ \end{pmatrix} = 6\\
  f_{VV} &= \begin{pmatrix} n_V\\ 2\\ \end{pmatrix} = 10\\
  f_{MM} &= \begin{pmatrix} n_M\\ 2\\ \end{pmatrix} = 1\\
  f_{CV} &= n_C \times n_V = 20\\
  f_{CM} &= n_C \times n_M = 8\\
  f_{VM} &= n_V \times n_M = 10\\
  q_{CC} &= \frac{f_{CC}}{f_{CC} + f_{VV} + f_{MM} + f_{CV} + f_{CM} + f_{VM}}
  = \frac{6}{55} = 0.1\\
  q_{VV} &= \frac{f_{VV}}{f_{CC} + f_{VV} + f_{MM} + f_{CV} + f_{CM} + f_{VM}}
  = \frac{10}{55} = 0.18\\
  q_{MM} &= \frac{f_{MM}}{f_{CC} + f_{VV} + f_{MM} + f_{CV} + f_{CM} + f_{VM}}
  = \frac{1}{55} = 0.018\\
  q_{CV} &= \frac{f_{CV}}{f_{CC} + f_{VV} + f_{MM} + f_{CV} + f_{CM} + f_{VM}}
  = \frac{20}{55} = 0.36\\
  q_{CM} &= \frac{f_{CM}}{f_{CC} + f_{VV} + f_{MM} + f_{CV} + f_{CM} + f_{VM}}
  = \frac{8}{55} = 0.145\\
  q_{VM} &= \frac{f_{VM}}{f_{CC} + f_{VV} + f_{MM} + f_{CV} + f_{CM} + f_{VM}}
  = \frac{10}{55} = 0.18\\
  p_C &= q_{CC} + \frac{1}{2}\left(q_{CV} + q_{CM}\right) = 0.35\\
  p_V &= q_{VV} + \frac{1}{2}\left(q_{VC} + q_{VM}\right) = 0.37\\
  p_M &= q_{MM} + \frac{1}{2}\left(q_{MC} + q_{MV}\right) = 0.18\\
\end{align*}

I might have made a mistake in my calculations here since the final probabilities
don't add up to one. My theory should be good though. I'll come back if I have time.\\

\section{Problem four: BLAST stats}
\textbf{A.)}
n = 200\\
m = $10^9$\\
E = 0.1\\

\begin{align*}
  S' &= -\log_2\left(\frac{E}{nm}\right)\\
  &= -\log_2\left(\frac{0.1}{200*10^9}\right)\\
  &= 40.8\\
\end{align*}

\textbf{B.)}
\begin{align*}
  P &= 1 - e^{-E}\\
  &= 1 - e^{-0.1}\\
  &= 0.095\\
\end{align*}

\section{Problem five: Duplex energy}
We put the 5’ and 3’ microRNA sequences into a fasta file and use the
command RNAduplex to find their duplex energies. Example command:
cat duplex.fa | RNAduplex\\

\begin{align*}
  \mbox{hsa-mir-17}&: -22.7\\
  \mbox{hsa-mir-17}&: -18.5\\
  \mbox{hsa-mir-17}&: -21.9\\
\end{align*}

\section{Problem six: BH correction}
First we rank the values:

\begin{align*}
  p_1, p_7, p_2, p_3, p_5, p_4, p_6, p_{10}, p_8, p_9\\
\end{alig*}

Maximum PrN/r under FDR of 0.05: p6, so everything below that is good.\\


\section{Problem seven: }

\section{Problem eight: MEME}
\textbf{A.)}meme family.fasta -protein -mod zoops -minw 12\\
\textbf{A.)}meme proteins.fasta -dna -mod anr -nmotifs 3 -minw 10 -maxw 20 --minites 5\\


\section{Problem nine: RNA structure}
\textbf{A.)} Using the command cat finalRNA.fa | RNAfold -T 27, we get a free energy
of -20.38.\\

\textbf{B.)}I counted a total of 47 BP in the sequence. Of this, 30 are involved
in bonds in the predicted secondary structure, leaving 17 unpaired nucleotides.\\

\section{Problem 10: BLAST formatDB tasks}
\textbf{A.)} formatdb -p T -t ``Protein Database'' -n ProteinDB\\
\textbf{B.)} blastall -p blastn -i transcripts.fa -d ProteinDB -g 5 -E 1\\
-o transcripts.txt

\section{Problem 11: Matching}
\textbf{A.)} Small RNA-Seq\\
\textbf{B.)} RNA-Seq\\
\textbf{C.)} ChIP-Seq\\
\textbf{D.)} Gro-Seq\\
\textbf{E.)} Pol-II ChIP-Seq\\

\section{Problem 12: More matchign}
\textbf{A.)} MEME \\
\textbf{B.)} cuffdiff\\
\textbf{C.)} BLAST\\
\textbf{D.)} TopHat\\
\textbf{E.)} MACS\\
\textbf{F.)} Trinity\\
\textbf{G.)} cufflinks\\

\section{Problem 13: Complexity}
\begin{align*}
  \frac{1}{10}\log_4\left(\frac{10!}{10!}\right) &= 0\\
  \frac{1}{10}\log_4\left(\frac{10!}{5!*5!}\right) &= 0.398\\
\end{align*}

\section{Problem 14: transcriptoin}
I did the program on another computer so I can't paste it in here. I followed
exactly the program outlined in Section 1.2.2. I got: MGIYL* as my protein.\\


\end{document}

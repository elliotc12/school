\documentclass[10pt]{article} % Font size - 10pt, 11pt or 12pt

\usepackage[hmargin=1.25cm, vmargin=1.5cm]{geometry} % Document margins

\usepackage{graphicx}
\usepackage{amsmath}
\usepackage{listings}

\usepackage[usenames,dvipsnames]{xcolor} % Allows the definition of hex colors

% Fonts and tweaks for XeLaTeX
\usepackage{fontspec,xltxtra,xunicode}
\defaultfontfeatures{Mapping=tex-text}
%\setmonofont[Scale=MatchLowercase]{Andale Mono}

% Colors for links, text and headings
\usepackage{hyperref}
\definecolor{linkcolor}{HTML}{506266} % Blue-gray color for links
\definecolor{shade}{HTML}{F5DD9D} % Peach color for the contact information box
\definecolor{text1}{HTML}{2b2b2b} % Main document font color, off-black
\definecolor{headings}{HTML}{701112} % Dark red color for headings
% Other color palettes: shade=B9D7D9 and linkcolor=A40000; shade=D4D7FE and linkcolor=FF0080

\hypersetup{colorlinks,breaklinks, urlcolor=linkcolor, linkcolor=linkcolor} % Set up links and colors

\usepackage{fancyhdr}
\pagestyle{fancy}
\fancyhf{}
% Headers and footers can be added with the \lhead{} \rhead{} \lfoot{} \rfoot{} commands
% Example footer:
%\rfoot{\color{headings} {\sffamily Last update: \today}. Typeset with Xe\LaTeX}

\renewcommand{\headrulewidth}{0pt} % Get rid of the default rule in the header

\usepackage{titlesec} % Allows creating custom \section's

\allowdisplaybreaks

% Format of the section titles
\titleformat{\section}{\color{headings}
\scshape\Large\raggedright}{}{0em}{}[\color{black}\titlerule]

\title{Bioinformatics Midterm}
\author{Elliott Capek}
\titlespacing{\section}{0pt}{0pt}{5pt} % Spacing around titles

\begin{document}

\maketitle{}
\section{One: DP Alignment Algorithms}
Smith-Waterman and Needleman-Wunsch are two very similar dynamic programming
algorithms for finding the optimal alignment of a set of text sequences. The
basic premise of the algorithms is filling in a table that corresponds to every
possible alignment that can be made, then choosing the optimal alignment based
on scoring guidelines. The reason this isn't a ridiculously slow process is
because the table represents not every possible alignment, but the branch points
between different alignments.\\

Smith-Waterman (SW) is used to find global alignments, so the entire sequences matter.
Needleman-Wunsch (NW) is used to find local alignments, where only a subset of the
sequence is to be considered. In order to only consider a subset of the sequence,
two modifications to SW are made to achieve NW. These changes effectively limit
the portion of the DP table being considered. The first change is that no section in
the table can have a score lower than the score of the top-left (start of global
alignment) element. This conceptually means that the alignment can begin anywhere
in the sequence, since having the same score as the global alignment start means
that any table element can be treated as the alignment start.\\

The second change is that the NW alignment can end anywhere, not at the bottom-right
corner of the table. This effectively means that the local alignmemnt can end before
the entire sequence is considered.\\

\section{Two: }


\end{document}

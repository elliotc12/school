\documentclass[10pt]{article} % Font size - 10pt, 11pt or 12pt

\usepackage[hmargin=1.25cm, vmargin=1.5cm]{geometry} % Document margins

\usepackage{graphicx}
\usepackage{amsmath}
\usepackage{listings}

\usepackage[usenames,dvipsnames]{xcolor} % Allows the definition of hex colors

% Fonts and tweaks for XeLaTeX
\usepackage{fontspec,xltxtra,xunicode}
\defaultfontfeatures{Mapping=tex-text}
%\setmonofont[Scale=MatchLowercase]{Andale Mono}

% Colors for links, text and headings
\usepackage{hyperref}
\definecolor{linkcolor}{HTML}{506266} % Blue-gray color for links
\definecolor{shade}{HTML}{F5DD9D} % Peach color for the contact information box
\definecolor{text1}{HTML}{2b2b2b} % Main document font color, off-black
\definecolor{headings}{HTML}{701112} % Dark red color for headings
% Other color palettes: shade=B9D7D9 and linkcolor=A40000; shade=D4D7FE and linkcolor=FF0080

\hypersetup{colorlinks,breaklinks, urlcolor=linkcolor, linkcolor=linkcolor} % Set up links and colors

\usepackage{fancyhdr}
\pagestyle{fancy}
\fancyhf{}
% Headers and footers can be added with the \lhead{} \rhead{} \lfoot{} \rfoot{} commands
% Example footer:
%\rfoot{\color{headings} {\sffamily Last update: \today}. Typeset with Xe\LaTeX}

\renewcommand{\headrulewidth}{0pt} % Get rid of the default rule in the header

\usepackage{titlesec} % Allows creating custom \section's

\allowdisplaybreaks

% Format of the section titles
\titleformat{\section}{\color{headings}
\scshape\Large\raggedright}{}{0em}{}[\color{black}\titlerule]

\title{Bioinformatics Midterm}
\author{Elliott Capek}
\titlespacing{\section}{0pt}{0pt}{5pt} % Spacing around titles

\begin{document}

\maketitle{}
\section{One: DP Alignment Algorithms}
Needleman-Wunsch and Smith-Waterman are two very similar dynamic programming
algorithms for finding the optimal alignment of a set of text sequences. The
basic premise of the algorithms is filling in a table that corresponds to every
possible alignment that can be made, then choosing the optimal alignment based
on scoring guidelines. The reason this isn't a ridiculously slow process is
because the table represents not every possible alignment, but the branch points
between different alignments.\\

Needleman-Wunsch (NW) is used to find global alignments, so the entire sequences
matter. Smith-Waterman (SW) is used to find local alignments, where only a subset of
the sequence is to be considered. In order to only consider a subset of the sequence,
two modifications to NW are made to achieve SW. These changes effectively limit
the portion of the DP table being considered. The first change is that no section in
the table can have a score lower than the score of the top-left (start of global
alignment) element. This conceptually means that the alignment can begin anywhere
in the sequence, since having the same score as the global alignment start means
that any table element can be treated as the alignment start.\\

The second change is that the SW alignment ``ends'' at the highest-scoring value,
not at the bottom-right corner of the table. This effectively means that the local
alignment can end before the entire sequence is considered.\\

\section{Two: BLAST Scores}
\textbf{E-Value:} The expected number of hits that score equal to or higher than a
particular hit for a given query sequence. Conceptually similar to how unreliable a
score is, but also takes into account the size of the database and sequence length.

\begin{align*}
  E &= Knme^{-\lambda S}\\
\end{align*}

Inverse exponential relationship to score means larger scores have lower E-values,
as expected since higher scores correspond to better alignments. Query sequence length
n also important, since longer sequences are more likely to have a random local
alignment in them. Database size m is important for the same reason: more sequences
in the database lead to a higher chance of getting a random hit.\\

The E-value is a weird number in that it isn't for a particular score, but for any
score greater than a particular number. It is useful for assessing how useful a hit
is. If the hit has a high E-value, then there is a good chance that it is a random
hit that shouldn't be trusted.\\

\textbf{p-Value:} The probability of getting a score equal to or higher than the score
of a given hit for a particular query due to chance. Given by:

\begin{align*}
  P(S\geq x) &= 1 - e^{-Knme^{-\lambda x}}\\
\end{align*}

As score threshold x gets larger, the probability gets smaller, as expected. The two
terms in this presumably come from integrating over the Poisson distribution, where
1 is the probability of having less than the maximum score and the term on the right
is the probability of having a score less than x.\\
The p-value is used to calculate E, which is more useful when evaluating the validity
of queries.\\

\section{Three: BLOSUM Matrix}
Here we calculate BLOSUM matrix elements for an alignment with
\textbf{5 R, 2H, 2 L}.\\

\begin{align*}
  f_{RR} &= \begin{pmatrix} n_R\\ 2\\ \end{pmatrix} = 10\\
  f_{HH} &= \begin{pmatrix} n_H\\ 2\\ \end{pmatrix} = 1\\
  f_{LL} &= \begin{pmatrix} n_L\\ 2\\ \end{pmatrix} = 1\\
  f_{HL} &= n_H \times n_L = 4\\
  f_{HR} &= n_H \times n_R = 10\\
  f_{LR} &= n_L \times n_R = 10\\
  q_{RR} &= \frac{f_{RR}}{f_{RR} + f_{HH} + f_{LL} + f_{HL} + f_{HR} + f_{LR}}
  = \frac{10}{36} = 0.277\\
  q_{HH} &= \frac{f_{HH}}{f_{RR} + f_{HH} + f_{LL} + f_{HL} + f_{HR} + f_{LR}}
  = \frac{1}{36} = 0.027\\
  q_{LL} &= \frac{f_{LL}}{f_{RR} + f_{HH} + f_{LL} + f_{HL} + f_{HR} + f_{LR}}
  = \frac{1}{36} = 0.027\\
  q_{HL} &= \frac{f_{HL}}{f_{RR} + f_{HH} + f_{LL} + f_{HL} + f_{HR} + f_{LR}}
  = \frac{4}{36} = 0.111\\
  q_{HR} &= \frac{f_{HR}}{f_{RR} + f_{HH} + f_{LL} + f_{HL} + f_{HR} + f_{LR}}
  = \frac{10}{36} = 0.277\\
  q_{LR} &= \frac{f_{LR}}{f_{RR} + f_{HH} + f_{LL} + f_{HL} + f_{HR} + f_{LR}}
  = \frac{10}{36} = 0.277\\
  p_R &= q_{RR} + \frac{1}{2}\left(q_{RH} + q_{RL}\right) = 0.277 +
  \frac{0.277 + 0.277}{2} = 0.558\\
  p_H &= q_{HH} + \frac{1}{2}\left(q_{HR} + q_{HL}\right) = 0.027 +
  \frac{0.111 + 0.277}{2} = 0.221\\
  p_L &= q_{LL} + \frac{1}{2}\left(q_{LH} + q_{LR}\right) = 0.027 +
  \frac{0.111 + 0.277}{2} = 0.221\\
\end{align*}

Here all q-terms and p-terms sum to one, as desired.


\end{document}

\documentclass[10pt]{article} % Font size - 10pt, 11pt or 12pt

\usepackage[hmargin=1.25cm, vmargin=1.5cm]{geometry} % Document margins

\usepackage{graphicx}
\usepackage{amsmath}
\usepackage{listings}

\usepackage[usenames,dvipsnames]{xcolor} % Allows the definition of hex colors

% Fonts and tweaks for XeLaTeX
\usepackage{fontspec,xltxtra,xunicode}
\defaultfontfeatures{Mapping=tex-text}
%\setmonofont[Scale=MatchLowercase]{Andale Mono}

% Colors for links, text and headings
\usepackage{hyperref}
\definecolor{linkcolor}{HTML}{506266} % Blue-gray color for links
\definecolor{shade}{HTML}{F5DD9D} % Peach color for the contact information box
\definecolor{text1}{HTML}{2b2b2b} % Main document font color, off-black
\definecolor{headings}{HTML}{701112} % Dark red color for headings
% Other color palettes: shade=B9D7D9 and linkcolor=A40000; shade=D4D7FE and linkcolor=FF0080

\hypersetup{colorlinks,breaklinks, urlcolor=linkcolor, linkcolor=linkcolor} % Set up links and colors

\usepackage{fancyhdr}
\pagestyle{fancy}
\fancyhf{}
% Headers and footers can be added with the \lhead{} \rhead{} \lfoot{} \rfoot{} commands
% Example footer:
%\rfoot{\color{headings} {\sffamily Last update: \today}. Typeset with Xe\LaTeX}

\renewcommand{\headrulewidth}{0pt} % Get rid of the default rule in the header

\usepackage{titlesec} % Allows creating custom \section's

\allowdisplaybreaks

% Format of the section titles
\titleformat{\section}{\color{headings}
\scshape\Large\raggedright}{}{0em}{}[\color{black}\titlerule]

\title{Bioinformatics Assignment 3}
\author{Elliott Capek}
\titlespacing{\section}{0pt}{0pt}{5pt} % Spacing around titles

\begin{document}

\maketitle{}

\section{Problem One: Sequence Complexity}
We use the Wooton-Federhen complexity score to quantify the complexity of four nucleotide sequences.
The formula is this:

\begin{align*}
  C_{wf} &= \frac{1}{N}\log_4\left(\frac{N!}{n_A!n_C!n_T!n_G!}\right)\\
\end{align*}

Complexity Scores:
\begin{align*}
  &ATATATATATATATATATAT: N=20, n_A=10, n_T=10 \rightarrow C_{wf}
  = \frac{1}{20}\log_4\left(\frac{20!}{10!10!}\right) = 0.43\\
  &GCGCGCGCGCGCGCGCGCGC: N=20, n_G=10, n_C=10 \rightarrow C_{wf}
  = \frac{1}{20}\log_4\left(\frac{20!}{10!10!}\right) = 0.43\\
  &ACGTACGTACGTACGTACGT: N=20, n_A=5, n_T=5 n_G=5, n_C=5 \rightarrow C_{wf}
  = \frac{1}{20}\log_4\left(\frac{20!}{5!5!5!5!}\right) = 0.83\\
  &AAAAAAAAAAAAAAAAAAAA: N=20, n_A=20 \rightarrow C_{wf}
  = \frac{1}{20}\log_4\left(\frac{20!}{20!}\right) = 0\\
\end{align*}

These results are as we expect. The first two sequences are medium complexity, and the one with
an even distribution of the four nucleotides is the highest complexity.\\

\section{Problem Two: }
Here we use the Benjamini-Hochberg procedure for an FDR of 0.01. First, we rank the p-values
ascendingly:

\begin{align*}
  &p_1 \leq p_7 \leq p_2 \leq p_3 \leq p_4 \leq p_5 \leq p_6 \leq p_{10} \leq p_8 \leq p_9\\
  &1\hspace{0.65cm}2\hspace{0.65cm}3\hspace{0.65cm}4\hspace{0.65cm}5\hspace{0.65cm}6\hspace{0.65cm}
  7\hspace{0.65cm}8\hspace{0.65cm}9\hspace{0.65cm}10\\
\end{align*}

We find the maximum significant rank through:

\begin{align*}
  \frac{p_{r*}N}{r*} \leq FDR\\
  \frac{p_{r*}}{r*} \leq 0.001\\
\end{align*}

\begin{align*}
  \frac{0.001 * 10}{1} \leq \frac{0.001 * 10}{2} \leq \frac{0.005 * 10}{3} \leq
  \frac{0.01 * 10}{4} \leq \frac{0.01 * 10}{5} \leq \frac{0.05 * 10}{6} \leq
  \frac{0.02 * 10}{7} \le \frac{0.03 * 10}{8} \leq \frac{0.2 * 10}{9} \leq
  \frac{0.5 * 10}{10}\\
\end{align*}

\begin{align*}
  0.01 - 0.005 - \frac{0.0}5{3} \leq
  \frac{0.01 * 10}{4} \leq \frac{0.01 * 10}{5} \leq \frac{0.05 * 10}{6} \leq
  \frac{0.02 * 10}{7} \le \frac{0.03 * 10}{8} \leq \frac{0.2 * 10}{9} \leq
  \frac{0.5 * 10}{10}\\
\end{align*}

we find that the $p_1$ and $p_7$ are the only p-values that make this cut, so they are the only
ones we consider significant.\\

\section{Problem 3: }
We download the sonic hedgehog gene and run \texttt{hmmscan} to find the location of likely domains.
We use Pfam database human domains from Lab 9 to do this. When we run \texttt{hmmscan} we get
the following (trimmed) results:

\begin{verbatim}
#                                      -------------- this domain ---
# target name        accession   tlen  i-Evalue  score  bias   description of target
#------------------- ---------- -----  -------- ------ ----- -----------------------
HH_signal            PF01085.15   161   2.7e-85  283.8   0.0   Hedgehog amino-terminal
Hint                 PF01079.17   214   7.8e-72  241.1   0.0   Hint module
Peptidase_M15_3      PF08291.8    111     0.001   19.0   0.0   Peptidase M15
Hint_2               PF13403.3    143         3    8.0   0.0   Hint domain
\end{verbatim}

Here we see the top two domains, the ``Hedgehog amino-terminal'' and ``Hint module'' have
pretty small E-values compared to the other two. The tlens are similar for all of these
domains, so we can compare their E-values and say the top two domains are significant and
the bottom two aren't.\\

\section{Problem 4: Phylogenetic Tree NN interchange}
Here we use the NNI procedure to come up with new ways of representing the following tree:

\begin{verbatim}
A    B
I----I
C    D
\end{verbatim}

For any four-subtree tree, there are only two ways it can be rearranged. These are the only
two possible rearrangements:

\begin{verbatim}
A    C        A    C
I----I  and   I----I
B    D        D    B
\end{verbatim}

Each tree can be simplified to a pairing between A and one of the other three subtrees, since
you only need one pair to define a tree of size 4. There are three possible pairs involving A,
so three possible trees.\\

\section{Problem Five: jnet domains}
Here we use the \texttt{jnet} program to find the predicted secondary structure of human
beta-defensin-1 sequence. Here are the results:\\

\begin{verbatim}
 RES	: MRTSYLLLFTLCLLLSEMASGGNFLTGLGHRSDHYNCVSSGGQCLYSACPIFTKIQGTCYRGKAK
 ALIGN	: ---HHHHHHHHHHHHHHHH----EEE---------EEE-----EE-E----EEEE----------
 CONF	: 83146888899999876405884010246668753030587632101255231211200023322
 FINAL	: ---HHHHHHHHHHHHHHHH----EEE---------EEE-----EEEE----EEEE----------
\end{verbatim}

Here we see there is likely a large alpha-helix (H) near the N-terminus and 4
beta-sheets evenly spaced throughout.\\

This compares pretty well to the RCSB structure prediction, which predicts an
alpha-helix near the N-term and three alpha-helices evenly spaced through the
protein.\\

\section{Problem Six: ChIPSeq}
To find the p-value of the ChIPSeq peak we are given, we use the following formula:\\

\begin{align*}
  \mbox{p-val} &= 1 - F(N_{ChIP}-1,N_{ChIP}+N_{cont}, 0.5)\\
  \mbox{p-val} &= 1 - F(46, 65, 0.5)\\
  &= 0.000211 = 2.1*10^{-4}\\
\end{align*}

This is a fairly small p-value, which means this is likely a legitimate peak.\\
\end{document}

\documentclass[10pt]{article} % Font size - 10pt, 11pt or 12pt

\usepackage[hmargin=1.25cm, vmargin=1.5cm]{geometry} % Document margins

\usepackage{graphicx}
\usepackage{amsmath}
\usepackage{marvosym} % Required for symbols in the colored box
\usepackage{ifsym} % Required for symbols in the colored box

\usepackage[usenames,dvipsnames]{xcolor} % Allows the definition of hex colors

% Fonts and tweaks for XeLaTeX
\usepackage{fontspec,xltxtra,xunicode}
\defaultfontfeatures{Mapping=tex-text}
%\setmonofont[Scale=MatchLowercase]{Andale Mono}

% Colors for links, text and headings
\usepackage{hyperref}
\definecolor{linkcolor}{HTML}{506266} % Blue-gray color for links
\definecolor{shade}{HTML}{F5DD9D} % Peach color for the contact information box
\definecolor{text1}{HTML}{2b2b2b} % Main document font color, off-black
\definecolor{headings}{HTML}{701112} % Dark red color for headings
% Other color palettes: shade=B9D7D9 and linkcolor=A40000; shade=D4D7FE and linkcolor=FF0080

\hypersetup{colorlinks,breaklinks, urlcolor=linkcolor, linkcolor=linkcolor} % Set up links and colors

\usepackage{fancyhdr}
\pagestyle{fancy}
\fancyhf{}
% Headers and footers can be added with the \lhead{} \rhead{} \lfoot{} \rfoot{} commands
% Example footer:
%\rfoot{\color{headings} {\sffamily Last update: \today}. Typeset with Xe\LaTeX}

\renewcommand{\headrulewidth}{0pt} % Get rid of the default rule in the header

\usepackage{titlesec} % Allows creating custom \section's

% Format of the section titles
\titleformat{\section}{\color{headings}
\scshape\Large\raggedright}{}{0em}{}[\color{black}\titlerule]

\title{Classical Mechanics Assignment 3}
\author{Elliott Capek}
\titlespacing{\section}{0pt}{0pt}{5pt} % Spacing around titles

\begin{document}

\maketitle{}

\section{Problem One}
\textbf{Consider an object that is thrown vertically up with initial speed $v_0$ in a linear medium.} \\

\textbf{a. Measuring y upward from the point of release, write expressions for the object velocity and position as a function of time.}

\begin{align*}
  m\ddot{y}(t) &= -mg - b\dot{y}(t)\\
  m\ddot{y}(t) &= -b\big(\frac{mg}{b} + \dot{y}(t))\\
  m\ddot{y}(t) &= -b\big(\dot{y}_{term} + \dot{y}(t))
  \hspace{1in}\ddot{y} = 0 \rightarrow \dot{y}_{term} = \frac{mg}{b}\\
  m\ddot{y}(t) &= -b\phi(t)
  \hspace{1.6in} \phi(t) = \dot{y}_{term} + \dot{y}(t)\\
  m\ddot{\phi(t)} &= -b\phi(t)
  \hspace{1.6in} \dot{\phi(t)} = \ddot{y}(t)\\
  \phi(t) &= Ae^{-\frac{bt}{m}}\\
  \dot{y}(t) &= Ae^{-\frac{b}{m}t} - \dot{y}_{term}
  \hspace{1.6in} \dot{y}(0) = \dot{y}_0 = A - \dot{y}_{term}\\
  \dot{y}(t) &= (\dot{y}_0 + \dot{y}_{term})e^{-\frac{b}{m}t} - \dot{y}_{term}\\
  \dot{y}(t) &= \dot{y}_0e^{-\frac{b}{m}t} + \dot{y}_{term}(e^{-\frac{b}{m}t} - 1)\\
  y(t) &= -\frac{m}{b}e^{-\frac{b}{m}t}\Big(\dot{y}_0 + \dot{y}_{term} \Big) - \dot{y}_{term}t\\
\end{align*}

\textbf{b. Find the time at which the object reaches its highest point and its position at that time.}

\begin{align*}
  \dot{y}(t) &= 0 = \dot{y}_0e^{-\frac{b}{m}t} + \dot{y}_{term}(e^{-\frac{b}{m}t} - 1)\\
  \dot{y}_{term}(1 - e^{-\frac{b}{m}t}) &= \dot{y}_0e^{-\frac{b}{m}t}\\
  \dot{y}_{term} - \dot{y}_{term}e^{-\frac{b}{m}t} &= \dot{y}_0e^{-\frac{b}{m}t}\\
  \dot{y}_{term}e^{\frac{b}{m}t} &= \dot{y}_0 + \dot{y}_{term}\\
  e^{\frac{b}{m}t} &= \frac{\dot{y}_0 + \dot{y}_{term}}{\dot{y}_{term}}\\
  t_{max} &= \frac{m}{b}\log{\frac{\dot{y}_0 + \dot{y}_{term}}{\dot{y}_{term}}}\\
  y(t_{max}) &= -\frac{m}{b}e^{-\frac{b}{m}\frac{m}{b}\log{\frac{\dot{y}_0 + \dot{y}_{term}}{\dot{y}_{term}}}}\Big(\dot{y}_0 + \dot{y}_{term} \Big) - \dot{y}_{term}\frac{m}{b}\log{\frac{\dot{y}_0 + \dot{y}_{term}}{\dot{y}_{term}}}\\
    y(t_{max}) &= -\frac{m}{b\frac{\dot{y}_0 + \dot{y}_{term}}{\dot{y}_{term}}}\Big(\dot{y}_0 + \dot{y}_{term} \Big) - \dot{y}_{term}\frac{m}{b}\log{\frac{\dot{y}_0 + \dot{y}_{term}}{\dot{y}_{term}}}\\
\end{align*}


\textbf{c. Show that the limiting case for very small drag coefficient coincides with the solution for the frictionless motion.}\\
For frictionless motion:
\begin{align*}
  \ddot{y}(t) &= -g\\
  \dot{y}(t) &= \dot{y}_0 - gt\\
  y(t) &= \dot{y}_0t - \frac{1}{2}gt^2\\
  \dot{y}(t_{max}) &= 0 \rightarrow t_{max} = \frac{\dot{y}}{g}\\
  y_{max} &= \frac{\dot{y}_0^2}{2g}\\
\end{align*}

For motion in a linear medium when $\lim_{b\rightarrow0}$:

\begin{align*}
  \lim_{b\rightarrow0}y(t_{max}) &= \lim_{b->0} \hspace{0.2cm} -\frac{m}{b\frac{\dot{y}_0 + \dot{y}_{term}}{\dot{y}_{term}}}\Big(\dot{y}_0 + \dot{y}_{term} \Big) - \dot{y}_{term}\frac{m}{b}\log{\frac{\dot{y}_0 + \dot{y}_{term}}{\dot{y}_{term}}}\\
  \lim_{b\rightarrow0}y(t_{max}) &= \lim_{b->0} \hspace{0.2cm} -\frac{m}{b\frac{\dot{y}_0 + \frac{mg}{b}}{\frac{mg}{b}}}\Big(\dot{y}_0 + \frac{mg}{b} \Big) - \frac{mg}{b}\frac{m}{b}\log{\frac{\dot{y}_0 + \frac{mg}{b}}{\frac{mg}{b}}}\\
  \lim_{b\rightarrow0}y(t_{max}) &= \lim_{b->0} \hspace{0.2cm} -\frac{m}{b\frac{\dot{y}_0 + \frac{mg}{b}}{\frac{mg}{b}}}\Big(\dot{y}_0 + \frac{mg}{b} \Big) - \frac{mg}{b}\frac{m}{b}\log{\frac{\dot{y}_0 + \frac{mg}{b}}{\frac{mg}{b}}}\\
\end{align*}


\vspace{1 cm}

\section{Problem Two}
\textbf{A point mass of mass m slides down an inclined plane under the influence of gravity. If the motion is resisted by a force $F=kmv^2$, show that the time required to move a distance d after starting from rest is:}

\begin{equation*}
  t = \frac{\cosh^{-1}(e^{kd})}{\sqrt{kg\sin(\Theta)}}\\
\end{equation*}

\textbf{Where $\Theta$ is the angle of inclination of the plane.} \\ \\





\vspace{1 cm}

\section{Problem Three}
\textbf{A badly designed rocket has an initial mass of 7x104 kg and on firing it burns its fuel at a rate of 250 kg/s. The exhaust speed is 2500 m/s. If the rocket has a vertical ascent from rest on the Earth, how long after the rocket engines fire will the rocket lift off?} \\ \\
\vspace{1 cm}

\section{Problem Four}
\textbf{A projectile of mass M explodes while in flight at speed v into three fragments. The first fragment, with mass m1=M/2 travels in the original direction of the projectile; the second, with mass m2=M/6 travels in the opposite direction, and the third comes to rest. The energy released in the explosion is equal to five times the kinetic energy of the projectile just before exploding. What are the velocities of the first two fragments?} \\ \\
\vspace{1 cm}

\section{Problem Five}
\textbf{Find the center of mass of a uniform hemispherical shell of inner and outer radii a and b and mass M. Comment on the limiting cases. (5 bonus points)} \\ \\
\vspace{1 cm}

\end{document}

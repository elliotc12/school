\documentclass[10pt]{article} % Font size - 10pt, 11pt or 12pt

\usepackage[hmargin=1.25cm, vmargin=1.5cm]{geometry} % Document margins

\usepackage{graphicx}
\usepackage{amsmath}
\usepackage{marvosym} % Required for symbols in the colored box
\usepackage{ifsym} % Required for symbols in the colored box

\usepackage[usenames,dvipsnames]{xcolor} % Allows the definition of hex colors

% Fonts and tweaks for XeLaTeX
\usepackage{fontspec,xltxtra,xunicode}
\defaultfontfeatures{Mapping=tex-text}
%\setmonofont[Scale=MatchLowercase]{Andale Mono}

% Colors for links, text and headings
\usepackage{hyperref}
\definecolor{linkcolor}{HTML}{506266} % Blue-gray color for links
\definecolor{shade}{HTML}{F5DD9D} % Peach color for the contact information box
\definecolor{text1}{HTML}{2b2b2b} % Main document font color, off-black
\definecolor{headings}{HTML}{701112} % Dark red color for headings
% Other color palettes: shade=B9D7D9 and linkcolor=A40000; shade=D4D7FE and linkcolor=FF0080

\hypersetup{colorlinks,breaklinks, urlcolor=linkcolor, linkcolor=linkcolor} % Set up links and colors

\usepackage{fancyhdr}
\pagestyle{fancy}
\fancyhf{}
% Headers and footers can be added with the \lhead{} \rhead{} \lfoot{} \rfoot{} commands
% Example footer:
%\rfoot{\color{headings} {\sffamily Last update: \today}. Typeset with Xe\LaTeX}

\renewcommand{\headrulewidth}{0pt} % Get rid of the default rule in the header

\usepackage{titlesec} % Allows creating custom \section's

\allowdisplaybreaks

% Format of the section titles
\titleformat{\section}{\color{headings}
\scshape\Large\raggedright}{}{0em}{}[\color{black}\titlerule]

\title{Classical Mechanics Assignment 6}
\author{Elliott Capek}
\titlespacing{\section}{0pt}{0pt}{5pt} % Spacing around titles

\begin{document}

\maketitle{}

\section{Problem One: Ball on an Inclined Plane}
\textbf{Write the Lagrangian for a cylinder (mass m, radius R, and moment of inertia I) that rolls without slipping straight down an inclined plane at an angle $\alpha$ from the horizontal. Use as your generalized coordinates the cylinder’s distance x measured down the plane from its starting point. Solve the Lagrange equation for the cylinder’s acceleration $\ddot{x}$. Remember that $T = \frac{1}{2} mv^2 + \frac{1}{2} I\omega^2$ where $v$ is the velocity of the center of mass and $\omega$ is the cylinder’s angular velocity.}

\begin{align*}
  x' &= x\cos(\alpha)\\
  y' &= -x\sin(\alpha)\\
  \omega &= \frac{v}{2\pi R}\\
  \vspace{1cm}\\
  U &= mgy' = -mgx\sin(\alpha)\\
  T &= \frac{1}{2}mv^2 + \frac{1}{2}I\omega^2\\
  &= \frac{1}{2}m\left(\dot{x}'^2 + \dot{y}'^2\right) + \frac{1}{2}I\left(\frac{\dot{x}'^2 + \dot{y}'^2}{2\pi R}\right)\\
  &= \frac{1}{2}m\left(\dot{x}^2\sin^2(\alpha) + \dot{x}^2\cos^2(\alpha)\right) + \frac{1}{2}I\left(\frac{\dot{x}^2\sin^2(\alpha) + \dot{x}^2\cos^2(\alpha)}{4\pi^2R^2}\right)\\
  &= m\left(\dot{x}^2\right) + I\left(\frac{\dot{x}^2}{4\pi^2R^2}\right)\\
  &= \dot{x}^2\left(\frac{I + 4\pi^2R^2m}{4\pi^2R^2}\right)\\
  L &= \dot{x}^2\left(\frac{I + 4\pi^2R^2m}{4\pi^2R^2}\right) + mgx\sin(\alpha)\\
  \vspace{1cm}\\
  &\frac{dL}{dx} - \frac{d}{dt}\frac{dL}{d\dot{x}} = 0\\
  &mg\sin(\alpha) - \ddot{x}\left(\frac{I + 4\pi^2R^2m}{\pi^2R^2}\right) = 0\\
  \ddot{x} &= \frac{\pi^2R^2mg\sin(\alpha)}{I + 4\pi^2R^2m}\\
  x &= \frac{\pi^2R^2mg\sin(\alpha)}{2I + 8\pi^2R^2m}t^2
\end{align*}

\section{Problem Two: Bead on a Spiral}
\textbf{A smooth wire is bent into the shape of a helix, with cylindrical coordinates $\rho=R$ and $z=\lambda\phi$, where $R$ and $\lambda$ are constants and the z axis is vertically up (and gravity is vertically down). Using z as your generalized coordinate, write down the Lagrangian for a bead of mass m threaded on the wire. Find the Lagrange equation and the bead’s vertical acceleration $\ddot{z}$. In the limit that $R → 0$, what is $\ddot{z}$? Does this make sense?}

\begin{align*}
  v &= \sqrt{\dot{\rho}^2 + \rho^2\dot{\phi}^2 + \dot{z}^2}\\
  &= \sqrt{\rho^2\frac{\dot{z}^2}{\lambda^2} + \dot{z}^2}\\
  \vspace{1cm}\\
  T &= \frac{1}{2}mv^2\\
  &= \frac{1}{2}m\left(\rho^2\frac{\dot{z}^2}{\lambda^2} + \dot{z}^2\right)\\
  U &= mgz\\
  L &= \frac{1}{2}m\left(\rho^2\frac{\dot{z}^2}{\lambda^2} + \dot{z}^2\right) - mgz\\
  \vspace{1cm}\\
  &\frac{dL}{dz} - \frac{d}{dt}\frac{dL}{d\dot{z}} = 0\\
  -mg - \ddot{z}m\left(\frac{\rho^2}{\lambda^2}+1\right) &= 0\\
  \ddot{z} &= \frac{-g}{\frac{\rho^2}{\lambda^2}+1}\\
\end{align*}

\section{Problem Three: }
\textbf{Write the Lagrangian for a cylinder (mass m, radius r, and moment of inertia $I=\frac{mr^2}{2}$) that rolls without slipping in the bottom of a circular shaped hole of equation $z = −\sqrt{R^2 − x^2}$ , where z is the vertical axis (see figure below). Find the angular frequency of small oscillations about the bottom of the hole.}

\end{document}

\documentclass[10pt]{article} % Font size - 10pt, 11pt or 12pt

\usepackage[hmargin=1.25cm, vmargin=1.5cm]{geometry} % Document margins

\usepackage{marvosym} % Required for symbols in the colored box
\usepackage{ifsym} % Required for symbols in the colored box

\usepackage[usenames,dvipsnames]{xcolor} % Allows the definition of hex colors

% Fonts and tweaks for XeLaTeX
\usepackage{fontspec,xltxtra,xunicode}
\defaultfontfeatures{Mapping=tex-text}
%\setmonofont[Scale=MatchLowercase]{Andale Mono}

% Colors for links, text and headings
\usepackage{hyperref}
\definecolor{linkcolor}{HTML}{506266} % Blue-gray color for links
\definecolor{shade}{HTML}{F5DD9D} % Peach color for the contact information box
\definecolor{text1}{HTML}{2b2b2b} % Main document font color, off-black
\definecolor{headings}{HTML}{701112} % Dark red color for headings
% Other color palettes: shade=B9D7D9 and linkcolor=A40000; shade=D4D7FE and linkcolor=FF0080

\hypersetup{colorlinks,breaklinks, urlcolor=linkcolor, linkcolor=linkcolor} % Set up links and colors

\usepackage{fancyhdr}
\pagestyle{fancy}
\fancyhf{}
% Headers and footers can be added with the \lhead{} \rhead{} \lfoot{} \rfoot{} commands
% Example footer:
%\rfoot{\color{headings} {\sffamily Last update: \today}. Typeset with Xe\LaTeX}

\renewcommand{\headrulewidth}{0pt} % Get rid of the default rule in the header

\usepackage{titlesec} % Allows creating custom \section's

% Format of the section titles
\titleformat{\section}{\color{headings}
\scshape\Large\raggedright}{}{0em}{}[\color{black}\titlerule]

\title{Classical Mechanics Assignment One}
\author{Elliott Capek}
\titlespacing{\section}{0pt}{0pt}{5pt} % Spacing around titles

\begin{document}

\maketitle{}

\section{Problem One}
\textbf{What do you expect from this course?} \\ \\
       {From this course I want to learn the physics techniques for modeling more complicated systems than what was taught in the introductory physics sequence. I want to get a better understanding for how physical modeling works and get a better intution for when to use particular techniques to model physical systems. I'd also like to learn how to use advanced techniques, like the Lagrangian, to characterizing physical systems.}

\vspace{1 cm}

\section{Problem Two}
\textbf{What do you expect from the instructor?} \\ \\
       {I expect the professor to answer questions during lecture and be available for office hours for help and clarification on homework assignments.}

\vspace{1 cm}       

\section{Problem Three}
\textbf{What do you expect from your classmates?} \\ \\
       {I expect my classmates to not be disruptive during lecture and work to create an environment conducive to learning. I expect them not to ask for homework solutions, but to ask for help on homework when needed.}

\vspace{1 cm}       

\section{Problem Four}
\textbf{How prepared do you feel that you are for this class?} \\ \\
       {I feel adequately prepared. I have taken the Paradigms series in the past year, in addition to the two linear algebra courses and calculus series recommended for this course. I am not worried about the material, and know that if anything challenging comes up, I can always do research, ask my classmates or the professor for help.}

\end{document}

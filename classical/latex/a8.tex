\documentclass[10pt]{article} % Font size - 10pt, 11pt or 12pt

\usepackage[hmargin=1.25cm, vmargin=1.5cm]{geometry} % Document margins

\usepackage{graphicx}
\usepackage{amsmath}
\usepackage{marvosym} % Required for symbols in the colored box
\usepackage{ifsym} % Required for symbols in the colored box

\usepackage[usenames,dvipsnames]{xcolor} % Allows the definition of hex colors

% Fonts and tweaks for XeLaTeX
\usepackage{fontspec,xltxtra,xunicode}
\defaultfontfeatures{Mapping=tex-text}
%\setmonofont[Scale=MatchLowercase]{Andale Mono}

% Colors for links, text and headings
\usepackage{hyperref}
\definecolor{linkcolor}{HTML}{506266} % Blue-gray color for links
\definecolor{shade}{HTML}{F5DD9D} % Peach color for the contact information box
\definecolor{text1}{HTML}{2b2b2b} % Main document font color, off-black
\definecolor{headings}{HTML}{701112} % Dark red color for headings
% Other color palettes: shade=B9D7D9 and linkcolor=A40000; shade=D4D7FE and linkcolor=FF0080

\hypersetup{colorlinks,breaklinks, urlcolor=linkcolor, linkcolor=linkcolor} % Set up links and colors

\usepackage{fancyhdr}
\pagestyle{fancy}
\fancyhf{}
% Headers and footers can be added with the \lhead{} \rhead{} \lfoot{} \rfoot{} commands
% Example footer:
%\rfoot{\color{headings} {\sffamily Last update: \today}. Typeset with Xe\LaTeX}

\renewcommand{\headrulewidth}{0pt} % Get rid of the default rule in the header

\usepackage{titlesec} % Allows creating custom \section's

\allowdisplaybreaks

% Format of the section titles
\titleformat{\section}{\color{headings}
\scshape\Large\raggedright}{}{0em}{}[\color{black}\titlerule]

\title{Classical Mechanics Assignment 8}
\author{Elliott Capek}
\titlespacing{\section}{0pt}{0pt}{5pt} % Spacing around titles

\begin{document}

\maketitle{}

\section{Problem One: Springy Pendulum}
\textbf{A pendulum consists of a mass m suspended by a massless spring with un-extended length $b$ and spring constant $k$. Find Hamilton’s equations of motion.}

\begin{align*}
  x &= (b+\ell)\sin\theta\\
  y &= -(b+\ell)\cos\theta\\
  \dot{x} &= \dot{\ell}\sin\theta + (b+\ell)\cos\theta\dot{\theta}\\
  \dot{y} &= -\dot{\ell}\cos\theta + (b+\ell)\sin\theta\dot{\theta}\\
  \vspace{1cm}\\
  T &= \frac{1}{2}m\left(\dot{x}^2 + \dot{y}^2\right)\\
  &= \frac{1}{2}m\left(\dot{\ell}^2 + (b+\ell)^2\dot{\theta}^2\right)\\
  U &= mgy = -mg(b + \ell)\cos\theta + \frac{1}{2}k\ell^2\\
  \mathcal{L} &= \frac{1}{2}m\left(\dot{\ell}^2 + (b+\ell)^2\dot{\theta}^2\right) + mg(b + \ell)\cos\theta - \frac{1}{2}k\ell^2\\
  \vspace{1cm}\\
  p_{\ell} &= \frac{d\mathcal{L}}{d\dot{\ell}} = m\dot{\ell}\\
  p_{\theta} &= \frac{d\mathcal{L}}{d\dot{\theta}} = m(b + \ell)\dot{\theta}\\
  \mathcal{H} &= m\dot{\ell}^2 + m(b + \ell)\dot{\theta}^2 - \frac{1}{2}m\left(\dot{\ell}^2 + (b+\ell)^2\dot{\theta}^2\right) - mg(b + \ell)\cos\theta + \frac{1}{2}k\ell^2\\
  &= \frac{p_{\ell}^2}{m} + \frac{p_{\theta}^2}{m(b + \ell)} - \frac{1}{2}m\left(\frac{p_{\ell}^2}{m^2} + \frac{p_{\theta}^2}{m^2}\right) - mg(b + \ell)\cos\theta + \frac{1}{2}k\ell^2\\
  \vspace{1cm}\\
  \dot{p_{\theta}} &= -\frac{d\mathcal{H}}{d\theta}\\
  &= m\ddot{\theta} = -mg(b + \ell)\sin\theta\\
  \dot{p_{\ell}} &= -\frac{d\mathcal{H}}{d\ell}\\
  &= m\ddot{\ell} = -m\dot{\theta}^2 + m\left(b+\ell\right)\dot{\theta}^2 + mg\cos\theta + 2k\ell\\
\end{align*}

This was an interesting usage of Hamiltonian mechanics to solve a very weird physical system. This would have been a difficult problem to solve using Newtonian mechanics. It probably would have been easier to solve with Lagrangian mechanics due to there being fewer steps, but Hamiltonian worked well. The end result is a set of coupled motion equations with no obvious solution, which is in line with what we were told about in class.

\section{Problem Two: Non-harmonic Oscillator}
\textbf{The potential for an non-harmonic oscillator is:}

\begin{equation*}
  U(x) = \frac{kx^2}{2} + \frac{bx^4}{4}\\
\end{equation*}

\textbf{where k and b are positive constants. Find Hamilton’s equations of motion. No final discussion needed.}

\begin{align*}
  T &= \frac{1}{2}m\dot{x}^2\\
  \mathcal{L} &= \frac{1}{2}m\dot{x}^2 - \frac{kx^2}{2} - \frac{bx^4}{4}\\
  \vspace{1cm}
  p &= \frac{d\mathcal{L}}{d\dot{x}} = m\dot{x}\\
  \mathcal{H} &= m\dot{x} - \frac{1}{2}m\dot{x}^2 + \frac{kx^2}{2} + \frac{bx^4}{4}\\
  \dot{p} &= -\frac{d\mathcal{H}}{dx} = -kx - bx^3\\
\end{align*}

We combine this equation with our knowledge of the definition of momentum $p = m\dot{x}$ to find the Hamilton Equation of Motion:

\begin{align*}
  m\ddot{x} &= -kx - bx^3\\
\end{align*}

\section{Problem Three: Two masses Three springs}
\textbf{Reconsider the problem of the two masses and three springs (seen in class) in the event that the three springs all have different force constants (but the masses are equal). Find the two characteristic frequencies and compare the magnitudes with the natural frequencies of the two oscillators in the absence of coupling.}

\begin{align*}
  T &= \frac{1}{2}m\dot{x}^2 + \frac{1}{2}m\dot{y}^2\\
  U &= \frac{1}{2}k_1x^2 + \frac{1}{2}k_3y^2 + \frac{1}{2}k_2\left(y - x\right)^2\\
  \mathcal{L} &= \frac{1}{2}\left(k_1x^2 + k_3y^2 + k_2\left(y-x\right)^2 + m\dot{x}^2 + m\dot{y}^2\right)\\
\end{align*}

By solving the Euler-Lagrange equation for this Lagrangian we find the following motion equations:

\begin{align*}
  m\ddot{x} &= k_1x - k_2\left(y-x\right)\\
  m\ddot{y} &= k_3y + k_2\left(y-x\right)\\
\end{align*}

We transfer these motion equations to matrix form and solve their determinant:

\begin{align*}
  det
  \begin{pmatrix}
    k_1+k_2-\omega^2m & -k_2\\
    -k_2 & k_2 + k_3-\omega^2m\\
  \end{pmatrix}
  = 0\\
  \left(k_1+k_2-\omega^2m\right)\left(k_2 + k_3 - \omega^2m\right) - k_2^2 = 0\\
  &\omega^4m^2 + \omega^2m\left(-k_1-2k_2-k_3\right) + \left(k_1k_2 + k_1k_3 + k_2k_3\right) = 0\\
  \omega^2 &= \frac{m\left(k_1+2k_2+k_3\right) \pm \sqrt{m^2\left(k_1+2k_2+k_3\right)^2 - 4m^2\left(k_1k_2+k_1k_3+k_2k_3\right)}}{2m^2}\\
  \omega_1 &= \sqrt{\frac{m\left(k_1+2k_2+k_3\right) + \sqrt{m^2\left(k_1+2k_2+k_3\right)^2 - 4m^2\left(k_1k_2+k_1k_3+k_2k_3\right)}}{2m^2}}\\
  &= \sqrt{\frac{\left(k_1+2k_2+k_3\right) + \sqrt{k_1^2+4k_2^2+k_3^2-2k_1k_3}}{2m}}\\
  \omega_2 &= \sqrt{\frac{m\left(k_1+2k_2+k_3\right) - \sqrt{m^2\left(k_1+2k_2+k_3\right)^2 - 4m^2\left(k_1k_2+k_1k_3+k_2k_3\right)}}{2m^2}}\\
  &= \sqrt{\frac{k_1+2k_2+k_3 - \sqrt{k_1^2+4k_2^2+k_3^2-2k_1k_3}}{2m}}\\
\end{align*}

We can compare the uncoupled frequencies (where $k_2=0$) to these:

\begin{align*}
  \lim_{k_2\rightarrow0}\omega_1 &= \sqrt{\frac{\left(k_1+k_3\right) + \sqrt{k_1^2+k_3^2-2k_1k_3}}{2m}} = \sqrt{\frac{k_1}{m}}\\
  \lim_{k_2\rightarrow0}\omega_2 &= \sqrt{\frac{\left(k_1+k_3\right) - \sqrt{k_1^2+k_3^2-2k_1k_3}}{2m}} = \sqrt{\frac{k_3}{m}}\\
\end{align*}

This is just what we expect, a system of two independent uncoupled harmonic oscillators.\\

This was a neat problem. There wasn't really anything different about it compared with other similar problems we've done, except perhaps the determinant equation being too difficult to factor, so having to use the quadratic formula on a fourth-order polynomial. It was nice that the uncoupled limit worked out so well.\\

\section{Problem Four: }
\textbf{A thin hoop of radius R and mass M oscillates in its own plane hanging from a single fixed point. Attached to the hoop is a small mass M constrained to move (in a frictionless manner) along the hoop. The inertia of the loop for rotation around a point on the loop is $I=2MR^2$. Consider only small oscillations, and show that the eigenfrequencies are:}

\begin{equation*}
  \omega_1 = \sqrt{\frac{2g}{R}}
\end{equation*}

\begin{equation*}
  \omega_2 = \sqrt{\frac{g}{2R}}
\end{equation*}

We begin by defining our generalized coordinates. $\Theta$ is the hoop's CM-connection point angle with the horizontal and $\phi$ is the mass-hoop CM angle with respect to the horizontal.

\begin{align*}
  x &= R\sin\theta + R\sin\phi\\
  y &= -R\cos\theta + -R\cos\phi\\
  \dot{x}^2 + \dot{y}^2 &= R^2\left(\dot{\theta}^2 + \dot{\phi}^2 + 2\dot{\theta}\dot{\phi}\cos\left(\theta-\phi\right)\right)\\
  \vspace{1cm}\\
  T &= \frac{1}{2}M\left(\dot{x}^2+\dot{y}^2\right) + \frac{1}{2}I\omega^2\\
  &= \frac{1}{2}MR^2\left(\dot{\theta}^2 + \dot{\phi}^2 + 2\dot{\phi}\dot{\theta}\cos(\theta-\phi) + 2\dot{\theta}^2\right)\\
  \mathcal{L} &= \frac{1}{2}MR^2\left(\dot{\theta}^2 + \dot{\phi}^2 + 2\dot{\phi}\dot{\theta}\cos(\theta-\phi) + 2\dot{\theta}^2\right) + mgR\left(\cos\theta + \cos\phi\right) + mgR\cos\theta\\
  &= \frac{1}{2}MR^2\left(3\dot{\theta}^2 + \dot{\phi}^2 + 2\dot{\phi}\dot{\theta}\cos(\theta-\phi)\right) + mgR\left(2\cos\theta + \cos\phi\right)\\
  &\hspace{1cm} cos(\theta) = 1 - \frac{\theta^2}{2},\hspace{1cm} cos(\phi) = 1 - \frac{\phi^2}{2}\\
  &= \frac{1}{2}MR^2\left(3\dot{\theta}^2 + \dot{\phi}^2 + 2\dot{\phi}\dot{\theta}\right) + mgR\left(3 - \theta^2 - \frac{\phi^2}{2}\right)\\
  \vspace{1cm}\\
  \frac{d}{dt}\frac{d\mathcal{L}}{d\dot{\theta}} &= \frac{d\mathcal{L}}{d\theta}\\
  \frac{1}{2}MR^2\left(6\ddot{\theta} + 2\ddot{\phi}\right) &= -2MgR\theta\\
  R\left(6\ddot{\theta} + 2\ddot{\phi}\right) &= -2g\theta\\
  3\ddot{\theta} + \ddot{\phi} &= \frac{-2g}{R}\theta\\
  \frac{d}{dt}\frac{d\mathcal{L}}{d\dot{\phi}} &= \frac{d\mathcal{L}}{d\phi}\\
  \frac{1}{2}MR^2\left(2\ddot{\phi} + 2\ddot{\theta}\right) &= -MgR\phi\\
  \ddot{\phi} + \ddot{\theta} &= \frac{-g}{R}\phi\\
\end{align*}

We have obtained our equations of motion using Lagrangian mechanics. We now guess the solution using the complex exponential function. This will give us a system of two equations and three unknowns. However, this particular system allows us to cancel out the A and B coefficients and solve for $\omega^2$.\\

\begin{align*}
  \theta &= Ae^{i\omega t}\\
  \phi &= Be^{i\omega t}\\
  3\omega^2A + \omega^2B &= \frac{-2g}{R}A\\
  \omega^2A + \omega^2B &= \frac{-g}{R}B\\
  A = B\left(\frac{g}{R\omega^2}-1\right)\\
  3\omega^2\left(\frac{g}{R\omega^2}-1\right) + \omega^2 &= \frac{2g}{R}\left(\frac{g}{R\omega^2}-1\right)\\
  \frac{3g}{R} - 3\omega^2 + \omega^2 = \frac{2g^2}{R^2}\frac{1}{\omega^2} - \frac{2g}{R}\\
  -2\omega^4 + \frac{5g}{R}\omega^2 - \frac{2g^2}{R^2} &= 0\\
  \omega^2&= \frac{-\frac{5g}{R} \pm \sqrt{\frac{25g^2}{R^2} - 8\frac{2g^2}{R^2}}}{-4} = \frac{g}{2R}, \frac{2g}{R}\\
  \omega_1 &= \sqrt{\frac{g}{2R}}\\
  \omega_2 &= \sqrt{\frac{2g}{R}}\\
\end{align*}

This was a very cool problem. We applied a bunch of different physics to solve it: Taylor approximations to the small-angle cosine, Lagrangian mechanics, and a complex set of generalized coordinates. This was a long and difficult problem, but I enjoyed thinking about how to set it up. The trick of using the small-angle approximation as a Taylor series, but only applying it to the large terms (not the terms involving the angular velocities) was very tricky. I think that thinking about how values in your equations influence the behavior of the equation, instead of just operating on them, is important to doing good physics.\\
\end{document}

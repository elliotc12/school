\documentclass[10pt]{article} % Font size - 10pt, 11pt or 12pt

\usepackage[hmargin=1.25cm, vmargin=1.5cm]{geometry} % Document margins

\usepackage{amsmath}
\usepackage{marvosym} % Required for symbols in the colored box
\usepackage{ifsym} % Required for symbols in the colored box

\usepackage[usenames,dvipsnames]{xcolor} % Allows the definition of hex colors

% Fonts and tweaks for XeLaTeX
\usepackage{fontspec,xltxtra,xunicode}
\defaultfontfeatures{Mapping=tex-text}
%\setmonofont[Scale=MatchLowercase]{Andale Mono}

% Colors for links, text and headings
\usepackage{hyperref}
\definecolor{linkcolor}{HTML}{506266} % Blue-gray color for links
\definecolor{shade}{HTML}{F5DD9D} % Peach color for the contact information box
\definecolor{text1}{HTML}{2b2b2b} % Main document font color, off-black
\definecolor{headings}{HTML}{701112} % Dark red color for headings
% Other color palettes: shade=B9D7D9 and linkcolor=A40000; shade=D4D7FE and linkcolor=FF0080

\hypersetup{colorlinks,breaklinks, urlcolor=linkcolor, linkcolor=linkcolor} % Set up links and colors

\usepackage{fancyhdr}
\pagestyle{fancy}
\fancyhf{}
% Headers and footers can be added with the \lhead{} \rhead{} \lfoot{} \rfoot{} commands
% Example footer:
%\rfoot{\color{headings} {\sffamily Last update: \today}. Typeset with Xe\LaTeX}

\renewcommand{\headrulewidth}{0pt} % Get rid of the default rule in the header

\usepackage{titlesec} % Allows creating custom \section's

% Format of the section titles
\titleformat{\section}{\color{headings}
\scshape\Large\raggedright}{}{0em}{}[\color{black}\titlerule]

\title{Classical Mechanics Assignment One}
\author{Elliott Capek}
\titlespacing{\section}{0pt}{0pt}{5pt} % Spacing around titles

\begin{document}

\maketitle{}

\section{Problem One}
\textbf{Consider  a  ball  that  moves  vertically  under  the  influences  of  both  gravity  and  air resistance.  For  the  purposes  of  this  problem,  take  vertically  upward  as  the  positive 
direction. (For instance, the gravitational force on the ball would be expressed as -mg in this case.) For each equation of motion below, determine whether that equation applies 
to (i) a situation in which the ball moves upward, (ii) a situation in which the ball moves 
downward, (iii) either of these, or (iv) neither of these. Explain your reasoning for each 
case. c is a positive constant. Justify each answer, no need  to add a  final paragraph of 
comment for the whole problem.} \\ \\

\begin{equation}
  m\frac{dv}{dt} = -mg + cv\\
\end{equation}
This motion equation corresponds to case (iii), either upwards or downwards motion. $\frac{dv}{dt}$ is positive when $cv > mg$, and negative when $cv < mg$. This means that in the case that v is positive and $cv$ has a larger magnitude than the downward acceleration $mg$, the acceleration will be positive and make velocity more positive, leading to upwards motion. When $cv$ is less than the magnitude of $mg$, gravity is either overcoming an upwards force or reinforcing a downward force and accelerating the ball downward.\\

\begin{equation}
  m\frac{dv}{dt} = -mg - cv\\
\end{equation}
This situation is the opposite of the above one: neither upwards nor downwards motion will ensue. This is because acceleration now moves opposite velocity. In the case where $v > \frac{-mg}{c}$, the ball will be accelerated downwards until $v < \frac{-mg}{c}$. Once this happens, the ball will be accelerated upwards until $v > \frac{-mg}{c}$ once again, and this will continue in a cycle.\\

\begin{equation}
  m\frac{dv}{dt} = -mg + cv^2\\
\end{equation}
This situation corresponds to (iii), either upwards or downwards motion. If $\sqrt{\frac{mg}{c}} < v_0$, then the acceleration will be upwards and an already upward-moving ball will move that way faster. If $\sqrt{\frac{mg}{c}} > v_0$, then the ball will be caught in a loop of its downward velocity being pushed upward or downward, but it will never be positive enough to have both v' and v positive at once.\\

\begin{equation}
  m\frac{dv}{dt} = -mg - cv^2\\
\end{equation}


\vspace{1 cm}

\section{Problem Two}
\textbf{Express  the  terminal  velocity  of  objects  falling  vertically  in  air  and  subject  to  the following (fictitious) drag  forces (no need to add a  final paragraph of comment  for the 
whole problem):} \\ \\

a.)
\begin{align}
  \vec{F}_{drag} &= -\alpha\log(v)\vec{v}\\
  \frac{dv}{dt} &= -mg + \vec{F}_{drag}\\
  \frac{dv}{dt} &= -mg - \alpha\log(v)v\\  
  \frac{dv}{dt} &= -mg - \alpha\log(v_{term})v_{term} = 0\\
  -mg &= \alpha\log(v_{term})v_{term}\\
\end{align}

b.)
\begin{align}
  \vec{F}_{drag} &= -\alpha v^3\vec{v}\\
  \frac{dv}{dt} &= -mg + \vec{F}_{drag}\\
  \frac{dv}{dt} &= -mg -\alpha v^3\vec{v}\\
  \frac{dv}{dt} &= -mg -\alpha v^4\\
  \frac{dv}{dt} &= -mg -\alpha v_{term}^4 = 0\\
  -\alpha v_{term}^4 &= -mg\\
  v_{term}^4 &= \frac{-mg}{\alpha}\\
  v_{term} &= \sqrt[4]{\frac{-mg}{\alpha}}\\
 \end{align}

\vspace{1 cm}       

\section{Problem Three}
\textbf{Consider a softball of diameter 10cm and mass 200g and one of Millikan’s oil droplets 
with  diameter  1µm  and  mass  1pg.  They  are  both subject  to  air  drag  of  equation
$\vec{F}_{drag} = −\beta D\vec{v} −\gamma D^2v\vec{v}$, with $\beta = 1.6 \times 10^{−4}$ and $\gamma = 0.25$ (SI units).} \\ \\

a. For  the  ball  and  the  droplet,  find  the  velocity  range  for  which  the  linear  (viscous) 
term of the drag force dominates and the one for which the quadratic (momentum) 
term dominates.\\

\begin{align}
  F_{vis} &> F_{mom}\\
  -\beta D\vec{v_{vis}} &> -\gamma D^2v_{vis}\vec{v}_{vis}\\
  \beta D\vec{v_{vis}} &< \gamma D^2v_{vis}\vec{v}_{vis}\\
  \beta D\vec{v_{vis}} &< \gamma D^2v_{vis}\vec{v}_{vis}\\
  \beta v_{vis} &< \gamma Dv^2_{vis}\\
  \frac{\beta}{\gamma D} &< v_{vis}\\
  \frac{\beta}{\gamma D} &< v_{vis}\\
\end{align}
Therefore, $\frac{\beta}{\gamma D}  > v_{mom}$.

Ball:
\begin{align}
  \frac{\beta}{\gamma D} < v_{vis}\\
  \frac{1.6*10^{-4}}{0.25*0.1} < v_{vis}\\
  0.0064 \frac{m}{s} < v_{vis}
\end{align}

Oil droplet:
\begin{align}
  \frac{\beta}{\gamma D} < v_{vis}\\
  \frac{1.6*10^{-4}}{0.25*10^{-6}} < v_{vis}\\
  640 \frac{m}{s} < v_{vis}
\end{align}

b. Calculate the terminal velocity of the ball and of the droplets when they are falling 
vertically.\\

\begin{align}
  \frac{dv}{dt} &= -mg −\beta D\vec{v} −\gamma D^2v\vec{v}\\
  0 &= -mg −\beta Dv_{term} −\gamma D^2v^2_{term}\\
  0 &= mg + \beta Dv_{term} + \gamma D^2v^2_{term}\\
  0 &= \gamma D^2v^2_{term} + \beta Dv_{term} + mg\\
  v_{term} &= \frac{-\beta D \pm \sqrt{\beta^2 D^2 - 4\gamma D^2mg}}{2\gamma D^2}\\
\end{align}

Ball:
\begin{align}
  v_{term} &= \frac{-\beta D \pm \sqrt{\beta^2 D^2 - 4\gamma D^2mg}}{2\gamma D^2}\\
  v_{term} &= \frac{-1.6*10^{-4}*0.1 \pm \sqrt{1.6^2*10^{-8}*0.1^2 - 4*0.25*0.1^2*0.2*9.8}}{2*0.25*0.1^2}\\  
\end{align}

c. Reflect on your result. What term of the drag  force is dominant  for the  free-falling 
softball? What term is dominant for the oil droplet?\\


\vspace{1 cm}       

\section{Problem Four}
\textbf{How prepared do you feel that you are for this class?} \\ \\

\end{document}

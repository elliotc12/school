\documentclass[10pt]{article} % Font size - 10pt, 11pt or 12pt

\usepackage[hmargin=1.25cm, vmargin=1.5cm]{geometry} % Document margins

\usepackage{graphicx}
\usepackage{amsmath}
\usepackage{marvosym} % Required for symbols in the colored box
\usepackage{ifsym} % Required for symbols in the colored box

\usepackage[usenames,dvipsnames]{xcolor} % Allows the definition of hex colors

% Fonts and tweaks for XeLaTeX
\usepackage{fontspec,xltxtra,xunicode}
\defaultfontfeatures{Mapping=tex-text}
%\setmonofont[Scale=MatchLowercase]{Andale Mono}

% Colors for links, text and headings
\usepackage{hyperref}
\definecolor{linkcolor}{HTML}{506266} % Blue-gray color for links
\definecolor{shade}{HTML}{F5DD9D} % Peach color for the contact information box
\definecolor{text1}{HTML}{2b2b2b} % Main document font color, off-black
\definecolor{headings}{HTML}{701112} % Dark red color for headings
% Other color palettes: shade=B9D7D9 and linkcolor=A40000; shade=D4D7FE and linkcolor=FF0080

\hypersetup{colorlinks,breaklinks, urlcolor=linkcolor, linkcolor=linkcolor} % Set up links and colors

\usepackage{fancyhdr}
\pagestyle{fancy}
\fancyhf{}
% Headers and footers can be added with the \lhead{} \rhead{} \lfoot{} \rfoot{} commands
% Example footer:
%\rfoot{\color{headings} {\sffamily Last update: \today}. Typeset with Xe\LaTeX}

\renewcommand{\headrulewidth}{0pt} % Get rid of the default rule in the header

\usepackage{titlesec} % Allows creating custom \section's

\allowdisplaybreaks

% Format of the section titles
\titleformat{\section}{\color{headings}
\scshape\Large\raggedright}{}{0em}{}[\color{black}\titlerule]

\title{Classical Mechanics Assignment 5}
\author{Elliott Capek}
\titlespacing{\section}{0pt}{0pt}{5pt} % Spacing around titles

\begin{document}

\maketitle{}

\section{Problem One}
\textbf{Find the shape assumed by an infinitely flexible rope of linear density $\rho$ and length L with both sides attached to the ceiling of a room. (Hint, the shape will minimize the gravitational potential energy of the rope).}

\begin{align*}
  PE_{aug} &= \int_{-a}^{a} \left[-gy\rho \sqrt{1 + x'^2} - \lambda\sqrt{1+x'^2}\right]dx\\
  f(y,y',x) &= -gy\rho \sqrt{1 + x'^2} - \lambda\sqrt{1+x'^2}\\
\end{align*}

We use the $\frac{df}{dx'} = k$ form of the Euler-Lagrange equation:

\begin{align*}
  \frac{-gy\rho x'}{\sqrt{1 + x'^2}} - \frac{\lambda x'}{\sqrt{1+x'^2}} &= k\\
  \frac{-gy\rho x' - \lambda x'}{\sqrt{1 + x'^2}} &= k\\
\end{align*}

\section{Problem Two}
\textbf{Demonstrate that the shortest distance between two points in 3D is a straight line.}

\begin{align*}
  L &= \int_{z_1}^{z_2} \sqrt{1 + x'^2 + y'^2} dz\\
  f(x, x', y, y', z) &= \sqrt{1 + x'^2 + y'^2}\\
\end{align*}

Since there is no x, y or z dependence in $f$, we can use the condensed Euler-Lagrange equation $\frac{df}{dn'} = k$:\\

\begin{align*}
  \frac{df}{dx'} &= K_1\\
  \frac{x'}{\sqrt{1 + x'^2 + y'^2}} &= K_1\\
  x'^2 &= K_1^2\left(1 + x'^2 + y'^2\right)\\
  x'^2 &= \frac{K_1^2\left(1 + y'^2\right)}{1-K_1}\\
  y'^2 &= \frac{K_2^2\left(1 + x'^2\right)}{1-K_2}
  \hspace*{1in} \mbox{Solutions for x' and y' are symmetric}\\
  x'^2 &= \frac{K_1^2\left(1 + \frac{K_2^2\left(1 + x'^2\right)}{1-K_2}\right)}{1-K_1}\\
  x'^2\left(1-K_1\right) &= K_1^2\left(1 + \frac{K_2^2\left(1 + x'^2\right)}{1-K_2}\right)\\
  x'^2\left(1-K_1\right) &= K_1^2 + \frac{K_1^2K_2^2\left(1 + x'^2\right)}{1-K_2}\\
  x'^2\left(1-K_1\right) &= K_1^2 + \frac{K_1^2K_2^2}{1-K_2} + \frac{x'^2K_1^2K_2^2}{1-K_2}\\
  x'^2\left(1-K_1 - \frac{K_1^2K_2^2}{1-K_2}\right) &= K_1^2 + \frac{K_1^2K_2^2}{1-K_2}\\
  x' &= \sqrt{\frac{K_1^2 + \frac{K_1^2K_2^2}{1-K_2}}{1-K_1 - \frac{K_1^2K_2^2}{1-K_2}}}\\
\end{align*}

Because $x'$ is constant and $y'^2 = \frac{K_2^2\left(1 + x'^2\right)}{1-K_2}$, y' is also constant. Thus we have the equation of a line!\\

This was a fun easy problem. It was interesting that it was so easy to add arbitrary numbers of variables to the Euler-Lagrange equation. To add a variable, you just add another Euler-Lagrange equation to the system. It is strange that the equations are not coupled at all. It kind of makes sense, since you are just minimizing the path in two dimensions for each equation. You could take the 3D plot and look at its XZ side and minimize that. Then you could minimize the YZ plot. The XZ and YZ plots describe the same curve, but are not related.
\end{document}

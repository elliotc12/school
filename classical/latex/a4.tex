\documentclass[10pt]{article} % Font size - 10pt, 11pt or 12pt

\usepackage[hmargin=1.25cm, vmargin=1.5cm]{geometry} % Document margins

\usepackage{graphicx}
\usepackage{amsmath}
\usepackage{marvosym} % Required for symbols in the colored box
\usepackage{ifsym} % Required for symbols in the colored box

\usepackage[usenames,dvipsnames]{xcolor} % Allows the definition of hex colors

% Fonts and tweaks for XeLaTeX
\usepackage{fontspec,xltxtra,xunicode}
\defaultfontfeatures{Mapping=tex-text}
%\setmonofont[Scale=MatchLowercase]{Andale Mono}

% Colors for links, text and headings
\usepackage{hyperref}
\definecolor{linkcolor}{HTML}{506266} % Blue-gray color for links
\definecolor{shade}{HTML}{F5DD9D} % Peach color for the contact information box
\definecolor{text1}{HTML}{2b2b2b} % Main document font color, off-black
\definecolor{headings}{HTML}{701112} % Dark red color for headings
% Other color palettes: shade=B9D7D9 and linkcolor=A40000; shade=D4D7FE and linkcolor=FF0080

\hypersetup{colorlinks,breaklinks, urlcolor=linkcolor, linkcolor=linkcolor} % Set up links and colors

\usepackage{fancyhdr}
\pagestyle{fancy}
\fancyhf{}
% Headers and footers can be added with the \lhead{} \rhead{} \lfoot{} \rfoot{} commands
% Example footer:
%\rfoot{\color{headings} {\sffamily Last update: \today}. Typeset with Xe\LaTeX}

\renewcommand{\headrulewidth}{0pt} % Get rid of the default rule in the header

\usepackage{titlesec} % Allows creating custom \section's

\allowdisplaybreaks

% Format of the section titles
\titleformat{\section}{\color{headings}
\scshape\Large\raggedright}{}{0em}{}[\color{black}\titlerule]

\title{Classical Mechanics Assignment 4}
\author{Elliott Capek}
\titlespacing{\section}{0pt}{0pt}{5pt} % Spacing around titles

\begin{document}

\maketitle{}

\section{Problem One}
\textbf{Consider a rocket accelerating in free space by burning its fuel at a constant rate. Find the velocity at which the momentum of the rocket (neglecting the already spent fuel) is maximized.} \\

We want to find the velocity at which $m\dot{y}$ is maximized. We find when the derivative of this with respect to time is zero, plug in for the values we find and go from there:

\begin{align*}
  \frac{\partial}{\partial t} m\dot{y} &= \dot{m}\dot{y} + m\ddot{y} = 0\\
  m\ddot{y} &= -mg - \dot{m}v_{ex}\\
  \dot{m}\dot{y} &= mg + \dot{m}v_{ex}\\
  \dot{y} &= \frac{mg + \dot{m}v_{ex}}{\dot{m}}\\
\end{align*}

\section{Problem Two}
\textbf{Consider a rocket subject to a linear resistive force $\vec{F} = -b\vec{v}$, but no other external forces. Find the velocity as a function of the remaining mass for a rocket that starts from rest and ejects mass at a constant rate $k = -\dot{m}$ and velocity $v_{ex}$.} \\

\begin{align*}
  m\ddot{y} &= -\frac{b}{m}\dot{y} - \dot{m}v_{ex}\\
\end{align*}

\section{Problem Three}
\textbf{Consider the surface generated by revolving a line connecting two points ($x_1$,$y_1$) and ($x_2$,$y_2$) about an axis coplanar with the two points.  Find the equation of the line connecting the points such that the surface area generated by the revolution is minimum.}

\begin{align*}
  SA &= \int_{y_1}^{y_2} dA = \int_{y_1}^{y_2} 2\pi\sqrt{x^2+y^2}\sqrt{x'^2+1}dy\\
  f(x, x', y) &= 2\pi\sqrt{x^2+y^2}\sqrt{x'^2+1}\\
\end{align*}

\begin{align*}
  \frac{\partial f}{\partial x} &- \frac{\partial}{\partial y} \frac{\partial f}{\partial x'} = 0\\
  \frac{\partial f}{\partial x} &= \frac{2\pi x\sqrt{x'^2+1}}{\sqrt{x^2+y^2}}\\
  \frac{\partial}{\partial y} \frac{\partial f}{\partial x'} &= 0\\
\end{align*}

So...
\begin{align*}
  \frac{2\pi x\sqrt{x'^2+1}}{\sqrt{x^2+y^2}} = 0\\
\end{align*}

\section{Problem Four}
\textbf{Consider a medium in which the refractive index $n$ is inversely proportional to $r_2$. That is, $n=a/r2$, where $r$ is the distance from the origin. Use Fermat’s principle to find the path of light traveling between two points that lie in the xy plane. Hint: use 2-dimensional polar coordinates. Show that the path found is part of a circle that intersects the origin of the reference frame.}

We use polar coordinates to describe our two points. Our differential is $ds = \sqrt{r^2+r'^2}d\theta$.

\begin{align*}
  T &= \int_{\theta_1}^{\theta_2} \frac{1}{v}ds = \int_{\theta_1}^{\theta_2} \frac{a}{cr^2}\sqrt{r^2+r'^2}d\theta\\
  f(r,r',\theta) &= \frac{a}{cr^2}\sqrt{r^2+r'^2}d\theta\\
\end{align*}

\begin{align*}
  \frac{\partial f}{\partial r} - \frac{\partial}{\partial \theta} \frac{\partial f}{\partial r'} &= 0\\
  \frac{\partial}{\partial \theta} \frac{\partial f}{\partial r'} &= 0\\
\end{align*}

And so:

\begin{align*}
    \frac{\partial f}{\partial r} = 0\\
\end{align*}

Meaning 

\end{document}

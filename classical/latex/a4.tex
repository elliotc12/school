\documentclass[10pt]{article} % Font size - 10pt, 11pt or 12pt

\usepackage[hmargin=1.25cm, vmargin=1.5cm]{geometry} % Document margins

\usepackage{graphicx}
\usepackage{amsmath}
\usepackage{marvosym} % Required for symbols in the colored box
\usepackage{ifsym} % Required for symbols in the colored box

\usepackage[usenames,dvipsnames]{xcolor} % Allows the definition of hex colors

% Fonts and tweaks for XeLaTeX
\usepackage{fontspec,xltxtra,xunicode}
\defaultfontfeatures{Mapping=tex-text}
%\setmonofont[Scale=MatchLowercase]{Andale Mono}

% Colors for links, text and headings
\usepackage{hyperref}
\definecolor{linkcolor}{HTML}{506266} % Blue-gray color for links
\definecolor{shade}{HTML}{F5DD9D} % Peach color for the contact information box
\definecolor{text1}{HTML}{2b2b2b} % Main document font color, off-black
\definecolor{headings}{HTML}{701112} % Dark red color for headings
% Other color palettes: shade=B9D7D9 and linkcolor=A40000; shade=D4D7FE and linkcolor=FF0080

\hypersetup{colorlinks,breaklinks, urlcolor=linkcolor, linkcolor=linkcolor} % Set up links and colors

\usepackage{fancyhdr}
\pagestyle{fancy}
\fancyhf{}
% Headers and footers can be added with the \lhead{} \rhead{} \lfoot{} \rfoot{} commands
% Example footer:
%\rfoot{\color{headings} {\sffamily Last update: \today}. Typeset with Xe\LaTeX}

\renewcommand{\headrulewidth}{0pt} % Get rid of the default rule in the header

\usepackage{titlesec} % Allows creating custom \section's

\allowdisplaybreaks

% Format of the section titles
\titleformat{\section}{\color{headings}
\scshape\Large\raggedright}{}{0em}{}[\color{black}\titlerule]

\title{Classical Mechanics Assignment 4}
\author{Elliott Capek}
\titlespacing{\section}{0pt}{0pt}{5pt} % Spacing around titles

\begin{document}

\maketitle{}

\section{Problem One}
\textbf{Consider a rocket accelerating in free space by burning its fuel at a constant rate. Find the velocity at which the momentum of the rocket (neglecting the already spent fuel) is maximized.} \\

\begin{align*}
  m\ddot{y} &= -\dot{m}V_{ex}\\
  m\frac{dv}{dt} &= -\frac{dm}{dt}V_{ex}\\
  dv &= -\frac{dm}{m}V_{ex}\\
  \int_0^v dv &= \int_{m_0}^m-\frac{dm}{m}V_{ex}\\
  v &= -\log(\frac{m}{m_0})V_{ex}\\
  m &= m_0e^{\frac{-v}{V_{ex}}}\\
\end{align*}

We want to find the velocity which minimizes momentum...

\begin{align*}
  p &= mv\\
  \frac{dp}{dv} &= m + v\frac{dm}{dv} = 0\\
  m + \frac{m_0v}{V_{ex}}e^{\frac{-v}{V_{ex}}} &= 0\\
  m &= \frac{m_0v}{V_{ex}}e^{\log(\frac{m}{m_0})}\\
  v &= V_{ex}\\
\end{align*}

This problem has an interesting result. It isn't intuitive to me why it is the case - it probably has something to do with momentum conservation. The derivation itself was fairly simple, just another application of separation of variables. I like getting practice using this and other methods to solve differential equations.\\

\section{Problem Two}
\textbf{Consider a rocket subject to a linear resistive force $\vec{F} = -b\vec{v}$, but no other external forces. Find the velocity as a function of the remaining mass for a rocket that starts from rest and ejects mass at a constant rate $k = -\dot{m}$ and velocity $v_{ex}$.} \\

\begin{align*}
  (m_0+\dot{m}t)\ddot{y} &= -b\dot{y} - \dot{m}v_{ex}\\
  \frac{d\dot{y}}{dt} &= \frac{-b\dot{y} - \dot{m}v_{ex}}{(m_0+\dot{m}t)}\\
  \int_0^v \frac{d\dot{y}}{-b\dot{y} - \dot{m}v_{ex}} &= \int_0^t \frac{1}{(m_0+\dot{m}t)}dt\\
  \frac{-1}{b}\log(\frac{bv + \dot{m}v_{ex}}{\dot{m}v_{ex}}) &= \frac{1}{\dot{m}}\log(\frac{m_0+\dot{m}t}{m_0})\\
  -\dot{m}\log(\frac{bv + \dot{m}v_{ex}}{\dot{m}v_{ex}}) &= b\log(\frac{m_0+\dot{m}t}{m_0})\\
  \frac{bv + \dot{m}v_{ex}}{\dot{m}v_{ex}} &= (\frac{m_0+\dot{m}t}{m_0})^{\frac{b}{\dot{m}}}\\
  v &= \frac{1}{b}\Big(\dot{m}v_{ex}(\frac{m_0+\dot{m}t}{m_0})^{\frac{-b}{\dot{m}}} - \dot{m}v_{ex}\Big)\\
  v &= \frac{kV_{ex}}{b}\Big(1 - (\frac{m}{m_0})^{\frac{b}{k}} \Big)\\
\end{align*}

This was another interesting problem. I think I'm getting really experienced at using separation of variables to solve equations. The equation for velocity makes sense - as the $m/m_0$ fraction decreases, the velocity increases. Since the fraction m/m0 is always less than one, the velocity is always possible. It is interesting that the terminal velocity once all the mass has been expended (if all the rocket is actually fuel) is $kV_{ex}/b$ - this is good to know.\\

\section{Problem Three}
\textbf{Consider the surface generated by revolving a line connecting two points ($x_1$,$y_1$) and ($x_2$,$y_2$) about an axis coplanar with the two points.  Find the equation of the line connecting the points such that the surface area generated by the revolution is minimum.}

\begin{align*}
  SA &= \int_{y_1}^{y_2} dA = \int_{y_1}^{y_2} 2\pi y\sqrt{y'^2+1}dx\\
  f(y, y', x) &= 2\pi y\sqrt{y'^2+1}\\
\end{align*}

\begin{align*}
  f - y'\frac{\partial f}{\partial y'} &= C\\
  y\sqrt{y'^2+1} - \frac{yy'^2}{\sqrt{y'^2+1}} &= C\\
  y\sqrt{y'^2+1} &= C + \frac{yy'^2}{\sqrt{y'^2+1}}\\
  y(y'^2+1) &= C\sqrt{y'^2+1} + yy'^2\\
  y &= C\sqrt{y'^2+1}\\
  \frac{y^2}{C^2} - 1 &= y'^2\\
  y^2 - C^2 &= y'^2C^2\\
  \frac{\sqrt{y^2 - C^2}}{C} &= y'\\
  \frac{\sqrt{y^2 - C^2}}{C} &= y'\\
  \frac{dx}{dy} = \frac{1}{y'} &= \frac{C}{\sqrt{y^2 - C^2}}\\
  x = \int \frac{dx}{dy}dy &= \int \frac{1}{y'} = \frac{C}{\sqrt{y^2 - C^2}}dy\\
  x = C\cosh^{-1}(\frac{y}{C})+b\\
  y = C\cosh(\frac{x-b}{C})\\
\end{align*}

So we can see that a hyperbolic cosine describes the smallest surface which connects two points. We can find the values of b and C by solving our result for the initial and final points, which must be done with a computer. This corresponds to a surface which starts at one point, bows inwards, then arcs outward again. This problem was a cool introduction to the power of using the Euler-Lagrange equation to find surfaces. It was actually fairly easy to get the solution, given the difficulty of solving the problem by any other means.\\

\section{Problem Four}
\textbf{Consider a medium in which the refractive index $n$ is inversely proportional to $r_2$. That is, $n=a/r2$, where $r$ is the distance from the origin. Use Fermat’s principle to find the path of light traveling between two points that lie in the xy plane. Hint: use 2-dimensional polar coordinates. Show that the path found is part of a circle that intersects the origin of the reference frame.}

We use polar coordinates to describe our two points. Our differential is $ds = \sqrt{r^2+r'^2}d\theta$.

\begin{align*}
  T &= \int_{\theta_1}^{\theta_2} \frac{1}{v}ds = \int_{\theta_1}^{\theta_2} \frac{a}{cr^2}\sqrt{r^2+r'^2}d\theta\\
  f(r,r',\theta) &= \frac{a}{cr^2}\sqrt{r^2+r'^2}\\
\end{align*}

\begin{align*}
  f - r'\frac{\partial f}{\partial r'} &= C\\
  \frac{a}{cr^2}\sqrt{r^2+r'^2} - \frac{ar'^2}{cr^2\sqrt{r^2+r'^2}} &= C\\
  \frac{1}{r^2}\sqrt{r^2+r'^2} - \frac{r'^2}{r^2\sqrt{r^2+r'^2}} &= \frac{Cc}{a}\\
  \sqrt{r^2+r'^2} &= \frac{Ccr^2}{a} + \frac{r'^2}{\sqrt{r^2+r'^2}}\\
  r^2+r'^2 &= \frac{Ccr^2}{a}\sqrt{r^2+r'^2} + r'^2\\
  \frac{a^2}{C^2c^2} &= r^2+r'^2\\
  \sqrt{\frac{a^2}{C^2c^2} - r^2} &= \frac{dr}{d\theta}\\
  \int d\theta &= \int \frac{dr}{\sqrt{\frac{a^2}{C^2c^2} - r^2}}\\
  \theta &= \tan^{-1}\Big( \frac{c^2r\sqrt(\frac{a^2C^2-c^2r^2}{c^2})}{-a^2C^2+c^2r^2}   \Big)\\
  \tan(\theta) &= \frac{c^2r\sqrt{\frac{a^2C^2-c^2r^2}{c^2}}}{-a^2C^2+c^2r^2}\\
  r(\theta) &= \frac{C\tan(\theta)}{\sqrt{1+\tan(\theta)^2}}
\end{align*}


This was a very interesting problem. We arrived at a concise solution for a very complicated physical problem using relatively straightforward techniques. It was interesting to note that the actual value of the physical constants is not important, since the path is minimized this particular way for any value of a and c. The C in our final expression is just a constant multiple that modifies the radius of the circle.
\end{document}

\documentclass[10pt]{article} % Font size - 10pt, 11pt or 12pt

\usepackage[hmargin=1.25cm, vmargin=1.5cm]{geometry} % Document margins

\usepackage{marvosym} % Required for symbols in the colored box
\usepackage{ifsym} % Required for symbols in the colored box

\usepackage[usenames,dvipsnames]{xcolor} % Allows the definition of hex colors

% Fonts and tweaks for XeLaTeX
\usepackage{fontspec,xltxtra,xunicode}
\defaultfontfeatures{Mapping=tex-text}
%\setmonofont[Scale=MatchLowercase]{Andale Mono}

% Colors for links, text and headings
\usepackage{hyperref}
\definecolor{linkcolor}{HTML}{506266} % Blue-gray color for links
\definecolor{shade}{HTML}{F5DD9D} % Peach color for the contact information box
\definecolor{text1}{HTML}{2b2b2b} % Main document font color, off-black
\definecolor{headings}{HTML}{701112} % Dark red color for headings
% Other color palettes: shade=B9D7D9 and linkcolor=A40000; shade=D4D7FE and linkcolor=FF0080

\hypersetup{colorlinks,breaklinks, urlcolor=linkcolor, linkcolor=linkcolor} % Set up links and colors

\usepackage{fancyhdr}
\usepackage{amsmath}
\usepackage{amssymb}
\pagestyle{fancy}
\fancyhf{}
% Headers and footers can be added with the \lhead{} \rhead{} \lfoot{} \rfoot{} commands
% Example footer:
%\rfoot{\color{headings} {\sffamily Last update: \today}. Typeset with Xe\LaTeX}

\renewcommand{\headrulewidth}{0pt} % Get rid of the default rule in the header

\usepackage{titlesec} % Allows creating custom \section's

% Format of the section titles
\titleformat{\section}{\color{headings}
\scshape\Large\raggedright}{}{0em}{}[\color{black}\titlerule]

\title{Solid State Assignment Two}
\author{Elliott Capek}
\titlespacing{\section}{0pt}{0pt}{5pt} % Spacing around titles

\begin{document}

\maketitle{}

\section{Problem 2: O-C-O Anti/Bonding energies}
This problem asks us to come up with the anti/bonding energies for an O-C-O molecule, given its
on-site and hopping energies.\\

First we construct the Hamiltonian for this system, using the rules defined in the problem:\\

\begin{align*}
  H &=
  \begin{pmatrix}
    E_O & \beta & 0\\
    \beta & E_C & \beta\\
    0 & \beta & E_O\\
  \end{pmatrix}
\end{align*}

We then find the eigenvalues of this equation:\\

\begin{align*}
  det(H) &=
  det\begin{bmatrix}
    E_O-\lambda & \beta & 0\\
    \beta & E_C-\lambda & \beta\\
    0 & \beta & E_O-\lambda\\
  \end{bmatrix}\\
  \mbox{Mathematica} \rightarrow
  &= E_O, \frac{1}{2}\left(\pm\sqrt{8\beta^2 + E_c^2 - 2E_cE_O + E_O^2} + E_c + E_O\right)\\
\end{align*}

This is an interesting result. As expected, there are three eigenvalues for a 3x3 matrix. One
corresponds to the original atomic oxygen energy level, which seems odd. This seems to be saying
that the oxygen atomic orbitals are retained in some way in the final molecule. The bonding
(lower energy) orbital has energy
$\frac{1}{2}\left(-\sqrt{8\beta^2 + E_c^2 - 2E_cE_O + E_O^2} + E_c + E_O\right)$, while the
antibonding (higher energy) orbital has
$\frac{1}{2}\left(\sqrt{8\beta^2 + E_c^2 - 2E_cE_O + E_O^2} + E_c + E_O\right)$.

\section{Problem Three: Types of Bonds}
Here we discuss different types of bonds and what they're made up of.\\

\textbf{sigma between $p_z$ orbitals: } Here two $p_z$ orbitals subtract from
each other (bonding) or add (antibonding) on the bonding axis to form new orbitals.\\
\vspace{3cm}

\textbf{pi between $p_z$ orbitals: } Here two $p_z$ orbitals add to each other (bonding)
or subtract (antibonding) perpendicular to the bonding axis to form new orbitals.\\
\vspace{3cm}

\textbf{sigma between $d_{z^2}$ orbitals: } Here two $d_{z^2}$ orbitals subtract from each other
(bonding) or add (antibonding) on the bonding axis to form new orbitals.\\
\vspace{3cm}

\textbf{pi between $d_{yz}$ orbitals: } Here two $d_{yz}$ orbitals subtract from each other
(bonding) or add (antibonding) on the bonding plane such that each of the two lobes
connect to form new orbitals.\\
\vspace{3cm}

\section{Problem Four: Heteronuclear Diatomic Molecule}
Here we follow the same process for finding the energies of a multiatomic system that we used
in problem Two: defining which atomic wave functions overlap on-site and off-site, deciding
the energies of these overlaps, creating a Hamiltonian and finding the energies of the
Hamiltonian. We are given that for this heteroatomic system atom A's on-site interaction
energy is $E_A$ and B's is $E_B$. Similarly B's is $E_B$, and $E_A > E_B$. We then let out
hopping integral take the value $H_{12} = H_{21} = \beta$. Thus our Hamiltonian is:

\begin{align*}
  H &=
  \begin{pmatrix}
    E_A & \beta\\
    \beta & E_B\\
  \end{pmatrix}
\end{align*}

Our next job is to find the eigenvalues of this matrix:

\begin{align*}
  \det\begin{pmatrix}
    E_A-\lambda & \beta\\
    \beta & E_B-\lambda\\
  \end{pmatrix}
  &= 0\\
  (E_A-\lambda)(E_B-\lambda) - \beta^2 &= 0\\
  \lambda^2 - \lambda(E_A + E_B )+ E_AE_B - \beta^2 &= 0\\
  \lambda &= \frac{(E_A + E_B) \pm \sqrt{(E_A + E_B)^2 - 4(E_AE_B - \beta^2)}}{2}\\
  &= \frac{(E_A + E_B) \pm \sqrt{E_A^2 + E_B^2 - 2E_AE_B + 4\beta^2}}{2}\\
  &= \frac{(E_A + E_B) \pm \sqrt{(E_A - E_B)^2 + 4\beta^2}}{2}\\
  &= \frac{(E_A + E_B)}{2} \pm \sqrt{\frac{(E_A - E_B)^2}{4} + \beta^2}\\
\end{align*}

This is the energy equation shown in Sutton 2.45. The case using addition is the antibonding (less
stable) orbital, and the case using subtraction is the more stable bonding orbital. We then plug
these equations back into our original matrix to find the eigenvectors:\\

\begin{align*}
  \det\begin{pmatrix}
    E_A - \frac{(E_A + E_B)}{2} + \sqrt{\frac{(E_A - E_B)^2}{4} + \beta^2} & \beta\\
    \beta & E_B - \frac{(E_A + E_B)}{2} + \sqrt{\frac{(E_A - E_B)^2}{4} + \beta^2}\\
  \end{pmatrix}
\end{align*}


\section{Problem Five: Silicon Crystal Structure}
Silicon forms a face-centered cubic diamond crystal lattice, where the center of each face, in
addition to each corner, has an atom at it. The length of a side of the cube is $5.431$ \AA
$= 5.431 * 10^{-10}m$. This is the lattice constant of Silicon crystal. Correspondingly the
bond lengths in silicon crystal are $5.431$ \AA (between corners) and
$5.431 * \frac{\sqrt{2}}{2} = 3.84$ \AA.
\hspace{2cm}
Silicon is a diamond FCC, so its space group is Fd$\overline{3}$m. This space group is a
Bravais lattice, meaning a lattice that is generated by adding any integer multiple of one
of three basis vectors to a starting point to generate an arbitrary member of the lattice.\\
\end{document}

\documentclass[10pt]{article} % Font size - 10pt, 11pt or 12pt

\usepackage[hmargin=1.25cm, vmargin=1.5cm]{geometry} % Document margins

\usepackage{marvosym} % Required for symbols in the colored box

\usepackage[usenames,dvipsnames]{xcolor} % Allows the definition of hex colors

% Fonts and tweaks for XeLaTeX
\usepackage{fontspec,xltxtra,xunicode}
\defaultfontfeatures{Mapping=tex-text}
%\setmonofont[Scale=MatchLowercase]{Andale Mono}

% Colors for links, text and headings
\usepackage{hyperref}
\definecolor{linkcolor}{HTML}{506266} % Blue-gray color for links
\definecolor{shade}{HTML}{F5DD9D} % Peach color for the contact information box
\definecolor{text1}{HTML}{2b2b2b} % Main document font color, off-black
\definecolor{headings}{HTML}{701112} % Dark red color for headings
% Other color palettes: shade=B9D7D9 and linkcolor=A40000; shade=D4D7FE and linkcolor=FF0080

\hypersetup{colorlinks,breaklinks, urlcolor=linkcolor, linkcolor=linkcolor} % Set up links and colors

\usepackage{fancyhdr}
\usepackage{amsmath}
\usepackage{braket}
\usepackage{amssymb}
\pagestyle{fancy}
\fancyhf{}
% Headers and footers can be added with the \lhead{} \rhead{} \lfoot{} \rfoot{} commands
% Example footer:
%\rfoot{\color{headings} {\sffamily Last update: \today}. Typeset with Xe\LaTeX}

\renewcommand{\headrulewidth}{0pt} % Get rid of the default rule in the header

\usepackage{titlesec} % Allows creating custom \section's

% Format of the section titles
\titleformat{\section}{\color{headings}
\scshape\Large\raggedright}{}{0em}{}[\color{black}\titlerule]

\title{Quantum Mechanics Assignment Five}
\author{Elliott Capek}
\titlespacing{\section}{0pt}{0pt}{5pt} % Spacing around titles

\begin{document}

\maketitle{}

\section{1: Effect of E-field on H spectrum}
Here we examine how a perturbing electric field influences the n=2 to n=1 transition for
the Hydrogen atom.\\

\textbf{a.)} First we find the unperturbed transition energy:\\

\begin{align*}
  E_2 - E_1 &= -13.6 \mbox{eV} \left(\frac{1}{2^2} - \frac{1}{1^2}\right) = 10.2 \mbox{eV}\\
\end{align*}

\textbf{b.)} Now we apply a 0.5 V/m electric field and see how the spectrum changes. The Stark
Effect tells us that the energy levels will be perturbed by either zero or
$\pm3e\epsilon a_0$, which for this electric field would be $\pm 7.94*10^{-2}eV$. This comes out
to be a change of roughly $0.77$ percent, which is a fairly small value.\\

The perturbation due to the electric field can be interpreted as the energy change of the
electric dipole moment interacting with the applied field. Some of the n=2 states are radially
symmetric, and so have no electric dipole to alter the energy. However certain combinations of
orbitals (ie $\bra{200}+\bra{210}$) do have a bias to part of the electron, and so do have an
energy change.\\

\section{11.2: Proving raising / lowering operators}
Here we show that the raising and lowering operators for angular momentum behave similarly
to the QHO raising and lowering operators.\\

First, we must show that the operators raise and lower the z-component momentum measurement
by one unit:\\

\begin{align*}
  J_z(J_+\ket{j,m_j}) &= (\hbar J_+ + J_+J_z)\ket{j,m_j}\\
  &= \hbar J_+\ket{j,m_j} + m_j \hbar J_+\ket{j,m_j}\\
  J_z(J_+\ket{j,m_j}) &= (m_j+1)\hbar(J_z\ket{j,m_j})
  \hspace{2cm}\mbox{J+ raises energy by $\hbar$}\\
\end{align*}

\begin{align*}
  J_z(J_-\ket{j,m_j}) &= (-\hbar J_- + J_-J_z)\ket{j,m_j}\\
  &= \hbar J_-\ket{j,m_j} + m_j \hbar J_-\ket{j,m_j}\\
  J_z(J_-\ket{j,m_j}) &= (m_j-1)\hbar(J_z\ket{j,m_j})
  \hspace{2cm}\mbox{J- lowers energy by $\hbar$}\\
\end{align*}

This shows that the operation of $J_\pm$ on a state before the operation of $J_z$ effectively
raises or lowers the measurement of that state by $\hbar$. We now need to find the relationship
between $J_\pm\ket{j,m_j}$ and $\ket{j,m_j\pm1}$.\\

\begin{align*}
  |J_+\ket{j,m_j}|^2 &= c^2\\
  &= \bra{j,m_j}J_-J_+\ket{j,m_j}\\
  &= \bra{j,m_j}(J_x-iJ_y)(J_x+iJ_y)\ket{j,m_j}\\
  &= \bra{j,m_j}J_xJ_x + J_yJ_y + i[J_x,J_y]\ket{j,m_j}\\
  &= \bra{j,m_j}J^2 - J_zJ_z + i[J_x,J_y]\ket{j,m_j}\\
  &= \bra{j,m_j}J^2 - J_zJ_z - \hbar J_z\ket{j,m_j}\\
  &= j(j+1)\hbar - m_j^2\hbar^2 - m_j\hbar^2\\
  &= \hbar^2\left(j(j+1)\hbar - m_j(m_j + 1)\right)\\
  &\rightarrow J_+\ket{j,m_j} = \hbar\sqrt{j(j+1) - m_j(m_j+1)}\ket{j,m_j+1}\\
\end{align*}

\begin{align*}
  |J_-\ket{j,m_j}|^2 &= c^2\\
  &= \bra{j,m_j}J_+J_-\ket{j,m_j}\\
  &= \bra{j,m_j}(J_x+iJ_y)(J_x-iJ_y)\ket{j,m_j}\\
  &= \bra{j,m_j}J_xJ_x + J_yJ_y - i[J_x,J_y]\ket{j,m_j}\\
  &= \bra{j,m_j}J^2 - J_zJ_z - i[J_x,J_y]\ket{j,m_j}\\
  &= \bra{j,m_j}J^2 - J_zJ_z + \hbar J_z\ket{j,m_j}\\
  &= j(j+1)\hbar^2 - m_j^2\hbar^2 + m_j\hbar&2\\
  &= \hbar^2\left(j(j+1) - m_j(m_j-1)\right)\\
  &\rightarrow J_-\ket{j,m_j} = \hbar\sqrt{j(j+1) - m_j(m_j-1)}\ket{j,m_j-1}\\
\end{align*}

We've just shown that the angular momentum ladder operators behave the same way the quantum
harmonic oscillator operators do. This is powerful, since it will allow us to simplify math
later on. It is important to note that the raising and lowering operators only act on the
Z-component of spin, and that they have two termination conditions: at the minimum and
maximum Z-component.\\

\textit{Inspired by Jared Cayton}\\

\section{11.7: $\vec{S} \cdot \vec{I}$ simplified}

Here we show how using angular momentum ladder operators can simplify the dot product
$\vec{S} \cdot \vec{I}$:\\

\begin{align*}
  \vec{S} \cdot \vec{I} &= \hat{S}_x\hat{I}_x + \hat{S}_y\hat{I}_y + \hat{S}_z\hat{I}_z\\
  &= \frac{1}{2}\left(\hat{S}_x\hat{I}_x -i\hat{S}_x\hat{I}_y + i\hat{S}_y\hat{I}_x
  + \hat{S}_y\hat{I}_y + \hat{S}_x\hat{I}_x + i\hat{S}_x\hat{I}_y
  + -i\hat{S}_y\hat{I}_x\hat{S}_y\hat{I}_y\right)
  + \hat{S}_z\hat{I}_z\\
  &= \frac{1}{2}\left(\hat{S}_+\hat{I}_- + \hat{S}_-\hat{I}_+\right) + \hat{S}_z\hat{I}_z\\
\end{align*}

This is a useful math trick with no physical meaning! Multiplying raising and lowering operators
together has no physical significance, let alone operators for different particles. Nevertheless
this will probably come in handy later.\\

\section{11.8: Hyperfine Hamiltonian}
Here we use ladder operators to find the hyperfine Hamiltonian for the Hydrogen atom:\\

\begin{align*}
  H_{hf}' &=
  \begin{pmatrix}
    \bra{++}H_{hf}'\ket{++} &
    \bra{++}H_{hf}'\ket{+-} &
    \bra{++}H_{hf}'\ket{-+} &
    \bra{++}H_{hf}'\ket{--} &\\
    \bra{+-}H_{hf}'\ket{++} &
    \bra{+-}H_{hf}'\ket{+-} &
    \bra{+-}H_{hf}'\ket{-+} &
    \bra{+-}H_{hf}'\ket{--} &\\
    \bra{-+}H_{hf}'\ket{++} &
    \bra{-+}H_{hf}'\ket{+-} &
    \bra{-+}H_{hf}'\ket{-+} &
    \bra{-+}H_{hf}'\ket{--} &\\
    \bra{--}H_{hf}'\ket{++} &
    \bra{--}H_{hf}'\ket{+-} &
    \bra{--}H_{hf}'\ket{-+} &
    \bra{--}H_{hf}'\ket{--} &\\
  \end{pmatrix}\\
  &=
  \frac{A}{\hbar^2}
  \begin{pmatrix}
    \bra{++}\vec{S}\cdot\vec{I}\ket{++} &
    \bra{++}\vec{S}\cdot\vec{I}\ket{+-} &
    \bra{++}\vec{S}\cdot\vec{I}\ket{-+} &
    \bra{++}\vec{S}\cdot\vec{I}\ket{--} &\\
    \bra{+-}\vec{S}\cdot\vec{I}\ket{++} &
    \bra{+-}\vec{S}\cdot\vec{I}\ket{+-} &
    \bra{+-}\vec{S}\cdot\vec{I}\ket{-+} &
    \bra{+-}\vec{S}\cdot\vec{I}\ket{--} &\\
    \bra{-+}\vec{S}\cdot\vec{I}\ket{++} &
    \bra{-+}\vec{S}\cdot\vec{I}\ket{+-} &
    \bra{-+}\vec{S}\cdot\vec{I}\ket{-+} &
    \bra{-+}\vec{S}\cdot\vec{I}\ket{--} &\\
    \bra{--}\vec{S}\cdot\vec{I}\ket{++} &
    \bra{--}\vec{S}\cdot\vec{I}\ket{+-} &
    \bra{--}\vec{S}\cdot\vec{I}\ket{-+} &
    \bra{--}\vec{S}\cdot\vec{I}\ket{--} &\\
  \end{pmatrix}\\
   = & \frac{A}{\hbar^2}
  \begin{pmatrix}
    \frac{\hbar^2}{4}&
    \bra{++}\frac{1}{2}S_-I_+\ket{+-}&
    \bra{++}\frac{1}{2}S_+I_-\ket{-+}&
    0&\\
    0&
    \bra{+-}\frac{1}{2}S_-I_+\ket{+-} - \frac{\hbar^2}{4}&
    \bra{+-}\frac{1}{2}S_+I_-\ket{-+}&
    0&\\
    0&
    \bra{-+}\frac{1}{2}S_-I_+\ket{+-}&
    \bra{-+}\frac{1}{2}S_+I_-\ket{-+} - \frac{\hbar^2}{4}&
    0&\\
    0&
    \bra{--}\frac{1}{2}S_-I_+\ket{+-}&
    \bra{--}\frac{1}{2}S_+I_-\ket{-+}&
    \frac{\hbar^2}{4} &\\
  \end{pmatrix}\\
  = & \frac{A}{\hbar^2}
  \begin{pmatrix}
    \frac{\hbar^2}{4}&
    0&
    0&
    0&\\
    0&
    -\frac{\hbar^2}{4}&
    \frac{2\hbar^2}{4}&
    0&\\
    0&
    \frac{2\hbar^2}{4}&
    -\frac{\hbar^2}{4}&
    0&\\
    0&
    0&
    0&
    \frac{\hbar^2}{4} &\\
  \end{pmatrix}\\
  = & \frac{A}{4}
  \begin{pmatrix}
    1&
    0&
    0&
    0&\\
    0&
    -1&
    2&
    0&\\
    0&
    2&
    -1&
    0&\\
    0&
    0&
    0&
    1&\\
  \end{pmatrix}\\
\end{align*}

We have succesfully shown how expanding the angular momentum dot product and using ladder
operators allows each element in our perturbation Hamiltonian to be solved for. Our end
product is a nondiagonal matrix with first and second-order corrections. Unfortunately our
original energies are degenerate, as this is the n=2 state. Thus we must diagonalize this
matrix to get a new diagonal perturbation matrix representing the changes to our original
system.\\

\end{document}

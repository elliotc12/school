\documentclass[10pt]{article} % Font size - 10pt, 11pt or 12pt

\usepackage[hmargin=1.25cm, vmargin=1.5cm]{geometry} % Document margins

\usepackage[usenames,dvipsnames]{xcolor} % Allows the definition of hex colors

% Fonts and tweaks for XeLaTeX
\usepackage{fontspec,xltxtra,xunicode}
\defaultfontfeatures{Mapping=tex-text}
%\setmonofont[Scale=MatchLowercase]{Andale Mono}

% Colors for links, text and headings
\usepackage{hyperref}
\definecolor{linkcolor}{HTML}{506266} % Blue-gray color for links
\definecolor{shade}{HTML}{F5DD9D} % Peach color for the contact information box
\definecolor{text1}{HTML}{2b2b2b} % Main document font color, off-black
\definecolor{headings}{HTML}{701112} % Dark red color for headings
% Other color palettes: shade=B9D7D9 and linkcolor=A40000; shade=D4D7FE and linkcolor=FF0080

\hypersetup{colorlinks,breaklinks, urlcolor=linkcolor, linkcolor=linkcolor} % Set up links and colors

\usepackage{titlesec}
\setcounter{secnumdepth}{4}
\titleformat{\paragraph}{\normalfont\normalsize\bfseries}{\theparagraph}{1em}{}
\titlespacing*{\paragraph}
{0px}{3.25ex plus 1ex minux .2ex}{1.5ex plus .2ex}

\usepackage{fancyhdr}
\usepackage{amssymb}
\usepackage{amsmath}
\pagestyle{fancy}
\fancyhf{}

\renewcommand{\headrulewidth}{0pt} % Get rid of the default rule in the header

\begin{document}

\newpage
\tableofcontents

\newpage

Table of Contents


\section{Chapter: Introduction}
	\subsection{What is dynein?}
		\subsubsection{How does it behave?}
		\subsubsection{What role does it play in the cell?}
		\subsubsection{Why does the cell need motor transport?}
		\subsubsection{Why is dynein special and interesting?}
	\subsection{How does dynein walk?}
		\subsubsection{Experimental results on dynein’s step}
		\subsubsection{Experimental features of dynein (ATPase, multi-state…)}
		\subsubsection{Powerstroke Theory}
		    \paragraph{a The theory}
		    \paragraph{b Evidence of theory}
		\subsubsection{c What is still needed to support theory}
		\subsubsection{Other theories?}
	\subsection{Past Simulations}
		\subsubsection{Why simulations are necessary}
		\subsubsection{Sarlah Winch Model}
		\subsubsection{-x Other models}
	\subsection{Brownian motion}
                \subsubsection{Introduction/derivation}
                \subsubsection{Justification}
\section{Chapter: Methods}
	\subsection{Our model}
		\subsubsection{Rigid point model}
		\subsubsection{Harmonic energies}
		\subsubsection{Two-state model}
		\subsubsection{D model}
	\subsection{Simulating our model}
		\subsubsection{Motion equations}
			\paragraph{a Onebound solution}
			\paragraph{b Bothbound solution}
		\subsubsection{Code}
			\paragraph{a Running, compiling, language choice?}
		\subsubsection{Time evolution}
			\paragraph{Euler’s method}
			\paragraph{Vs other methods (Runge Kutta, etc)}
			\paragraph{Finding the right dt}
			\paragraph{State transitions}
		        \paragraph{Gibbs energy of transition}
		\subsection{Parameters}
			\subsubsection{a-n Lengths, angles, binding constants, etc…}
	\subsection{Tuning our model}
		\subsubsection{Searching for optimal parameters}
\section{Chapter: Results}
	\subsection{Histograms}
		\subsubsection{Step length histograms}
		\subsubsection{Step time histograms}
	\subsection{Other ways to represent data?}


\section{Chapter: Discussion}
	\subsection{Why our model worked, didn’t work}
	\subsection{Future work on this project}
	\subsection{Possible further projects}

\end{document}

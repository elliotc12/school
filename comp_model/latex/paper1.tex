\documentclass[10pt]{article} % Font size - 10pt, 11pt or 12pt

\usepackage{amsmath}
\usepackage[hmargin=1.25cm, vmargin=1.5cm]{geometry} % Document margins

\usepackage[usenames,dvipsnames]{xcolor} % Allows the definition of hex colors

% Fonts and tweaks for XeLaTeX
\usepackage{fontspec,xltxtra,xunicode}
\defaultfontfeatures{Mapping=tex-text}
%\setmonofont[Scale=MatchLowercase]{Andale Mono}

% Colors for links, text and headings
\usepackage{hyperref}
\definecolor{linkcolor}{HTML}{506266} % Blue-gray color for links
\definecolor{shade}{HTML}{F5DD9D} % Peach color for the contact information box
\definecolor{text1}{HTML}{2b2b2b} % Main document font color, off-black
\definecolor{headings}{HTML}{701112} % Dark red color for headings
% Other color palettes: shade=B9D7D9 and linkcolor=A40000; shade=D4D7FE and linkcolor=FF0080

\hypersetup{colorlinks,breaklinks, urlcolor=linkcolor, linkcolor=linkcolor} % Set up links and colors

\usepackage{fancyhdr}
\pagestyle{fancy}
\fancyhf{}
% Headers and footers can be added with the \lhead{} \rhead{} \lfoot{} \rfoot{} commands
% Example footer:
%\rfoot{\color{headings} {\sffamily Last update: \today}. Typeset with Xe\LaTeX}

\renewcommand{\headrulewidth}{0pt} % Get rid of the default rule in the header

\usepackage{titlesec} % Allows creating custom \section's

% Format of the section titles
\titleformat{\section}{\color{headings}
\scshape\Large\raggedright}{}{0em}{}[\color{black}\titlerule]

\title{Effect of Parameters in Phase-Field Model for simulating Solid-Liquid Transitions}
\author{Elliott Capek}

\begin{document}
\maketitle{}

\section{Introduction}
The Phase-Field model is a method for simulating phase transitions in systems. Phase transitions in real life are discontinuous, making them hard to represent analytically. The Phase-Field method overcomes this by representing the phase of a region with a space- and time-dependent function $\phi(r,t)$. $\phi$ ranges from $0$ to $1$, with either end of the range being one phase or the other, and numbers between $0$ and $1$ representing transitional regions. These transitional regions, while not physical, can approximate reality well if the distance between a 0-zone and a 1-zone is as small as possible. \\

In this paper we seek to reproduce the results of Sanal [1], where a solid-liquid phase transition with anisotropic dependence is time-evolved. The anisotropy (angular dependence) of the equations used in this paper allow for radially-symmetric crystals to form. In this formulation, $\phi=1$ corresponds to solid phase and $\phi=0$ corresponds to liquid phase. The equation for time-evolving $\phi$ is given in Sanal as:

\begin{equation}  \label{eq:dPhidt}
  \tau \frac{\partial \phi}{\partial t} = -\frac{\partial}{\partial x} \left(\epsilon \frac{d\epsilon}{d\theta}\frac{\partial\phi}{\partial y}\right) + \frac{\partial}{\partial y} \left(\epsilon \frac{d\epsilon}{d\theta}\frac{\partial \phi}{\partial x}\right) + \nabla \cdot \left(\epsilon^2\nabla\phi\right) + \phi\left(1-\phi\right)\left(\phi-0.5+m\right)
\end{equation}

Here $\tau$ is the relaxation time, or the time it takes $\phi$ to come into equilibrium with its neighbors. $\epsilon$ is a small $\theta$-dependent parameter which controls how quickly the solid-liquid interface moves. $\theta$ is the angle from the origin measured from the horizontal. Making $\epsilon$ depend on $\theta$ introduces anisotropy into the model, allowing different angles to spread at different rates. $m$ is the thermodynamic driving force, a temperature-dependent number which determines how favorable solidification is. Larger values of temperature (and hence $m$) correspond to quicker solidification. \\

The equation for $\epsilon(\theta)$ is given as follows:

\begin{equation}
  \epsilon = 1 + \delta\cos\left(j\left(\theta_0-\theta\right)\right)
\end{equation}

where $\delta$ is the strength of anisotropy, $j$ is the harmonic number and $\theta_0$ is the angular offset. The higher $j$ is, the more angular extrema will exist in the system.\\

Temperature is also important to this model, as it decides the value of m, the rate of interface front movement. Temperature's time-evolution is given by:

\begin{equation}  \label{eq:dTdt}
  \frac{\partial T}{\partial t} = \nabla^2T + k\frac{\partial \phi}{\partial t}
\end{equation}

where k is the latent heat of the phase change. Liquid freezing into a solid is exergonic, so heat is evolved when $\frac{\partial\phi}{\partial t}$ is positive. \\

\section{Method}
To simulate the behavior of the differential equation given by Sanal, we use the forwards-Euler finite difference method to time-evolve Equations (\ref{eq:dPhidt}) and (\ref{eq:dTdt}). We implement this in a C program (see Code section) and a Python script using matplotlib to plot our results.

\section{Results}

\section{Next Steps}

\end{document}

\documentclass[a4paper, 12pt]{article}
\usepackage{fullpage} % changes the margin

\begin{document}
\noindent
\huge\textbf{Project proposal}\\

\large Elliott Capek, \today
\normalsize

\section*{Summary}
Coherence potentials are activity patterns which propagate with high fidelity through space. Their transmission fidelity and information content suggests that they are not the result of a few neurons firing stochastically, but rather the synchronous dynamics of microcircuits. If microcircuit ``modules'' could be made to take on stable, synchronous, whole-module firing states, and if these firing states could be entrained onto distant modules through long-range connections, then this could be an explanation for coherence potentials. I propose to test this idea by creating a leaky integrate-and-fire network with modular connectivity, implementing plasticity rules known to produce modular synchrony, then studying the pseudo-LFP dynamics of the model using a heuristic method. By varying simulation conditions such as external currents, model parameters, connectivities between modules and plasticity rules, I hope to create a network with the following properties:

\begin{itemize}
\item Modules within the network can be entrained into stable synchronous states through external currents
\item These stable states are varied in nature enough that their average activity is as complex as real coherence potential signals
\item These states are ``contagious'' and can entrain other distantly-connected modules into the same state, perpetuating the coherence potential
\end{itemize}

If a network meets these three requirements, then it should demonstrate coherence potentials.

\section*{Objectives}
\begin{itemize}
\item Create a biologically realistic \cite{objectworkingmemory, biologicalsoc} leaky integrate-and-fire (LIF) model of a cortical area which can reproduce basic single-neuron details from 2PI \cite{bellaypaper}, such as firing rate distribution, branching parameter, and time-binned avalanche dynamics.
  \begin{itemize}
  \item Reproduce the theoretical synchronous-to-asynchronous phase transition \cite{munozlg} using a synaptic resources parameter tuned to avoid the bistable up-down regime and non-critical (neutral) avalanches. Other models have found a continuous (presumably not up/down) phase transition \cite{rubinov}, so it is likely we will be able to find one as well. Other models have remained within the bistable regime and thus see only an up/down transition with neutral avalanches \cite{neutraltheory}.
  \end{itemize}
\item Explore parameter values and model features, such as hierarchical modularity \cite{rubinov, mountcastle}, short-term plasticity \cite{dynamicsynapses} and spike time-dependent plasticity \cite{heterogeneousnetwork, biologicalsoc, rubinov} to achieve module (but not network)-wide synchronous behavior, such as oscillations \cite{munozlg} or module spikes \cite{rubinov}.
  \begin{itemize}
  \item Use the ``exact'' solution method to very rapidly simulate models at different parameter values, allowing a large parameter-space to be explored quickly \cite{exactsolution}
  \item This is the step which will decide the project. If a module can be made to enter into stable, high-information states merely through input currents (externally or from other modules), then the hardest bridge will be crossed
  \end{itemize}
\item Implement a heuristic-LFP method for estimating the electrode measurements in the model network; perhaps starting with simply averaging local module activity. Yahya would probably know more about this.
\item Devise a method for ``entraining'' desired modules to enter into stable states, for example by feeding the same custom waveform into all the neurons in a single module as an external current. It has been shown that ``tetanizing'' a network can modify it through plasticity to reproduce the tetanized signal in a phenomena known as ``ommitted stimulus potential'' \cite{heterogeneousnetwork}; perhaps this phenomena would work for other inputs
\end{itemize}

%% \section*{Alternate Objectives}
%% \begin{itemize}
%% \item Create a biologically realistic leaky integrate-and-fire (LIF) model of a cortical section which can reproduce single-neuron details from 2PI \cite{bellaypaper}, such as firing rate distribution and quiescent time, correlations (average cross-correlation, distance-dependent correlation and autocorrelation), branching parameter, and time-binned avalanche dynamics
%% \end{itemize}

\section*{Methods}

\subsection*{Modular hierarchy}

Recent models for self-organized criticality have used networks of many neural modules with hundreds to thousands of tightly-connected neurons each \cite{munozlg, rubinov}. We will use a similar scheme to compare with these results.\\

Each module will follow the 80/20 principle of excitatory and inhibitory neurons, which is important for information processing \cite{larremorerestrepo, entropyinhibition}. Brunel and Wang used a model which used biologically realistic values for membrane capacitance and channel time constant to describe neurons with excitatory, GABA or NMDA/AMPA channels \cite{objectworkingmemory}. Using realistic proportions of these neurons to organize a module would be a nice way to make the model biologically realistic.\\

\subsection*{Connectivity}
Rubinov \textit{et al} start with 100-neuron modules which are uniformly connected, then use different connection rules to link neurons inter-module \cite{rubinov}. This could be a good place to start.\\

\subsection*{Plasticity \& synaptic resources}
It seems that many different papers which achieve self-organised criticality use ``dynamical variables'' of some sort (eg synapse weight, synaptic resource or gain) \cite{munozlg, rubinov, kinouchi}. These variables have lead to interesting dynamics by their ability to store and integrate information at a slower timescale than voltage. In particular, Rubinov \textit{et al} describe how dynamic synapses allow slow mesoscopic oscillations which are abolished when synapses are frozen \cite{rubinov}. Dynamic synapses have also been shown to be important for self-organized criticality \cite{dynamicsynapses}. It has also been shown that a combination of short-term plasticity (STP) and spike-time-dependent plasticity (STDP) can have synergistic effects and create very robust networks with intersting dynamics \cite{biologicalsoc}, and STDP may be important for the formation of neural communities and nonperiodic synchronicity \cite{heterogeneousnetwork}. A recent report showed SOC in a model with a supposedly biologically realistic dynamic ``gain'' instead of dynamic synapses, and claimed it was much faster to simulate. Perhaps this would be a thing to look into \cite{kinouchi}.\\

It may also be important to have non-identical neurons, since this creates nonperiodicity \cite{heterogeneousnetwork}.

\subsection*{Voltage equations}
TBD, many useful papers: \cite{objectworkingmemory, rubinov, kinouchi, larremorerestrepo, entropyinhibition, dynamicsynapses, biologicalsoc, heterogeneousnetwork}

\section*{Todo}
\begin{itemize}
\item Find more models which create synchronous network states
\end{itemize}

\bibliographystyle{plain}
\bibliography{report}

\end{document}

\documentclass[a4paper, 12pt]{article}
\usepackage{fullpage} % changes the margin

\begin{document}
\noindent
\huge\textbf{Project report}
\normalsize

\section*{Introduction}

\section*{Objectives}

\begin{itemize}
\item Create a biologically realistic leaky integrate-and-fire (LIF) model of a cortical section which can reproduce single-neuron details from 2PI \cite{bellaypaper}, such as firing rate distribution and quiescent time, correlations (average cross-correlation, distance-dependent correlation and autocorrelation), branching parameter, and time-binned avalanche dynamics
  
\item Use biophysical hierarchy of neural connections \cite{rubinov} and represent the main pyramidal and interneuron types within cortex using realistic parameters \cite{objectworkingmemory}
  
\item Reproduce the theoretical synchronous-to-asynchronous phase transition \cite{munozlg} using a synaptic resources parameter tuned to avoid the bistable up-down regime and non-critical (neutral) avalanches. Other models have found a continuous (presumably not up/down) phase transition \cite{rubinov}, so it is likely we will be able to find one as well. Other models have remained within the bistable regime and thus see only an up/down transition with neutral avalanches \cite{neutraltheory}.
  
\item Use the 'exact' solution method \cite{exactsolution} to quickly simulate large networks, possibly making the model fast enough to explore the parameter space to find different network modes (eg oscillation, up/down bistability \cite{munozlg}), or to optimize via gradient descent
  
\item schizophrenia - something something either simulate PCP problems biophysically and try to reproduce the PCP phenotype OR manually just increase the repeat-firing of cortical neurons
  
\item Possibly introduce a spatial component to neurons to estimate LFP electrode measurements from the LIF network, allowing us to see if the model can be made to reproduce coherence potentials
\end{itemize}

\section*{Methods}

\subsection*{Modular hierarchy}
\subsection*{Connectivity}
\subsection*{Plasticity & synaptic resources}
\subsection*{Voltage equations}

\section*{TODO}
\begin{itemize}
\item Find papers on LIF cortical organization - how should we connect the modules? Should we do modules? What should they look like?
\end{itemize}

\cite{lippert}

\bibliographystyle{plain}
\bibliography{report}

\end{document}

\documentclass[10pt]{article} % Font size - 10pt, 11pt or 12pt

\nonstopmode

\usepackage[hmargin=1.25cm, vmargin=1.5cm]{geometry} % Document margins

\usepackage{amsmath}

\usepackage[usenames,dvipsnames]{xcolor} % Allows the definition of hex colors

% Fonts and tweaks for XeLaTeX
\usepackage{fontspec,xltxtra,xunicode}
\defaultfontfeatures{Mapping=tex-text}
%\setmonofont[Scale=MatchLowercase]{Andale Mono}

% Colors for links, text and headings
\usepackage{hyperref}
\definecolor{linkcolor}{HTML}{506266} % Blue-gray color for links
\definecolor{shade}{HTML}{F5DD9D} % Peach color for the contact information box
\definecolor{text1}{HTML}{2b2b2b} % Main document font color, off-black
\definecolor{headings}{HTML}{701112} % Dark red color for headings
% Other color palettes: shade=B9D7D9 and linkcolor=A40000; shade=D4D7FE and linkcolor=FF0080

\hypersetup{colorlinks,breaklinks, urlcolor=linkcolor, linkcolor=linkcolor} % Set up links and colors

\usepackage{fancyhdr}
\usepackage{amssymb}
\pagestyle{fancy}
\fancyhf{}
% Headers and footers can be added with the \lhead{} \rhead{} \lfoot{} \rfoot{} commands
% Example footer:
%\rfoot{\color{headings} {\sffamily Last update: \today}. Typeset with Xe\LaTeX}

\renewcommand{\headrulewidth}{0pt} % Get rid of the default rule in the header

\usepackage{titlesec} % Allows creating custom \section's

% Format of the section titles
\titleformat{\section}{\color{headings}
\scshape\Large\raggedright}{}{0em}{}[\color{black}\titlerule]

\title{PH651 Assignment Five}
\author{Elliott Capek}
\titlespacing{\section}{0pt}{0pt}{5pt} % Spacing around titles

\newcommand{\bra}[1]{\big<#1\big|}
\newcommand{\ket}[1]{\big|#1\big>}
\newcommand{\braket}[2]{\big<#1\big|#2\big>}

\begin{document}

\maketitle{}

\begin{align*}
  A =
  \begin{pmatrix}
    7 & 0 & 0\\
    0 & 1 & -i\\
    0 & i & -1\\
  \end{pmatrix}
  \hspace{2cm}
  &
  B =
  \begin{pmatrix}
    1 & 0 & 3\\
    0 & 2i & 0\\
    i & 0 & -5i\\
  \end{pmatrix}
\end{align*}

\textbf{a.)}
A is Hermitian:
\begin{align*}
  A^\dagger &=
  \begin{pmatrix}
    7 & 0 & 0\\
    0 & 1 & -i\\
    0 & i & -1\\
  \end{pmatrix}^{*T}
  =
  \begin{pmatrix}
    7 & 0 & 0\\
    0 & 1 & i\\
    0 & -i & -1\\
  \end{pmatrix}^{T}
  =
  \begin{pmatrix}
    7 & 0 & 0\\
    0 & 1 & -i\\
    0 & i & -1\\
  \end{pmatrix}
  = A
\end{align*}

B cannot be Hermetian because it has complex diagonals, so $A_{nn} \neq A_{nn}^*$:

\begin{align*}
  B^\dagger = 
  \begin{pmatrix}
    1 & 0 & -i\\
    0 & -2i & 0\\
    3 & 0 & 5i\\
  \end{pmatrix}
\end{align*}

\textbf{b.)}
First we find the eigenvalues of A:
\begin{align*}
  \det
  \begin{bmatrix}
    7 -\lambda& 0 & 0\\
    0 & 1 -\lambda& -i\\
    0 & i & -1 -\lambda\\
  \end{bmatrix}
  &= (7-\lambda)\left((1-\lambda)(-1-\lambda) - (-i*i)\right) = \left(7-\lambda\right)\left(\lambda^2-2\right)\\
  &\rightarrow \lambda = 7, \pm\sqrt{2} 
\end{align*}

Then the eigenvectors of A:

\begin{align*}
  \begin{pmatrix}
    0 & 0 & 0\\
    0 & -6 & -i\\
    0 & i & -8\\
  \end{pmatrix}
  \begin{pmatrix}
    x\\
    y\\
    z\\
  \end{pmatrix}
  =
  \begin{pmatrix}
    0\\
    0\\
    0\\
  \end{pmatrix}
  &
  \rightarrow
  \begin{pmatrix}
    x = x\\
    -6y-iz = 0\\
    iy - 8z = 0\\
  \end{pmatrix}
  \rightarrow
  \ket{7} =
  \begin{pmatrix}
    x = 1\\
    y = 0\\
    z = 0\\
  \end{pmatrix}\\
  \begin{pmatrix}
    7 - \sqrt{2} & 0 & 0\\
    0 & 1-\sqrt{2} & -i\\
    0 & i & -1-\sqrt{2}\\
  \end{pmatrix}
  \begin{pmatrix}
    x\\
    y\\
    z\\
  \end{pmatrix}
  =
  \begin{pmatrix}
    0\\
    0\\
    0\\
  \end{pmatrix}
  &
  \rightarrow
  \begin{pmatrix}
    x = 0\\
    \left(1-\sqrt{2}\right)y -iz = 0\\
    iy - \left(1+\sqrt{2}\right)z = 0\\
  \end{pmatrix}
  \rightarrow
  \ket{\sqrt{2}} = \frac{1}{\sqrt{4+2\sqrt{2}}}
  \begin{pmatrix}
    x = 0\\
    y = -\left(1+\sqrt{2}\right)i\\
    z = 1\\
  \end{pmatrix}\\
  \begin{pmatrix}
    7+\sqrt{2} & 0 & 0\\
    0 & 1+\sqrt{2} & -i\\
    0 & i & -1+\sqrt{2}\\
  \end{pmatrix}
  \begin{pmatrix}
    x\\
    y\\
    z\\
  \end{pmatrix}
  =
  \begin{pmatrix}
    0\\
    0\\
    0\\
  \end{pmatrix}
  &
  \rightarrow
  \begin{pmatrix}
    x = x\\
    \left(1+\sqrt{2}\right)y = iz\\
    iy = \left(\sqrt{2}-1\right)z\\
  \end{pmatrix}
  \rightarrow
  \ket{-\sqrt{2}} = \frac{1}{\sqrt{4-2\sqrt{2}}}
  \begin{pmatrix}
    x = 0\\
    y = \left(\sqrt{2}-1\right)i\\
    z = 1\\
  \end{pmatrix}
\end{align*}

Tr(A) always equals the sum of eigenvalues of a matrix regardless of basis. Thus the sum of A's eigenvalues ($7 + \sqrt{2} - \sqrt{2}$)
is the same as the sum of diagonals $(7 + 1 - 1)$.\\

\textbf{c}
To show vectors form a complete and orthonormal basis, we show that the sum of their projection operators equals the identity matrix:

\textbf{CHECK IF TIME}

\begin{align*}
  \begin{pmatrix}
    1\\
    0\\
    0\\
  \end{pmatrix}
  \begin{pmatrix}
    1 & 0 & 0\\
  \end{pmatrix} +
  \frac{1}{4+2\sqrt{2}}
  \begin{pmatrix}
    0\\
    -\left(1+\sqrt{2}\right)i\\
    1\\
  \end{pmatrix}
  \begin{pmatrix}
    0 & \left(1+\sqrt{2}\right)i & 1\\
  \end{pmatrix} +
  \frac{1}{4-2\sqrt{2}}
  \begin{pmatrix}
    0\\ \left(\sqrt{2}-1\right)i \\ 1\\
  \end{pmatrix}
  \begin{pmatrix}
    0 & -\left(\sqrt{2}-1\right)i & 1\\
  \end{pmatrix}
\end{align*}

\begin{align*}
  =
  \begin{pmatrix}
    1 + 0 + 0 & 0 + 0 + 0 & 0 + 0 + 0\\
    0 + 0 + 0 & 0 + \left(1+\sqrt{2}\right)^2 + \left(\sqrt{2}-1\right)^2& 0 -\left(1+\sqrt{2}\right)i + \left(\sqrt{2}-1\right)i \\
    0 + 0 + 0 & 0 + \left(1+\sqrt{2}\right)i - \left(\sqrt{2}-1\right)i & 0 + 0 + 1\\
  \end{pmatrix}
  =
  \begin{pmatrix}
    1 & 0 & 0\\
    0 & 1 & 0\\
    0 & 0 & 1\\
  \end{pmatrix}
\end{align*}

\textbf{d}

\begin{align*}
  AB =
  \begin{pmatrix}
    7 & 0 & 0\\
    0 & 1 & -i\\
    0 & i & -1\\
  \end{pmatrix}
  \begin{pmatrix}
    1 & 0 & 3\\
    0 & 2i & 0\\
    i & 0 & -5i\\
  \end{pmatrix}
  &=
  \begin{pmatrix}
    7 & 0 & 21\\
    1 & 2i & -5\\
    -i & -2 & 5i\\
  \end{pmatrix}\\
  \det(AB) = 7\left(2i*5i - 10\right) + 21\left(-2-2\right) &= -224\\
\end{align*}

\begin{align*}
  BA =
  \begin{pmatrix}
    1 & 0 & 3\\
    0 & 2i & 0\\
    i & 0 & -5i\\
  \end{pmatrix}
  \begin{pmatrix}
    7 & 0 & 0\\
    0 & 1 & -i\\
    0 & i & -1\\
  \end{pmatrix}
  &=
  \begin{pmatrix}
    7 & 0 & 21\\
    1 & 2i & -5\\
    -i & -2 & 5i\\
  \end{pmatrix}\\
  \det(BA) = 7\left(2i*5i - 10\right) - 3i(-2*7i) + -3(2i*7i) = -224\\
\end{align*}

\end{document}

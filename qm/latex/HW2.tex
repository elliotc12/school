\documentclass[10pt]{article} % Font size - 10pt, 11pt or 12pt

\nonstopmode

\usepackage[hmargin=1.25cm, vmargin=1.5cm]{geometry} % Document margins

\usepackage{amsmath}

\usepackage[usenames,dvipsnames]{xcolor} % Allows the definition of hex colors

% Fonts and tweaks for XeLaTeX
\usepackage{fontspec,xltxtra,xunicode}
\defaultfontfeatures{Mapping=tex-text}
%\setmonofont[Scale=MatchLowercase]{Andale Mono}

% Colors for links, text and headings
\usepackage{hyperref}
\definecolor{linkcolor}{HTML}{506266} % Blue-gray color for links
\definecolor{shade}{HTML}{F5DD9D} % Peach color for the contact information box
\definecolor{text1}{HTML}{2b2b2b} % Main document font color, off-black
\definecolor{headings}{HTML}{701112} % Dark red color for headings
% Other color palettes: shade=B9D7D9 and linkcolor=A40000; shade=D4D7FE and linkcolor=FF0080

\hypersetup{colorlinks,breaklinks, urlcolor=linkcolor, linkcolor=linkcolor} % Set up links and colors

\usepackage{fancyhdr}
\usepackage{amssymb}
\pagestyle{fancy}
\fancyhf{}
% Headers and footers can be added with the \lhead{} \rhead{} \lfoot{} \rfoot{} commands
% Example footer:
%\rfoot{\color{headings} {\sffamily Last update: \today}. Typeset with Xe\LaTeX}

\renewcommand{\headrulewidth}{0pt} % Get rid of the default rule in the header

\usepackage{titlesec} % Allows creating custom \section's

% Format of the section titles
\titleformat{\section}{\color{headings}
\scshape\Large\raggedright}{}{0em}{}[\color{black}\titlerule]

\title{PH651 Assignment Five}
\author{Elliott Capek}
\titlespacing{\section}{0pt}{0pt}{5pt} % Spacing around titles

\newcommand{\bra}[1]{\big<#1\big|}
\newcommand{\ket}[1]{\big|#1\big>}
\newcommand{\braket}[2]{\big<#1\big|#2\big>}

\begin{document}

\maketitle{}

\begin{align*}
  A =
  \begin{pmatrix}
    7 & 0 & 0\\
    0 & 1 & -i\\
    0 & i & -1\\
  \end{pmatrix}
  \hspace{2cm}
  &
  B =
  \begin{pmatrix}
    1 & 0 & 3\\
    0 & 2i & 0\\
    i & 0 & -5i\\
  \end{pmatrix}
\end{align*}

\textbf{a.)}
A is Hermitian:
\begin{align*}
  A^\dagger &=
  \begin{pmatrix}
    7 & 0 & 0\\
    0 & 1 & -i\\
    0 & i & -1\\
  \end{pmatrix}^{*T}
  =
  \begin{pmatrix}
    7 & 0 & 0\\
    0 & 1 & i\\
    0 & -i & -1\\
  \end{pmatrix}^{T}
  =
  \begin{pmatrix}
    7 & 0 & 0\\
    0 & 1 & -i\\
    0 & i & -1\\
  \end{pmatrix}
  = A
\end{align*}

B cannot be Hermetian because it has complex diagonals, so $A_{nn} \neq A_{nn}^*$:

\begin{align*}
  B^\dagger = 
  \begin{pmatrix}
    1 & 0 & -i\\
    0 & -2i & 0\\
    3 & 0 & 5i\\
  \end{pmatrix}
\end{align*}

\textbf{b.)}
First we find the eigenvalues of A:
\begin{align*}
  \det
  \begin{bmatrix}
    7 -\lambda& 0 & 0\\
    0 & 1 -\lambda& -i\\
    0 & i & -1 -\lambda\\
  \end{bmatrix}
  &= (7-\lambda)\left((1-\lambda)(-1-\lambda) - (-i*i)\right) = \left(7-\lambda\right)\left(-2-\lambda^2\right)\\
  &\rightarrow (Wolfram) \rightarrow \lambda = 7, \pm\sqrt{2} 
\end{align*}

Then the eigenvalues of B:
\begin{align*}
  \det
  \begin{bmatrix}
    1 -\lambda& 0 & 3\\
    0 & 2i-\lambda & 0\\
    i & 0 & -5i-\lambda\\
  \end{bmatrix}
  &= \left(1-\lambda\right)\left(2i-\lambda\right)\left(-5i-\lambda\right) + 3\left(-\left(2i-\lambda\right)i\right)\\
  &\rightarrow (Wolfram) \rightarrow \lambda = 2i, \frac12\left(\left(1-5i\right)\pm\sqrt{-24+22i}\right)
\end{align*}

Then the eigenvectors of A:

\begin{align*}
  \begin{pmatrix}
    7 & 0 & 0\\
    0 & 1 & -i\\
    0 & i & -1\\
  \end{pmatrix}
  \begin{pmatrix}
    x\\
    y\\
    z\\
  \end{pmatrix}
  = 3
  \begin{pmatrix}
    x\\
    y\\
    z\\
  \end{pmatrix}
  &
  \rightarrow
  \begin{pmatrix}
    7x = 3x\\
    y-iz = 3y\\
    iy - z = 3z\\
  \end{pmatrix}
  \rightarrow
  \ket{3} =
  \begin{pmatrix}
    x = 0\\
    y-iz = 3y\\
    iy - z = 3z\\
  \end{pmatrix}
\end{align*}

\end{document}

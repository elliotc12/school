\documentclass[10pt]{article} % Font size - 10pt, 11pt or 12pt

\nonstopmode

\usepackage[hmargin=1.25cm, vmargin=1.5cm]{geometry} % Document margins

\usepackage{amsmath}

\usepackage[usenames,dvipsnames]{xcolor} % Allows the definition of hex colors

% Fonts and tweaks for XeLaTeX
\usepackage{fontspec,xltxtra,xunicode}
\defaultfontfeatures{Mapping=tex-text}
%\setmonofont[Scale=MatchLowercase]{Andale Mono}

% Colors for links, text and headings
\usepackage{hyperref}
\definecolor{linkcolor}{HTML}{506266} % Blue-gray color for links
\definecolor{shade}{HTML}{F5DD9D} % Peach color for the contact information box
\definecolor{text1}{HTML}{2b2b2b} % Main document font color, off-black
\definecolor{headings}{HTML}{701112} % Dark red color for headings
% Other color palettes: shade=B9D7D9 and linkcolor=A40000; shade=D4D7FE and linkcolor=FF0080

\hypersetup{colorlinks,breaklinks, urlcolor=linkcolor, linkcolor=linkcolor} % Set up links and colors

\usepackage{fancyhdr}
\usepackage{amssymb}
\pagestyle{fancy}
\fancyhf{}
% Headers and footers can be added with the \lhead{} \rhead{} \lfoot{} \rfoot{} commands
% Example footer:
%\rfoot{\color{headings} {\sffamily Last update: \today}. Typeset with Xe\LaTeX}

\renewcommand{\headrulewidth}{0pt} % Get rid of the default rule in the header

\usepackage{titlesec} % Allows creating custom \section's

% Format of the section titles
\titleformat{\section}{\color{headings}
\scshape\Large\raggedright}{}{0em}{}[\color{black}\titlerule]

\title{Elegromagnetism Assignment Five}
\author{Elliott Capek}
\titlespacing{\section}{0pt}{0pt}{5pt} % Spacing around titles

\newcommand{\bra}[1]{\big<#1\big|}
\newcommand{\ket}[1]{\big|#1\big>}
\newcommand{\braket}[2]{\big<#1\big|#2\big>}

\begin{document}

\maketitle{}

\section{Problem One}
To solve for the current in a capacitor/nested inductor circuit, derive a differential equation for current using rules about inductors and capacitors, then solve it. First the inductance. Flux for two antiparallel superimposed inductors with loop densities $n_a$ and $n_b$, length $\ell$ and radius $a$ is given by:\\

\begin{align*}
  \Phi &= BA = \frac{\mu \left(N_a^2-N_b^2\right) I}{\ell}\pi a^2 = \mu \left(n_a^2-n_b^2\right)I\ell\pi a^2\\
\end{align*}

From one of Maxwell's equations:\\

\begin{align*}
  V_L &= \frac{d\Phi}{dt} = \mu \left(n_a^2-n_b^2\right)\ell\pi a^2\frac{d^2Q}{dt^2}\\
\end{align*}

We also know the ovltage drop across a parallel plate capacitor:\\

\begin{align*}
  V_c &= \frac{Q}{C} = \frac{Qd}{\epsilon_0 A} = \frac{Qd}{\epsilon_0 \pi a^2}\\
\end{align*}

Using Kirchoff's Loop Law:\\

\begin{align*}
  V_c &= -V_L\\
  \frac{Qd}{\epsilon_0 \pi a^2} &= \mu \left(n_a^2-n_b^2\right)\ell\pi a^2\frac{d^2Q}{dt^2}\\
  -\frac{d}{\pi^2 a^4\mu\left(n_a^2-n_b^2\right)\ell}Q &= \frac{d^2Q}{dt^2}\\
\end{align*}

The solution to which is, noting that Q must be real:\\

\begin{align*}
  Q(t) &= Q_0\cos(\sqrt{\frac{d}{\epsilon_0\pi^2 a^4\mu\left(n_a^2-n_b^2\right)\ell}}t) = \cos(\omega t)\\
  \omega &= \sqrt{\frac{d}{\epsilon_0\pi^2 a^4\mu\left(n_a^2-n_b^2\right)\ell}} = \sqrt{\frac{1}{LC}}\\
\end{align*}

\textbf{b.) Energy in circuit}
The individual component energies are:\\

\begin{align*}
  U_L &= \frac{1}{2}LI^2 = \left(\mu(n_a^2-n_b^2)\ell\pi a^2\right)\left(\frac{d}{\pi^2 a^4\mu\left(n_a^2-n_b^2\right)\ell}\right)Q_0^2\sin^2(\omega t)\\
  &= \left(\frac{d}{2\epsilon_0\pi a^2}\right)Q_0^2\sin^2(\omega t)\\
  U_C &= \frac{1}{2}\frac{Q^2}{C} = \frac{d}{2\epsilon_0 \pi a^2}Q_0^2\cos^2(\omega t)\\
\end{align*}

Thus:\\

\begin{align*}
  U &= \frac{Q_0^2d}{\epsilon_0 \pi a^2}(\cos^2(\omega t)+\sin^2(\omega t))\\
  &= \frac{Q_0^2d}{\epsilon_0 \pi a^2}\\
\end{align*}

\textbf{c.) Poynting vector}

Find the capacitor Poynting vector using oscillating E-field:

\begin{align*}
  V &= \frac{Q}{C}Ed = \frac{Q_0\cos(\omega t)d}{\epsilon_0 A}\\
  \vec{E} &= \frac{Q_0\cos(\omega t)}{\epsilon_0 A}
\end{align*}

The B-field in the capacitor can be found using Maxwell's Equations:\\

\begin{align*}
  \nabla \times B &= \mu_0\epsilon_0 \frac{dE}{dt} = -\frac{\mu_0\omega Q\sin(\omega t)\hat{z}}{A}\\
  \frac{1}{r}(\frac{d}{dr}(rB_\phi)-\frac{dB_r}{d\phi} &= \frac{d}{dr}(rB_\phi)\\
  \vec{B} &= \frac{r\mu_0\omega Q_0 \sin(\omega t)\hat{\vec}}{2A}\\
\end{align*}

Thus the Poynting vector is:\\

\begin{align*}
  \vec{S} &= \frac{1}{\mu_0}EB = \frac{Q^2}{2\epsilon_0A^2}r\omega \cos(\omega t)\sin(\omega t)\vec{r}
\end{align*}

The vector points radially; thus light will radiate from the capacitor during its operation, leaking energy.\\

\end{document}

\documentclass[10pt]{article} % Font size - 10pt, 11pt or 12pt

\nonstopmode

\usepackage[hmargin=1.25cm, vmargin=1.5cm]{geometry} % Document margins

\usepackage{amsmath}

\usepackage[usenames,dvipsnames]{xcolor} % Allows the definition of hex colors

% Fonts and tweaks for XeLaTeX
\usepackage{fontspec,xltxtra,xunicode}
\defaultfontfeatures{Mapping=tex-text}
%\setmonofont[Scale=MatchLowercase]{Andale Mono}

% Colors for links, text and headings
\usepackage{hyperref}
\definecolor{linkcolor}{HTML}{506266} % Blue-gray color for links
\definecolor{shade}{HTML}{F5DD9D} % Peach color for the contact information box
\definecolor{text1}{HTML}{2b2b2b} % Main document font color, off-black
\definecolor{headings}{HTML}{701112} % Dark red color for headings
% Other color palettes: shade=B9D7D9 and linkcolor=A40000; shade=D4D7FE and linkcolor=FF0080

\hypersetup{colorlinks,breaklinks, urlcolor=linkcolor, linkcolor=linkcolor} % Set up links and colors

\usepackage{fancyhdr}
\usepackage{amssymb}
\pagestyle{fancy}
\fancyhf{}
% Headers and footers can be added with the \lhead{} \rhead{} \lfoot{} \rfoot{} commands
% Example footer:
%\rfoot{\color{headings} {\sffamily Last update: \today}. Typeset with Xe\LaTeX}

\renewcommand{\headrulewidth}{0pt} % Get rid of the default rule in the header

\usepackage{titlesec} % Allows creating custom \section's

% Format of the section titles
\titleformat{\section}{\color{headings}
\scshape\Large\raggedright}{}{0em}{}[\color{black}\titlerule]

\title{PH651 Assignment Five}
\author{Elliott Capek}
\titlespacing{\section}{0pt}{0pt}{5pt} % Spacing around titles

\newcommand{\bra}[1]{\big<#1\big|}
\newcommand{\ket}[1]{\big|#1\big>}
\newcommand{\braket}[2]{\big<#1\big|#2\big>}

\begin{document}

\maketitle{}

\section{Problem One}
To find the bound charges on this cross section, we use the following equations:

\begin{align*}
  \rho_b &= -\nabla \vec{P} = 0\hspace{2cm} \mbox{P is constant}\\
  \sigma_b &= \vec{P} \cdot \vec{n}\\
\end{align*}

The top two faces each make a $60^\circ$ angle with $\vec{P}$, and the bottom one is antiparallel. Thus we see:

\begin{align*}
  \sigma_{b,top} &= \vec{P} \cdot \hat{n} = P\cos\left(60^\circ\right) = \frac P2\\
  \sigma_{b,bot} &= \vec{P} \cdot \hat{n} = P\cos\left(180^\circ\right) = -P\\
\end{align*}

To find total charge we just sum these densities over the surface area of the prism. We notice that there are two top faces and one
bottom face, thus the two surface charges cancel out and the net charge on the object is zero. We expect this for a dielectric with
no free charge.

\section{Problem Two}
We let the radius of the internal dielectric be \textbf{a} and the outer dielectric have radius \textbf{b}. We then list the
boundary conditions for inner and outer potentials $\phi_1$ and $\phi_2$ and outer potential $\phi_o$.\\

\textbf{Boundaries}
\begin{align*}
  \phi_1(a) &= \phi_2(a)\\
  \phi_2(b) &= \phi_o(b)\\
  \frac{\partial \phi_1(a)}{\partial r} &= \frac{\partial \phi_2(a)}{\partial r}\\
  \frac{\partial \phi_2(b)}{\partial r} &= \frac{\partial \phi_o(b)}{\partial r}\\
  \phi(r>>b) &= -E_or\cos\theta
\end{align*}

Our solutions will take the general form:

\begin{align*}
  \phi &= \sum_{\ell = 0}^\infty \left(A_\ell r^\ell + \frac{B_\ell}{r^{\ell+1}}\right)P_\ell(\cos\theta)
\end{align*}

For the inner potential $\phi_1$ we don't permit any B-terms, since the inverse r would blow up at zero. For the middle potential $\phi_2$ we can apply no constraints. For the outer potential $\phi_o$ we say that all A are zero, else the potential would blow up at infinity. Thus:

\begin{align*}
  \phi_1 &= \sum_{\ell = 0}^\infty A^1_\ell r^\ell P_\ell(\cos\theta)\\
  \phi_2 &= \sum_{\ell = 0}^\infty \left(A^2_\ell r^\ell + \frac{B^2_\ell}{r^{\ell+1}}\right)P_\ell(\cos\theta)\\
  \phi_o &= -E_0r\cos\theta + \sum_{\ell = 0}^\infty \frac{B^o_\ell}{r^{\ell+1}}P_\ell(\cos\theta)
\end{align*}

We then apply our first two boundary conditions, noting that only the individual $\ell$ terms need to equal, not the whole summation:

\begin{align*}
  A^1_\ell a^\ell &= A^2_\ell a^\ell + \frac{B^2_\ell}{a^{\ell+1}}\\
  A^2_\ell b^\ell +& \frac{B^2_\ell}{b^{\ell+1}} = \frac{B^o_\ell}{b^{\ell+1}}\\
  A^2_1 b^1 +& \frac{B^2_1}{b^2} = -E_ob^1
\end{align*}

We then apply our derivative boundary conditions:

\begin{align*}
  \epsilon_r^1\ell A^1_\ell a^{\ell-1} =& \epsilon_r^2\ell A^2_\ell a^{\ell-1} - \epsilon_r^2\frac{\left(\ell+1\right)B^2_\ell}{a^\ell}\\
  \epsilon_r^2\ell A^2_\ell b^{\ell-1} -& \epsilon_r^2\frac{\left(\ell+1\right)B^2_\ell}{b^\ell}
  = -\epsilon_0\frac{\left(\ell+1\right)B^o_\ell}{b^\ell}\\
  \epsilon_r^2A^2_1 -2\frac{B_1^2}{b} &= -E_0
\end{align*}

If we solve for $A_1$ we get:

\end{document}

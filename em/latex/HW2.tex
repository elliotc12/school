\documentclass[10pt]{article} % Font size - 10pt, 11pt or 12pt

\nonstopmode

\usepackage[hmargin=1.25cm, vmargin=1.5cm]{geometry} % Document margins

\usepackage{amsmath}

\usepackage[usenames,dvipsnames]{xcolor} % Allows the definition of hex colors

% Fonts and tweaks for XeLaTeX
\usepackage{fontspec,xltxtra,xunicode}
\defaultfontfeatures{Mapping=tex-text}
%\setmonofont[Scale=MatchLowercase]{Andale Mono}

% Colors for links, text and headings
\usepackage{hyperref}
\definecolor{linkcolor}{HTML}{506266} % Blue-gray color for links
\definecolor{shade}{HTML}{F5DD9D} % Peach color for the contact information box
\definecolor{text1}{HTML}{2b2b2b} % Main document font color, off-black
\definecolor{headings}{HTML}{701112} % Dark red color for headings
% Other color palettes: shade=B9D7D9 and linkcolor=A40000; shade=D4D7FE and linkcolor=FF0080

\hypersetup{colorlinks,breaklinks, urlcolor=linkcolor, linkcolor=linkcolor} % Set up links and colors

\usepackage{fancyhdr}
\usepackage{amssymb}
\pagestyle{fancy}
\fancyhf{}
% Headers and footers can be added with the \lhead{} \rhead{} \lfoot{} \rfoot{} commands
% Example footer:
%\rfoot{\color{headings} {\sffamily Last update: \today}. Typeset with Xe\LaTeX}

\renewcommand{\headrulewidth}{0pt} % Get rid of the default rule in the header

\usepackage{titlesec} % Allows creating custom \section's

% Format of the section titles
\titleformat{\section}{\color{headings}
\scshape\Large\raggedright}{}{0em}{}[\color{black}\titlerule]

\title{Elegromagnetism Assignment Two}
\author{Elliott Capek}
\titlespacing{\section}{0pt}{0pt}{5pt} % Spacing around titles

\newcommand{\bra}[1]{\big<#1\big|}
\newcommand{\ket}[1]{\big|#1\big>}
\newcommand{\braket}[2]{\big<#1\big|#2\big>}

\begin{document}

\maketitle{}

\section{Problem One}
To find the bound charges on this cross section, we use the following equations:

\begin{align*}
  \rho_b &= -\nabla \vec{P} = 0\hspace{2cm} \mbox{P is constant}\\
  \sigma_b &= \vec{P} \cdot \vec{n}\\
\end{align*}

The top two faces each make a $60^\circ$ angle with $\vec{P}$, and the bottom one is antiparallel. Thus we see:

\begin{align*}
  \sigma_{b,top} &= \vec{P} \cdot \hat{n} = P\cos\left(60^\circ\right) = \frac P2\\
  \sigma_{b,bot} &= \vec{P} \cdot \hat{n} = P\cos\left(180^\circ\right) = -P\\
\end{align*}

To find total charge we just sum these densities over the surface area of the prism. We notice that there are two top faces and one
bottom face, thus the two surface charges cancel out and the net charge on the object is \textbf{zero}. We expect this for a dielectric with no free charge.\\

We then find the far field by treating the triangle as a point dipole extending infinitely through the z axis:

\begin{align*}
  \vec{E}(\vec{r}, \theta) &= \int_\infty^\infty \frac{\vec{P}\cdot\vec{r}}{\sqrt{z^2+r^2}}dz
  = P\cos(\theta)\int_\infty^\infty \frac{\sqrt{z^2+r^2}}{\sqrt{z^2+r^2}^3}dz
  = P\cos(\theta)\int_\infty^\infty \frac{1}{z^2+r^2}dz\\
  &= \frac{P\cos(\theta)\pi}{r}
\end{align*}

This result almost makes sense. Normally a dipole falls off by $\frac{1}{r^3}$, but this one is generated by an infinite line, so it
falls off lots slower.\\

\section{Problem Two}
The example solved in Griffiths Ex 4.7 can be generalized to a two-dielectrics-in-all-space system by letting $\epsilon_r = \epsilon_1 / \epsilon_2$:

\textbf{a.)}\\
\begin{align*}
	V_{in} &= -\frac{3E_0r\cos\theta}{\epsilon_r+2}\\
	E_{in} &= \frac{3E_0\cos\theta}{\epsilon_r+2}\hat{r} + \frac{-3E_0\sin\theta}{\epsilon_r+2}\hat{\theta}\\
        V_{out} &= -E_0r\cos\theta + \frac{(\epsilon_r-1)R^3E_0\cos\theta}{(\epsilon_r+2)r^2}\\
        E_{out} &= \left(E_0\cos\theta+\frac{2(\epsilon_r-1)R^3E_0\cos\theta}{(\epsilon_r+2)r^3}\right)\hat{r} + \left(-E_0\sin\theta+\frac{-(\epsilon_r-1)R^3E_0\sin\theta}{(\epsilon_r+2)r^3}\right)\hat{\theta}\\
        &= \left(\frac{E_0\cos\theta(3\epsilon_r-1)}{\epsilon_r+2}\right)\hat{r} + \left(\frac{-3E_0\sin\theta}{\epsilon_r+2}\right)\hat{\theta}\\
\end{align*}

\textbf{b.)}\\
Solving for free and bound surface charge:

\begin{align*}
  P &= \epsilon_1E\\
  \sigma_b &= P \cdot \hat{r} = \frac{3E_0\cos\theta}{\epsilon_r+2}\\
  \sigma_f &= \epsilon_1E_{in,r} - \epsilon_2E_{out,r} = \frac{-E_0\cos\theta\epsilon_2}{\epsilon_r+2}\\
  \sigma &= \frac{-E_0\cos\theta(3-\epsilon_2)}{\epsilon_r+2}\\
\end{align*}

This charge is positive at the top of the sphere and negative at the bottom, as expected for a bound charge.\\

\textbf{c.)}\\
Tangential E-fields are equal at $r=R$:

\begin{align*}
  E_{in,\theta} - E_{out,\theta} =& \frac{-3E_0\sin\theta}{\epsilon_r+2} - \frac{-3E_0\sin\theta}{\epsilon_r+2} = 0\\
\end{align*}

\textbf{d.)}
E-fields are curl-free. Because the dielectrics are linear, D is proportional to E, and so it seems like the difference should be zero. However, we see that:\\

\begin{align*}
  D_{out} - D_{in} &= \left(\epsilon_0-\epsilon_1\right)\frac{-3E_0\sin\theta}{\epsilon_r+2}
\end{align*}

The change in permittivity constants causes a change in curl.\\

\end{document}

\documentclass[10pt]{article} % Font size - 10pt, 11pt or 12pt

\nonstopmode

\usepackage[hmargin=1.25cm, vmargin=1.5cm]{geometry} % Document margins

\usepackage{amsmath}

\usepackage[usenames,dvipsnames]{xcolor} % Allows the definition of hex colors

% Fonts and tweaks for XeLaTeX
\usepackage{fontspec,xltxtra,xunicode}
\defaultfontfeatures{Mapping=tex-text}
%\setmonofont[Scale=MatchLowercase]{Andale Mono}

% Colors for links, text and headings
\usepackage{hyperref}
\definecolor{linkcolor}{HTML}{506266} % Blue-gray color for links
\definecolor{shade}{HTML}{F5DD9D} % Peach color for the contact information box
\definecolor{text1}{HTML}{2b2b2b} % Main document font color, off-black
\definecolor{headings}{HTML}{701112} % Dark red color for headings
% Other color palettes: shade=B9D7D9 and linkcolor=A40000; shade=D4D7FE and linkcolor=FF0080

\hypersetup{colorlinks,breaklinks, urlcolor=linkcolor, linkcolor=linkcolor} % Set up links and colors

\usepackage{fancyhdr}
\usepackage{amssymb}
\pagestyle{fancy}
\fancyhf{}
% Headers and footers can be added with the \lhead{} \rhead{} \lfoot{} \rfoot{} commands
% Example footer:
%\rfoot{\color{headings} {\sffamily Last update: \today}. Typeset with Xe\LaTeX}

\renewcommand{\headrulewidth}{0pt} % Get rid of the default rule in the header

\usepackage{titlesec} % Allows creating custom \section's

% Format of the section titles
\titleformat{\section}{\color{headings}
\scshape\Large\raggedright}{}{0em}{}[\color{black}\titlerule]

\title{Quantum Mechanics Assignment One}
\author{Elliott Capek}
\titlespacing{\section}{0pt}{0pt}{5pt} % Spacing around titles

\begin{document}

\maketitle{}

\section{Problem One}
\textbf{Boundaries:}

\begin{align*}
  \phi(\infty) &= 0\\
  \phi(x,0) &= 0\\
  \phi(0,y) &= 0\\
\end{align*}

Guess the solution is four charges in a rectangle. If this solution works, by the Uniqueness Theorem it is the only solution.

\begin{align*}
  \phi &= \phi_1 + \phi_2 + \phi_3 + \phi_4\\
\end{align*}

The potential due to the real point charge is:

\begin{align*}
  \phi_1(x,y) &= \frac{q}{4\pi\epsilon\sqrt{(x-d_1)^2+(y-d_2)^2}}\\
\end{align*}

We let the second and fourth-quadrant points have charge -q, so potential:

\begin{align*}
  \phi_2(x,y) &= \frac{-q}{4\pi\epsilon\sqrt{(x+d_1)^2+(y-d_2)^2}}\\
  \phi_4(x,y) &= \frac{-q}{4\pi\epsilon\sqrt{(x-d_1)^2+(y+d_2)^2}}\\
\end{align*}

and the third-quadrant point have charge +q, so potential:

\begin{align*}
  \phi_3(x,y) &= \frac{q}{4\pi\epsilon\sqrt{(x+d_1)^2+(y+d_2)^2}}\\
\end{align*}

Thus the total potential is given by:

\begin{align*}
  \phi(x,y) &= \frac{q}{4\pi\epsilon}\Big(
  \left((x-d_1)^2+(y-d_2)^2\right)^{-1/2} + \left((x+d_1)^2+(y+d_2)^2\right)^{-1/2}\\
  &-\left((x+d_1)^2+(y-d_2)^2\right)^{-1/2} - \left((x-d_1)^2+(y+d_2)^2\right)^{-1/2}\Big)
\end{align*}

This satisfies our two boundary conditions:

\begin{align*}
  \phi(0,y) &= \frac{q}{4\pi\epsilon}\Big(
  \left(d_1^2+(y-d_2)^2\right)^{-1/2} + \left(d_1^2+(y+d_2)^2\right)^{-1/2}
  -\left(d_1^2+(y-d_2)^2\right)^{-1/2} - \left(d_1^2+(y+d_2)^2\right)^{-1/2}\Big)
  = 0\\
\end{align*}

\begin{align*}
  \phi(x,0) &= \frac{q}{4\pi\epsilon}\Big(
  \left((x-d_1)^2+d_2^2\right)^{-1/2} + \left((x+d_1)^2+d_2^2\right)^{-1/2}
  -\left((x+d_1)^2+d_2^2\right)^{-1/2} - \left((x-d_1)^2+d_2^2\right)^{-1/2}\Big)
  = 0\\
\end{align*}

To find the force on the first-quadrant charge, we use Coulomb's Equation:

%% \begin{align*}
%%   F &= -q\left(\frac{d}{dx} + \frac{d}{dy}\right) \frac{q}{4\pi\epsilon}\Big(
%%   \left((x-d_1)^2+(y-d_2)^2\right)^{-1/2} + \left((x+d_1)^2+(y+d_2)^2\right)^{-1/2}\\
%%   &-\left((x+d_1)^2+(y-d_2)^2\right)^{-1/2} - \left((x-d_1)^2+(y+d_2)^2\right)^{-1/2}\Big)\\
%%   &= \frac{-q^2}{4\pi\epsilon} \Big(
%%   (x-d_1 + y-d_2)\left((x-d_1)^2+(y-d_2)^2\right)^{-3/2} - (x+d_1 + y+d_2)\left((x+d_1)^2+(y+d_2)^2\right)^{-3/2}\\
%%   &+ (x+d_1 + y-d_2)\left((x+d_1)^2+(y-d_2)^2\right)^{-3/2} - (x-d_1 + y+d_2)\left((x-d_1)^2+(y+d_2)^2\right)^{-3/2}\Big)\\
%%   F(d_1,d_2) &= \frac{-q^2}{4\pi\epsilon} \Big(
%%   - (2d_1 + 2d_2)\left(4d_1^2 + 4d_2^2\right)^{-3/2} + 2d_1\left(4d_1^2\right)^{-3/2} - 2d_2\left(4d_2^2\right)^{-3/2}\Big)\\
%%   &= \frac{q^2}{4\pi\epsilon} \left(\frac{(2d_1 + 2d_2)}{\sqrt{\left(4d_1^2 + 4d_2^2\right)}^3} - \frac{1}{4d_1^2} - \frac{1}{4d_2^2}\right)\\
%% \end{align*}

\begin{align*}
  \phi(\vec{r}) &= \frac{q}{\sqrt{2}4\pi\epsilon_0\sqrt{4d_1^2 + 4d_2^2}}\left(\hat{i} + \hat{j}\right) + \frac{-q}{4\pi\epsilon_0\sqrt{4d_1^2}}\hat{i} + \frac{-q}{4\pi\epsilon_0\sqrt{4d_2^2}}\hat{j}\\
  &= \frac{q}{4\pi\epsilon_0}\left(\frac{1}{\sqrt{8d_1^2 + 8d_2^2}} - \frac{1}{\sqrt{4d_1^2}}\right)\hat{i}
  + \frac{q}{4\pi\epsilon_0}\left(\frac{1}{\sqrt{8d_1^2 + 8d_2^2}} - \frac{1}{\sqrt{4d_2^2}}\right)\hat{j}
\end{align*}

If expanded, at large r, $\phi$ should look like a quadrupole scaling with $\frac{1}{r^3}$.\\

\section{Problem Two}
We want to show the following equality is always true:

\begin{align*}
  \int\rho_1\phi_2d^3\vec{r} &= \int\rho_2\phi_1d^3\vec{r}
\end{align*}

We first apply the Laplace Equation:

\begin{align*}
  -\epsilon_0\int\nabla^2\phi_1\phi_2d^3\vec{r} &= -\epsilon_0\int\nabla^2\phi_2\phi_1d^3\vec{r}
\end{align*}

We then expand this equation into its cartesian representation:

\begin{align*}
  \int\int\int\sum_{n=x,y,z}\frac{\partial^2}{\partial n^2} (\phi_1) \phi_2 dxdydz &=
  \int\int\int\sum_{n=x,y,z}\frac{\partial^2}{\partial n^2} (\phi_2) \phi_1 dxdydz
\end{align*}

We then use Integration by Parts:

\begin{align*}
  \int\int\int\sum_{n=x,y,z}\left(\frac{d}{dn}\phi_1\phi_2\Big|_{-\infty}^{\infty} - \int\frac{\partial}{\partial n}\phi_1\frac{\partial}{\partial n}\phi_2dn \right) dxdydz &=
  \int\int\int\sum_{n=x,y,z}\left(\frac{d}{dn}\phi_2\phi_1\Big|_{-\infty}^{\infty} - \int\frac{\partial}{\partial n}\phi_2\frac{\partial}{\partial n}\phi_1dn \right) dxdydz
\end{align*}

\section{Problem Three}
The solution to the full sphere in an external field $E_0$ was shown in class to be:

\begin{align*}
  \phi(r,\theta) &= \left(-E_0r + \frac{E_0a^3}{r^2}\right)\cos\theta
\end{align*}

given boundaries $\phi(r=a) = 0$ and $\phi(r\rightarrow\infty) = 0$. We notice that our boundaries for the half sphere problem are the same, plus one addition: $\phi(r<a, z=0) = 0$. The above solution satisfies this condition as well. Then by the Uniqueness Theorem this solution must also be valid for our half sphere.\\

The first term $-E_0r\cos\theta$ is the potential from the external field. The second term $\frac{E_0a^3\cos\theta}{r^2}$ is the term due
to the sphere. This term can be rewritten as the dipole term in the multipole expansion:\\

\begin{align*}
  \frac{E_0a\cos\theta}{r^2} &= \frac{1}{4\pi\epsilon_0}\frac{1}{r^2} \hat{r} \cdot \vec{p}\\
\end{align*}

where $\vec{p} = 4\pi\epsilon_0E_0 a^3 \hat{z}$. Thus the half sphere can be exactly expressed as a dipole.\\

\end{document}

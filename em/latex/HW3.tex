\documentclass[10pt]{article} % Font size - 10pt, 11pt or 12pt

\nonstopmode

\usepackage[hmargin=1.25cm, vmargin=1.5cm]{geometry} % Document margins

\usepackage{amsmath}

\usepackage[usenames,dvipsnames]{xcolor} % Allows the definition of hex colors

% Fonts and tweaks for XeLaTeX
\usepackage{fontspec,xltxtra,xunicode}
\defaultfontfeatures{Mapping=tex-text}
%\setmonofont[Scale=MatchLowercase]{Andale Mono}

% Colors for links, text and headings
\usepackage{hyperref}
\definecolor{linkcolor}{HTML}{506266} % Blue-gray color for links
\definecolor{shade}{HTML}{F5DD9D} % Peach color for the contact information box
\definecolor{text1}{HTML}{2b2b2b} % Main document font color, off-black
\definecolor{headings}{HTML}{701112} % Dark red color for headings
% Other color palettes: shade=B9D7D9 and linkcolor=A40000; shade=D4D7FE and linkcolor=FF0080

\hypersetup{colorlinks,breaklinks, urlcolor=linkcolor, linkcolor=linkcolor} % Set up links and colors

\usepackage{fancyhdr}
\usepackage{amssymb}
\pagestyle{fancy}
\fancyhf{}
% Headers and footers can be added with the \lhead{} \rhead{} \lfoot{} \rfoot{} commands
% Example footer:
%\rfoot{\color{headings} {\sffamily Last update: \today}. Typeset with Xe\LaTeX}

\renewcommand{\headrulewidth}{0pt} % Get rid of the default rule in the header

\usepackage{titlesec} % Allows creating custom \section's

% Format of the section titles
\titleformat{\section}{\color{headings}
\scshape\Large\raggedright}{}{0em}{}[\color{black}\titlerule]

\title{Elegromagnetism Assignment Three}
\author{Elliott Capek}
\titlespacing{\section}{0pt}{0pt}{5pt} % Spacing around titles

\newcommand{\bra}[1]{\big<#1\big|}
\newcommand{\ket}[1]{\big|#1\big>}
\newcommand{\braket}[2]{\big<#1\big|#2\big>}

\newcommand{\b}[1]{\boldsymbol{#1}}
\newcommand{\bv}[1]{\boldsymbol{\vec{#1}}}

\begin{document}

\maketitle{}

\section{Problem One}
Worked with Jared Cayton.\\
Here we use two image charges, one for when z > 0 and one for when z < 0. They are both located at (0, 0, $\pm$ d).

Thus:

\begin{align*}
	V &= \frac{1}{4\pi\epsilon_1}\left(\frac{q_f}{r} + \frac{q_1}{r})
\end{align*}

We use the following boundary conditions:

\begin{align*}
	V_{z>0}(z=0) = V_{z<0}(z=0)\\
	\epsilon_1\frac{dV_{top}}{dz}(z=0) = \epsilon_2\frac{dV_{bottom}}{dz}(z=0)\\
\end{align*}

We express our top voltage in Cartesian coordinates:

\begin{align*}
  V_{top} &= \frac{1}{4\pi\epsilon_1} \left(\frac{q_f}{\sqrt{x^2+y^2+(z-d)^2}} + \frac{q_1}{\sqrt{x^2+y^2+(z-d)^2}}\right)\\
  \frac{dV_{top}}{dz} &= \frac{1}{4\pi\epsilon_1}\left(\frac{-q_f\sqrt{x^2+y^2+(z-d)^2}(z+d)}{x^2+y^2+(z-d)^2}
  \frac{-q_1\sqrt{x^2+y^2+(z-d)^2}(z+d)}{x^2+y^2+(z+d)^2} - \right)\\
  \epsilon_1 \frac{dV_{top}}{dz}(z=0) &= \frac{-1}{4\pi}\left(\frac{q_fd-q_1d}{\left(x^2+y^2+d^2\right)^{\frac32}}\right)\\
\end{align*}

We do the same for bottom voltage:
\begin{align*}
  V_{bot} &= \frac{1}{4\pi\epsilon_2} \left(\frac{q_2}{\sqrt{x^2+y^2+(z-d)^2}}\right)\\
  \frac{dV_{bot}}{dz} &= \frac{1}{4\pi\epsilon_2}\left(\frac{-q_2\left(z-d\right)}{\left(x^2+y^2+(z-d)^2\right)^{\frac32}}\right)\\
  \epsilon_2 \frac{dV_{bot}}{dz}(z=0) &= \frac{1}{4\pi}\left(\frac{-q_2d}{\left(x^2+y^2+d^2\right)^{\frac32}}\right)\\
\end{align*}

We then apply our boundary conditions due to charges:

\begin{align*}
  \frac{q_f+q_1}{\epsilon_1} &= \frac{q_2}{\epsilon_2}\\
  q_f - q_1 &= q_2\\
\end{align*}

We now have enough information to find the value of the image charges:

\begin{align*}
  q_f - q_2 &= q_1\\
  q_f - \frac{\epsilon_2}{\epsilon_1}q_f - \frac{\epsilon_2}{\epsilon_1}q_1 = q_1\\
  q_f\left(1-\frac{\epsilon_2}{\epsilon_1}\right) &= q_1\left(1+\frac{\epsilon_2}{\epsilon_1}\right)\\
  q_1 &= q_f\frac{\epsilon_1-\epsilon_2}{\epsilon_1+\epsilon_2}
\end{align*}

\begin{align*}
  q_2 &= q_f-q_1 = q_f - q_f\frac{\epsilon_1-\epsilon_2}{\epsilon_1+\epsilon_2}\\
  &= q_f\left(1-\frac{\epsilon_1-\epsilon_2}{\epsilon_1+\epsilon_2}\right) = q_f\frac{2\epsilon_2}{\epsilon_1+\epsilon_2}\\
\end{align*}

We then express our voltage in terms of these found charges:

\begin{align*}
  V_{top} &= \frac{q_f}{4\pi\epsilon_1}\left(\frac{1}{\sqrt{x^2+y^2+(z-d)^2}} + \frac{\left(\frac{\epsilon_1-\epsilon_2}{\epsilon_1+\epsilon_2}\right)}{\sqrt{x^2+y^2+(z+d)^2}}\right)
\end{align*}

Thus

\begin{align*}
  E_{top} &= -\nabla V_{top} = \frac{-1f}{4\pi\epsilon_1}\left(\frac{x\hat{x}+y\hat{y}+(z-d)\hat{z}}{\left(x^2+y^2+(z-d)^2\right)^{\frac32}} + \left(\frac{\epsilon_1-\epsilon_2}{\epsilon_1+\epsilon_2}\right)\left(\frac{x\hat{x}+y\hat{y}+(z+d)\hat{z}}{\left(x^2+y^2+(z+d)^2\right)^{\frac32}}\right)\right)\\
\end{align*}

And similarly for $E_{bot}$:

\begin{align*}
  E_{bot} = -\nabla V_{bot} &= \frac{q_f}{4\pi\left(\epsilon_1+\epsilon_2\right)}\left(\frac{x\hat{x}+y\hat{y}+\left(z-d\right)\hat{z}}{\left(x^2+y^2+(z-d)^2\right)^{\frac32}}\right)
\end{align*}


\section{Problem Two}
\textbf{Part I}
\textbf{Part I}\\

We begin with $\bv{m} = \frac{I}{2}\oint \bv{r} \times d\bv{l}$ and want to show it is equivalent to $\bv{m} = IS\hat{n}$ when integrating over a flat surface.\\

We argue that Because $\bv{r}$ and $d\bv{l}$ are coplanar, the cross product always points in the upwards direction, $\hat{n}$. Each segment $d\bv{l}$ represents a small part of the loop which is a distance $\bv{r}$ from the origin. Thus each value $\bv{r} \times d\bv{l}$ will be a small rectangle with area $rdl$, or $2\hat{n}dA$, since dA is a triangle one half the size of $\bv{r}$-$d\bv{l}$.\\

Thus we see that:

\begin{align*}
	\bv{m} &= \frac{I}{2}\oint \bv{r} \times d\bv{l}\\
	&= \frac{I}{2} \oint 2\hat{n}dA = IS\hat{n}
\end{align*}

\textbf{Part II}

We begin with the definition of torque, plug in the definition of force on a current, apply the cross product triple product rule,
argue that one of the terms goes to zero, then apply the equation from Griffiths 4th ed 1.62e.

\begin{align*}
  \boldsymbol{N} &= \oint d\bv{r} \times \bv{F}\\
  &= \oint \bv{r} \times \left(\bv{I} \times \bv{B}\right)dl\\
  &= \oint \bv{I}\left(\bv{r} \cdot \bv{B}\right) - \bv{B}\left(\bv{r} \cdot \bv{I}\right)dl \hspace{1.2cm}\mbox{cross triple product}\\
  &= \oint \bv{I}\left(\bv{r} \cdot \bv{B}\right)dl - \oint\bv{B}\left(\bv{r} \cdot \bv{I}\right)dl\\
  &= \oint \bv{I}\left(\bv{r} \cdot \bv{B}\right)dl \hspace{3cm}\mbox{since }\Big(\oint \bv{r} \cdot d\bv{l} = 0\Big)\\
  &= \oint \bv{I}\left(\bv{B} \cdot \bv{r}\right)dl\\
  &= I\oint \left(\bv{B} \cdot \bv{r}\right)d\boldsymbol{l}\\
  &= I\bv{a} \times \bv{B} = \bv{m} \times \bv{B}\\
\end{align*}

as desired.

\section{Problem Three}

We solve this problem by guessing forces.\\

For the magnetic force, we know several things:

\begin{itemize}
\item Force should have two magnetic dipole terms for each sphere
\item Wikipedia says dipole-dipole interactions depend on distance with $r^{-4}$
\item There should be a $\mu_0$ in our magnetic force
\end{itemize}

We know that magnetic dipoles look like current-areas and should depend on $\omega, Q, a$. We guess:

\begin{align*}
	I \approx Q\omega\pi a^2
\end{align*}

We know $\mu_0$ has units of force per current squared. We guess a solution which involves a magnetic dipole squared, $d^4$ and $\mu_0$ in a way which leads to force units:

\begin{align*}
	F_B \approx \frac{Q^2\omega^2\pi^2a^4\mu_0}{d^4}
\end{align*}

We then use Coulomb's force equation to guess out electric force:

\begin{align*}
	F_E \approx \frac{Q^2}{4\pi\epsilon_0 d^2}
\end{align*}

Thus
\begin{align*}
	\frac{F_B}{F_E} \approx \frac{4\pi^3\omega^2a^4\mu_0\epsilon_0}{d^2}
\end{align*}

\end{document}

\documentclass[10pt]{article} % Font size - 10pt, 11pt or 12pt

\nonstopmode

\usepackage[hmargin=1.25cm, vmargin=1.5cm]{geometry} % Document margins

\usepackage{amsmath}

\usepackage[usenames,dvipsnames]{xcolor} % Allows the definition of hex colors

% Fonts and tweaks for XeLaTeX
\usepackage{fontspec,xltxtra,xunicode}
\defaultfontfeatures{Mapping=tex-text}
%\setmonofont[Scale=MatchLowercase]{Andale Mono}

% Colors for links, text and headings
\usepackage{hyperref}
\definecolor{linkcolor}{HTML}{506266} % Blue-gray color for links
\definecolor{shade}{HTML}{F5DD9D} % Peach color for the contact information box
\definecolor{text1}{HTML}{2b2b2b} % Main document font color, off-black
\definecolor{headings}{HTML}{701112} % Dark red color for headings
% Other color palettes: shade=B9D7D9 and linkcolor=A40000; shade=D4D7FE and linkcolor=FF0080

\hypersetup{colorlinks,breaklinks, urlcolor=linkcolor, linkcolor=linkcolor} % Set up links and colors

\usepackage{fancyhdr}
\usepackage{amssymb}
\pagestyle{fancy}
\fancyhf{}
% Headers and footers can be added with the \lhead{} \rhead{} \lfoot{} \rfoot{} commands
% Example footer:
%\rfoot{\color{headings} {\sffamily Last update: \today}. Typeset with Xe\LaTeX}

\renewcommand{\headrulewidth}{0pt} % Get rid of the default rule in the header

\usepackage{titlesec} % Allows creating custom \section's

% Format of the section titles
\titleformat{\section}{\color{headings}
\scshape\Large\raggedright}{}{0em}{}[\color{black}\titlerule]

\title{Elegromagnetism Assignment Three}
\author{Elliott Capek}
\titlespacing{\section}{0pt}{0pt}{5pt} % Spacing around titles

\newcommand{\bra}[1]{\big<#1\big|}
\newcommand{\ket}[1]{\big|#1\big>}
\newcommand{\braket}[2]{\big<#1\big|#2\big>}

\newcommand{\bb}[1]{\boldsymbol{#1}}
\newcommand{\bv}[1]{\boldsymbol{\vec{#1}}}

\begin{document}

\maketitle{}

\section{Problem Two}
Use an Ampere loop to calculate the H-field in the two magnetic materials:\\

\begin{align*}
  2\pi r H &= I\hat{z}\\
  H &= \frac{I\hat{z}}{2\pi r}
\end{align*}

We can use this field, along with $M=\chi H$, to find M:\\

\begin{align*}
  M_1 &= \frac{\chi_1I\hat{z}}{2\pi r}\\
  M_2 &= \frac{\chi_2I\hat{z}}{2\pi r}
\end{align*}

We then find the bound currents:

\begin{align*}
  J_{b} &= 0\\
  K_{b,1,in} &= \frac{\chi_1I\hat{z}}{2\pi a_1}
  K_{b,2,in} &= \frac{-\chi_2I\hat{z}}{2\pi a_1}
  K_{b,2,out} &= \frac{\chi_2I\hat{z}}{2\pi a_2}
\end{align*}

Use an Amperian loop to find the B-field:\\

\begin{align*}
  B_{1} &= 0\\
  B_{2} &= \frac{I(1+\chi_1-\chi_2)}{2\pi r}\hat{\theta}\\
  B_{2} &= \frac{I(1+\chi_1-\chi_2)}{2\pi r}\hat{\theta} + \frac{\chi_2I\hat{z}}{2\pi r}\\
\end{align*}

\end{document}

\documentclass[10pt]{article} % Font size - 10pt, 11pt or 12pt

\nonstopmode

\usepackage[hmargin=1.25cm, vmargin=1.5cm]{geometry} % Document margins

\usepackage{amsmath}

\usepackage[usenames,dvipsnames]{xcolor} % Allows the definition of hex colors

% Fonts and tweaks for XeLaTeX
\usepackage{fontspec,xltxtra,xunicode}
\defaultfontfeatures{Mapping=tex-text}
%\setmonofont[Scale=MatchLowercase]{Andale Mono}

% Colors for links, text and headings
\usepackage{hyperref}
\definecolor{linkcolor}{HTML}{506266} % Blue-gray color for links
\definecolor{shade}{HTML}{F5DD9D} % Peach color for the contact information box
\definecolor{text1}{HTML}{2b2b2b} % Main document font color, off-black
\definecolor{headings}{HTML}{701112} % Dark red color for headings
% Other color palettes: shade=B9D7D9 and linkcolor=A40000; shade=D4D7FE and linkcolor=FF0080

\hypersetup{colorlinks,breaklinks, urlcolor=linkcolor, linkcolor=linkcolor} % Set up links and colors

\usepackage{fancyhdr}
\usepackage{amssymb}
\pagestyle{fancy}
\fancyhf{}
% Headers and footers can be added with the \lhead{} \rhead{} \lfoot{} \rfoot{} commands
% Example footer:
%\rfoot{\color{headings} {\sffamily Last update: \today}. Typeset with Xe\LaTeX}

\renewcommand{\headrulewidth}{0pt} % Get rid of the default rule in the header

\usepackage{titlesec} % Allows creating custom \section's

% Format of the section titles
\titleformat{\section}{\color{headings}
\scshape\Large\raggedright}{}{0em}{}[\color{black}\titlerule]

<<<<<<< HEAD
\title{Elegromagnetism Assignment Four}
=======
\title{Elegromagnetism Assignment Three}
>>>>>>> 5eaa6e59275b067a33afbb12a64f6453f48e27fd
\author{Elliott Capek}
\titlespacing{\section}{0pt}{0pt}{5pt} % Spacing around titles

\newcommand{\bra}[1]{\big<#1\big|}
\newcommand{\ket}[1]{\big|#1\big>}
\newcommand{\braket}[2]{\big<#1\big|#2\big>}

<<<<<<< HEAD
=======
\newcommand{\bb}[1]{\boldsymbol{#1}}
\newcommand{\bv}[1]{\boldsymbol{\vec{#1}}}

>>>>>>> 5eaa6e59275b067a33afbb12a64f6453f48e27fd
\begin{document}

\maketitle{}

<<<<<<< HEAD
\section{Problem One}
We solve this problem through the method of images. Define a current $I_g\hat{y}$ which flows at $(0,0,-d)$. First find the bound surface current, assuminig linear magnetic material:\\

\begin{align*}
  K_{1,b}(0,0,0) &= \vec{M} \mbox{x} \vec{n} = \frac{\chi_1}{\mu_1}\vec{B} \mbox{x} \vec{n} = \frac{\chi_1}{\mu_1}\frac{\mu_1I}{2\pi d} = \frac{\chi_1I}{2\pi d}\\
  K_{2,b}(0,0,0) &= \frac{-\chi_2I}{2\pi d}\\
  K_{b} &= \frac{(\chi_1-\chi_2)I}{2\pi d}\\
\end{align*}

The A-field will be that due to infinite wires of constant current. The non-mirror magnetic material will see currents of $I$ in the material and $I_g$ in the other material. There cannot be any image charge in the mirror material, so we note that $I+I_g$ in the non-mirror material is the only current which satisfies the continuity boundary conditions for the mirror material's A-field:\\

\begin{align*}
  A_1 &= \frac{\mu_1I\ln{\sqrt{x^2+y^2+(z-d)^2}}}{4\pi} + \frac{\mu_1I_g\ln{\sqrt{x^2+y^2+(z+d)^2}}}{4\pi}\\
  A_2 &= \frac{\mu_1(I+I_g)\ln{\sqrt{x^2+y^2+(z-d)^2}}}{4\pi}\\
  &A_1(x,y,0) = A_2(x,y,0)\\
\end{align*}

We can calculate the value of $I_g$ through the derivative continuity boundary condition of A:

\begin{align*}
  &\frac{\partial A_1}{\partial n}_{z=0} - \frac{\partial A_2}{\partial n}_{z=0} = -\mu_1 K\\
  &\frac{\partial A_1}{\partial n}(0,0,z) = \frac{\mu_1I}{4\pi(z-d)} + \frac{\mu_1I_g}{4\pi(z+d)}\\
  &\frac{\partial A_2}{\partial n}(0,0,z) = \frac{\mu_1(I+I_g)}{4\pi(z-d)}\\
  &\frac{-\mu_1I}{4\pi(d)} + \frac{\mu_1I_g}{4\pi(d)} + \frac{\mu_1(I+I_g)}{4\pi(d)} = -\mu_0K_b\\
  &\frac{\mu_1I_g}{2\pi d} = \frac{(\chi_1-\chi_2)I}{2\pi d}\\
  &I_g = \frac{-\mu_0(\chi_1-\chi_2)I}{\mu_1} = \frac{-\mu_0}{\mu_1}\left(\frac{\mu_1}{\mu_0}-1-\frac{\mu_2}{\mu_0}+1\right)I\\
  &I_g = \frac{\mu_2-\mu_1}{\mu_1}I\\
\end{align*}

Thus the B-fields can be computed as:\\

\begin{align*}
  B_1(x,y,z) &= \frac{\mu_1I}{2\pi\sqrt{x^2+y^2+(z-d)^2}} + \frac{(\mu_2-\mu_1)I}{2\pi\sqrt{x^2+y^2+(z+d)^2}}\\
  B_2(x,y,z) &= \frac{\mu_2I}{2\pi\sqrt{x^2+y^2+(z-d)^2}}\\
\end{align*}


\section{Problem 2}
Use an Amperian loop of radius $s$ to calculate the H-field in and out of the cable:\\

\begin{align*}
  2\pi sH &= \pi s^2 I\\
  H &= \frac{sI}{2} \hspace{1cm} s<a_2\\
  H &= \frac{a_2^2I}{2r} \hspace{1cm} r>a_2\\
\end{align*}

By the definition $M=\chi H$, we can find the M-field:\\

\begin{align*}
  M_1 &= \frac{\chi_1 sI}{2}\\
  M_2 &= \frac{\chi_2 sI}{2}\\
  M_{out} &= 0\\
\end{align*}

This field is curl-free, thus there is no bound volume current. The surface bound current is given by:\\

\begin{align*}
  K_{b1} &= \frac{\chi_1 a_1 I\hat{z}}{2}\\
  K_{b2,in} &= \frac{-\chi_2 a_1 I\hat{z}}{2}\\
  K_{b2,out} &= \frac{\chi_2 a_2 I\hat{z}}{2}\\
\end{align*}

We can use an Amperian loop to find the B-fields:

\begin{align*}
  B_{in} &= 0\\
  B_{mid} &= \left(\frac{I}{2\pi s}\right)\left(K_{b1}*2\pi a_1 + K_{b2,in}*2\pi a_1\right)\\
  &= \frac{1}{2\pi s}\left(\frac{\chi_1 a_1 I\hat{z}}{2} * 2\pi a_1 + \frac{-\chi_2 a_1 I \hat{z}}{2}*2\pi a_1\right)\\
  &= \frac{a_1^2I}{2s}\left(\chi_1-\chi_2\right)\hat{\theta}\\
  B_{out} &= \frac{I}{2s}\left(a_1^2\chi_1 - a_1^2\chi_2 + a_2^2\chi_2\right)\\
=======
\section{Problem Two}
Use an Ampere loop to calculate the H-field in the two magnetic materials:\\

\begin{align*}
  2\pi r H &= I\hat{z}\\
  H &= \frac{I\hat{z}}{2\pi r}
\end{align*}

We can use this field, along with $M=\chi H$, to find M:\\

\begin{align*}
  M_1 &= \frac{\chi_1I\hat{z}}{2\pi r}\\
  M_2 &= \frac{\chi_2I\hat{z}}{2\pi r}
\end{align*}

We then find the bound currents:

\begin{align*}
  J_{b} &= 0\\
  K_{b,1,in} &= \frac{\chi_1I\hat{z}}{2\pi a_1}
  K_{b,2,in} &= \frac{-\chi_2I\hat{z}}{2\pi a_1}
  K_{b,2,out} &= \frac{\chi_2I\hat{z}}{2\pi a_2}
\end{align*}

Use an Amperian loop to find the B-field:\\

\begin{align*}
  B_{1} &= 0\\
  B_{2} &= \frac{I(1+\chi_1-\chi_2)}{2\pi r}\hat{\theta}\\
  B_{2} &= \frac{I(1+\chi_1-\chi_2)}{2\pi r}\hat{\theta} + \frac{\chi_2I\hat{z}}{2\pi r}\\
>>>>>>> 5eaa6e59275b067a33afbb12a64f6453f48e27fd
\end{align*}

\end{document}

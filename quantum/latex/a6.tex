\documentclass[10pt]{article} % Font size - 10pt, 11pt or 12pt

\usepackage[hmargin=1.25cm, vmargin=1.5cm]{geometry} % Document margins

\usepackage{marvosym} % Required for symbols in the colored box

\usepackage[usenames,dvipsnames]{xcolor} % Allows the definition of hex colors

% Fonts and tweaks for XeLaTeX
\usepackage{fontspec,xltxtra,xunicode}
\defaultfontfeatures{Mapping=tex-text}
%\setmonofont[Scale=MatchLowercase]{Andale Mono}

% Colors for links, text and headings
\usepackage{hyperref}
\definecolor{linkcolor}{HTML}{506266} % Blue-gray color for links
\definecolor{shade}{HTML}{F5DD9D} % Peach color for the contact information box
\definecolor{text1}{HTML}{2b2b2b} % Main document font color, off-black
\definecolor{headings}{HTML}{701112} % Dark red color for headings
% Other color palettes: shade=B9D7D9 and linkcolor=A40000; shade=D4D7FE and linkcolor=FF0080

\hypersetup{colorlinks,breaklinks, urlcolor=linkcolor, linkcolor=linkcolor} % Set up links and colors

\usepackage{fancyhdr}
\usepackage{amsmath}
\usepackage{braket}
\usepackage{amssymb}
\pagestyle{fancy}
\fancyhf{}
% Headers and footers can be added with the \lhead{} \rhead{} \lfoot{} \rfoot{} commands
% Example footer:
%\rfoot{\color{headings} {\sffamily Last update: \today}. Typeset with Xe\LaTeX}

\renewcommand{\headrulewidth}{0pt} % Get rid of the default rule in the header

\usepackage{titlesec} % Allows creating custom \section's

% Format of the section titles
\titleformat{\section}{\color{headings}
\scshape\Large\raggedright}{}{0em}{}[\color{black}\titlerule]

\title{Quantum Mechanics Assignment Six}
\author{Elliott Capek}
\titlespacing{\section}{0pt}{0pt}{5pt} % Spacing around titles

\begin{document}

\maketitle{}

\section{11.12: Spin-1, Spin-$\frac{1}{2}$}
This system is made up of two particles, one with spin-1 and the other with spin-$\frac{1}{2}$.\\

\textbf{a.)} The total states possible for the system are:\\

\begin{align*}
  \ket{1,\frac{1}{2},-\hbar,-\frac{\hbar}{2}}\\
  \ket{1,\frac{1}{2},0,-\frac{\hbar}{2}}\\
  \ket{1,\frac{1}{2},\hbar,-\frac{\hbar}{2}}\\
  \ket{1,\frac{1}{2},-\hbar,\frac{\hbar}{2}}\\
  \ket{1,\frac{1}{2},0,\frac{\hbar}{2}}\\
  \ket{1,\frac{1}{2},\hbar,\frac{\hbar}{2}}\\
\end{align*}

\textbf{b.)} The highest z-projection of these states, the stretched state, is
$\ket{1,\frac{1}{2},\hbar,\frac{\hbar}{2}}$.\\

\textbf{c.)} Here we just solve for the coupled state wavefunctions using raising
and lowering operators:\\

\begin{align*}
  \ket{\frac{3}{2},\frac{3}{2}} &= \ket{1,\frac{1}{2},1,\frac{1}{2}}\\
  S_-\ket{\frac{3}{2},\frac{3}{2}} &= \left(S_{1-}+S_{2-}\right)\ket{1,\frac{1}{2},1,\frac{1}{2}}\\
  \sqrt{3}\hbar\ket{\frac{3}{2},\frac{1}{2}} &= \sqrt{2}\hbar\ket{1,\frac{1}{2},0,\frac{1}{2}} + \hbar\ket{1,\frac{1}{2},1,-\frac{1}{2}}\\
  \ket{\frac{3}{2},\frac{1}{2}} &= \sqrt{\frac{2}{3}}\ket{1,\frac{1}{2},0,\frac{1}{2}} + \frac{1}{\sqrt{3}}\ket{1,\frac{1}{2},1,-\frac{1}{2}}\\
\end{align*}

\begin{align*}
  S_-\ket{\frac{3}{2},\frac{1}{2}} &=
  \left(S_{1-}+S_{2-}\right)\sqrt{\frac{2}{3}}\ket{1,\frac{1}{2},0,\frac{1}{2}} + \frac{1}{\sqrt{3}}\ket{1,\frac{1}{2},1,-\frac{1}{2}}\\
  \sqrt{2}\hbar{\frac{3}{2},\frac{1}{2}} &= \frac{2\hbar}{\sqrt{3}}\ket{1,1/2,-1,1/2}
  + \frac{\sqrt{2}\hbar}{\sqrt{3}}\ket{1,1/2,0,-1/2} + \frac{\hbar\sqrt{2}}{\sqrt{3}}\ket{1,1/2,0,-1/2}\\
  \ket{3/2,-1/2} &= \frac{1}{\sqrt{3}}\ket{1,1/2,-1,1/2} + \sqrt{\frac{2}{3}}\ket{1,1/2,0,-1/2}\\
\end{align*}

\begin{align*}
  S_-\ket{3/2,-1/2} &= \left(S_{1-}+S_{2-}\right)\left(\frac{1}{\sqrt{3}}\ket{1,1/2,-1,1/2} + \sqrt{\frac{2}{3}}\ket{1,1/2,0,-1/2}\right)\\
  \ket{3/2,-3/2} &= \ket{1,1/2,-1,-1/2}\\
\end{align*}

We then use the method described on pg 371 of McIntyre to find the 1/2-states: creating a linear equation and solving it.\\

\begin{align*}
  \ket{1/2,1/2} &= A\ket{1,1/2,0,1/2} + B\ket{1,1/2,1,-1/2} + C\ket{1,1/2,-1,1/2}\\
  \bra{3/2,1/2}\bra{1/2,1/2} = 0 \rightarrow c = 0\\
  \bra{3/2,1/2}\ket{1/2,1/2} = 0 \rightarrow a\sqrt{\frac{2}{3}} + b\sqrt{\frac{1}{3}} \rightarrow -\frac{b}{\sqrt{2}} = a\\
  3a^2 = 1 \rightarrow a = \frac{1}{\sqrt{3}}\\
  \ket{1/2,1/2} = -\frac{1}{\sqrt{3}}\ket{1,1/2,0,-1/2} + \sqrt{\frac{2}{3}}\ket{1,1/2,1,-1/2}\\
\end{align*}

\begin{align*}
  S_-\ket{1/2,1/2} &= \left(S_{1-}+S_{2-}\right)\left(-\frac{1}{\sqrt{3}}\ket{1,1/2,0,-1/2} + \sqrt{\frac{2}{3}}\ket{1,1/2,1,-1/2}\right)\\
  \ket{1/2,-1/2} = \frac{1}{\sqrt{3}}\ket{1,1/2,0,-1/2} - \sqrt{\frac{2}{3}}\ket{1,1/2,-1,1/2}\\
\end{align*}

We then construct a Clebsch-Gordan matrix out of our states: (sorry for the bad formatting,
the formatting is identical to that found in table 11.3 in McIntyre)

\begin{align*}
  \begin{pmatrix}
    1 & 0 & 0 & 0\\
    0 & \frac{1}{\sqrt{3}} & 0 & 0\\
    0 & \frac{\sqrt{2}}{\sqrt{3}} & 0 & 0\\
    0 & 0 & \frac{\sqrt{2}}{\sqrt{3}} & 0\\
    0 & 0 & \frac{1}{\sqrt{3}} & 0\\
    0 & 0 & 0 & 1\\
  \end{pmatrix}
\end{align*}







\section{11.15: Coupled / Uncoupled bases}
This system consists of two particles, one spin-$\frac{3}{2}$ and the other
spin-$\frac{1}{2}$.\\

\textbf{a.)} First we list all the possible states this system can take on:

\begin{align*}
  \ket{\frac{3}{2},\frac{1}{2},\frac{3\hbar}{2},\frac{\hbar}{2}}\hspace{1cm}&
  \ket{\frac{3}{2},\frac{1}{2},\frac{1\hbar}{2},\frac{\hbar}{2}}\\
  \ket{\frac{3}{2},\frac{1}{2},\frac{-1\hbar}{2},\frac{\hbar}{2}}\hspace{1cm}&
  \ket{\frac{3}{2},\frac{1}{2},\frac{-3\hbar}{2},\frac{\hbar}{2}}\\
  \ket{\frac{3}{2},\frac{1}{2},\frac{3\hbar}{2},\frac{-\hbar}{2}}\hspace{1cm}&
  \ket{\frac{3}{2},\frac{1}{2},\frac{1\hbar}{2},\frac{-\hbar}{2}}\\
  \ket{\frac{3}{2},\frac{1}{2},\frac{-1\hbar}{2},\frac{-\hbar}{2}}\hspace{1cm}&
  \ket{\frac{3}{2},\frac{1}{2},\frac{-3\hbar}{2},\frac{-\hbar}{2}}\\
\end{align*}

\textbf{b.)} J can take on two values: $J = j_1 \pm j_2 = 2, 1$. $J=2$ corresponds to
$M = -2, -1, 0, 1, 2$, while $J=1$ corresponds to $M=-1, 0, 1$.\\

\textbf{c.)} Here we enumerate all the possible combinations of S and M:

\begin{align*}
  \ket{2,-2}, \ket{2,-1}, \ket{2,0}&, \ket{2,1}, \ket{2,2}\\
  \ket{1,-1}, \ket{1,0}&, \ket{1,-1}\\
\end{align*}

\textbf{d.)} Then we use a Clebsch-Gordan table to find how the coupled basis states
relate to the uncoupled basis:

\begin{align*}
  \ket{2,-2} &= \ket{\frac{3}{2},\frac{1}{2},-\frac{3}{2}, -\frac{1}{2}}\\
  \ket{2,-1} &=
  \frac{\sqrt{3}}{2}\ket{\frac{3}{2},\frac{1}{2},-\frac{1}{2},-\frac{1}{2}}
  + \frac{1}{2}\ket{\frac{3}{2},\frac{1}{2},-\frac{3}{2},\frac{1}{2}}\\
  \ket{2,0} &=
  \frac{1}{\sqrt{2}}\ket{\frac{3}{2},\frac{1}{2},\frac{1}{2},-\frac{1}{2}}
  + \frac{1}{\sqrt{2}}\ket{\frac{3}{2},\frac{1}{2},-\frac{1}{2},\frac{1}{2}}\\
  \ket{2,1} &=
  \frac{1}{2}\ket{\frac{3}{2},\frac{1}{2},\frac{3}{2},-\frac{1}{2}}
  + \frac{\sqrt{3}}{2}\ket{\frac{3}{2},\frac{1}{2},\frac{1}{2},\frac{1}{2}}
  \ket{2,2} &= \ket{\frac{3}{2},\frac{1}{2},\frac{3}{2},\frac{1}{2}}\\
  \ket{1,-1} &=
  \frac{1}{2}\ket{\frac{3}{2},\frac{1}{2},-\frac{1}{2},-\frac{1}{2}}
  + -\frac{\sqrt{3}}{2}\ket{\frac{3}{2},\frac{1}{2},-\frac{3}{2},\frac{1}{2}}\\
  \ket{1,0} &=
  \frac{1}{\sqrt{2}}\ket{\frac{3}{2},\frac{1}{2},\frac{1}{2},-\frac{1}{2}}
    - \frac{1}{\sqrt{2}}\ket{\frac{3}{2},\frac{1}{2},-\frac{1}{2},\frac{1}{2}}\\
  \ket{1,-1} &=
  \frac{\sqrt{3}}{2}\ket{\frac{3}{2},\frac{1}{2},\frac{3}{2},-\frac{1}{2}}
  - \frac{1}{2} \ket{\frac{3}{2},\frac{1}{2},\frac{1}{2},\frac{1}{2}}\\
\end{align*}

This was a neat problem that showed how easy it is to related the coupled and
uncoupled states when one is given a CG table.\\

\section{11.16: Hyperfine Deuterium}
We can either add or subtract the electron spin: $\frac{1}{2}$ from the nuclear
spin: 1. Thus we get total system angular momenta of either $\frac{3}{2}$ or
$\frac{1}{2}$. The former momentum supports 4 z-measurement states, while the latter
supports 2. This means we will be dealing with a 6x6 matrix in our calculations.\\

We use the definition $F = \left(S+I\right)^2$ to create an expression for $S\cdot I$:
$S\cdot I = \frac{1}{2}\left(F^2 - S^2 - I^2\right)$. Our calculation of the
perturbation energies then becomes a simple matter of coming up with the matrices for
$F^2, S^2$ and $I^2$:\\

\begin{align*}
  H'_{hf} &= \frac{A}{\hbar^2}S\cdot I\\
  S\cdot I &= \frac{1}{2}\left(F^2 - S^2 - I^2\right)\\
  &=
  \frac{\hbar^2}{2}
  \begin{pmatrix}
    \frac{15}{4} & 0 & 0 & 0 & 0 & 0\\
    0 & \frac{15}{4} & 0 & 0 & 0 & 0\\
    0 & 0 & \frac{15}{4} & 0 & 0 & 0\\
    0 & 0 & 0 & \frac{15}{4} & 0 & 0\\
    0 & 0 & 0 & 0 & \frac{3}{2} & 0\\
    0 & 0 & 0 & 0 & 0 & \frac{3}{2}\\
  \end{pmatrix}
  -
  \frac{\hbar^2}{2}
  \begin{pmatrix}
    \frac{3}{2} & 0 & 0 & 0 & 0 & 0\\
    0 & \frac{3}{2} & 0 & 0 & 0 & 0\\
    0 & 0 & \frac{3}{2} & 0 & 0 & 0\\
    0 & 0 & 0 & \frac{3}{2} & 0 & 0\\
    0 & 0 & 0 & 0 & \frac{3}{2} & 0\\
    0 & 0 & 0 & 0 & 0 & \frac{3}{2}\\
  \end{pmatrix}
  -
  \frac{\hbar^2}{2}
  \begin{pmatrix}
    2 & 0 & 0 & 0 & 0 & 0\\
    0 & 2 & 0 & 0 & 0 & 0\\
    0 & 0 & 2 & 0 & 0 & 0\\
    0 & 0 & 0 & 2 & 0 & 0\\
    0 & 0 & 0 & 0 & 2 & 0\\
    0 & 0 & 0 & 0 & 0 & 2\\
  \end{pmatrix}\\
  &= \frac{\hbar^2}{2}
  \begin{pmatrix}
    \frac{1}{4} & 0 & 0 & 0 & 0 & 0\\
    0 & \frac{1}{4} & 0 & 0 & 0 & 0\\
    0 & 0 & \frac{1}{4} & 0 & 0 & 0\\
    0 & 0 & 0 & \frac{1}{4} & 0 & 0\\
    0 & 0 & 0 & 0 & -\frac{11}{4} & 0\\
    0 & 0 & 0 & 0 & 0 & -\frac{11}{4}\\
  \end{pmatrix}\\
  H'_{hf} &= \frac{A}{2}
  \begin{pmatrix}
    \frac{1}{4} & 0 & 0 & 0 & 0 & 0\\
    0 & \frac{1}{4} & 0 & 0 & 0 & 0\\
    0 & 0 & \frac{1}{4} & 0 & 0 & 0\\
    0 & 0 & 0 & \frac{1}{4} & 0 & 0\\
    0 & 0 & 0 & 0 & -\frac{11}{4} & 0\\
    0 & 0 & 0 & 0 & 0 & -\frac{11}{4}\\
  \end{pmatrix}\\
\end{align*}

Where the above matrices are indexed as:
$\ket{\frac{3}{2}, \frac{3}{2}},\ket{\frac{3}{2}, \frac{1}{2}},
\ket{\frac{3}{2}, -\frac{1}{2}},\ket{\frac{3}{2}, -\frac{3}{2}},
\ket{\frac{1}{2}, \frac{1}{2}},\ket{\frac{1}{2}, -\frac{1}{2}}$. This corresponds
to energies of $\frac{A}{16}$ and $\frac{-11}{4}A$. The first energy has a
degeneracy of five and the latter has a degeneracy of two. This is interestingly
different from the hydrogen atom, since the neutron is contributing to the nuclear
spin! Though we see that the nucleus-electron spin coupling only splits the energies
in two.\\

\vspace{5cm}

\end{document}

\documentclass[10pt]{article} % Font size - 10pt, 11pt or 12pt

\usepackage[hmargin=1.25cm, vmargin=1.5cm]{geometry} % Document margins

\usepackage{marvosym} % Required for symbols in the colored box
\usepackage{ifsym} % Required for symbols in the colored box

\usepackage[usenames,dvipsnames]{xcolor} % Allows the definition of hex colors

% Fonts and tweaks for XeLaTeX
\usepackage{fontspec,xltxtra,xunicode}
\defaultfontfeatures{Mapping=tex-text}
%\setmonofont[Scale=MatchLowercase]{Andale Mono}

% Colors for links, text and headings
\usepackage{hyperref}
\definecolor{linkcolor}{HTML}{506266} % Blue-gray color for links
\definecolor{shade}{HTML}{F5DD9D} % Peach color for the contact information box
\definecolor{text1}{HTML}{2b2b2b} % Main document font color, off-black
\definecolor{headings}{HTML}{701112} % Dark red color for headings
% Other color palettes: shade=B9D7D9 and linkcolor=A40000; shade=D4D7FE and linkcolor=FF0080

\hypersetup{colorlinks,breaklinks, urlcolor=linkcolor, linkcolor=linkcolor} % Set up links and colors

\usepackage{fancyhdr}
\usepackage{physics}
\usepackage{amssymb}
\pagestyle{fancy}
\fancyhf{}
% Headers and footers can be added with the \lhead{} \rhead{} \lfoot{} \rfoot{} commands
% Example footer:
%\rfoot{\color{headings} {\sffamily Last update: \today}. Typeset with Xe\LaTeX}

\renewcommand{\headrulewidth}{0pt} % Get rid of the default rule in the header

\usepackage{titlesec} % Allows creating custom \section's

% Format of the section titles
\titleformat{\section}{\color{headings}
\scshape\Large\raggedright}{}{0em}{}[\color{black}\titlerule]

\title{Quantum Mechanics Assignment One}
\author{Elliott Capek}
\titlespacing{\section}{0pt}{0pt}{5pt} % Spacing around titles

\begin{document}

\maketitle{}

\section{Problem 2.17}
This problem asks us to take the following spin-1 state and figure out the probabilities of
measuring z- and x-component spins, and also to find their expectation value.\\

\begin{align*}
  \ket{\Psi} \doteq &\frac{1}{\sqrt{30}}
  \begin{pmatrix}
    1\\
    2\\
    5i\\
  \end{pmatrix}
\end{align*}

We use the normal $P_x = \abs{\braket{\Psi_x}{\Psi}}^2$ to find probabilities. It is simple to
calculate the braket since both states are in the $S_z$ representation.

\begin{align*}
  P_{1} = \abs{\braket{\Psi_{1}}{\Psi}}^2 &=
  \abs{\frac{1}{\sqrt{30}}\begin{pmatrix}1&0&0\end{pmatrix}\begin{pmatrix}1\\2\\5i\\\end{pmatrix}}^2
  = \frac{1}{30}\\
  P_{0} = \abs{\braket{\Psi_{0}}{\Psi}}^2 &=
  \abs{\frac{1}{\sqrt{30}}\begin{pmatrix}0&1&0\end{pmatrix}\begin{pmatrix}1\\2\\5i\\\end{pmatrix}}^2
  = \frac{4}{30}\\
  P_{-1} = \abs{\braket{\Psi_{-1}}{\Psi}}^2 &=
  \abs{\frac{1}{\sqrt{30}}\begin{pmatrix}0&0&1\end{pmatrix}\begin{pmatrix}1\\2\\5i\\\end{pmatrix}}^2
  = \frac{25}{30}\\
\end{align*}
\begin{align*}
  \expval{S_z}{\Psi} &= \frac{1}{30}\begin{pmatrix}1&2&-5i\end{pmatrix}\begin{pmatrix}
      \hslash&0&0\\0&0&0\\0&0&-\hslash\\\end{pmatrix}\begin{pmatrix}1\\2\\5i\end{pmatrix}
        = \frac{1}{30}\begin{pmatrix}1&2&-5i\end{pmatrix}
          \begin{pmatrix}\hslash\\0\\5\hslash i\end{pmatrix}
            = \frac{1}{30}\left(1 + -25\right) = \frac{-4}{5}\\
\end{align*}

We then want to compute the expectation value of the spin in the x-direction. We use the
$S_z$ representation of $S_x$:

\begin{align*}
  \expval{S_x}{\Psi} &= \frac{1}{30}\begin{pmatrix}1&2&-5i\end{pmatrix}\frac{\hslash}{\sqrt{2}}
    \begin{pmatrix}0&1&0\\1&0&1\\0&1&0\\\end{pmatrix}\begin{pmatrix}1\\2\\5i\end{pmatrix}
        = \frac{1}{30\sqrt{2}}\begin{pmatrix}1&2&-5i\end{pmatrix}
          \begin{pmatrix}2\\1+5i\\2\end{pmatrix}\\
            = \frac{4}{30\sqrt{2}}\\
\end{align*}

\section{Problem 5.11}
This problem deals with an infinite square well wave which suddenly has its right potential wall
move from $L$ to $3L$. We are asked to find the probability of finding the wave in the ground or
first excited state. This problem is basically asking us to find the $c_1$ and $c_2$ coefficients
of the old wave function in the new well, and from them find probabilities.

\begin{align*}
  \Psi(x) &= \sqrt{\frac{2}{L}}\sin\left(\frac{\pi x}{L}\right)\\
  \ket{E_1} &= \sqrt{\frac{2}{3L}}\sin\left(\frac{\pi x}{3L}\right)\\
  \ket{E_2} &= \sqrt{\frac{2}{3L}}\sin\left(\frac{2\pi x}{3L}\right)\\
\end{align*}

We wish to find the $c_1$ and $c_2$ coefficients for this system.

\begin{align*}
  c_1 &= \braket{E_1}{\Psi} = \int_0^{3L} \sqrt{\frac{2}{3L}}\sin\left(\frac{\pi x}{3L}\right)*
  \sqrt{\frac{2}{L}}\sin\left(\frac{\pi x}{L}\right)dx\\
  &= \sqrt{\frac{4}{3L^3}}\int_0^{3L}
  \sin\left(\frac{\pi x}{3L}\right)\sin\left(\frac{\pi x}{L}\right)\\
  &= \sqrt{\frac{2}{3L^3}}\left(\int_0^{3L} \cos\left(\frac{2\pi x}{3L}\right)dx
  - \int_0^{3L} \cos\left(\frac{4\pi x}{3L}\right)dx\right)\\
  &= 0\\
\end{align*}

\begin{align*}
  c_2 &= \braket{E_2}{\Psi} = \int_0^{3L} \sqrt{\frac{2}{3L}}\sin\left(\frac{2\pi x}{3L}\right)*
  \sqrt{\frac{2}{L}}\sin\left(\frac{\pi x}{L}\right)dx\\
  &= \sqrt{\frac{4}{3L^3}}\int_0^{3L}
  \sin\left(\frac{2\pi x}{3L}\right)\sin\left(\frac{\pi x}{L}\right)\\
  &= \sqrt{\frac{2}{3L^3}}\left(\int_0^{3L} \cos\left(\frac{\pi x}{3L}\right)dx
  - \int_0^{3L} \cos\left(\frac{5\pi x}{3L}\right)dx\right)\\
  &= 0\\
\end{align*}

From this we can see that the probability of measuring the wave in the new tripled well to be in
$\ket{E_1}$ or $\ket{E_2}$ is zero, since $P_n = \abs{c_n}^2$. This result is to be expected, since
the wave should be completely in the second excited state, $\ket{E_3}$. This is because extending
the $\Psi(x)$ wave function from $0$ to $3L$ results in a two-node wave function identical to the
$\ket{E_3}$ state.\\

\section{8.7}
We begin with the potential and find the r-values at which $E=V$. We then calculate
the probability using $P = \int_r^\infty \psi^* \psi dr$.

\begin{align*}
  V(r) &= \frac{-e^2}{4\pi\epsilon_0r}\\
\end{align*}

We first solve for the forbidden distance:

\begin{align*}
  V(r) &= E_2\\
  -\frac{Z\alpha\hslash c}{r} &= -\frac{1}{2n^2}\alpha^2m_ec^2\\
  \frac{-1.44 \mbox{eV nm}}{r} &= -3.4 \mbox{eV}\\
  r &= 2.36 \mbox{nm}
\end{align*}

Next we find the probability of our particles being further than this:

\begin{align*}
  P_{200} &= \int_{2.36}^\infty \int_0^{2\pi} \int_0^\pi
  \frac{1}{\sqrt{8\sqrt{2}\pi a_0^{3/2}}}\left(2-\frac{r}{a_0}\right)e^{\frac{-r}{2a_0}}
  r^2\sin(\theta)drd\phi d\theta = \\
  P_{21-1} &= \int_{2.36}^\infty \int_0^{2\pi} \int_0^\pi
  \left(\frac{3r}{16\pi\sqrt{6}a_0^{5/2}}\right)^{1/2}\sin(\theta)e^{-i\phi}e^{-\frac{r}{2a_0}}
  r^2\sin(\theta)drd\phi d\theta = \\
  P_{210} &= \int_{2.36}^\infty \int_0^{2\pi} \int_0^\pi
  \left(\frac{3r}{8\pi\sqrt{6}a_0^{5/2}}\right)^{1/2}\cos(\theta)e^{-\frac{r}{2a_0}}
  r^2\sin(\theta)drd\phi d\theta = \\
  P_{211} &= \int_{2.36}^\infty \int_0^{2\pi} \int_0^\pi
  -\left(\frac{3r}{16\pi\sqrt{6}a_0^{5/2}}\right)^{1/2}\sin(\theta)e^{i\phi}e^{-\frac{r}{2a_0}}
  r^2\sin(\theta)drd\phi d\theta = \\
\end{align*}


\end{document}

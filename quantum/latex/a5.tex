\documentclass[10pt]{article} % Font size - 10pt, 11pt or 12pt

\usepackage[hmargin=1.25cm, vmargin=1.5cm]{geometry} % Document margins

\usepackage{marvosym} % Required for symbols in the colored box

\usepackage[usenames,dvipsnames]{xcolor} % Allows the definition of hex colors

% Fonts and tweaks for XeLaTeX
\usepackage{fontspec,xltxtra,xunicode}
\defaultfontfeatures{Mapping=tex-text}
%\setmonofont[Scale=MatchLowercase]{Andale Mono}

% Colors for links, text and headings
\usepackage{hyperref}
\definecolor{linkcolor}{HTML}{506266} % Blue-gray color for links
\definecolor{shade}{HTML}{F5DD9D} % Peach color for the contact information box
\definecolor{text1}{HTML}{2b2b2b} % Main document font color, off-black
\definecolor{headings}{HTML}{701112} % Dark red color for headings
% Other color palettes: shade=B9D7D9 and linkcolor=A40000; shade=D4D7FE and linkcolor=FF0080

\hypersetup{colorlinks,breaklinks, urlcolor=linkcolor, linkcolor=linkcolor} % Set up links and colors

\usepackage{fancyhdr}
\usepackage{amsmath}
\usepackage{braket}
\usepackage{amssymb}
\pagestyle{fancy}
\fancyhf{}
% Headers and footers can be added with the \lhead{} \rhead{} \lfoot{} \rfoot{} commands
% Example footer:
%\rfoot{\color{headings} {\sffamily Last update: \today}. Typeset with Xe\LaTeX}

\renewcommand{\headrulewidth}{0pt} % Get rid of the default rule in the header

\usepackage{titlesec} % Allows creating custom \section's

% Format of the section titles
\titleformat{\section}{\color{headings}
\scshape\Large\raggedright}{}{0em}{}[\color{black}\titlerule]

\title{Quantum Mechanics Assignment Five}
\author{Elliott Capek}
\titlespacing{\section}{0pt}{0pt}{5pt} % Spacing around titles

\begin{document}

\maketitle{}

\section{11.7: $\vec{S} \cdot \vec{I}$ simplified}

Here we show how using angular momentum ladder operators can simplify the dot product
$\vec{S} \cdot \vec{I}$:\\

\begin{align*}
  \vec{S} \cdot \vec{I} &= \hat{S}_x\hat{I}_x + \hat{S}_y\hat{I}_y + \hat{S}_z\hat{I}_z\\
  &= \frac{1}{2}\left(\hat{S}_x\hat{I}_x -i\hat{S}_x\hat{I}_y + i\hat{S}_y\hat{I}_x
  + \hat{S}_y\hat{I}_y + \hat{S}_x\hat{I}_x + i\hat{S}_x\hat{I}_y
  + -i\hat{S}_y\hat{I}_x\hat{S}_y\hat{I}_y\right)
  + \hat{S}_z\hat{I}_z\\
  &= \frac{1}{2}\left(\hat{S}_+\hat{I}_- + \hat{S}_-\hat{I}_+\right) + \hat{S}_z\hat{I}_z\\
\end{align*}

This is a useful math trick with no physical meaning! Multiplying raising and lowering operators
together has no physical significance, let alone operators for different particles. Nevertheless
this will probably come in handy later.\\

\section{11.8: Ladder operators to find perturabation Hamiltonian}

Here we

\begin{align*}
  
\end{align*}

\end{document}

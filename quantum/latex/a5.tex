\documentclass[10pt]{article} % Font size - 10pt, 11pt or 12pt

\usepackage[hmargin=1.25cm, vmargin=1.5cm]{geometry} % Document margins

\usepackage{marvosym} % Required for symbols in the colored box

\usepackage[usenames,dvipsnames]{xcolor} % Allows the definition of hex colors

% Fonts and tweaks for XeLaTeX
\usepackage{fontspec,xltxtra,xunicode}
\defaultfontfeatures{Mapping=tex-text}
%\setmonofont[Scale=MatchLowercase]{Andale Mono}

% Colors for links, text and headings
\usepackage{hyperref}
\definecolor{linkcolor}{HTML}{506266} % Blue-gray color for links
\definecolor{shade}{HTML}{F5DD9D} % Peach color for the contact information box
\definecolor{text1}{HTML}{2b2b2b} % Main document font color, off-black
\definecolor{headings}{HTML}{701112} % Dark red color for headings
% Other color palettes: shade=B9D7D9 and linkcolor=A40000; shade=D4D7FE and linkcolor=FF0080

\hypersetup{colorlinks,breaklinks, urlcolor=linkcolor, linkcolor=linkcolor} % Set up links and colors

\usepackage{fancyhdr}
\usepackage{amsmath}
\usepackage{braket}
\usepackage{amssymb}
\pagestyle{fancy}
\fancyhf{}
% Headers and footers can be added with the \lhead{} \rhead{} \lfoot{} \rfoot{} commands
% Example footer:
%\rfoot{\color{headings} {\sffamily Last update: \today}. Typeset with Xe\LaTeX}

\renewcommand{\headrulewidth}{0pt} % Get rid of the default rule in the header

\usepackage{titlesec} % Allows creating custom \section's

% Format of the section titles
\titleformat{\section}{\color{headings}
\scshape\Large\raggedright}{}{0em}{}[\color{black}\titlerule]

\title{Quantum Mechanics Assignment Five}
\author{Elliott Capek}
\titlespacing{\section}{0pt}{0pt}{5pt} % Spacing around titles

\begin{document}

\maketitle{}

\section{1: Effect of E-field on H spectrum}
Here we examine how a perturbing electric field influences the n=2 to n=1 transition for
the Hydrogen atom.\\

\textbf{a.)} First we find the unperturbed transition energy:\\

\begin{align*}
  E_2 - E_1 &= -13.6 \mbox{eV} \left(\frac{1}{2^2} - \frac{1}{1^2}\right) = 10.2 \mbox{eV}\\
\end{align*}

\textbf{b.)} Now we apply a 0.5 V/m electric field and see how the spectrum changes. The Stark
Effect tells us that the energy levels will be perturbed by either zero or
$\pm3e\epsilon a_0$, which for this electric field would be $\pm 2.64*10^{-2}eV$. This comes out
to be a change of roughly $0.25$ percent, which is a fairly small value.\\

The perturbation due to the electric field can be interpreted as the energy change of the
electric dipole moment interacting with the applied field. Some of the n=2 states are radially
symmetric, and so have no electric dipole to alter the energy. However certain combinations of
orbitals (ie $\bra{200}+\bra{210}$) do have a bias to part of the electron, and so do have an
energy change.\\

\section{11.7: $\vec{S} \cdot \vec{I}$ simplified}

Here we show how using angular momentum ladder operators can simplify the dot product
$\vec{S} \cdot \vec{I}$:\\

\begin{align*}
  \vec{S} \cdot \vec{I} &= \hat{S}_x\hat{I}_x + \hat{S}_y\hat{I}_y + \hat{S}_z\hat{I}_z\\
  &= \frac{1}{2}\left(\hat{S}_x\hat{I}_x -i\hat{S}_x\hat{I}_y + i\hat{S}_y\hat{I}_x
  + \hat{S}_y\hat{I}_y + \hat{S}_x\hat{I}_x + i\hat{S}_x\hat{I}_y
  + -i\hat{S}_y\hat{I}_x\hat{S}_y\hat{I}_y\right)
  + \hat{S}_z\hat{I}_z\\
  &= \frac{1}{2}\left(\hat{S}_+\hat{I}_- + \hat{S}_-\hat{I}_+\right) + \hat{S}_z\hat{I}_z\\
\end{align*}

This is a useful math trick with no physical meaning! Multiplying raising and lowering operators
together has no physical significance, let alone operators for different particles. Nevertheless
this will probably come in handy later.\\

\section{11.9: Commutators}
Here we show that the spin magnitude operators \textbf{$S^2$} and \textbf{$I^2$} commute with
the perturbation Hamiltonian, while the component operators \textbf{$S_z$} and \textbf{$I_z$}
do not.

\begin{align*}
  \left[S^2, H'_{hf}\right] &= \left(S^2H'_{hf} - H'_{hf}S^2\right)\ket{m_p,m_s}\\
  &= S^2H'_{hf}\ket{m_p,m_s} - H'_{hf}S^2\ket{m_p,m_s}\\
  &= S^2E_{m_p,m_s}\ket{m_p,m_s} - H'_{hf}S^2\frac{3\hbar^2}{4}\ket{m_p,m_s}\\
  &= \frac{3\hbar^2}{4}E_{m_p,m_s}\ket{m_p,m_s} - E_{m_p,m_s}S^2\frac{3\hbar^2}{4}\ket{m_p,m_s}\\
  &= 0\\
\end{align*}

\end{document}

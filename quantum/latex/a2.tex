\documentclass[10pt]{article} % Font size - 10pt, 11pt or 12pt

\usepackage[hmargin=1.25cm, vmargin=1.5cm]{geometry} % Document margins

\usepackage{marvosym} % Required for symbols in the colored box

\usepackage[usenames,dvipsnames]{xcolor} % Allows the definition of hex colors

% Fonts and tweaks for XeLaTeX
\usepackage{fontspec,xltxtra,xunicode}
\defaultfontfeatures{Mapping=tex-text}
%\setmonofont[Scale=MatchLowercase]{Andale Mono}

% Colors for links, text and headings
\usepackage{hyperref}
\definecolor{linkcolor}{HTML}{506266} % Blue-gray color for links
\definecolor{shade}{HTML}{F5DD9D} % Peach color for the contact information box
\definecolor{text1}{HTML}{2b2b2b} % Main document font color, off-black
\definecolor{headings}{HTML}{701112} % Dark red color for headings
% Other color palettes: shade=B9D7D9 and linkcolor=A40000; shade=D4D7FE and linkcolor=FF0080

\hypersetup{colorlinks,breaklinks, urlcolor=linkcolor, linkcolor=linkcolor} % Set up links and colors

\usepackage{fancyhdr}
\usepackage{amsmath}
\usepackage{physics}
\usepackage{amssymb}
\pagestyle{fancy}
\fancyhf{}
% Headers and footers can be added with the \lhead{} \rhead{} \lfoot{} \rfoot{} commands
% Example footer:
%\rfoot{\color{headings} {\sffamily Last update: \today}. Typeset with Xe\LaTeX}

\renewcommand{\headrulewidth}{0pt} % Get rid of the default rule in the header

\usepackage{titlesec} % Allows creating custom \section's

% Format of the section titles
\titleformat{\section}{\color{headings}
\scshape\Large\raggedright}{}{0em}{}[\color{black}\titlerule]

\title{Quantum Mechanics Assignment Two}
\author{Elliott Capek}
\titlespacing{\section}{0pt}{0pt}{5pt} % Spacing around titles

\begin{document}

\maketitle{}

\section{9.4: Forbidden region probability for QHO}
For this problem we want to calculate the probability of measuring a QHO ground-state object outside
of the classically allowed region. We do this by calculating the turning points of a classical HO,
ie finding the x-values where energy equals potential, and then integrating $\psi^*\psi$ over the
forbidden region to find the chance of forbidden observation.\\

We first find the turning point for the ground state $n=0$:
\begin{align*}
  V_{HO} &= E_0\\
  \frac{1}{2}m\omega^2x^2 &= \hbar\omega\left(0+\frac{1}{2}\right)\\
  x_0 &= \sqrt{\frac{\hbar}{m\omega}}\\
\end{align*}

We then find the probability of finding the particle in this region:

\begin{align*}
  P_{forb} &= \int_{\infty}^{-x_0} \psi_0^*\psi_0 dx + \int_{x_0}^{\infty} \psi_0^*\psi_0 dx\\
  &= 2\int_{x_0}^{\infty} \sqrt{\frac{m\omega}{\pi\hbar}}e^{-m\omega x^2 / \hbar} dx\\
  &= 0.16\\
\end{align*}

Thus we see that there is a small chance of finding the particle in the classically
forbidden region, which is expected for the ground state of the QHO. To do the
calculation we exploit the fact that even functions can be integrated on only one
side of the y-axis and doubled to get the total integral answer.\\

\section{9.9: QHO expectation values via integration, operator method}
In this problem we use both wave function integration and the operator method to find the
expectation values of various observables: x, p, $x^2$, $p^2$.

\textbf{a.) Ground state wave function calculations}
\begin{align*}
  <x> &= \int_{-\infty}^{\infty} \psi^*x\psi dx\\
  &= \int_{-\infty}^{\infty} \sqrt{\frac{m\omega}{\pi\hbar}}e^{-m\omega x^2 / \hbar}x dx = 0\\
\end{align*}
We don't need to perform the $<x>$ integral because it involves integrating an odd function over
an even domain, which always results in zero. This is to be expected, because the wave ground-state
wave function is symmetric across zero.\\

\begin{align*}
  <p> &= \int_{-\infty}^{\infty} \psi^* -i\hbar \frac{d}{dx}\psi dx\\
  &= \int_{-\infty}^{\infty}
  -i\sqrt{\frac{m\omega}{\pi\hbar}}(-m\omega x)e^{\frac{-m\omega x^2}{2\hbar}}dx = 0\\
\end{align*}
This integral is zero since we are again integrating an odd function over an even domain. This
makes sense because, since the spacial distribution of the wave function is equal on both sides
of the x-axis, the velocities (and hence momenta) should be equal as well.\\

\begin{align*}
  <x^2> &= \int_{-\infty}^{\infty} \psi^*x^2\psi dx\\
  &= \int_{-\infty}^{\infty} \sqrt{\frac{m\omega}{\pi\hbar}}e^{-m\omega x^2 / \hbar}x^2 dx\\
  &= 2\sqrt{\frac{m\omega}{\pi\hbar}}\int_{0}^{\infty} e^{-m\omega x^2 / \hbar}x^2 dx\\
  &= 2\sqrt{\frac{m\omega}{\pi\hbar}} *
  \frac{\sqrt{\pi}}{4(\frac{m\omega}{\hbar})^{3/2}} \hspace{2cm}\mbox{Wolfram}\\
  &= \frac{\hbar}{2m\omega}\\
\end{align*}
Here we exploit the fact that integrating an even function over an even domain is equivalent to
doubling the integral of the function over half the domain. The answer makes sense, since increasing
$m\omega$ corresponds to making the quadratic potential steeper, hence making the wave function
tighter.\\

\begin{align*}
  <p^2> &= \int_{-\infty}^{\infty} \psi^* \left(-\hbar^2 \frac{d^2}{dx^2}\right)\psi dx\\
  &= \sqrt{\frac{m\omega}{\pi\hbar}}\int_{-\infty}^{\infty} -\hbar^2e^{\frac{-m\omega x^2}{\hbar}}
  \left(\frac{m^2\omega^2x^2}{\hbar^2} - \frac{m\omega}{\hbar}\right)dx\\
  &= -\hbar^2\sqrt{\frac{m\omega}{\pi\hbar}}\left(
  \int_{-\infty}^{\infty} e^{\frac{-m\omega x^2}{\hbar}}\frac{m^2\omega^2x^2}{\hbar^2}dx -
  \int_{-\infty}^{\infty} e^{\frac{-m\omega x^2}{\hbar}}\frac{m\omega}{\hbar}\right)dx\\
  &= -\hbar^2\sqrt{\frac{m\omega}{\pi\hbar}}
  \left(\frac{1}{2}\sqrt{\frac{m\pi\omega}{\hbar}} -
  \sqrt{\frac{m\pi\omega}{\hbar}}\right) \hspace{2cm}\mbox{Wolfram}\\
  &= \frac{\hbar m\omega}{2}\\
\end{align*}
This answer is intuitive, since increasing mass as well as oscillatory speed should both increase
the expected momentum.

\textbf{b.) Any-state operator method calculations}
Here we define the position and momentum operators in terms of the raising and lowering operators:
\begin{align*}
  \hat{x} &= \sqrt{\frac{\hbar}{2m\omega}}\left(a + a^\dagger\right)\\
  \hat{p} &= i\sqrt{\frac{m\omega\hbar}{2}}\left(a^\dagger - a\right)\\
\end{align*}
We then solve for the various expectation values using these definitions:

\begin{align*}
  <x> &= \bra{n}x\ket{n}\\
  &= \sqrt{\frac{\hbar}{2m\omega}}\bra{n}\left(a + a^\dagger\right)\ket{n}\\
  &= \sqrt{\frac{\hbar}{2m\omega}}
  \left(\sqrt{n}\bra{n}\ket{n-1} + \sqrt{n+1}\bra{n}\ket{n+1}\right) = 0\\
\end{align*}
This answer indicates that all pure QHO wave functions are centered around zero. This makes
sense since the QHO wave functions are all made up of either even or odd functions. When squared,
as is done when finding the probability, the equation becomes entirely even, meaning it is
symmetric across the y-axis. From this we can see that all QHO wave functions have an expectation
value of zero for x-measurement.

\begin{align*}
  <p> &= \bra{n}\hat{p}\ket{n}\\
  &= i\sqrt{\frac{m\hbar\omega}{2}}\bra{n}\left(a^\dagger-a\right)\ket{n}\\
  &= i\sqrt{\frac{m\hbar\omega}{2}}
  \left(\sqrt{n+1}\bra{n}\ket{n+1} - \sqrt{n}\bra{n}\ket{n-1}\right) = 0\\
\end{align*}
This is an interesting result. It is intuitive that any wave function should have an expectation of
zero for momentum, else the wave function would move through space due to a nonzero average
velocity.

\begin{align*}
  <x^2> &= \bra{n}x^2\ket{n}\\
  &= \frac{\hbar}{2m\omega}\bra{n}\left(a + a^\dagger\right)\left(a + a^\dagger\right)\ket{n}\\
  &= \frac{\hbar}{2m\omega}\bra{n}
  \left(aa + aa^\dagger + a^\dagger a + a^\dagger a^\dagger\right)\ket{n}\\
  &= \frac{\hbar}{2m\omega}
  \left(aa + aa^\dagger + a^\dagger a + a^\dagger a^\dagger\right)\ket{n}\\
  &= \frac{\hbar}{2m\omega} \left(\sqrt{n-1}\sqrt{n-2}\bra{n}\ket{n-2} + (n+1)\bra{n}\ket{n} +
  (n)\bra{n}\ket{n} + \sqrt{n+1}\sqrt{n+2}\bra{n}\ket{n+2}\right)\\
  &= \frac{\hbar}{2m\omega} \left(2n+1\right)\\
\end{align*}
This is an interesting result which indicates that for each extra rung on the ladder, the
expectation of the squared position increases linearly, meaning the expectation of the magnitude
of x increases as $\sqrt{n}$. This makes sense, since increasing energy levels have less and less
room to expand in the quadratic potential well. The $n=0$ state corresponds to the wave function
solution for the ground state found above.\\

\begin{align*}
  <p^2> &= \bra{n}p^2\ket{n}\\
  &= -\frac{m\hbar\omega}{2}\bra{n}\left(a^\dagger-a\right)\left(a^\dagger-a\right)\ket{n}\\
  &= -\frac{m\hbar\omega}{2}\bra{n}
  \left(a^\dagger a^\dagger - a^\dagger a - aa^\dagger + aa\right)\ket{n}\\
  &= -\frac{m\hbar\omega}{2}\bra{n}\left(\sqrt{n-1}\sqrt{n-2}\bra{n}\ket{n-2}
  - (n)\bra{n}\ket{n} - (n+1)\bra{n}\ket{n} + \sqrt{n+1}\sqrt{n+2}\bra{n}\ket{n+2}\right)\\
  &= \frac{\hbar m\omega}{2}\left(2n + 1\right)
\end{align*}
This result is similar to the above one for $x^2$ with different units. It is still a $\sqrt{n}$
relationship for momentum, which makes sense.\\

\textbf{c.) Uncertainty principle}\\
Since the above two cases agree with each other, we check that the more general case of part
\textbf{b.} agrees with the uncertainty principle.

\begin{align*}
  \Delta x &= \sqrt{<x^2>-<x>^2}\\
  &= \sqrt{\frac{\hbar}{2m\omega} \left(2n+1\right)}\\
  \Delta p &= \sqrt{<p^2>-<p>^2}\\
  &= \sqrt{\frac{\hbar m\omega}{2}\left(2n + 1\right)}\\
  \Delta x \Delta p &= \sqrt{\frac{\hbar}{2m\omega} \left(2n+1\right)}
  \sqrt{\frac{\hbar m\omega}{2}\left(2n + 1\right)}\\
  &= \frac{\hbar}{2}(2n+1)\\
\end{align*}
This is an interesting result which indicates that uncertainty increases as the pure wave function
gets more energetic. Ground state is the state with maximum certainty. The higher states get
progressively more uncertain, which makes sense since both the velocity and the well get larger
for higher states.\\

\section{9.14: Unknown QHO wavefunction}
We are given the wave function $\psi(x,0) = A\left[1 - 3\sqrt{\frac{m\omega}{\hbar}}x +
  2\frac{m\omega}{\hbar}x^2\right]$. Because its highest-order polynomial term is $x^2$, we know
that it must be made up of the $n = 0, 1, 2$ energy states. We now figure out the coefficients for
each:

\begin{align*}
  c_0 &= \int_{-\infty}^{\infty} \phi_0^*\psi dx\\
  &= A\int_{-\infty}^{\infty} \left(\frac{m\omega}{\pi\hbar}\right)^{1/2}e^{-m\omega x^2/\hbar}
  \left(1 - 3\sqrt{\frac{m\omega}{\hbar}}x +
  2\frac{m\omega}{\hbar}x^2\right)e^{-m\omega x^2/(2\hbar)}dx\\
  &= 2A\\
  c_1 &= A\int_{-\infty}^{\infty} \frac{2\sqrt{\frac{m\omega}{\hbar}x}}{\sqrt{2}}
    \left(\frac{m\omega}{\pi\hbar}\right)^{1/2} \left(1 - 3\sqrt{\frac{m\omega}{\hbar}}x +
    2\frac{m\omega}{\hbar}x^2\right)e^{-m\omega x^2/\hbar}dx\\
 &= -\frac{3}{\sqrt{2}}A\\
 c_2 &= A\int_{-\infty}^{\infty} \frac{\frac{4m\omega}{\hbar}x^2 - 2}{\sqrt{8}}
    \left(\frac{m\omega}{\pi\hbar}\right)^{1/2} \left(1 - 3\sqrt{\frac{m\omega}{\hbar}}x +
    2\frac{m\omega}{\hbar}x^2\right)e^{-m\omega x^2/\hbar}dx\\
 &= \sqrt{2}A\\
\end{align*}

From this we can see that $\psi(x,0) = \left(\frac{2\sqrt{2}}{\sqrt{21}}\ket{0}
+ \frac{-3}{\sqrt{21}}\ket{1} + \frac{sqrt{2}}{\sqrt{21}}\ket{2}\right)$. It is then easy to compute
the expectation value: \\

\begin{align*}
  <E> &= \sum \hbar\omega\left(n+\frac{1}{2}\right)|c_n|^2\\
  &= \frac{2 + \frac{27}{4} + 5}{\frac{21}{2}} = 1.3\hbar\omega\\
\end{align*}

\section{9.13: Half-QHO well}
(Inspired by http://www.physicspages.com/2012/08/18/half-harmonic-oscillator/)

This system is identical to the 1D QHO, except all positive x values have infinite
potential. So the system is a mixture of the infinite well and SHO potentials. We
thus expect the wave function to have characteristics of both systems. For positive
x values we expect the wave function to be zero, as it is in the infinite well.
Because the negative x-values have a potential identical to QHO, we expect the wave
function to be identical as well. So our total wave function is a Hermite/Gaussian
for negative x values and zero for positive x values.\\

However, we're not done. We notice that the x=0 boundary condition is not obeyed
for all Hermite/Gaussian wave functions. We require the wave function to be zero
at x=0, which is only true for odd QHO wave functions. Thus our energy spectrum
is identical to the QHO spectru, but only for odd values of n:

\begin{align*}
  E_n &= \left(n + \frac{1}{2}\right)\hbar\omega\hspace{2cm}\mbox{n=1,3,5,...}\\
\end{align*}

For this problem we just applied the basic logic that the value of a wave function
is only based on its current location, not the overall shape of the potential. The
wave function for a system that is identical to the QHO will have a wave function
identical to QHO in the similar regions. This is a powerful result.\\

\section{5: 2D QHO}
(Derivation inspired by
$\mbox{nyu.edu/classes/tuckerman/adv.chem/lectures/lecture\_18/node6.html)}$\\
To take the QHO to two dimensions, we use the following Hamiltonian:

\begin{align*}
  H &= \frac{\hat{p}_x^2}{2m} + \frac{\hat{p}_y^2}{2m}
  + \frac{1}{2}m\omega^2\left(x^2+y^2\right)\\
\end{align*}

We notice that none of the cartesian terms in this Hamiltonian are coupled, so
our energy eigenvalue equation then takes the form:

\begin{align*}
  H_{xy}X(x)Y(y) &= (E_x + E_y)X(x)Y(y)\\
\end{align*}

This equation is basically the sum of two uncoupled QHOs. Because of this, we
can calculate the energy fairly easily:

\begin{align*}
  E_{xy} &= E_x + E_y = \left(n_x+n_y + 1\right)\hbar\omega\\
\end{align*}

From this we can see that the system is very degenerate. For the $n^{th}$-highest
energy level there are $\frac{n}{2}$ possible states that have that energy: for
the tenth energy level, the possible n-numbers are $1,9$, $2,8$, $3,7$, $4,6$ and
$5,5$.\\

We can construct the ground level energy state by taking the product of the two
uncoupled QHOs:

\begin{align*}
  \psi_0(x,y) &= \left(\frac{m\omega}{\pi\hbar}\right)^{1/4}
  e^{-m\omega\left(x^2+y^2\right)/2\hbar}\\
\end{align*}

The ease with which we go from the 1D to 2D harmonic oscillators comes from the fact
that the x and y coordinates of the oscillator are not coupled. The x and y cross
sections of the wave can take on any allowed 1D state. It is not intuitive to me
why this method exactly works. Why are the energies of the x and y components
just summed together to get total energy, as opposed to being multiplied or
something? The degeneracy of the system is interesting. Because a sum is used,
for large energies there are a huge number of ways to configure the system.\\

\section{6: Position, momentum commutator}
Here we find the value of the position, momentum commutator:
\begin{align*}
  [\hat{x}, \hat{p}] &= \hat{x}\hat{p} - \hat{p}\hat{x}\\
  &= \left(\hat{x}\hat{p}\psi(x) - \hat{p}\hat{x}\psi(x)\right)\frac{1}{\psi(x)}\\
  &= \left(\hat{x}\left(-i\hbar\frac{d}{dx}\right)\psi(x) - \hat{p}x\psi(x)\right)\frac{1}{\psi(x)}\\
  &= \left(-i\hbar x \frac{d\psi}{dx} + i\hbar\left(\psi(x)
  + x\frac{d\psi}{dx}\right)\right)\frac{1}{\psi(x)}\\
  &= \left(i\hbar\psi(x)\right)\frac{1}{\psi(x)}\\
  &= i\hbar\\
\end{align*}
Here we use the definitions of the position and momentum operators to find the value of their
commutator. We need to operate on a generic wave function in order to do algebra on the
operators, since the differential in the momentum operator is meaningless without a function
to act on. This result is critical, since it is at the heart of the Heisenberg Uncertainty
Principle: position and momentum don't commute, so they do not share a basis and they are not
co-measurable. This is the mathematical representation of the popular saying ``you cannot know
the position and momentum of a particle simultaneously.''

\section{7: Commutator algebra}

\begin{align*}
  \left[\hat{A},\hat{A}^n\right] &= AA^n - A^nA = A^{n+1} - A^{n+1} = 0\\
  \left[\hat{A},\hat{B}\right] &= \hat{A}\hat{B} - \hat{B}\hat{A}
  = -\left(\hat{B}\hat{A} - \hat{A}\hat{B}\right) = -\left[\hat{B},\hat{A}\right]\\
  \left[\hat{A},c\hat{B}\right] &= \hat{A}c\hat{B} - c\hat{B}\hat{A}
  = c\left(\hat{A}\hat{B} - \hat{B}\hat{A}\right) = c\left[\hat{A},\hat{B}\right]\\
  \left[\hat{A},\left(\hat{B}+\hat{C}\right)\right] &= \hat{A}\left(\hat{B} + \hat{C}\right)
  - \left(\hat{B} + \hat{C}\right)\hat{A} = \hat{A}\hat{B} - \hat{B}\hat{A}
  + \hat{A}\hat{C} + \hat{C}\hat{A} = \left[\hat{A},\hat{B}\right] + \left[\hat{A},\hat{C}\right]\\
  \left[\hat{A},\hat{B}\hat{C}\right] &= \hat{A}\hat{B}\hat{C} - \hat{B}\hat{C}\hat{A}
  = \hat{A}\hat{B}\hat{C} - \hat{B}\hat{A}\hat{C} + \hat{B}\hat{A}\hat{C} - \hat{B}\hat{C}\hat{A}
  = \left[\hat{A},\hat{B}\right]\hat{C} + \hat{B}\left[\hat{A},\hat{C}\right]\\
  \left[\hat{A},\left[\hat{B},\hat{C}\right]\right]
  + \left[\hat{B},\left[\hat{C},\hat{A}\right]\right]
  &+ \left[\hat{C},\left[\hat{A},\hat{B}\right]\right] =
  \left[\hat{A},\hat{B}\hat{C} - \hat{C}\hat{B}\right]
  + \left[\hat{B},\hat{C}\hat{A} - \hat{A}\hat{C}\right]
  + \left[\hat{C},\hat{A}\hat{B} - \hat{B}\hat{A}\right] = \\
  &= \hat{A}\hat{B}\hat{C} - \hat{A}\hat{C}\hat{B}
  + \hat{B}\hat{C}\hat{A} - \hat{B}\hat{A}\hat{C}
  + \hat{C}\hat{A}\hat{B} - \hat{C}\hat{B}\hat{A}\\
\end{align*}

\end{document}

\documentclass[10pt]{article} % Font size - 10pt, 11pt or 12pt

\usepackage[hmargin=1.25cm, vmargin=1.5cm]{geometry} % Document margins

\usepackage{marvosym} % Required for symbols in the colored box
\usepackage{ifsym} % Required for symbols in the colored box

\usepackage[usenames,dvipsnames]{xcolor} % Allows the definition of hex colors

% Fonts and tweaks for XeLaTeX
\usepackage{fontspec,xltxtra,xunicode}
\defaultfontfeatures{Mapping=tex-text}
%\setmonofont[Scale=MatchLowercase]{Andale Mono}

% Colors for links, text and headings
\usepackage{hyperref}
\definecolor{linkcolor}{HTML}{506266} % Blue-gray color for links
\definecolor{shade}{HTML}{F5DD9D} % Peach color for the contact information box
\definecolor{text1}{HTML}{2b2b2b} % Main document font color, off-black
\definecolor{headings}{HTML}{701112} % Dark red color for headings
% Other color palettes: shade=B9D7D9 and linkcolor=A40000; shade=D4D7FE and linkcolor=FF0080

\hypersetup{colorlinks,breaklinks, urlcolor=linkcolor, linkcolor=linkcolor} % Set up links and colors

\usepackage{fancyhdr}
\usepackage{physics}
\usepackage{amssymb}
\pagestyle{fancy}
\fancyhf{}
% Headers and footers can be added with the \lhead{} \rhead{} \lfoot{} \rfoot{} commands
% Example footer:
%\rfoot{\color{headings} {\sffamily Last update: \today}. Typeset with Xe\LaTeX}

\renewcommand{\headrulewidth}{0pt} % Get rid of the default rule in the header

\usepackage{titlesec} % Allows creating custom \section's

% Format of the section titles
\titleformat{\section}{\color{headings}
\scshape\Large\raggedright}{}{0em}{}[\color{black}\titlerule]

\title{Quantum Mechanics Assignment Two}
\author{Elliott Capek}
\titlespacing{\section}{0pt}{0pt}{5pt} % Spacing around titles

\begin{document}

\maketitle{}

\section{9.4: Forbidden region probability for QHO}
For this problem we want to calculate the probability of measuring a QHO ground-state object outside
of the classically allowed region. We do this by calculating the turning points of a classical HO,
ie finding the x-values where energy equals potential, and then integrating $\psi^*\psi$ over the
forbidden region to find the chance of forbidden observation.\\

We first find the turning point for the ground state $n=0$:
\begin{align*}
  V_{HO} &= E_0\\
  \frac{1}{2}m\omega^2x^2 &= \hbar\omega\left(0+\frac{1}{2}\right)\\
  x_0 &= \sqrt{\frac{\hbar}{m\omega}}\\
\end{align*}

We then find the probability of finding the particle in this region:

\begin{align*}
  P_{forb} &= \int_{\infty}^{-x_0} \psi_0^*\psi_0 dx + \int_{x_0}^{\infty} \psi_0^*\psi_0 dx\\
  &= \int_{\infty}^{-x_0} \sqrt{\frac{m\omega}{\pi\hbar}}e^{-m\omega x^2 / \hbar}dx
  + \int_{x_0}^{\infty} \sqrt{\frac{m\omega}{\pi\hbar}}e^{-m\omega x^2 / \hbar} dx\\
  &= \sqrt{\frac{m\omega}{\pi\hbar}} \left(\frac{-\hbar}{m\omega}\right)\left(
    \frac{1}{x}e^{-m\omega x^2 / \hbar} \Bigg|_{-\infty}^{-\sqrt{\frac{\hbar}{m\omega}}} +
    \frac{1}{x}e^{-m\omega x^2 / \hbar} \Bigg|_{\sqrt{\frac{\hbar}{m\omega}}}^{\infty}\right)\\
  &= \frac{2}{\sqrt{\pi}} = 0.41\\
\end{align*}

\section{9.9: QHO expectation values via integration, operator method}
In this problem we use both wave function integration and the operator method to find the
expectation values of various observables: x, p, $x^2$, $p^2$.

\textbf{a.) Ground state wave function calculations}
\begin{align*}
  <x> &= \int_{-\infty}^{\infty} \psi^*x\psi dx\\
  &= \int_{-\infty}^{\infty} \sqrt{\frac{m\omega}{\pi\hbar}}e^{-m\omega x^2 / \hbar}x dx = 0\\
\end{align*}
We don't need to perform the $<x>$ integral because it involves integrating an odd function over
an even domain, which always results in zero. This is to be expected, because the wave ground-state
wave function is symmetric across zero.\\

\begin{align*}
  <p> &= \int_{-\infty}^{\infty} \psi^* -i\hbar \frac{d}{dx}\psi dx\\
  &= \int_{-\infty}^{\infty}
  -i\sqrt{\frac{m\omega}{\pi\hbar}}(-m\omega x)e^{\frac{-m\omega x^2}{2\hbar}}dx = 0\\
\end{align*}
This integral is zero since we are again integrating an odd function over an even domain. This
makes sense because, since the spacial distribution of the wave function is equal on both sides
of the x-axis, the velocities (and hence momenta) should be equal as well.\\

\begin{align*}
  <x^2> &= \int_{-\infty}^{\infty} \psi^*x^2\psi dx\\
  &= \int_{-\infty}^{\infty} \sqrt{\frac{m\omega}{\pi\hbar}}e^{-m\omega x^2 / \hbar}x^2 dx\\
  &= 2\sqrt{\frac{m\omega}{\pi\hbar}}\int_{0}^{\infty} e^{-m\omega x^2 / \hbar}x^2 dx\\
  &= 2\sqrt{\frac{m\omega}{\pi\hbar}} *
  \frac{\sqrt{\pi}}{4(\frac{m\omega}{\hbar})^{3/2}} \hspace{2cm}\mbox{Wolfram}\\
  &= \frac{\hbar}{2m\omega}\\
\end{align*}
Here we exploit the fact that integrating an even function over an even domain is equivalent to
doubling the integral of the function over half the domain. The answer makes sense, since increasing
$m\omega$ corresponds to making the quadratic potential steeper, hence making the wave function
tighter.\\

\begin{align*}
  <p^2> &= \int_{-\infty}^{\infty} \psi^* \left(-\hbar^2 \frac{d^2}{dx^2}\right)\psi dx\\
  &= \sqrt{\frac{m\omega}{\pi\hbar}}\int_{-\infty}^{\infty} -\hbar^2e^{\frac{-m\omega x^2}{\hbar}}
  \left(\frac{m^2\omega^2x^2}{\hbar^2} - \frac{m\omega x}{\hbar}\right)dx\\
  &= -\hbar^2\sqrt{\frac{m\omega}{\pi\hbar}}\left(
  \int_{-\infty}^{\infty} e^{\frac{-m\omega x^2}{\hbar}}\frac{m^2\omega^2x^2}{\hbar^2}dx -
  \int_{-\infty}^{\infty} e^{\frac{-m\omega x^2}{\hbar}}\frac{m\omega}{\hbar}\right)dx\\
  &= -\hbar^2\sqrt{\frac{m\omega}{\pi\hbar}}
  \left(\frac{1}{4}\sqrt{\frac{m\pi\omega}{\hbar}} -
  \frac{1}{2}\sqrt{\frac{m\pi\omega}{\hbar}}\right) \hspace{2cm}\mbox{Wolfram}\\
  &= \frac{\hbar m\omega}{4}\\
\end{align*}
This answer is intuitive, since increasing mass as well as oscillatory speed should both increase
the expected momentum.

\textbf{b.) Any-state operator method calculations}
Here we define the position and momentum operators in terms of the raising and lowering operators:
\begin{align*}
  \hat{x} &= \sqrt{\frac{\hbar}{2m\omega}}\left(a + a^\dagger\right)\\
  \hat{p} &= i\sqrt{\frac{m\omega\hbar}{2}}\left(a^\dagger - a\right)\\
\end{align*}
We then solve for the various expectation values using these definitions:

\begin{align*}
  <x> &= \bra{n}x\ket{n}\\
  &= \sqrt{\frac{\hbar}{2m\omega}}\bra{n}\left(a + a^\dagger\right)\ket{n}\\
  &= \sqrt{\frac{\hbar}{2m\omega}}
  \left(\sqrt{n}\bra{n}\ket{n-1} + \sqrt{n+1}\bra{n}\ket{n+1}\right) = 0\\
\end{align*}
This answer indicates that all pure QHO wave functions are centered around zero. This makes
sense since the QHO wave functions are all made up of either even or odd functions. When squared,
as is done when finding the probability, the equation becomes entirely even, meaning it is
symmetric across the y-axis. From this we can see that all QHO wave functions have an expectation
value of zero for x-measurement.

\begin{align*}
  <p> &= \bra{n}\hat{p}\ket{n}\\
  &= i\sqrt{\frac{m\hbar\omega}{2}}\bra{n}\left(a^\dagger-a\right)\ket{n}\\
  &= i\sqrt{\frac{m\hbar\omega}{2}}
  \left(\sqrt{n+1}\bra{n}\ket{n+1} - \sqrt{n}\bra{n}\ket{n-1}\right) = 0\\
\end{align*}
This is an interesting result. It is intuitive that any wave function should have an expectation of
zero for momentum, else the wave function would move through space.

How does this reconcile w/ fact that p involves a spacial derivative?

\end{document}

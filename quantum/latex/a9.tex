\documentclass[10pt]{article} % Font size - 10pt, 11pt or 12pt

\usepackage[hmargin=1.25cm, vmargin=1.5cm]{geometry} % Document margins

\usepackage{marvosym} % Required for symbols in the colored box

\usepackage[usenames,dvipsnames]{xcolor} % Allows the definition of hex colors

% Fonts and tweaks for XeLaTeX
\usepackage{fontspec,xltxtra,xunicode}
\defaultfontfeatures{Mapping=tex-text}
%\setmonofont[Scale=MatchLowercase]{Andale Mono}

% Colors for links, text and headings
\usepackage{hyperref}
\definecolor{linkcolor}{HTML}{506266} % Blue-gray color for links
\definecolor{shade}{HTML}{F5DD9D} % Peach color for the contact information box
\definecolor{text1}{HTML}{2b2b2b} % Main document font color, off-black
\definecolor{headings}{HTML}{701112} % Dark red color for headings
% Other color palettes: shade=B9D7D9 and linkcolor=A40000; shade=D4D7FE and linkcolor=FF0080

\hypersetup{colorlinks,breaklinks, urlcolor=linkcolor, linkcolor=linkcolor} % Set up links and colors

\usepackage{fancyhdr}
\usepackage{amsmath}
\usepackage{braket}
\usepackage{amssymb}
\pagestyle{fancy}
\fancyhf{}
% Headers and footers can be added with the \lhead{} \rhead{} \lfoot{} \rfoot{} commands
% Example footer:
%\rfoot{\color{headings} {\sffamily Last update: \today}. Typeset with Xe\LaTeX}

\renewcommand{\headrulewidth}{0pt} % Get rid of the default rule in the header

\usepackage{titlesec} % Allows creating custom \section's

% Format of the section titles
\titleformat{\section}{\color{headings}
\scshape\Large\raggedright}{}{0em}{}[\color{black}\titlerule]


\title{Quantum Mechanics Assignment Nine}
\author{Elliott Capek}
\titlespacing{\section}{0pt}{0pt}{5pt} % Spacing around titles

\begin{document}

\maketitle{}

\section{Problem One: Gaussian Perturbation of Infinite Square Well}
Here we deal with an object in an infinite square well subject to a perturbation $V_0x^2e^{-t\tau}$.
The particle begins at $t=0$ in the ground state $E_1$. We wish to find the probability of a
transition to $E_2$ as a function of time $t$. To do this, we first calculate the perturbed
state coefficient $c_2(t)$:

\begin{align*}
  c_2(t) &= \frac{1}{i\hbar}\int_0^t\bra{2}H'\ket{1}e^{i(E_2-E_1)t'/\hbar}dt'\\
  &= \frac{2}{i\hbar L}\int_0^t
  \left(\int_0^L\sin\left(\frac{2\pi x}{L}\right)
  V_0x^2e^{-t/\tau}\sin\left(\frac{\pi x}{L}\right)dx\right)
  e^{\frac{i\hbar\pi^2(2^2-1^2)}{2mL^2}t}dt'\\
  &= \frac{2V_0}{i\hbar L}\int_0^t
  \left(\int_0^L\sin\left(\frac{2\pi x}{L}\right)
  x^2\sin\left(\frac{\pi x}{L}\right)dx\right)
  e^{-t/\tau}e^{\frac{i3\hbar\pi^2}{2mL^2}t}dt'\\
  &= \frac{2V_0}{i\hbar L}\int_0^t\frac{-8L^3}{\pi^2}e^{-t/\tau}e^{\frac{i3\hbar\pi^2}{2mL^2}t}dt'\\
  &= \frac{-16V_0L^2}{i\hbar\pi^2}
  \int_0^te^{\left(\frac{i3\hbar\pi^2}{2mL^2} - \frac{1}{\tau}\right)t}dt'\\
  &= i\frac{16V_0L^2}{\hbar\pi^2}\left(\frac{i3\hbar\pi^2}{2mL^2} - \frac{1}{\tau}\right)^{-1}
  e^{\frac{i3\hbar\pi^2}{2mL^2}t}e^{-\frac{t}{\tau}}\\
\end{align*}

We then find the probability of measurement:

\begin{align*}
  P_{1\rightarrow2} &= |c_2(t)|^2\\
  &= \left(\frac{16V_0L^2}{\hbar\pi^2}\right)^2
  \left(\frac{i3\hbar\pi^2}{2mL^2} - \frac{1}{\tau}\right)^{-2}
  e^{i\frac{6\hbar\pi^2}{2mL^2}t}e^{-2\frac{t}{\tau}}\\
\end{align*}

Because

\begin{align*}
  \lim_{t\rightarrow\infty} P_{1\rightarrow2}(t) = 0\\
\end{align*}

we can say that it becomes less and less likely for the system to be measured with energy
$E_2$ as time goes on.\\

This was a cool introduction to applying time-dependent perturbation theory equations to solving
problems. The answer is intuitive. The perturbation decays exponentially, so at large times it
should behave exactly like the unperturbed infinite square well and stay in the ground state with
100\% probability.\\

\section{Problem Two: Cubed Perturbation}
Here we deal with a time-dependent perturbation $H'(x,t) = Ax^3e^{-\gamma t}$
of a harmonic oscillator. We wish to find the probability of transitioning from the ground
state after a long time.\\

\textbf{a.)} First we find the probability of transitioning, using energies given by the unperturbed
harmonic oscillator:

\begin{align*}
  P_{i\rightarrow f}(t)
  &= \frac{1}{\hbar^2}\left|\int_0^t\bra{f}H'(t')\ket{0}e^{i(E_f-E_0)t/\hbar}dt'\right|^2\\
  &= \frac{1}{\hbar^2}\left|\int_0^t
  \left(\int_{-\infty}^\infty Ax^3e^{-\gamma t}dx\right)e^{if\omega t}dt'\right|^2\\
  &= \frac{A}{\hbar^2}\left|\int_0^t\bra{f}x^3\ket{0}e^{-\gamma t}e^{if\omega t}dt'\right|^2\\
  &= \frac{A^2}{\hbar^2}\left|\left(\frac{\hbar}{2m\omega}\right)^{3/2}
  \int_0^t\bra{f}\left(a^\dagger+a\right)^3\ket{0}e^{-\gamma t}e^{if\omega t}dt'\right|^2\\
  &= \frac{A^2\hbar}{8m^3\omega^3}
  \left|\int_0^t\bra{f}
  \left(a^\dagger a^\dagger a^\dagger + a^\dagger a^\dagger a + a^\dagger aa^\dagger
  + aa^\dagger a^\dagger + aaa^\dagger + aa^\dagger a + a^\dagger aa + aaa\right)
  \ket{0}e^{-\gamma t}e^{if\omega t}dt'\right|^2\\
\end{align*}

From this we can see that the only possible values of $\ket{f}$ are $\ket{1}$ and $\ket{3}$. We
solve for their transition probabilities:

\begin{align*}
  P_{0\rightarrow 1}(t) &= \frac{A^2\hbar}{8m^3\omega^3}
  \left|\int_0^t\bra{1}\left(a^\dagger aa^\dagger + aa^\dagger a^\dagger\right)
  \ket{0}e^{-\gamma t}e^{if\omega t}dt'\right|^2\\
  &= \frac{A^2\hbar}{8m^3\omega^3}\left|\int_0^t3e^{-\gamma t}e^{if\omega t}dt'\right|^2\\
\end{align*}

\begin{align*}
  P_{0\rightarrow 3}(t) &= \frac{A^2\hbar}{8m^3\omega^3}
  \left|\int_0^t\bra{3}a^\dagger a^\dagger a^\dagger\ket{0}e^{-\gamma t}e^{if\omega t}dt'\right|^2\\
  &= \frac{A^2\hbar}{8m^3\omega^3}\left|\int_0^t\sqrt{6}e^{-\gamma t}e^{if\omega t}dt'\right|^2\\
\end{align*}

\end{document}

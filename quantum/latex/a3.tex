\documentclass[10pt]{article} % Font size - 10pt, 11pt or 12pt

\usepackage[hmargin=1.25cm, vmargin=1.5cm]{geometry} % Document margins

\usepackage{marvosym} % Required for symbols in the colored box

\usepackage[usenames,dvipsnames]{xcolor} % Allows the definition of hex colors

% Fonts and tweaks for XeLaTeX
\usepackage{fontspec,xltxtra,xunicode}
\defaultfontfeatures{Mapping=tex-text}
%\setmonofont[Scale=MatchLowercase]{Andale Mono}

% Colors for links, text and headings
\usepackage{hyperref}
\definecolor{linkcolor}{HTML}{506266} % Blue-gray color for links
\definecolor{shade}{HTML}{F5DD9D} % Peach color for the contact information box
\definecolor{text1}{HTML}{2b2b2b} % Main document font color, off-black
\definecolor{headings}{HTML}{701112} % Dark red color for headings
% Other color palettes: shade=B9D7D9 and linkcolor=A40000; shade=D4D7FE and linkcolor=FF0080

\hypersetup{colorlinks,breaklinks, urlcolor=linkcolor, linkcolor=linkcolor} % Set up links and colors

\usepackage{fancyhdr}
\usepackage{amsmath}
\usepackage{physics}
\usepackage{amssymb}
\pagestyle{fancy}
\fancyhf{}
% Headers and footers can be added with the \lhead{} \rhead{} \lfoot{} \rfoot{} commands
% Example footer:
%\rfoot{\color{headings} {\sffamily Last update: \today}. Typeset with Xe\LaTeX}

\renewcommand{\headrulewidth}{0pt} % Get rid of the default rule in the header

\usepackage{titlesec} % Allows creating custom \section's

% Format of the section titles
\titleformat{\section}{\color{headings}
\scshape\Large\raggedright}{}{0em}{}[\color{black}\titlerule]

\title{Quantum Mechanics Assignment Three}
\author{Elliott Capek}
\titlespacing{\section}{0pt}{0pt}{5pt} % Spacing around titles

\begin{document}

\maketitle{}

\section{9.16: Time evolving QHO state}
We are given the probabilities of measuring certain energies for a QHO state $\ket{\psi}$. We
are asked to find the energy basis representation for this state and compute the expectations for
momentum and energy.\\

\textbf{Finding energy representation}
\begin{align*}
  |\bra{1}\ket{\psi}|^2 = 0.36 \rightarrow |c_1|^2 &= 0.36 \rightarrow c_1 = e^{ai}0.6\\
  |\bra{2}\ket{\psi}|^2 = 0.64 \rightarrow |c_1|^2 &= 0.64 \rightarrow c_1 = e^{bi}0.8\\
\end{align*}

\textbf{Time evolution}
\begin{align*}
  \ket{\psi} &= 0.6e^{ai}e^{-iE_1t/\hbar}\ket{1} + 0.8e^{bi}e^{-iE_2t/\hbar}\ket{2}\\
  &= 0.6e^{ai}e^{-3i\omega t/2}\ket{1} + 0.8e^{bi}e^{-5i\omega t/2}\ket{2}\\
\end{align*}

We are told that the expectation value for position is minimized at $t=0$. We can use this to find
the relative phases of the wave function coefficients:\\

\textbf{Calculating relative phase}
\begin{align*}
  <x> &= \bra{\psi}\left(\sqrt{\frac{\hbar}{2m\omega}}
  \left(a + a^\dagger\right)\right)\ket{\psi}\\
  &= \sqrt{\frac{\hbar}{2m\omega}}\left(\bra{\psi}a\ket{\psi}
  + \bra{\psi}a^\dagger\ket{\psi}\right)\\
  &= \sqrt{\frac{\hbar}{2m\omega}}\left(\sqrt{2}c_1^*(t)c_2(t)\bra{1}\ket{1}
  + \sqrt{2}c_1(t)c_2^*(t)\bra{2}\ket{2}\right)\\
  &= \sqrt{\frac{\hbar}{m\omega}}\left(c_1^*(t)c_2(t) + c_1(t)c_2^*(t)\right)\\
  &= \sqrt{\frac{\hbar}{m\omega}}\left(0.6*0.8*e^{i(b-a)} + 0.6*0.8e^{i(a-b)}\right)
  \hspace{2cm}\mbox{Time-dependent terms are 1 when t=0}\\
  &= 2*0.6*0.8*\sqrt{\frac{\hbar}{m\omega}}\left(\cos(b-a) + i\sin(b-a)\right)\\
\end{align*}

To minimize this expectation value, $a-b = b-a = \pi$. Thus we represent our system as follows:

\begin{align*}
  \psi(t) &= 0.6e^{i\pi}e^{-3i\omega t/2}\ket{1} + 0.8e^{-5i\omega t/2}\ket{2}\\
  &= -0.6e^{-3i\omega t/2}\ket{1} + 0.8e^{-5i\omega t/2}\ket{2}\\
\end{align*}

We then calculate expectation values for the system:\\

\textbf{Momentum}
\begin{align*}
  <p> &= -i\sqrt{\frac{\hbar}{2m\omega}}\bra{\psi}\left(a - a^\dagger)\right)\ket{\psi}\\
  &= -i\sqrt{\frac{\hbar m\omega}{2}}\left(\bra{\psi}a\ket{\psi}
  - \bra{\psi}a^\dagger\ket{\psi}\right)\\
  &= -i\sqrt{\frac{\hbar m\omega}{2}}\left(\sqrt{2}c_1^*(t)c_2(t)\bra{1}\ket{1}
  - \sqrt{2}c_1(t)c_2^*(t)\bra{2}\ket{2}\right)\\
  &= -i*0.6*0.8*\sqrt{\hbar m\omega}\left(e^{i\omega t} - e^{-i\omega t}\right)\\
  &= -0.48\sqrt{\hbar m\omega}\sin(\omega t)\\
\end{align*}

An interesting result! First, the units look good. Second, the magnitude of the expected momentum
integrated over time is zero, meaning that the particle won't travel anywhere, which is good. The
trigonometric function suggests that the particle's momentum oscillates sinusoidally at the
classical frequency, just as we expect for a classical oscillator.\\

\textbf{Energy}
\begin{align*}
  <H> &= \bra{1}(\frac{3}{2}\hbar\omega)\ket{1} + \bra{2}(\frac{5}{2}\hbar\omega)\ket{2}\\
  &= 0.36 * \frac{3}{2}\hbar\omega + 0.64 * \frac{5}{2}\hbar\omega\\
  &= 2.08\hbar\omega\\
\end{align*}

This is an expected result closest to the most-represented eigenstate in the system. It makes sense
that there is no time-dependence, since that would mean energy would be entering and leaving the
system, and energy is conserved for QM expectation values.

\section{9.17: More Expectation Values}
Similar to the above problem we are given a wavefunction and asked to calculate expectation values
from it. We begin by figuring out its normalization constant:

\textbf{Normalization constant}
\begin{align*}
  \psi(x,0) &= A\left[\phi_0(x) + 2\phi_1(x) + 2\phi_2(x)\right]\\
  \braket{\psi} &= 1
  A^2\left(1 + 4 + 4\right) &= 1 \rightarrow A = \frac{1}{3}\\
\end{align*}

Therefore our full time-evolved wavefunction is:

\begin{align*}
  \psi(x,t) &= \frac{1}{3}e^{\frac{-i\omega t}{2}}\phi_0(x)
  + \frac{2}{3}e^{\frac{-3i\omega t}{2}}\phi_1(x)
  + \frac{2}{3}e^{\frac{-5i\omega t}{2}}\phi_2(x)\\
\end{align*}

\textbf{Energy measurements}
By inspection we see that the first three energy levels of the QHO spectrum are represented in the
wave function, so these are the only ones we calculate the energy for.

\begin{align*}
  \mathbb{P}(\frac{1}{2}\omega) &= |\bra{0}\ket{\psi}|^2 = |c_0|^2 = \frac{1}{9}\\
  \mathbb{P}(\frac{3}{2}\omega) &= |\bra{1}\ket{\psi}|^2 = |c_1|^2 = \frac{4}{9}\\
  \mathbb{P}(\frac{5}{2}\omega) &= |\bra{2}\ket{\psi}|^2 = |c_2|^2 = \frac{4}{9}\\
\end{align*}

We can see the probabilities sum up to 1, as desired.\\

\textbf{Momentum expectation value}
\begin{align*}
  <p> &= -i\sqrt{\frac{\hbar}{2m\omega}}\bra{\psi}\left(a - a^\dagger)\right)\ket{\psi}\\
  &= -i\sqrt{\frac{\hbar m\omega}{2}}\left(\bra{\psi}a\ket{\psi}
  - \bra{\psi}a^\dagger\ket{\psi}\right)\\
  &= -i\sqrt{\frac{\hbar m\omega}{2}}
  \Bigg(
  \sqrt{1}\frac{2}{9}e^{\frac{i\omega t}{2}}e^{\frac{-3i\omega t}{2}}\bra{0}\ket{0}
  + \sqrt{2}\frac{4}{9}e^{\frac{3i\omega t}{2}}e^{\frac{-5i\omega t}{2}}\bra{1}\ket{1}\\
  &-\sqrt{1}\frac{2}{9}e^{\frac{3i\omega t}{2}}e^{\frac{-i\omega t}{2}}\bra{1}\ket{1}
  - \sqrt{2}\frac{4}{9}e^{\frac{5i\omega t}{2}}e^{\frac{-3i\omega t}{2}}\bra{2}\ket{2}
  \Bigg)\\
  &= -i\sqrt{\frac{\hbar m\omega}{2}}
  \left(\frac{2}{9}e^{-i\omega t} + \sqrt{2}\frac{4}{9}e^{-i\omega t}
  - \frac{2}{9}e^{i\omega t} - \sqrt{2}\frac{4}{9}e^{i\omega t}\right)\\
  &= \frac{-2i}{9}\sqrt{\frac{\hbar m\omega}{2}}
  \Bigg(
  \cos(\omega t) - i\sin(\omega t)
  + \sqrt{8}\cos(\omega t) - i\sqrt{8}\sin(\omega t)\\
  &- \cos(\omega t) - i\sin(\omega t)
  - \sqrt{8}\cos(\omega t) - i\sqrt{8}\sin(\omega t)
  \Bigg)\\
  &= \frac{4i}{9}\sqrt{\frac{\hbar m\omega}{2}}\left(1 + \sqrt{8}\right)\sin(\omega t)\\
\end{align*}

This is another cool result. It is messier than the previous momentum expectation value due to
a larger number of states, but still has proper units and displays the sinusoidal oscillation
typical of systems with adjacent energies. I wonder why systems without adjacent energies have a
zero expectation value for momentum. What is special about adjacent states that makes them
``couple''?\\

\textbf{Energy expectation}
\begin{align*}
  <H> &= \frac{1}{9}\frac{1}{2}\hbar\omega + \frac{4}{9}\hbar{3}{2}\hbar\omega
  + \frac{4}{9}\frac{5}{2}\hbar\omega\\
  &= \frac{33}{18}\hbar\omega = 1.83\hbar\omega\\
\end{align*}

A cool and expected result.\\

\textbf{Energy standard deviation}
We must first find the value of $<H^2>$:
\begin{align*}
  <H^2> &= \frac{1}{9}\frac{1}{4}\hbar^2\omega^2 + \frac{4}{9}\hbar{9}{4}\hbar^2\omega^2
  + \frac{4}{9}\frac{25}{4}\hbar^2\omega^2\\
  &= \frac{137}{36}\hbar^2\omega^2 = 3.8\hbar^2\omega^2\\
\end{align*}

We then calculate the standard deviation:
\begin{align*}
  \Delta E &= \sqrt{<H^2> - <H>^2} = \sqrt{(3.8 - 3.34)\hbar^2\omega^2}\\
  &= \sqrt{0.46\hbar^2\omega^2} \approx \frac{2}{3}\hbar\omega\\
\end{align*}

Standard deviations in QHO systems are interesting because all the energies are evenly spaced. This
value represents how spread-out the states are. It is expected that because the states are close
together for this system, the standard deviation is small.
\end{document}

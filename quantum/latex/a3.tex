\documentclass[10pt]{article} % Font size - 10pt, 11pt or 12pt

\usepackage[hmargin=1.25cm, vmargin=1.5cm]{geometry} % Document margins

\usepackage{marvosym} % Required for symbols in the colored box

\usepackage[usenames,dvipsnames]{xcolor} % Allows the definition of hex colors

% Fonts and tweaks for XeLaTeX
\usepackage{fontspec,xltxtra,xunicode}
\defaultfontfeatures{Mapping=tex-text}
%\setmonofont[Scale=MatchLowercase]{Andale Mono}

% Colors for links, text and headings
\usepackage{hyperref}
\definecolor{linkcolor}{HTML}{506266} % Blue-gray color for links
\definecolor{shade}{HTML}{F5DD9D} % Peach color for the contact information box
\definecolor{text1}{HTML}{2b2b2b} % Main document font color, off-black
\definecolor{headings}{HTML}{701112} % Dark red color for headings
% Other color palettes: shade=B9D7D9 and linkcolor=A40000; shade=D4D7FE and linkcolor=FF0080

\hypersetup{colorlinks,breaklinks, urlcolor=linkcolor, linkcolor=linkcolor} % Set up links and colors

\usepackage{fancyhdr}
\usepackage{amsmath}
\usepackage{physics}
\usepackage{amssymb}
\pagestyle{fancy}
\fancyhf{}
% Headers and footers can be added with the \lhead{} \rhead{} \lfoot{} \rfoot{} commands
% Example footer:
%\rfoot{\color{headings} {\sffamily Last update: \today}. Typeset with Xe\LaTeX}

\renewcommand{\headrulewidth}{0pt} % Get rid of the default rule in the header

\usepackage{titlesec} % Allows creating custom \section's

% Format of the section titles
\titleformat{\section}{\color{headings}
\scshape\Large\raggedright}{}{0em}{}[\color{black}\titlerule]

\title{Quantum Mechanics Assignment Three}
\author{Elliott Capek}
\titlespacing{\section}{0pt}{0pt}{5pt} % Spacing around titles

\begin{document}

\maketitle{}

\section{9.16: Time evolving QHO state}
We are given the probabilities of measuring certain energies for a QHO state $\ket{\psi}$. We
are asked to find the energy basis representation for this state and compute the expectations for
momentum and energy.\\

\textbf{Finding energy representation}
\begin{align*}
  |\bra{1}\ket{\psi}|^2 = 0.36 \rightarrow |c_1|^2 &= 0.36 \rightarrow c_1 = e^{ai}0.6\\
  |\bra{2}\ket{\psi}|^2 = 0.64 \rightarrow |c_1|^2 &= 0.64 \rightarrow c_1 = e^{bi}0.8\\
\end{align*}

\textbf{Time evolution}
\begin{align*}
  \ket{\psi} &= 0.6e^{ai}e^{-iE_1t/\hbar}\ket{1} + 0.8e^{bi}e^{-iE_2t/\hbar}\ket{2}\\
  &= 0.6e^{ai}e^{-3i\omega t/2}\ket{1} + 0.8e^{bi}e^{-5i\omega t/2}\ket{2}\\
\end{align*}

We are told that the expectation value for position is minimized at $t=0$. We can use this to find
the relative phases of the wave function coefficients:\\

\textbf{Calculating relative phase}
\begin{align*}
  <x> &= \bra{\psi}\left(\sqrt{\frac{\hbar}{2m\omega}}
  \left(a + a^\dagger\right)\right)\ket{\psi}\\
  &= \sqrt{\frac{\hbar}{2m\omega}}\left(\bra{\psi}a\ket{\psi}
  + \bra{\psi}a^\dagger\ket{\psi}\right)\\
  &= \sqrt{\frac{\hbar}{2m\omega}}\left(\sqrt{2}c_1^*(t)c_2(t)\bra{1}\ket{1}
  + \sqrt{2}c_1(t)c_2^*(t)\bra{2}\ket{2}\right)\\
  &= \sqrt{\frac{\hbar}{m\omega}}\left(c_1^*(t)c_2(t) + c_1(t)c_2^*(t)\right)\\
  &= \sqrt{\frac{\hbar}{m\omega}}\left(0.6*0.8*e^{i(b-a)} + 0.6*0.8e^{i(a-b)}\right)
  \hspace{2cm}\mbox{Time-dependent terms are 1 when t=0}\\
  &= 2*0.6*0.8*\sqrt{\frac{\hbar}{m\omega}}\left(\cos(b-a) + i\sin(b-a)\right)\\
\end{align*}

To minimize this expectation value, $a-b = b-a = \pi$. Thus we represent our system as follows:

\begin{align*}
  \psi(t) &= 0.6e^{i\pi}e^{-3i\omega t/2}\ket{1} + 0.8e^{-5i\omega t/2}\ket{2}\\
  &= -0.6e^{-3i\omega t/2}\ket{1} + 0.8e^{-5i\omega t/2}\ket{2}\\
\end{align*}

We then calculate expectation values for the system:\\

\textbf{Momentum}
\begin{align*}
  <p> &= -i\sqrt{\frac{\hbar}{2m\omega}}\bra{\psi}\left(a - a^\dagger)\right)\ket{\psi}\\
  &= -i\sqrt{\frac{\hbar m\omega}{2}}\left(\bra{\psi}a\ket{\psi}
  - \bra{\psi}a^\dagger\ket{\psi}\right)\\
  &= -i\sqrt{\frac{\hbar m\omega}{2}}\left(\sqrt{2}c_1^*(t)c_2(t)\bra{1}\ket{1}
  - \sqrt{2}c_1(t)c_2^*(t)\bra{2}\ket{2}\right)\\
  &= -i*0.6*0.8*\sqrt{\hbar m\omega}\left(e^{i\omega t} - e^{-i\omega t}\right)\\
  &= -0.48\sqrt{\hbar m\omega}\sin(\omega t)\\
\end{align*}

An interesting result! First, the units look good. Second, the magnitude of the expected momentum
integrated over time is zero, meaning that the particle won't travel anywhere, which is good. The
trigonometric function suggests that the particle's momentum oscillates sinusoidally at the
classical frequency, just as we expect for a classical oscillator.\\

\textbf{Energy}
\begin{align*}
  <H> &= \bra{1}(\frac{3}{2}\hbar\omega)\ket{1} + \bra{2}(\frac{5}{2}\hbar\omega)\ket{2}\\
  &= 0.36 * \frac{3}{2}\hbar\omega + 0.64 * \frac{5}{2}\hbar\omega\\
  &= 2.14\hbar\omega\\
\end{align*}

This is an expected result closest to the most-represented eigenstate in the system. It makes sense
that there is no time-dependence, since that would mean energy would be entering and leaving the
system, and energy is conserved for QM expectation values.

\section{9.17: More Expectation Values}
Similar to the above problem we are given a wavefunction and asked to calculate expectation values
from it. We begin by figuring out its normalization constant:

\textbf{Normalization constant}
\begin{align*}
  \psi(x,0) &= A\left[\phi_0(x) + 2\phi_1(x) + 2\phi_2(x)\right]\\
  \braket{\psi} &= 1\\
  A^2\left(1 + 4 + 4\right) &= 1 \rightarrow A = \frac{1}{3}\\
\end{align*}

Therefore our full time-evolved wavefunction is:

\begin{align*}
  \psi(x,t) &= \frac{1}{3}e^{\frac{-i\omega t}{2}}\phi_0(x)
  + \frac{2}{3}e^{\frac{-3i\omega t}{2}}\phi_1(x)
  + \frac{2}{3}e^{\frac{-5i\omega t}{2}}\phi_2(x)\\
\end{align*}

\textbf{Energy measurements}
By inspection we see that the first three energy levels of the QHO spectrum are represented in the
wave function, so these are the only ones we calculate the energy for.

\begin{align*}
  \mathbb{P}(\frac{1}{2}\omega) &= |\bra{0}\ket{\psi}|^2 = |c_0|^2 = \frac{1}{9}\\
  \mathbb{P}(\frac{3}{2}\omega) &= |\bra{1}\ket{\psi}|^2 = |c_1|^2 = \frac{4}{9}\\
  \mathbb{P}(\frac{5}{2}\omega) &= |\bra{2}\ket{\psi}|^2 = |c_2|^2 = \frac{4}{9}\\
\end{align*}

We can see the probabilities sum up to 1, as desired.\\

\textbf{Momentum expectation value}
\begin{align*}
  <p> &= -i\sqrt{\frac{\hbar}{2m\omega}}\bra{\psi}\left(a - a^\dagger)\right)\ket{\psi}\\
  &= -i\sqrt{\frac{\hbar m\omega}{2}}\left(\bra{\psi}a\ket{\psi}
  - \bra{\psi}a^\dagger\ket{\psi}\right)\\
  &= -i\sqrt{\frac{\hbar m\omega}{2}}
  \Bigg(
  \sqrt{1}\frac{2}{9}e^{\frac{i\omega t}{2}}e^{\frac{-3i\omega t}{2}}\bra{0}\ket{0}
  + \sqrt{2}\frac{4}{9}e^{\frac{3i\omega t}{2}}e^{\frac{-5i\omega t}{2}}\bra{1}\ket{1}\\
  &-\sqrt{1}\frac{2}{9}e^{\frac{3i\omega t}{2}}e^{\frac{-i\omega t}{2}}\bra{1}\ket{1}
  - \sqrt{2}\frac{4}{9}e^{\frac{5i\omega t}{2}}e^{\frac{-3i\omega t}{2}}\bra{2}\ket{2}
  \Bigg)\\
  &= -i\sqrt{\frac{\hbar m\omega}{2}}
  \left(\frac{2}{9}e^{-i\omega t} + \sqrt{2}\frac{4}{9}e^{-i\omega t}
  - \frac{2}{9}e^{i\omega t} - \sqrt{2}\frac{4}{9}e^{i\omega t}\right)\\
  &= \frac{-2i}{9}\sqrt{\frac{\hbar m\omega}{2}}
  \Bigg(
  \cos(\omega t) - i\sin(\omega t)
  + \sqrt{8}\cos(\omega t) - i\sqrt{8}\sin(\omega t)\\
  &- \cos(\omega t) - i\sin(\omega t)
  - \sqrt{8}\cos(\omega t) - i\sqrt{8}\sin(\omega t)
  \Bigg)\\
  &= \frac{-4}{9}\sqrt{\frac{\hbar m\omega}{2}}\left(1 + \sqrt{8}\right)\sin(\omega t)\\
\end{align*}

This is another cool result. It is messier than the previous momentum expectation value due to
a larger number of states, but still has proper units and displays the sinusoidal oscillation
typical of systems with adjacent energies. I wonder why systems without adjacent energies have a
zero expectation value for momentum. What is special about adjacent states that makes them
``couple''?\\

\textbf{Energy expectation}
\begin{align*}
  <H> &= \frac{1}{9}\frac{1}{2}\hbar\omega + \frac{4}{9}\hbar{3}{2}\hbar\omega
  + \frac{4}{9}\frac{5}{2}\hbar\omega\\
  &= \frac{33}{18}\hbar\omega = 1.83\hbar\omega\\
\end{align*}

A cool and expected result.\\

\textbf{Energy standard deviation}
We must first find the value of $<H^2>$:
\begin{align*}
  <H^2> &= \frac{1}{9}\frac{1}{4}\hbar^2\omega^2 + \frac{4}{9}\hbar{9}{4}\hbar^2\omega^2
  + \frac{4}{9}\frac{25}{4}\hbar^2\omega^2\\
  &= \frac{137}{36}\hbar^2\omega^2 = 3.8\hbar^2\omega^2\\
\end{align*}

We then calculate the standard deviation:
\begin{align*}
  \Delta E &= \sqrt{<H^2> - <H>^2} = \sqrt{(3.8 - 3.34)\hbar^2\omega^2}\\
  &= \sqrt{0.46\hbar^2\omega^2} \approx \frac{2}{3}\hbar\omega\\
\end{align*}

Standard deviations in QHO systems are interesting because all the energies are evenly spaced. This
value represents how spread-out the states are. It is expected that because the states are close
together for this system, the standard deviation is small.

\section{9.22: Fictional System}
Here we are given the energy spectrum $E_n = n^3\hbar\omega$ and asked to give some information
about the system: its Hamiltonian and eigenstates in energy representation (matrix form)
and the matrix representation of the operator $A\ket{n} = 3n^2\ket{n+2}$.\\

Because any operator in its own basis is diagonal and because any eigenvector in its own basis is
a unit vector, we represent the Hamiltonian and eigenvectors as so:

\begin{align*}
  H \doteq
  \begin{pmatrix}
    \hbar\omega & 0 & 0 & ...\\
    0 & 8\hbar\omega & 0 & ...\\
    0 & 0 & 27\hbar\omega & ...\\
    ... & ... & ... & ...\\
  \end{pmatrix}
\end{align*}

\begin{align*}
  \ket{1} \doteq
  \begin{pmatrix}
    1\\
    0\\
    0\\
    ...\\
  \end{pmatrix}
  \hspace{1cm}
  \ket{2} \doteq
  \begin{pmatrix}
    0\\
    1\\
    0\\
    ...\\
  \end{pmatrix}
  \hspace{1cm}
  \ket{3} \doteq
  \begin{pmatrix}
    0\\
    0\\
    1\\
    ...\\
  \end{pmatrix}
\end{align*}

To find the matrix representation of A we exploit the fact that in the energy basis eigenstates
are represented by vectors with a single unit entry. Thus the first column of our matrix will only
interact with the n=1 vector, the second column with the n=2 vector, etc. We also know that the
result of our matrix-vector multiplication should produce a vector two energies higher than the
input state. Thus it is easy to see that we will construct a matrix with elements two spaces below
diagonal with values of $3n^2$, where n is the current column.

\begin{align*}
  A \doteq
  \begin{pmatrix}
    0 & 0 & 0 & 0 & 0 & ...  \\
    0 & 0 & 0 & 0 & 0 & ...  \\
    3 & 0 & 0 & 0 & 0 & ...  \\
    0 & 12 & 0 & 0 & 0 & ... \\
    0 & 0 & 27 & 0 & 0 & ... \\
    0 & 0 & 0 & 48 & 0 & ... \\
    ... & ... & ... & ... & ... & ... \\
  \end{pmatrix}
\end{align*}

This was a cool problem that shows that it is very easy to construct a Hamiltonian for a given
quantum system in matrix representation (not necessarily position representation). It is
important to note that operator A is not an actual physical operator, since it has no eigenvalue
equation.\\

\clearpage

\section{Paper}
The paper \textit{Quantum ground state and single-phonon control of a mechanical resonator} is a
demonstration that a macroscopic, harmonic-oscillating ``quantum drum'' can be cooled enough to
display behavior normally only predicted for microscopic systems. This is interesting because it
is an easy validation of quantum mechanics. The project entails cooling a resonator until its
heat energy is much smaller than that predicted for its oscillation energy, ensuring that
energy measurements are only of the oscillation. The energy is then measured, and it is shown that
only discrete values are taken on, not a continuum, as predicted by QM.\\

It is interesting to me that the microscopic details of the system need not be taken into account.
It seems like the individual wave functions of each atom in the oscillator can just be summed
into one giant wave function for the oscillator. Maybe that simplification can only be done at
very cold temperatures?\\

\section{10.1: Diagonalize a perturbed Hamiltonian}
We begin with the perturbed Hamiltonian in the $S_z$ basis and convert it into its own perturbed
basis. We then show that the eigenvalues in this basis are identical to those found by redefining
the direction of Z, aka $E_n = \pm \frac{\hbar}{2} \sqrt{(\omega_0 + \omega_1)^2 + \omega_2^2}$.

We begin with the perturbed matrix and use the fact that the eigenvalues of a matrix will be the
diagonal elements in the diagonalized form:\\

\begin{align*}
  H &\doteq
  \pm\frac{\hbar}{2}
  \begin{pmatrix}
    \omega_0 + \omega_1 & \omega_2\\
    \omega_2 & -\omega_0 - \omega_1\\
  \end{pmatrix}\\
\end{align*}

\begin{align*}
  \det\left(H - \lambda I\right) &= 0\\
  \left(\omega_0 + \omega_1 - \lambda\right)
  \left(-\omega_0 - \omega_1 - \lambda\right) - \omega_2^2 &= 0\\
  \lambda^2 - \left(\omega_0 + \omega_1\right)^2 - \omega_2^2 &= 0\\
  \lambda = \pm\sqrt{\left(\omega_0 + \omega_1\right)^2 + \omega_2^2}\\
\end{align*}

We then use these eigenvalues to construct a diagonal matrix. We know that the basis for the
perturbed Hamiltonian will be very similar to the unperturbed Hamiltonian, so we arrange our
perturbed Hamiltonian with the top left element positive and bottom right negative:

\begin{align*}
  H_p &\doteq
  \pm\frac{\hbar}{2}
  \begin{pmatrix}
    \sqrt{\left(\omega_0 + \omega_1\right)^2 + \omega_2^2} & 0\\
    0 & -\sqrt{\left(\omega_0 + \omega_1\right)^2 + \omega_2^2}\\
  \end{pmatrix}\\
\end{align*}

We have already shown that the eigenvalues agree with those in McIntyre (10.9). What we have done in
this problem is taken a matrix in the unperturbed energy basis and transformed it into its own
perturbed basis. We know this is a new basis because the perturbed Hamiltonian in the original
basis was not diagonal.\\

It is a good confirmation that the method used in McIntyre to find the eigenvalues of the perturbed
system, redefining the direction of Z to point in the direction of the new B-field, yielded the same
result as creating a new basis for the perturbed Hamiltonian.\\

\end{document}

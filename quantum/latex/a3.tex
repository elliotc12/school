\documentclass[10pt]{article} % Font size - 10pt, 11pt or 12pt

\usepackage[hmargin=1.25cm, vmargin=1.5cm]{geometry} % Document margins

\usepackage{marvosym} % Required for symbols in the colored box

\usepackage[usenames,dvipsnames]{xcolor} % Allows the definition of hex colors

% Fonts and tweaks for XeLaTeX
\usepackage{fontspec,xltxtra,xunicode}
\defaultfontfeatures{Mapping=tex-text}
%\setmonofont[Scale=MatchLowercase]{Andale Mono}

% Colors for links, text and headings
\usepackage{hyperref}
\definecolor{linkcolor}{HTML}{506266} % Blue-gray color for links
\definecolor{shade}{HTML}{F5DD9D} % Peach color for the contact information box
\definecolor{text1}{HTML}{2b2b2b} % Main document font color, off-black
\definecolor{headings}{HTML}{701112} % Dark red color for headings
% Other color palettes: shade=B9D7D9 and linkcolor=A40000; shade=D4D7FE and linkcolor=FF0080

\hypersetup{colorlinks,breaklinks, urlcolor=linkcolor, linkcolor=linkcolor} % Set up links and colors

\usepackage{fancyhdr}
\usepackage{amsmath}
\usepackage{physics}
\usepackage{amssymb}
\pagestyle{fancy}
\fancyhf{}
% Headers and footers can be added with the \lhead{} \rhead{} \lfoot{} \rfoot{} commands
% Example footer:
%\rfoot{\color{headings} {\sffamily Last update: \today}. Typeset with Xe\LaTeX}

\renewcommand{\headrulewidth}{0pt} % Get rid of the default rule in the header

\usepackage{titlesec} % Allows creating custom \section's

% Format of the section titles
\titleformat{\section}{\color{headings}
\scshape\Large\raggedright}{}{0em}{}[\color{black}\titlerule]

\title{Quantum Mechanics Assignment Three}
\author{Elliott Capek}
\titlespacing{\section}{0pt}{0pt}{5pt} % Spacing around titles

\begin{document}

\maketitle{}

\section{9.16: Time evolving QHO state}
We are given the probabilities of measuring certain energies for a QHO state $\ket{\psi}$. We
are asked to find the energy basis representation for this state and compute the expectations for
momentum and energy.\\

\textbf{Finding energy representation}
\begin{align*}
  |\bra{1}\ket{\psi}|^2 = 0.36 \rightarrow |c_1|^2 &= 0.36 \rightarrow c_1 = e^{ai}0.6\\
  |\bra{2}\ket{\psi}|^2 = 0.64 \rightarrow |c_1|^2 &= 0.64 \rightarrow c_1 = e^{bi}0.8\\
\end{align*}

\textbf{Time evolution}
\begin{align*}
  \ket{\psi} &= 0.6e^{ai}e^{-iE_1t/\hbar}\ket{1} + 0.8e^{bi}e^{-iE_2t/\hbar}\ket{2}\\
  &= 0.6e^{ai}e^{-3i\omega t/2}\ket{1} + 0.8e^{bi}e^{-5i\omega t/2}\ket{2}\\
\end{align*}

We are told that the expectation value for position is minimized at $t=0$. We can use this to find
the relative phases of the wave function coefficients:\\

\textbf{Calculating relative phase}
\begin{align*}
  <x> &= \bra{\psi}\left(\sqrt{\frac{\hbar}{2m\omega}}
  \left(a + a^\dagger\right)\right)\ket{\psi}\\
  &= \sqrt{\frac{\hbar}{2m\omega}}\left(\bra{\psi}a\ket{\psi}
  + \bra{\psi}a^\dagger\ket{\psi}\right)\\
  &= \sqrt{\frac{\hbar}{2m\omega}}\left(\sqrt{2}c_1^*(t)c_2(t)\bra{1}\ket{1}
  + \sqrt{2}c_1(t)c_2^*(t)\bra{2}\ket{2}\right)\\
  &= \sqrt{\frac{\hbar}{m\omega}}\left(c_1^*(t)c_2(t) + c_1(t)c_2^*(t)\right)\\
  &= \sqrt{\frac{\hbar}{m\omega}}\left(0.6*0.8*e^{i(b-a)} + 0.6*0.8e^{i(a-b)}\right)
  \hspace{2cm}\mbox{Time-dependent terms are 1 when t=0}\\
  &= 2*0.6*0.8*\sqrt{\frac{\hbar}{m\omega}}\left(\cos(b-a) + i\sin(b-a)\right)\\
\end{align*}

To minimize this expectation value, $a-b = b-a = \pi$. Thus we represent our system as follows:

\begin{align*}
  \psi(t) &= 0.6e^{i\pi}e^{-3i\omega t/2}\ket{1} + 0.8e^{-5i\omega t/2}\ket{2}\\
  &= -0.6e^{-3i\omega t/2}\ket{1} + 0.8e^{-5i\omega t/2}\ket{2}\\
\end{align*}

We then calculate expectation values for the system:\\

\textbf{Momentum}
\begin{align*}
  <p> &= -i\sqrt{\frac{\hbar}{2m\omega}}\bra{\psi}\left(a - a^\dagger)\right)\ket{\psi}\\
  &= -i\sqrt{\frac{\hbar}{2m\omega}}\left(\bra{\psi}a\ket{\psi}
  - \bra{\psi}a^\dagger\ket{\psi}\right)\\
  &= -i\sqrt{\frac{\hbar}{2m\omega}}\left(\sqrt{2}c_1^*(t)c_2(t)\bra{1}\ket{1}
  - \sqrt{2}c_1(t)c_2^*(t)\bra{2}\ket{2}\right)\\
  &= -i*0.6*0.8*\sqrt{\frac{\hbar}{m\omega}}\left(e^{i\omega t} - e^{-i\omega t}\right)\\
  &= -0.48\sqrt{\frac{\hbar}{m\omega}}\sin(\omega t)\\
\end{align*}

\textbf{Energy}
\begin{align*}
  \braket{p} &= 
\end{align*}

\end{document}

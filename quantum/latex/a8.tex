\documentclass[10pt]{article} % Font size - 10pt, 11pt or 12pt

\usepackage[hmargin=1.25cm, vmargin=1.5cm]{geometry} % Document margins

\usepackage{marvosym} % Required for symbols in the colored box

\usepackage[usenames,dvipsnames]{xcolor} % Allows the definition of hex colors

% Fonts and tweaks for XeLaTeX
\usepackage{fontspec,xltxtra,xunicode}
\defaultfontfeatures{Mapping=tex-text}
%\setmonofont[Scale=MatchLowercase]{Andale Mono}

% Colors for links, text and headings
\usepackage{hyperref}
\definecolor{linkcolor}{HTML}{506266} % Blue-gray color for links
\definecolor{shade}{HTML}{F5DD9D} % Peach color for the contact information box
\definecolor{text1}{HTML}{2b2b2b} % Main document font color, off-black
\definecolor{headings}{HTML}{701112} % Dark red color for headings
% Other color palettes: shade=B9D7D9 and linkcolor=A40000; shade=D4D7FE and linkcolor=FF0080

\hypersetup{colorlinks,breaklinks, urlcolor=linkcolor, linkcolor=linkcolor} % Set up links and colors

\usepackage{fancyhdr}
\usepackage{amsmath}
\usepackage{braket}
\usepackage{amssymb}
\pagestyle{fancy}
\fancyhf{}
% Headers and footers can be added with the \lhead{} \rhead{} \lfoot{} \rfoot{} commands
% Example footer:
%\rfoot{\color{headings} {\sffamily Last update: \today}. Typeset with Xe\LaTeX}

\renewcommand{\headrulewidth}{0pt} % Get rid of the default rule in the header

\usepackage{titlesec} % Allows creating custom \section's

% Format of the section titles
\titleformat{\section}{\color{headings}
\scshape\Large\raggedright}{}{0em}{}[\color{black}\titlerule]


\title{Quantum Mechanics Assignment Eight}
\author{Elliott Capek}
\titlespacing{\section}{0pt}{0pt}{5pt} % Spacing around titles

\begin{document}

\maketitle{}

\section{Problem 1: Two Spin-1 Particle states}
Here we use a Clebsch-Gordan table and our knowledge of the P interchange operator
to list the spin states a system of two spin-1 particles could take on, and whether
they are symmetric or antisymmetric with respect to particle interchange.\\

Two spin particles combine to have a total angular momentum of 0, 1 or 2. We use a CG
table to list the states, in the format $\ket{J,M_J} = \ket{M_{S1}, M_{S2}}$:\\

\begin{align*}
  \mbox{Symmetric}\hspace{2cm} &\ket{2,2} = \ket{1,1} = \ket{1,1}\\
  \mbox{Symmetric}\hspace{2cm}&\ket{2,1} = \frac{1}{\sqrt{2}}\left(\ket{1,0}
  + \ket{0,1}\right)
  = \frac{1}{\sqrt{2}}\left(\ket{0,1} + \ket{1,0}\right)\\
  \mbox{Symmetric}\hspace{2cm} &\ket{2,0} =
  \frac{1}{\sqrt{2}}\left(\ket{1,-1} + \ket{-1,1}\right) + \sqrt{\frac{2}{3}}\ket{0,0}
  = \frac{1}{\sqrt{2}}\left(\ket{-1,1} + \ket{1,-1}\right)
  + \sqrt{\frac{2}{3}}\ket{0,0}\\
  \mbox{Symmetric}\hspace{2cm} &\ket{2,-1} =
  \frac{1}{\sqrt{2}}\left(\ket{-1,0} + \ket{0,-1}\right)
  = \frac{1}{\sqrt{2}}\left(\ket{0,-1} + \ket{-1,0}\right)\\
  \mbox{Symmetric}\hspace{2cm} &\ket{2,-2} = \ket{-1,-1} = \ket{-1,-1}\\
\end{align*}

\begin{align*}
  \mbox{Antisymmetric}\hspace{2cm} &\ket{1,1}
  = \frac{1}{\sqrt{2}}\left(\ket{1,0} - \ket{0,1}\right)
  = -\frac{1}{\sqrt{2}}\left(\ket{0,1} - \ket{1,0}\right)\\
  \mbox{Antisymmetric}\hspace{2cm} &\ket{1,0}
  = \frac{1}{\sqrt{2}}\left(\ket{1,-1} - \ket{-1,1}\right)
  = -\frac{1}{\sqrt{2}}\left(\ket{-1,1} - \ket{1,-1}\right)\\
  \mbox{Antisymmetric}\hspace{2cm} &\ket{1,-1}
  = \frac{1}{\sqrt{2}}\left(\ket{-1,0} - \ket{0,-1}\right)
  = -\frac{1}{\sqrt{2}}\left(\ket{0,-1} - \ket{-1,0}\right)\\
\end{align*}

\begin{align*}
  \mbox{Symmetric}\hspace{2cm} &\ket{0,0} = \ket{-1,-1} =
  \frac{1}{\sqrt{3}}\left(\ket{1,-1} - \ket{0,0} \ket{-1,1}\right)
  = \frac{1}{\sqrt{3}}\left(\ket{-1,1} - \ket{0,0} \ket{1,-1}\right)\\
\end{align*}

This is a demonstration of how the coupled basis states for 2x spin-1 particles
obey the Symmetrization Postulate. This is an advantage that the uncoupled basis
does not share: $\ket{1,-1}$ does not obey the SP, since the first particle is always
in the $M_S = 1$ state, and so is effectively ``labelled''. This is disallowed
by QM.\\

An interesting result we can glean from this is that two fermions with
spin-1 (eg protons) will always go to the F=1 state when they have a symmetric
wave function (ie they are in the ground state), since all wavefunctions must be
antisymmetric for fermions. This is a weird case that is not intuitive - why is
there such an arbitrary set of rules on what the spin can and cannot be?\\

\section{Problem 2: Neutron Star}
Neutron stars are celestial objects made up almost exclusively of neutrons, or spin-1/2 fermions.\\

\section{Problem 13.7: Noninteracting particles in a HO potential}
Here we find the interparticle spacing for three types of systems: a distinguishable,
fermonic and bosonic pair of particles in a QHO potential where one particle is in state $n$
and the other in state $k$. We construct our states as follows, noting that fermions have AS
descriptions, bosons have S descriptions and distinguishable particles have neither:

\begin{align*}
  \mbox{Distinguishable}&\hspace{1cm} \ket{\psi} = \ket{n}_1\ket{k}_2\\
  \mbox{Fermions}&\hspace{1cm} \ket{\psi}
  = \frac{1}{\sqrt{2}}\left(\ket{n}_1\ket{k}_2 - \ket{k}_1\ket{n}_2\right)\\
  \mbox{Bosons}&\hspace{1cm} \ket{\psi}
  = \frac{1}{\sqrt{2}}\left(\ket{n}_1\ket{k}_2 + \ket{k}_1\ket{n}_2\right)\\
\end{align*}

\textbf{a.) Distinguishable}

\begin{align*}
  \braket{(x_1-x_2)^2} &= \bra{n}_1\bra{k}_2(x_1-x_2)^2\ket{n}_1\ket{k}_2\\
  &= \bra{n}_1\bra{k}_2(x_1^2 + x_2^2 - 2x_1x_2)\ket{n}_1\ket{k}_2\\
  &= \bra{n}_1\bra{k}_2x_1^2\ket{n}_1\ket{k}_2 +
  \bra{n}_1\bra{k}_2x_2^2\ket{n}_1\ket{k}_2 - 2\bra{n}_1\bra{k}_2x_1x_2\ket{n}_1\ket{k}_2\\
  &= \bra{n}_1x_1^2\ket{n}_1\bra{k}_2\ket{k}_2 +
  \bra{n}_1\ket{n}_1\bra{k}_2x_2^2\ket{k}_2 - 2\bra{n}_1\bra{k}_2x_1x_2\ket{n}_1\ket{k}_2\\
  &= \bra{n}_1x_1^2\ket{n}_1 + \bra{k}_2x_2^2\ket{k}_2
  - 2\bra{n}_1\bra{k}_2x_1x_2\ket{n}_1\ket{k}_2\\
  &= \frac{\hbar}{2m\omega}\bra{n}_1
  \left(a^\dagger a^\dagger + aa + aa^\dagger + a^\dagger a\right)\ket{n}_1
  + \frac{\hbar}{2m\omega}\bra{k}_2
  \left(a^\dagger a^\dagger + aa + aa^\dagger + a^\dagger a\right)\ket{k}_2
  - 2\bra{n}_1\bra{k}_2x_1x_2\ket{n}_1\ket{k}_2\\
  &= \frac{\hbar}{2m\omega}\bra{n}_1\left(aa^\dagger + a^\dagger a\right)\ket{n}_1
  + \frac{\hbar}{2m\omega}\bra{k}_2\left(aa^\dagger + a^\dagger a\right)\ket{k}_2
  - 2\bra{n}_1\bra{k}_2x_1x_2\ket{n}_1\ket{k}_2\\
  &= \frac{\hbar}{2m\omega}\left(2n + 2k + 2\right)
  - \frac{2\hbar}{2m\omega}\bra{n}_1\bra{k}_2
  \left(a_1+a_1^\dagger\right)\left(a_2+a_2^\dagger\right)\ket{n}_1\ket{k}_2\\
  &= \frac{\hbar}{2m\omega}\left(2n + 2k + 2\right)
  - \frac{2\hbar}{2m\omega}\bra{n}_1\bra{k}_2
  \left(a_1a_2 + a_1^\dagger a_2^\dagger + a_1a_2^\dagger + a_2a_1^\dagger\right)\ket{k}_2\\
  &= \frac{\hbar}{2m\omega}\left(2n + 2k + 2\right)
  - \frac{2\hbar}{2m\omega}\bra{n}_1\bra{k}_2
  \left(a_1a_2 + a_1^\dagger a_2^\dagger + a_1a_2^\dagger + a_2a_1^\dagger\right)\bra{n}_1\ket{k}_2\\
  &= \frac{\hbar}{2m\omega}\left(2n + 2k + 2\right)\\
  &- \frac{2\hbar}{2m\omega}
  \left(\bra{n}_1a_1\ket{n}_1\ket{k}_2a_2\ket{k}_2
  + \bra{n}_1a_1^\dagger\ket{n}_1\bra{k}_2a_2^\dagger\ket{k}_2
  + \bra{n}_1a_1\ket{n}_1\bra{k}_2a_2^\dagger\ket{k}_2
  + \bra{k}_2a_2\ket{k}_2\bra{n}_1a_1^\dagger\ket{n}_1\right)\\
  &= \frac{\hbar}{m\omega}(2n+2k+2)\\
\end{align*}

\textbf{b.) Boson}

\begin{align*}
  \braket{(x_1-x_2)^2} &=
  \frac{1}{2}\left(\bra{n}_1\bra{k}_2 + \bra{k}_1\bra{n}_2\right)
  \left(x_1^2 + x_2^2 - 2x_1x_2\right)
  \left(\ket{n}_1\ket{k}_2 + \ket{k}_1\ket{n}_2\right)\\
  &= \frac{1}{2}\left(
  \bra{n}_1x_1^2\ket{n}_1 + \bra{k}_2x_2^2\ket{k}_2
  + \bra{k}_1x_1^2\ket{k}_1 + \bra{n}_2x_2^2\ket{n}_2
  - 2\bra{n}_1\bra{k}_2x_1x_2\ket{n}_1\ket{k}_2 - 2\bra{n}_2\bra{k}_1x_1x_2\ket{n}_2\ket{k}_1\right)\\
  &= \frac{1}{2}\left(4n + 4k + 4\right)
  - \bra{n}_1\bra{k}_2x_1x_2\ket{n}_1\ket{k}_2 - \bra{n}_2\bra{k}_1x_1x_2\ket{n}_2\ket{k}_1\\
  &= \frac{\hbar}{2m\omega}\left(4n + 4k + 4\right)\\
\end{align*}

\textbf{c.) Fermion}

\begin{align*}
  \braket{(x_1-x_2)^2} &=
  \frac{1}{2}\left(\bra{n}_1\bra{k}_2 - \bra{k}_1\bra{n}_2\right)
  \left(x_1^2 + x_2^2 - 2x_1x_2\right)
  \left(\ket{n}_1\ket{k}_2 - \ket{k}_1\ket{n}_2\right)\\
  &= \frac{1}{2}\left(
  \bra{n}_1x_1^2\ket{n}_1 + \bra{k}_2x_2^2\ket{k}_2
  - \bra{k}_1x_1^2\ket{k}_1 - \bra{n}_2x_2^2\ket{n}_2
  - 2\bra{n}_1\bra{k}_2x_1x_2\ket{n}_1\ket{k}_2 + 2\bra{n}_2\bra{k}_1x_1x_2\ket{n}_2\ket{k}_1\right)\\
  &= \frac{\hbar}{2m\omega}\frac{1}{2}\left(2n+1 + 2k+1 - 2n-1 - 2k-1 - 0\right)\\
  &= 0\\
\end{align*}

This is a really interesting problem that demonstrates how systems that are identical save
for their symmetry rules can behave totally differently. The boson system has twice the expectation
value of the distinguishable particle system, and the fermion system has zero expectation value
for particle distance. This is a very unintuitive result, since it predicts that on average the
two particles are on top of each other.\\

\section{Problem 4: Three-particle system}
In this problem we practice writing the wave functions for three-particle systems of particles
with different symmetry rules:\\

\section{Problem 5: Zeeman Effect}
When doing Zeeman Effect calculations and deciding whether the Zeeman corrections should go in the
first-order Hamiltonian correction or be put in the unperturbed Hamiltonian, a ``strong'' B-field
is one that is too big to be a first-order correction. For the n=2 state of Hydrogen, the
first-order fine structure corrections have a size of:

\begin{equation*}
  E_{fs}^{(1)}
  = -\frac{1}{2}\alpha^4mc^2\frac{1}{2^3}\left(\frac{1}{j+\frac{1}{2}}-\frac{3}{8}\right)\\
  \approx -\frac{1}{16}\alpha^4mc^2\\
\end{equation*}

We say that Zeeman corrections of this size and larger should be considered due to ``strong''
magnetic fields. The energy perturbation due to the B-field is given by:\\

\begin{equation*}
  \Delta E_{Z} = \frac{\mu_B B}{\hbar}\left(g_\ell L_z + g_e S_z\right)
  \approx \frac{\mu_B B}{\hbar}\\
\end{equation*}

We equate these expressions to find the smallest ``strong'' B-field:\\

\begin{align*}
  B &\approx \frac{-\hbar}{16mu_B}\alpha^4mc^2
  = \frac{-6.5*10^{-16}}{16*6*10^{-5}}*7^4*10^{-12}*5*10^5\\
  \approx 8*10^{-26}T\\
\end{align*}

We now find the B-field which perturbes the H atom enough to equate the energies of the 2p and 3d
states:\\

\begin{align*}
  E_{2p} + \Delta E_{Zeeman, 2p} + E_{fs}^{(1)} &= E_{3d} + \Delta E_{Zeeman, 3d} + E_{fs}^{(1)}\\
  \frac{-13.6eV}{4} + \mu_BB(m_\ell+2m_s)
  + \frac{1}{2}\alpha^4mc^2\left(\frac{3}{4n^3}
  -\frac{\ell(\ell+1)-m_\ell m_s}{n^3\ell(\ell+\frac{1}{2})(\ell+1)}\right)
  &=\\ \frac{-13.6eV}{8} + \mu_BB(m_\ell+2m_s) + \frac{1}{2}\alpha^4mc^2\left(\frac{3}{4n^3}
  -\frac{\ell(\ell+1)-m_\ell m_s}{n^3\ell(\ell+\frac{1}{2})(\ell+1)}\right)\\
  \frac{-13.6eV}{4} + 2\mu_BB
  + \alpha^4mc^2\left(\frac{3}{64}
  -\frac{2-\frac{1}{2}}{24}\right)
  &= \frac{-13.6eV}{8} + -3\mu_BB + \frac{1}{2}\alpha^4mc^2\left(\frac{3}{108}
  -\frac{6+1}{405}\right)\\
  \frac{13.6eV}{4} + 5\mu_BB
  &= \alpha^4mc^2
  \left(\frac{3}{216} - \frac{7}{405} - \frac{3}{64} + \frac{3}{48}\right)\\
  B &= \frac{\alpha^4mc^2
    \left(\frac{3}{216} - \frac{7}{405} - \frac{3}{64} + \frac{3}{48}\right) - \frac{13.6eV}{4}}{5\mu_B}\\
  &= \frac{\frac{317\alpha^4mc^2}{25920} - \frac{13.6eV}{4}}{5\mu_B}\\
  &\approx \frac{0.68eV}{5.7*10^{-5}eV/T} = 10^4 T\\
\end{align*}

This is a huge magnetic field!\\

\end{document}

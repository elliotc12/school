\documentclass[10pt]{article} % Font size - 10pt, 11pt or 12pt

\usepackage[hmargin=1.25cm, vmargin=1.5cm]{geometry} % Document margins

\usepackage{marvosym} % Required for symbols in the colored box

\usepackage[usenames,dvipsnames]{xcolor} % Allows the definition of hex colors

% Fonts and tweaks for XeLaTeX
\usepackage{fontspec,xltxtra,xunicode}
\defaultfontfeatures{Mapping=tex-text}
%\setmonofont[Scale=MatchLowercase]{Andale Mono}

% Colors for links, text and headings
\usepackage{hyperref}
\definecolor{linkcolor}{HTML}{506266} % Blue-gray color for links
\definecolor{shade}{HTML}{F5DD9D} % Peach color for the contact information box
\definecolor{text1}{HTML}{2b2b2b} % Main document font color, off-black
\definecolor{headings}{HTML}{701112} % Dark red color for headings
% Other color palettes: shade=B9D7D9 and linkcolor=A40000; shade=D4D7FE and linkcolor=FF0080

\hypersetup{colorlinks,breaklinks, urlcolor=linkcolor, linkcolor=linkcolor} % Set up links and colors

\usepackage{fancyhdr}
\usepackage{amsmath}
\usepackage{braket}
\usepackage{amssymb}
\pagestyle{fancy}
\fancyhf{}
% Headers and footers can be added with the \lhead{} \rhead{} \lfoot{} \rfoot{} commands
% Example footer:
%\rfoot{\color{headings} {\sffamily Last update: \today}. Typeset with Xe\LaTeX}

\renewcommand{\headrulewidth}{0pt} % Get rid of the default rule in the header

\usepackage{titlesec} % Allows creating custom \section's

% Format of the section titles
\titleformat{\section}{\color{headings}
\scshape\Large\raggedright}{}{0em}{}[\color{black}\titlerule]


\title{Quantum Mechanics Assignment Eight}
\author{Elliott Capek}
\titlespacing{\section}{0pt}{0pt}{5pt} % Spacing around titles

\begin{document}

\maketitle{}

\section{Problem 1: Two Spin-1 Particle states}
Here we use a Clebsch-Gordan table and our knowledge of the P interchange operator
to list the spin states a system of two spin-1 particles could take on, and whether
they are symmetric or antisymmetric with respect to particle interchange.\\

Two spin particles combine to have a total angular momentum of 0, 1 or 2. We use a CG
table to list the states, in the format $\ket{J,M_J} = \ket{M_{S1}, M_{S2}}$:\\

\begin{align*}
  \mbox{Symmetric}\hspace{2cm} &\ket{2,2} = \ket{1,1} = \ket{1,1}\\
  \mbox{Symmetric}\hspace{2cm}&\ket{2,1} = \frac{1}{\sqrt{2}}\left(\ket{1,0}
  + \ket{0,1}\right)
  = \frac{1}{\sqrt{2}}\left(\ket{0,1} + \ket{1,0}\right)\\
  \mbox{Symmetric}\hspace{2cm} &\ket{2,0} =
  \frac{1}{\sqrt{2}}\left(\ket{1,-1} + \ket{-1,1}\right) + \sqrt{\frac{2}{3}}\ket{0,0}
  = \frac{1}{\sqrt{2}}\left(\ket{-1,1} + \ket{1,-1}\right)
  + \sqrt{\frac{2}{3}}\ket{0,0}\\
  \mbox{Symmetric}\hspace{2cm} &\ket{2,-1} =
  \frac{1}{\sqrt{2}}\left(\ket{-1,0} + \ket{0,-1}\right)
  = \frac{1}{\sqrt{2}}\left(\ket{0,-1} + \ket{-1,0}\right)\\
  \mbox{Symmetric}\hspace{2cm} &\ket{2,-2} = \ket{-1,-1} = \ket{-1,-1}\\
\end{align*}

\begin{align*}
  \mbox{Antisymmetric}\hspace{2cm} &\ket{1,1}
  = \frac{1}{\sqrt{2}}\left(\ket{1,0} - \ket{0,1}\right)
  = -\frac{1}{\sqrt{2}}\left(\ket{0,1} - \ket{1,0}\right)\\
  \mbox{Antisymmetric}\hspace{2cm} &\ket{1,0}
  = \frac{1}{\sqrt{2}}\left(\ket{1,-1} - \ket{-1,1}\right)
  = -\frac{1}{\sqrt{2}}\left(\ket{-1,1} - \ket{1,-1}\right)\\
  \mbox{Antisymmetric}\hspace{2cm} &\ket{1,-1}
  = \frac{1}{\sqrt{2}}\left(\ket{-1,0} - \ket{0,-1}\right)
  = -\frac{1}{\sqrt{2}}\left(\ket{0,-1} - \ket{-1,0}\right)\\
\end{align*}

\begin{align*}
  \mbox{Symmetric}\hspace{2cm} &\ket{0,0} = \ket{-1,-1} =
  \frac{1}{\sqrt{3}}\left(\ket{1,-1} - \ket{0,0} \ket{-1,1}\right)
  = \frac{1}{\sqrt{3}}\left(\ket{-1,1} - \ket{0,0} \ket{1,-1}\right)\\
\end{align*}

This is a demonstration of how the coupled basis states for 2x spin-1 particles
obey the Symmetrization Postulate. This is an advantage that the uncoupled basis
does not share: $\ket{1,-1}$ does not obey the SP, since the first particle is always
in the $M_S = 1$ state, and so is effectively ``labelled''. This is disallowed
by QM.\\

An interesting result we can glean from this is that two fermions with
spin-1 (eg protons) will always go to the F=1 state when they have a symmetric
wave function (ie they are in the ground state), since all wavefunctions must be
antisymmetric for fermions. This is a weird case that is not intuitive - why is
there such an arbitrary set of rules on what the spin can and cannot be?\\

\end{document}

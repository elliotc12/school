\documentclass[10pt]{article} % Font size - 10pt, 11pt or 12pt

\usepackage[hmargin=1.25cm, vmargin=1.5cm]{geometry} % Document margins

\usepackage{marvosym} % Required for symbols in the colored box

\usepackage[usenames,dvipsnames]{xcolor} % Allows the definition of hex colors

% Fonts and tweaks for XeLaTeX
\usepackage{fontspec,xltxtra,xunicode}
\defaultfontfeatures{Mapping=tex-text}
%\setmonofont[Scale=MatchLowercase]{Andale Mono}

% Colors for links, text and headings
\usepackage{hyperref}
\definecolor{linkcolor}{HTML}{506266} % Blue-gray color for links
\definecolor{shade}{HTML}{F5DD9D} % Peach color for the contact information box
\definecolor{text1}{HTML}{2b2b2b} % Main document font color, off-black
\definecolor{headings}{HTML}{701112} % Dark red color for headings
% Other color palettes: shade=B9D7D9 and linkcolor=A40000; shade=D4D7FE and linkcolor=FF0080

\hypersetup{colorlinks,breaklinks, urlcolor=linkcolor, linkcolor=linkcolor} % Set up links and colors

\usepackage{fancyhdr}
\usepackage{amsmath}
\usepackage{braket}
\usepackage{amssymb}
\pagestyle{fancy}
\fancyhf{}
% Headers and footers can be added with the \lhead{} \rhead{} \lfoot{} \rfoot{} commands
% Example footer:
%\rfoot{\color{headings} {\sffamily Last update: \today}. Typeset with Xe\LaTeX}

\renewcommand{\headrulewidth}{0pt} % Get rid of the default rule in the header

\usepackage{titlesec} % Allows creating custom \section's

% Format of the section titles
\titleformat{\section}{\color{headings}
\scshape\Large\raggedright}{}{0em}{}[\color{black}\titlerule]

\title{Quantum Mechanics Assignment Five}
\author{Elliott Capek}
\titlespacing{\section}{0pt}{0pt}{5pt} % Spacing around titles

\begin{document}

\maketitle{}

\section{Problem 1: Relativistic Fine Correction}

This problem asks us to work through parts of the derivation of perturbation theory applied to
the relativistic correction to the Hydrogen energy spectrum.\\

\textbf{a.)}Here we show how to find the expectation value of $p^4$ more easily:\\

\begin{align*}
  &\bra{n,\ell,m}p^4\ket{n,\ell,m}\\
  &= \bra{n,\ell,m}p^2p^2\ket{n,\ell,m}\\
  &= 4m^2\bra{n,\ell,m}\left(H_0 + \frac{e^2}{4\pi\epsilon r}\right)
  \left(H_0 + \frac{e^2}{4\pi\epsilon r}\right)\ket{n,\ell,m}\\
  &= 4m^2\left(\bra{n,\ell,m}H_0^2\ket{n,\ell,m}
  + \bra{n,\ell,m}H_0\frac{e^2}{4\pi\epsilon r}\ket{n,\ell,m}
  + \bra{n,\ell,m}\frac{e^2}{4\pi\epsilon r}H_0\ket{n,\ell,m}
  + \bra{n,\ell,m}\frac{e^4}{16\pi^2\epsilon^2r^2}\ket{n,\ell,m}\right)\\
  &= 4m^2\left((E_n^{(0)})^2
  + E_n^{(0)}\bra{n,\ell,m}\frac{e^2}{4\pi\epsilon r}\ket{n,\ell,m}
  + \bra{n,\ell,m}\frac{e^2}{4\pi\epsilon r}\ket{n,\ell,m}E_n^{(0)}
  + \frac{e^4}{16\pi^2\epsilon^2r^2}\bra{n,\ell,m}\frac{1}{r^2}\ket{n,\ell,m}\right)\\
  &= 4m^2\left((E_n^{(0)})^2
  + \frac{e^2E_n^{(0)}}{4\pi\epsilon}\braket{\frac{1}{r}}
  + \frac{e^2}{4\pi\epsilon}\braket{\frac{1}{r}}E_n^{(0)}
  + \frac{e^4}{16\pi^2\epsilon^2r^2}\braket{\frac{1}{r^2}}\right)\\
  &= 4m^2\left((E_n^{(0)})^2
  + \frac{e^2E_n^{(0)}}{2\pi\epsilon}\braket{\frac{1}{r}}
  + \frac{e^4}{16\pi^2\epsilon^2r^2}\braket{\frac{1}{r^2}}\right)\\
\end{align*}

\textbf{b.)}\\

\textbf{c.)} We perform the following integrals using Mathematica:

\begin{align*}
  \braket{\frac{1}{r}}_{20} &= \left(\frac{Z}{2a}\right)^3\int_0^\infty
  4r\left(1-\frac{Zr}{2a}\right)^2e^{-Zr/a}dr = \frac{Z}{4a}\\
  \braket{\frac{1}{r^2}}_{20} &= \left(\frac{Z}{2a}\right)^3\int_0^\infty
  4\left(1-\frac{Zr}{2a}\right)^2e^{-Zr/a}dr = \frac{Z^2}{4a^2}\\
  \braket{\frac{1}{r}}_{21} &= \left(\frac{Z}{2a}\right)^3\int_0^\infty
  \frac{Z^2r}{3a^2}e^{Zr/a}dr = \frac{Z}{4a}\\
  \braket{\frac{1}{r^2}}_{21} &= \left(\frac{Z^5}{24a^5}\right)^3\int_0^\infty
  r^2e^{-Zr/a}dr = \frac{Z^2}{12a^2}\\
\end{align*}

These are the desired results. We then assume this is true for all n, $\ell$ quantum numbers.\\

\textbf{d.)}

\end{document}

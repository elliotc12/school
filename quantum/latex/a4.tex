\documentclass[10pt]{article} % Font size - 10pt, 11pt or 12pt

\usepackage[hmargin=1.25cm, vmargin=1.5cm]{geometry} % Document margins

\usepackage{marvosym} % Required for symbols in the colored box

\usepackage[usenames,dvipsnames]{xcolor} % Allows the definition of hex colors

% Fonts and tweaks for XeLaTeX
\usepackage{fontspec,xltxtra,xunicode}
\defaultfontfeatures{Mapping=tex-text}
%\setmonofont[Scale=MatchLowercase]{Andale Mono}

% Colors for links, text and headings
\usepackage{hyperref}
\definecolor{linkcolor}{HTML}{506266} % Blue-gray color for links
\definecolor{shade}{HTML}{F5DD9D} % Peach color for the contact information box
\definecolor{text1}{HTML}{2b2b2b} % Main document font color, off-black
\definecolor{headings}{HTML}{701112} % Dark red color for headings
% Other color palettes: shade=B9D7D9 and linkcolor=A40000; shade=D4D7FE and linkcolor=FF0080

\hypersetup{colorlinks,breaklinks, urlcolor=linkcolor, linkcolor=linkcolor} % Set up links and colors

\usepackage{fancyhdr}
\usepackage{amsmath}
\usepackage{braket}
\usepackage{amssymb}
\pagestyle{fancy}
\fancyhf{}
% Headers and footers can be added with the \lhead{} \rhead{} \lfoot{} \rfoot{} commands
% Example footer:
%\rfoot{\color{headings} {\sffamily Last update: \today}. Typeset with Xe\LaTeX}

\renewcommand{\headrulewidth}{0pt} % Get rid of the default rule in the header

\usepackage{titlesec} % Allows creating custom \section's

% Format of the section titles
\titleformat{\section}{\color{headings}
\scshape\Large\raggedright}{}{0em}{}[\color{black}\titlerule]

\title{Quantum Mechanics Assignment Three}
\author{Elliott Capek}
\titlespacing{\section}{0pt}{0pt}{5pt} % Spacing around titles

\begin{document}

\maketitle{}

\section{10.8: Magnetic Perturbation Spin 1}
For this problem we want to find the energy perturbations due to a magnetic field perturbation
in the x direction, while remaining in the z basis. To do this we first find the
perturbed Hamiltonian, which is just the x spin-1 matrix times a factor, and
then use the formula
$E_n^{(2)} = \sum_{m\neq n}\frac{|\bra{n^{(0)}}H'\ket{n^{(0)}}|^2}{(E_n^{(0)}-E_m^{(0)})}$
to find the second-order energy corrections.\\

We first find the perturbed Hamiltonian:

\begin{align*}
  H' &= \vec{\mu} \cdot \vec{B_2}\\
  &= \frac{g e}{2m} \hat{\mu} \cdot B_2 \hat{x}\\
  &= \frac{\hbar\omega_2}{2} S_x
  \hspace{2cm}\omega_2 = \frac{geB}{2m}\\
  &=
  \begin{pmatrix}
    0 & \frac{\hbar\omega_2}{2} & 0\\
    \frac{\hbar\omega_2}{2} & 0 & \frac{\hbar\omega_2}{2}\\
    0 & \frac{\hbar\omega_2}{2} & 0\\
  \end{pmatrix}
\end{align*}

Then we just apply the second-order perturbation energy equation to find our
energy changes, noticing that there will be no first-order corrections because
$H'$ has zeroes along its diagonal.

\begin{align*}
  E_n^{(2)} = \sum_{m\neq n}\frac{|\bra{n^{(0)}}H'\ket{n^{(0)}|^2}}{(E_n^{(0)}-E_m^{(0)})}
\end{align*}

\begin{align*}
  E_1^{(2)} &=
  \frac{|H'_{1,2}|^2}{E_2 - E_1} +
  \frac{|H'_{1,3}|^2}{E_{3} - E_{1}}\\
  &= \frac{\hbar^2\omega_2^2}{4\hbar\omega_0}\\
  &= \frac{\hbar\omega_2^2}{4\omega_0}\\
\end{align*}

\begin{align*}
  E_2^{(2)} &=
  \frac{|H'_{2,1}|^2}{E_1 - E_2} +
  \frac{|H'_{2,3}|^2}{E_3 - E_2}\\
  &= -\frac{\hbar^2\omega_2^2}{4\hbar\omega_0} + \frac{\hbar^2\omega_2^2}{4\hbar\omega_0}\\
  &= 0\\
\end{align*}

\begin{align*}
  E_3^{(2)} &=
  \frac{|H'_{3,1}|^2}{E_1 - E_3} +
  \frac{|H'_{3,2}|^2}{E_2 - E_3}\\
  &= -\frac{\hbar\omega_2^2}{4\omega_0}\\
\end{align*}

These are nicely symmetric energy corrections typical of physics problems. From this we can see
that the ratio between the initial and perturbation Larmor frequencies is the only factor in
deciding the magnitude of the perturbation. It is kind of weird that having a big initial
Larmor frequency decreases the magnitude of the perturbation energy.\\

We now plot our results:\\
\vspace{7cm}


As expected, we see quadratic dependence of the energy change on on perturbation strength for the
second-order correction.
\section{9.17: Ramp correction to Infinite Square Well}
Here we find the first-order energy corrections for an infinite square well wavefunction where a
ramp potential $H' = V'(x) = \beta x$ is applied. Because the correction Hamiltonian is in the
x basis, we'll need to use wave function notation to find the ground-state energy correction:

\begin{align*}
  E_1^{(1)} &= \bra{1}H'\ket{1}\\
  &= \frac{2\beta}{L}\int_0^L \sin^2\left(\frac{\pi x}{L}\right)x dx\\
  &= \frac{\beta L}{2}\\
\end{align*}

This shows how easy it is to apply perturbation theory to find the energy corrections for simple
systems. The units make sense and as we expect, both $\beta$ and L increase the energy shift.\\

\section{10.18: Dirac delta Infinite Square Well}
In this problem we deal with an infinite square well which has been perturbed by a delta function:
$H' = V(x) = LV_0\delta(x-L/2)$\\

\textbf{a.} First, we calculate the nth energy perturbation $E_n^1$ using wave function notation:

\begin{align*}
  E_n^{(1)} &= \bra{n}H'\ket{n}\\
  &= \int_0^L 2\sin^2\left(\frac{n\pi x}{L}\right)V_0\delta(x-L/2)dx\\
  &= 2V_0\sin^2\left(\frac{n\pi}{2}\right)\\
  &= \begin{cases}
    2V_0 & \mbox{n is odd}\\
    0 & \mbox{n is even}\\
  \end{cases}\\
\end{align*}

\textbf{b.}
This result is based on the parity of the function. Even-parity functions have zero density at the
center of the well, whereas odd-parity functions have nonzero density there. Therefore conceptually
we can say that the delta potential only interacts with the odd functions; the even functions tunnel
right through.\\

\textbf{c.}
The corrected wave function is given by:

\begin{align*}
  \ket{1} = \sum_{m\neq1} \frac{2mL^2\bra{m}H'\ket{m}}{\left(m^2-1\right)\hbar^2\pi^2}\\
\end{align*}

This value will be maximized when m is the smallest odd number larger than 1, i.e, 3. Thus the
ground state is corrected most by the $\ket{3^{(0)}}$ state. This makes sense, since in general
similar-energy waves correct each other the most. This conceptually makes sense, since the wave
would take on characteristics of those similar to it in energy.\\

\section{10.24: Degenerate Perturbation}
Here we are given just the perturbed Hamiltonian of a system and are asked to find the first
nonvanishing energy correction to each of the four states. The perturbed Hamiltonian is:

\begin{equation*}
  H \doteq
  \begin{pmatrix}
    3 & \epsilon & 0 & 0\\
    \epsilon & 3 & 2\epsilon & 0\\
    0 & 2\epsilon & 5 & \epsilon\\
    0 & 0 & \epsilon & 7\\
  \end{pmatrix}
\end{equation*}

\textbf{a.}
We are told that there are no diagonal perturbation elements, and so because the Hamiltonian is in
the original basis we write our states as follows:

\begin{equation*}
  3V_0,
  \begin{pmatrix}
    1 \\ 0 \\ 0 \\ 0\\
  \end{pmatrix}
  \hspace{2cm}
  3V_0,
  \begin{pmatrix}
    0 \\ 1 \\ 0 \\ 0\\
  \end{pmatrix}
  \hspace{2cm}
  5V_0,
  \begin{pmatrix}
    0 \\ 0 \\ 1 \\ 0\\
  \end{pmatrix}
  \hspace{2cm}
  7V_0,
  \begin{pmatrix}
    0 \\ 0 \\ 0 \\ 1\\
  \end{pmatrix}
  \hspace{2cm}
\end{equation*}

\textbf{b.}
First-order corrections always manifest as terms on the Hamiltonian's diagonal. Our diagonal has
no such terms, and so all corrections are 2nd-order or higher. We first solve for the nondegenerate
$\ket{3}$ and $\ket{4}$ state energy corrections:

\begin{align*}
  E_3^{(2)} &=
  \frac{|H'_{3,1}|^2}{E_1 - E_3} +
  \frac{|H'_{3,2}|^2}{E_2 - E_3} +
  \frac{|H'_{3,4}|^2}{E_4 - E_3}\\
  &= \frac{4\epsilon^2}{-2V_0} + \frac{\epsilon^2}{-4V_0}\\
  &= \frac{-9\epsilon^2}{4V_0}
\end{align*}

\begin{align*}
  E_4^{(2)} &=
  \frac{|H'_{4,1}|^2}{E_1 - E_4} +
  \frac{|H'_{4,2}|^2}{E_2 - E_4} +
  \frac{|H'_{4,3}|^2}{E_3 - E_4}\\
  &= \frac{-\epsilon^2}{2V_0}
\end{align*}

We then solve for the perturbed states: 

\end{document}

\documentclass[10pt]{article} % Font size - 10pt, 11pt or 12pt

\nonstopmode

\usepackage[hmargin=1.25cm, vmargin=1.5cm]{geometry} % Document margins

\usepackage{amsmath}

\usepackage[usenames,dvipsnames]{xcolor} % Allows the definition of hex colors

% Fonts and tweaks for XeLaTeX
\usepackage{fontspec,xltxtra,xunicode}
\defaultfontfeatures{Mapping=tex-text}
%\setmonofont[Scale=MatchLowercase]{Andale Mono}

% Colors for links, text and headings
\usepackage{hyperref}
\definecolor{linkcolor}{HTML}{506266} % Blue-gray color for links
\definecolor{shade}{HTML}{F5DD9D} % Peach color for the contact information box
\definecolor{text1}{HTML}{2b2b2b} % Main document font color, off-black
\definecolor{headings}{HTML}{701112} % Dark red color for headings
% Other color palettes: shade=B9D7D9 and linkcolor=A40000; shade=D4D7FE and linkcolor=FF0080

\hypersetup{colorlinks,breaklinks, urlcolor=linkcolor, linkcolor=linkcolor} % Set up links and colors

\usepackage{fancyhdr}
\usepackage{amssymb}
\pagestyle{fancy}
\fancyhf{}
% Headers and footers can be added with the \lhead{} \rhead{} \lfoot{} \rfoot{} commands
% Example footer:
%\rfoot{\color{headings} {\sffamily Last update: \today}. Typeset with Xe\LaTeX}

\renewcommand{\headrulewidth}{0pt} % Get rid of the default rule in the header

\usepackage{titlesec} % Allows creating custom \section's

% Format of the section titles
\titleformat{\section}{\color{headings}
\scshape\Large\raggedright}{}{0em}{}[\color{black}\titlerule]

\title{Advanced Calc Homework 5}
\author{Elliott Capek}
\titlespacing{\section}{0pt}{0pt}{5pt} % Spacing around titles

\begin{document}

\maketitle{}

\section{Problem 6.4.2}
In search of a contradiction, say $f(x)$ is integrable. Then $\forall \epsilon > 0$ $\exists \delta>0$ such that for an arbitrary $\{x_i^1\}$ and $\{x_i^2\}$, $width(\{x_i^1\})<\delta$ and $width(\{x_i^2\})<\delta$ implies $|S_1-S_2|<\epsilon$.\\

Consider a regular partition $\{x_i\}$ such that $x_i = \frac{i}{N}$ for some $N$. Define sequences $\{x_i'\}$ and $\{x_i''\}$ such that $x_i \leq x_i' \leq x_{i+1}$ and $x_i \leq x_i'' \leq x_{i+1}$, and $S_1 = \sum_{i=1}^{i=N}f(x_i')\left(x_i-x_{i-1}\right)$ and $S_2 = \sum_{i=1}^{i=N}f(x_i'')\left(x_i-x_{i-1}\right)$. Note that:

\begin{align*}
  S_1 &= \sum_{i=1}^{i=N}f(x_i')(x_i-x_{i-1}) = \frac{1}{N}\sum_{i=1}^{i=N}x_i' \geq \frac{1}{N}\sum_{i=1}^{i=N}x_i\\
  &= \frac{1}{N^2}\sum_{i=1}^{i=N}i = \frac{N(N+1)}{2N^2} = \frac{1}{2} + \frac{1}{2N}\\
  S_2 &= \sum_{i=1}^{i=N}f(x_i'')(x_i-x_{i-1}) = \frac{-1}{N}\sum_{i=1}^{i=N}x_i'' \leq \frac{-1}{N}\sum_{i=1}^{i=N}x_i\\
  &= \frac{-1}{N^2}\sum_{i=1}^{i=N}i = \frac{-N(N+1)}{2N^2} = \frac{-1}{2} + \frac{-1}{2N}\\
\end{align*}

Thus $S_1 \geq \frac{1}{2} + \frac{1}{2N}$ and $S_2 \leq \frac{-1}{2} + \frac{-1}{2N}$, so $|S_1-S_2| = |1 + \frac{1}{N}|$. Note that the width of $\{x_i\}$ is given by $\frac{1}{N}$. Thus $\forall \epsilon>0$, $\exists \delta > 0$ such that $\frac{1}{N}<\delta$ implies $|S_1-S_2|<\epsilon$ for our chosen $S_1$ and $S_2$. Consider $\epsilon = \frac{1}{2}$, and the associated $\delta$. Consider $N$ such that $\frac{1}{N}<\delta$. Note that $|S_1-S_2|>1>\frac{1}{2}$, a contradiction. Thus $f(x)$ is not integrable.\\

\section{Problem 6.4.9}
%% Let $\delta=\frac{\epsilon}{2}$ and f, g be integrable functions. Then for all $\epsilon > 0$ there exist $\delta_f, \delta_g$ such that for partitions $\{x_i^f\}$ and $\{x_i^g\}$ where $S_f$ and $S_g$ are the Riemann sums using these partitions, then $|S_1 - \int_a^bf(x)dx|<\epsilon$ and $|S_2 - \int_a^bg(x)dx|<\epsilon$.

By Theorem 3.5 because $f:[a,b]\rightarrow\mathbb{R}$ and $g:[a,b]\rightarrow\mathbb{R}$ are continuous, they are both integrable. By the triangle inequality, $\forall x, |f(x) + g(x)| \leq |f(x) + |g(x)|$. By Corollary 2.3, $\int_a^b |f(x)+g(x)| \leq \int_a^b |f(x)|+|g(x)| = \int_a^b |f(x)| + \int_a^b |g(x)|$, as desired.\\

\section{Problem 3}
Because $f(x) = x$ is integrable, then for all $\epsilon > 0$, it is the case that for some $\delta > 0$, for all $\{x_i\}$, $max(x_i-x_{i-1})<\delta$, $|S - A| < \epsilon$. Consider $\{x_i\}$ such that $x_i = a + \frac{b-a}{N}i$ for an arbitrary $N$. Then $max(x_i-x_{i-1}) = \frac{b-a}{N}$. Consider the following algebraic manipulation, where $x_i' = x_i$ and $x_i$ is a regular partition and $a=0, b=1$:

\begin{align*}
  S &= \sum_{i=1}{i=N}f(x_i')(x_{i+1}-x_i)\\
  &= \sum_{i=1}{i=N}\left(a+\frac{b-a}{N}i\right)\left(a+\frac{b-a}{N}i - a -\frac{b-a}{N}(i-1)\right)\\
  &= \sum_{i=1}{i=N}\left(\frac{i}{N}\right)\left(\frac{1}{N}\right)\\
  &= \frac{1}{N^2}\sum_{i=1}{i=N}i\\
  &= \frac{1}{N^2}\frac{N(N+1)}{2} = \frac{1}{2} + \frac{1}{2N}\\
\end{align*}

Then $\lim_{N\rightarrow0} max(x_i-x_{i-1}) = 0$, so $\lim_{width(\{x_i\})} S = \frac{1}{2}$. Then $\int_0^1xdx = \frac{1}{2}$.\\

%% Now consider $A = \frac{1}{2}$. Then $|S - A| = |\frac{1}{2N}|$. Now for some arbitrary $\epsilon$ by the Archimedean Principle there exists some $N'$ such that $\frac{1}{2N'} < \epsilon$. Because $N$ is arbitrary, consider $N = N'$. Then for any $\delta > \frac{b - a}{N}$, $||$

\section{Problem 4}
We want to show that $f$ is integrable. That is, for all $\epsilon>0$ there exist some step functions $f_1:[a,b]\rightarrow\mathbb{R}$ and $f_2:[a,b]\rightarrow\mathbb{R}$ where $f_1(x) \leq f(x) \leq f_2(x)$ and $\int_a^b f_1(x) - f_2(x)dx \leq \epsilon$ for all $\epsilon$. We will do this by defining a partition and step functions such that the difference between $f_2$ and $f_1$ is always less than $\frac{\epsilon}{N}$, then showing that the integral of $f_1-f_2$ is zero. This proof works for a monotone increasing function, but would work equally well for a monotone decreasing function with minimal switching.\\

Define a partition $\{x_i\}$ on $[a,b]$. Define step functions $f_1$ and $f_2$ such that for each $x' \in [a,b]$ where $x_i \leq x' \leq x_{i+1}$, $f_1(x') = f(x_i)$ and $f_2(x') = f(x_{i+1})$. Define the partition $\{x_i\}$ such that $x_0 = a$ and $x_i$ is the value in the domain such that $f(x_i) = f(x_{i-1})+\frac{\epsilon}{N}$ for some $N$ and $\epsilon$, OR $b$, whichever is smaller. Note that this value must necessarily exist by the IVT, because $f$ is continuous on $[a,b]$ and $f(x_{i-1}) \leq f(x_i) \leq f(b)$ by f's positive monotonicity. Notice that $f(x_{i+1}) - f(x_i) \leq \frac{\epsilon}{N}$. Consider the following Riemann sum for $x_i \leq x'_i \leq x_{i+1}$:

\begin{align*}
  S &= \sum_{i=1}^{i=N} \left(f_1(x'_i) - f_2(x'_i)\right)\left(x_i-x_{i-1}\right)\\
  &\leq \sum_{i=1}^{i=N} \frac{\epsilon}{N}\left(x_i-x_{i-1}\right)\\
  &= \frac{\epsilon(b-a)}{N}\\
\end{align*}

Thus for arbitrary $\epsilon>0$ and $N > (b-a)$, $|S| < \epsilon$ and so $|S-0|<\epsilon$. Thus we can say $\int_a^bf_1(x)-f_2(x)dx = 0$. By extension $\int_a^bf_1(x)-f_2(x)dx < \epsilon$ for all $\epsilon>0$.

\end{document}

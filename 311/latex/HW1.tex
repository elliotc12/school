\documentclass[10pt]{article} % Font size - 10pt, 11pt or 12pt

\nonstopmode

\usepackage[hmargin=1.25cm, vmargin=1.5cm]{geometry} % Document margins

\usepackage{amsmath}

\usepackage[usenames,dvipsnames]{xcolor} % Allows the definition of hex colors

% Fonts and tweaks for XeLaTeX
\usepackage{fontspec,xltxtra,xunicode}
\defaultfontfeatures{Mapping=tex-text}
%\setmonofont[Scale=MatchLowercase]{Andale Mono}

% Colors for links, text and headings
\usepackage{hyperref}
\definecolor{linkcolor}{HTML}{506266} % Blue-gray color for links
\definecolor{shade}{HTML}{F5DD9D} % Peach color for the contact information box
\definecolor{text1}{HTML}{2b2b2b} % Main document font color, off-black
\definecolor{headings}{HTML}{701112} % Dark red color for headings
% Other color palettes: shade=B9D7D9 and linkcolor=A40000; shade=D4D7FE and linkcolor=FF0080

\hypersetup{colorlinks,breaklinks, urlcolor=linkcolor, linkcolor=linkcolor} % Set up links and colors

\usepackage{fancyhdr}
\usepackage{amssymb}
\pagestyle{fancy}
\fancyhf{}
% Headers and footers can be added with the \lhead{} \rhead{} \lfoot{} \rfoot{} commands
% Example footer:
%\rfoot{\color{headings} {\sffamily Last update: \today}. Typeset with Xe\LaTeX}

\renewcommand{\headrulewidth}{0pt} % Get rid of the default rule in the header

\usepackage{titlesec} % Allows creating custom \section's

% Format of the section titles
\titleformat{\section}{\color{headings}
\scshape\Large\raggedright}{}{0em}{}[\color{black}\titlerule]

\title{Advanced Calc Homework \large 1: 1.1: 1,3,5,7,8,11,17, 1.2: 1,4}
\author{Elliott Capek}
\titlespacing{\section}{0pt}{0pt}{5pt} % Spacing around titles

\begin{document}

\maketitle{}

\section{Problem 1.1.1}
\textbf{A} False, since $1$ isn't irrational.\\
\textbf{B} False, since $\sqrt{2}$ is irrational, and thus $2$ cannot be inductive stepped to from $1$.\\
\textbf{C} False. Consider $a = \sqrt{2}$ and $b = 2 - \sqrt{2}$. Both are irrational, since the sum of a rational and irrational is irrational, but their sum is a rational.\\
\textbf{D} False for a similar reason: consider $\pi$ and $\frac{1}{\pi}$. Both irrational since rational and irrational numbers combine via multiplication to be irrational, but have a rational product.\\
\textbf{E} True. ISOAC, let $n^2$ be odd but $n$ even. Thus $\exists k\in\mathbb{N} st n=2k$, so $n^2 = 4k^2 = 2*2k^2$. This is a contradiction, since $n^2$ was supposed to be odd.\\

\section{Problem 1.1.3}
\textbf{WTS} For any natural number $n$:

\begin{equation*}
  \sum_{j=1}^{n} j^2 = \frac{n(n+1)(2n+1)}{6}
\end{equation*}

\textbf{Proof}
We will prove this using the Principle of Induction. We will first show that the statement S(k), where S(k) is the above equality with $n=k$, is true for $S(1)$, then show that $S(k)$ being true implies $S(k+1)$ is true.\\

\textbf{Base case k=1}
Notice that for $k=1$:

\begin{align*}
  &\sum_{j=1}^{1} j^2 = 1\\
  &\frac{1(1+1)(2+1)}{6} = \frac66 = 1\\
\end{align*}

Thus the base case holds, so $S(1)$ is true.\\

\textbf{Inductive step}
Now we let S(k) be true, so:

\begin{equation*}
  \sum_{j=1}^{k} j^2 = \frac{k(k+1)(2k+1)}{6}
\end{equation*}

We want to show S(m) is true, where $m=k+1$. Add $(k+1)^2$ to both sides and do algebra:

\begin{align*}
  \sum_{j=1}^{k} j^2 + (k+1)^2 &= \frac{k(k+1)(2k+1)}{6} + (k+1)^2\\
  \sum_{j=1}^{m} j^2 &= \frac{k(k+1)(2k+1)}{6} + k^2 + 2k + 1\\
  &= \frac{2k^3+3k^2+k}{6} + \frac{6k^2 + 12k + 6}{6}\\
  &= \frac{2k^3+8k^2+13k+6}{6}\\
  &= \frac{(k+1)(2k^2+7k+6)}{6}\\
  &= \frac{(k+1)(k+2)(2k+3)}{6}\\
  &= \frac{m(m+1)(2m+1)}{6}\\
\end{align*}

thus $S(k+1)$. We have shown the base case S(1) and inductive step $S(k)$ implies $S(k+1)$, so the statement $S(n)$ must be true for all $n\in \mathbb{N}$.\\

\section{Problem 1.1.5}
Before we do this proof, we'll need to prove a subclaim:\\
\textbf{WTS} S(k): $(1 + 2 + ... + k) = \frac{k(k+1)}{2}$\\

\textbf{Proof} We will use the Principle of Induction to prove the above claim for all S(k). First, the base case: S(1): $1 = \frac{1(2)}{2} = 1$. Then the inductive step:

\begin{align*}
  (1 + 2 + ... + k) &= \frac{k(k+1)}{2}\\
  1 + 2 + ... + k + k + 1 &= \frac{k(k+1)}{2} + k + 1\\
  &= \frac{k^2 + 3k + 2}{2}\\
  &= \frac{(k+1)(k+2)}{2}\\
\end{align*}

thus $S(k) \rightarrow S(k+1)$, so S(n) is true for all $n \in \mathbb{N}$. Now to prove the problem claim.\\

\textbf{WTS} S(k): $(1 + 2 + ... + k)^2 = 1^3 + 2^3 + ... + k^3$\\

\textbf{Proof} We will use the Principle of Induction to prove that S(k) is true for all $k \in \mathbb{N}$.\\

\textbf{Base case:} $1^2 = 1^3 = 1$.\\

\textbf{Inductive step:} Let $(1 + 2 + ... + k)^2 = 1^3 + 2^3 + ... + k^3$. Then by algebra:

\begin{align*}
  (1 + 2 + ... + k + k + 1)^2 &= (1 + 2 + ... + k)(1 + 2 + ... + k + k + 1) + (k + 1)(1 + 2 + ... + k + k + 1)\\
  &= (1 + 2 + ... + k)^2 + 2(k+1)(1 + 2 + ... + k) + (k + 1)^2\\
  &= (1 + 2 + ... + k)^2 + 2(k+1)\frac{k(k+1)}{2} + (k + 1)^2\\
  &= (1 + 2 + ... + k)^2 + k(k+1)^2 + (k + 1)^2\\
  &= (1 + 2 + ... + k)^2 + (k+1)^3\\
  &= 1^3 + 2^3 + ... + k^3 + (k+1)^3\\
\end{align*}

as desired. Thus since S(1) and $S(k) \rightarrow S(k+1)$, the claim S(n) must be true for all $n \in \mathbb{N}$.\\


\section{Problem 1.1.7}
\textbf{WTS} $n \in \mathbb{N}$ implies $n-1 \in \mathbb{N}$\\

\textbf{Proof} First we will prove that set $S = {n | n = 1 \mbox{ or } n, n-1 \in \mathbb{N}}$ is inductive. First note that $S$ holds the base case property, since $1 \in S$. Also note that if $s = n - 1$, then $s \in S \rightarrow s + 1 \in S$ by the requirement that both $n$ and $n+1$ be in S simultaneously. Thus S is an inductive set.\\

Because set $S$ is inductive, if $n \in \mathbb{N}$ and $n > 1$ then $n \in S$, and also $n-1 \in S$ by definition of membership in $S$. Thus $n > 1 \in \mathbb{N} \rightarrow n-1\in\mathbb{N}$.

\section{Problem 1.1.8}
\textbf{WTS} For $n,m \in \mathbb{N}$ where $n > m$, $n-m \in \mathbb{N}$.

\textbf{Proof} We will prove this using the Principle of Induction. We will show that the statement $S(k)$ is true for all $k\in\mathbb{N}$, where S(k) is the statement $n-k\in\mathbb{N}$.

\textbf{Base case} The base case S(1) states that for $n\in\mathbb{N}$, $n-1\in{mathbb}$. This is true by Problem 1.1.7.\\

\textbf{Inductive step} Statement S(k) says that for $n\in\mathbb{N}$, $n-k\in\mathbb{N}$. Note that, by Problem 1.1.7, $n-k-1\in\mathbb{N}$.\\

Thus both $S(1)$ and $S(k)\rightarrow S(k+1)$ are true. Thus $S(n)$ is true for all $n\in\mathbb{N}$, so for $n,m \in \mathbb{N}$ where $n > m$, $n-m \in \mathbb{N}$.\\

\section{Problem 1.1.11a}
\textbf{WTS} If $a \in \mathbb{Q}$ and $b \in \mathbb{R}\setminus \mathbb{Q}$, then $a+b \in \mathbb{R}\setminus \mathbb{Q}$.

\textbf{Proof} In search of a contradiction, assume for $a \in \mathbb{Q}$ and $b \in \mathbb{R}\setminus \mathbb{Q}$, that $a+b \in \mathbb{Q}$. Then by the definition of rational numbers there exist $m,n \in \mathbb{Z}$ such that $a+b = \frac{m}{n}$. $a$ can be similarly expanded using $x,y \in \mathbb{Z}: \frac{x}{y}+b = \frac{m}{n}$. We will perform algebra and show this leads to a contradiction:

\begin{align*}
  &\frac{x}{y}+b = \frac{m}{n}\\
  &b = \frac{m}{n} - \frac{x}{y}\\
  &b = \frac{my}{ny} - \frac{xn}{ny}\\
  &b = \frac{my-xn}{ny}\\
\end{align*}

Because the product and difference of two integers is also an integer, this means b can be expressed as a fraction of two integers. This is the desired contradiction. Thus, if $a \in \mathbb{Q}$ and $b \in \mathbb{R}\setminus \mathbb{Q}$, then $a+b \in \mathbb{R}\setminus \mathbb{Q}$.

\section{Problem 1.1.11b}
\textbf{WTS} If $a \in \mathbb{Q}$ and $b \in \mathbb{R}\setminus \mathbb{Q}$, then $ab \in \mathbb{R}\setminus \mathbb{Q}$.

\textbf{Proof} In search of a contradiction, assume for $a \in \mathbb{Q}$ and $b \in \mathbb{R}\setminus \mathbb{Q}$, that $ab \in \mathbb{Q}$. Then by the definition of rational numbers there exist $m,n \in \mathbb{Z}$ such that $ab = \frac{m}{n}$. $a$ can be similarly expanded using $x,y \in \mathbb{Z}: \frac{x}{y}b = \frac{m}{n}$. This can be rewritten as $b = \frac{my}{nx}$. Because the product of two integers is itself an integer, we have expressed b as a fraction of two integers. This is the desired contradiction. Thus, if $a \in \mathbb{Q}$ and $b \in \mathbb{R}\setminus \mathbb{Q}$, then $ab \in \mathbb{R}\setminus \mathbb{Q}$.

\section{Problem 1.1.17}
Consider set $S = \{x | x \in \mathbb{R}, x \geq 0, x^2 < c\}$.

\textbf{A}\\
We wish to show that $\forall s \in S$, $s \leq c + 1$. Consider $s \in S$. If $s > 1$ then, by multiplying by s, $s^2 > s$. Thus by the definition of S, $s < s^2 < c$ and so $s < c+1$. If $s \leq 1$, then since $c$ must be positive, $c + 1 > 1$ and so $s < c + 1$.\\

Then, by the Completeness Axiom, there must exist some $b \in R$ such that $b = Sub(S)$.\\

\textbf{D}\\
We know $b^2 \leq c$, so $b^2 - c$ cannot be in $\mathbb{P}$. We also know $b^2 \geq c$, so $b^2 - c$ cannot be in $-\mathbb{P}$. Thus by the second Positivity axiom, $b^2 = c$.\\


\section{Problem 1.2.1}
\textbf{A} False. Consider real numbers $1.6$ and $1.8$. There are no integers between these two, thus the integers are not dense.\\
\textbf{B} False. Consider real numbers $-2.0 and -1.0$. There are no positive reals between these two, and thus the positive reals are not dense.\\
\textbf{C} True. The rational numbers are dense in the reals. For reals a, b, where $a > b$, by the Archimedean principle there exists a number $n \in \mathbb{N}$ such that $\frac{1}{n} \leq a-b$. Then $b + \frac{1}{n}$ is between the two. If it is an integer, then $b + \frac{1}{n^2}$ shouldn't be, and is also between the two.


\section{Problem 1.2.4}
\textbf{A} Set $\{1/n | n \in \mathbb{N}\}$\\
\textit{Maximum} - 1, all $n > 1$ are smaller.\\
\textit{Minimum} - None. For any $1/n$ in the set, $1/(n+1)$ is smaller.\\
\textit{Supremum} - 1. $1/1$ is the largest value in the set.\\
\textit{Infimum} - 0. Any number $\epsilon > 0$ has a number $1\m$ smaller than it, by the Archimedean Principle, so 0 is the largest lower bound.\\

\textbf{B} Set $\{x | x \in \mathbb{R} \mbox{ and } x^2 < 2\}$\\
\textit{Maximum} - None. For any number $e < \sqrt{2}$, $e + \frac{\sqrt{2}-e}{2}$ is bigger\\
\textit{Minimum} - Same argument with $-\sqrt{2}$.\\
\textit{Supremum} - $\sqrt{2}$. Anything smaller is in the set, and it isn't in the set.\\
\textit{Infimum} - $-\sqrt{2}$ by the same argument.\\
\end{document}

\documentclass[10pt]{article} % Font size - 10pt, 11pt or 12pt

\nonstopmode

\usepackage[hmargin=1.25cm, vmargin=1.5cm]{geometry} % Document margins

\usepackage{amsmath}

\usepackage[usenames,dvipsnames]{xcolor} % Allows the definition of hex colors

% Fonts and tweaks for XeLaTeX
\usepackage{fontspec,xltxtra,xunicode}
\defaultfontfeatures{Mapping=tex-text}
%\setmonofont[Scale=MatchLowercase]{Andale Mono}

% Colors for links, text and headings
\usepackage{hyperref}
\definecolor{linkcolor}{HTML}{506266} % Blue-gray color for links
\definecolor{shade}{HTML}{F5DD9D} % Peach color for the contact information box
\definecolor{text1}{HTML}{2b2b2b} % Main document font color, off-black
\definecolor{headings}{HTML}{701112} % Dark red color for headings
% Other color palettes: shade=B9D7D9 and linkcolor=A40000; shade=D4D7FE and linkcolor=FF0080

\hypersetup{colorlinks,breaklinks, urlcolor=linkcolor, linkcolor=linkcolor} % Set up links and colors

\usepackage{fancyhdr}
\usepackage{amssymb}
\pagestyle{fancy}
\fancyhf{}
% Headers and footers can be added with the \lhead{} \rhead{} \lfoot{} \rfoot{} commands
% Example footer:
%\rfoot{\color{headings} {\sffamily Last update: \today}. Typeset with Xe\LaTeX}

\renewcommand{\headrulewidth}{0pt} % Get rid of the default rule in the header

\usepackage{titlesec} % Allows creating custom \section's

% Format of the section titles
\titleformat{\section}{\color{headings}
\scshape\Large\raggedright}{}{0em}{}[\color{black}\titlerule]

\title{Advanced Calc Homework 2 \large 1.3.4, 1.3.7, 1.3.14, 2.1.2a, 2.1.6, 2.1.7, 2.1.15a}
\author{Elliott Capek}
\titlespacing{\section}{0pt}{0pt}{5pt} % Spacing around titles

\begin{document}

\maketitle{}

\section{Problem 1.3.14}
We want to show the relation $ab \leq \frac12\left(a^2+b^2\right)$. for any $a,b\in\mathbb{R}$\\

Note that since the square of any real number is nonnegative, $\left(a-b\right) \geq 0$. Squaring both sides yields $a^2 + b^2 - 2ab \geq 0$, and so $ab \leq \frac12\left(a^2+b^2\right$, as desired.\\

\section{Problem 2.1.2a}
We want to show $\forall \epsilon,$ $\exists N$ such that $\forall n \geq N,$ $\frac{1}{\sqrt{n}} < \epsilon$ using the Archimedian principle.\\

By the Archimedian principle, $\exists c$ such that $\frac{1}{c} < c$. Pick a $c$ with this property, and let $N = c^2$. Then $\frac{1}{\sqrt{N}} < \epsilon$. Note that $\forall n \geq N$, $\frac{1}{\sqrt{n}} \leq \frac{1}{\sqrt{N}}$. Thus $\frac{1}{\sqrt{n}} < \epsilon$, as desired.\\

\section{Problem 2.1.6}
We want to show that $\exists N$ such that $\forall n \geq N$, $a_n > 0$.\\

We are given that $\{a_n\}$ converges to $a$. Consider the $N$ such that $\forall n \geq N$, $\left|a_n - a\right| < a$. By Proposition 1.2, $\left|x - a\right| < r$ and $a-r < x < a+r$ are equivalent, thus $0 < a_n < 2a$. Thus $a_n > 0$, as desired.\\

\section{Problem 2.1.7}
We want to show that $\{b_n\}$ converges to $\ell$.\\

We are given that $\exists N$ such that $\forall n \geq N$, $a_n = b_n$. Consider the following algebraic manipulation:

\begin{align*}
  a_n &= b_n\\
  a_n - \ell &= b_n - \ell\\
  \left|a_n - \ell\right| &= \left|b_n - \ell\right|\\
  c\left|a_n - \ell\right| &= \left|b_n - \ell\right|\hspace{1cm}\mbox{c=1}\\
  \left|b_n - \ell\right| &\leq c\left|a_n - \ell\right|\\
\end{align*}

Thus, by the Comparison Lemma, $\{b_n\}$ converges to $\ell$ as desired.\\

\end{document}

\documentclass[10pt]{article} % Font size - 10pt, 11pt or 12pt

\nonstopmode

\usepackage[hmargin=1.25cm, vmargin=1.5cm]{geometry} % Document margins

\usepackage{amsmath}

\usepackage[usenames,dvipsnames]{xcolor} % Allows the definition of hex colors

% Fonts and tweaks for XeLaTeX
\usepackage{fontspec,xltxtra,xunicode}
\defaultfontfeatures{Mapping=tex-text}
%\setmonofont[Scale=MatchLowercase]{Andale Mono}

% Colors for links, text and headings
\usepackage{hyperref}
\definecolor{linkcolor}{HTML}{506266} % Blue-gray color for links
\definecolor{shade}{HTML}{F5DD9D} % Peach color for the contact information box
\definecolor{text1}{HTML}{2b2b2b} % Main document font color, off-black
\definecolor{headings}{HTML}{701112} % Dark red color for headings
% Other color palettes: shade=B9D7D9 and linkcolor=A40000; shade=D4D7FE and linkcolor=FF0080

\hypersetup{colorlinks,breaklinks, urlcolor=linkcolor, linkcolor=linkcolor} % Set up links and colors

\usepackage{fancyhdr}
\usepackage{amssymb}
\pagestyle{fancy}
\fancyhf{}
% Headers and footers can be added with the \lhead{} \rhead{} \lfoot{} \rfoot{} commands
% Example footer:
%\rfoot{\color{headings} {\sffamily Last update: \today}. Typeset with Xe\LaTeX}

\renewcommand{\headrulewidth}{0pt} % Get rid of the default rule in the header

\usepackage{titlesec} % Allows creating custom \section's

% Format of the section titles
\titleformat{\section}{\color{headings}
\scshape\Large\raggedright}{}{0em}{}[\color{black}\titlerule]

\title{Advanced Calc Homework 2\\ \large 1.3.4, 1.3.7, 1.3.14, 2.1.2a, 2.1.6, 2.1.7, 2.1.15a}
\author{Elliott Capek}
\titlespacing{\section}{0pt}{0pt}{5pt} % Spacing around titles

\begin{document}

\maketitle{}

\section{Problem 1.3.2}
For $a > 0$, we want to show that $|x - a| < \frac a2$ then $x > \frac a2$.\\

Let $|x - a| < \frac a2$. Then by Proposition 1.2 this is equivalent to statement $a - \frac a2 < x < a + \frac{3a}{2}$, thus $x > \frac a2$, as desired.\\

\section{Problem 1.3.7}
Starting with statement $(a+b) + (-b)$, apply the triangle inequality to get $|a+b| + |-b| \geq |a + b -b| = |a|$. By rearranging terms, we get $|a| - |b| \leq |a+b|$.\\

Now interchanging $a$ and $b$ yields $|b| - |a| \leq |b+a|$. By algebra, $-\left(|a| - |b|\right) \leq |a+b|$. Then $\left||a| - |b|\right| \leq |a+b|$.\\

\section{Problem 1.3.14}
We want to show the relation $ab \leq \frac12\left(a^2+b^2\right)$. for any $a,b\in\mathbb{R}$\\

Note that since the square of any real number is nonnegative, $\left(a-b\right) \geq 0$. Squaring both sides yields $a^2 + b^2 - 2ab \geq 0$, and so $ab \leq \frac12\left(a^2+b^2\right)$, as desired.\\

\section{Problem 2.1.2a}
We want to show $\forall \epsilon,$ $\exists N$ such that $\forall n \geq N,$ $\frac{1}{\sqrt{n}} < \epsilon$ using the Archimedian principle.\\

By the Archimedian principle, $\exists c$ such that $\frac{1}{c} < c$. Pick a $c$ with this property, and let $N = c^2$. Then $\frac{1}{\sqrt{N}} < \epsilon$. Note that $\forall n \geq N$, $\frac{1}{\sqrt{n}} \leq \frac{1}{\sqrt{N}}$. Thus $\frac{1}{\sqrt{n}} < \epsilon$, as desired.\\

\section{Problem 2.1.6}
We want to show that $\exists N$ such that $\forall n \geq N$, $a_n > 0$.\\

We are given that $\{a_n\}$ converges to $a$. Consider the $N$ such that $\forall n \geq N$, $\left|a_n - a\right| < a$. By Proposition 1.2, $\left|x - a\right| < r$ and $a-r < x < a+r$ are equivalent, thus $0 < a_n < 2a$. Thus $a_n > 0$, as desired.\\

\newpage

\section{Problem 2.1.7}
We want to show that $\{b_n\}$ converges to $\ell$.\\

We are given that $\exists N$ such that $\forall n \geq N$, $a_n = b_n$. Consider the following algebraic manipulation:

\begin{align*}
  a_n &= b_n\\
  a_n - \ell &= b_n - \ell\\
  \left|a_n - \ell\right| &= \left|b_n - \ell\right|\\
  c\left|a_n - \ell\right| &= \left|b_n - \ell\right|\hspace{1cm}\mbox{c=1}\\
  \left|b_n - \ell\right| &\leq c\left|a_n - \ell\right|\\
\end{align*}

Thus, by the Comparison Lemma, $\{b_n\}$ converges to $\ell$ as desired.\\

\section{Problem 2.1.15a}
We want to show the polynomial $a_n = n^3 - 4n^2 - 100n$ converge to infinity, or that $\forall c \in \mathbb{R}, c> 0$, $\exists N$ such that $\forall n \geq N$, $a_n > c$.\\

Consider $a_n$ and its derivatives:\\

\begin{align*}
  a_n &= n^3 - 4n^2 - 100n\\
  a_n' &= 3n^2 - 8n - 100\\
  a_n'' &= 6n - 8\\
  a_n''' &= 6\\
  a_n'''' &= 0
\end{align*}

Notice that, at $n=100$, it is the case that $a_n >0$, $a_n' > 0$, $a_n'' > 0$, $a_n''' > 0$ and $a_n'''' = 0$. Because the third derivative is a constant positive number, $a_n$ where $n > 100$ will always increase at an increasing rate. Thus for any $c < 100$, $N = 100$ satisfies the claim.\\

Now consider for any $c > 100$, because $a_n$ is increasing at a steadily increasing rate, there must exist some $n'$ such that $a_{n'} = c$. Now for all $n > n'$, $a_n > c$, so the claim is also satisfied for $c > 100$. Thus $a_n$ converges at infinity.\\

\end{document}

\documentclass[10pt]{article} % Font size - 10pt, 11pt or 12pt

\nonstopmode

\usepackage[hmargin=1.25cm, vmargin=1.5cm]{geometry} % Document margins

\usepackage{amsmath}

\usepackage[usenames,dvipsnames]{xcolor} % Allows the definition of hex colors

% Fonts and tweaks for XeLaTeX
\usepackage{fontspec,xltxtra,xunicode}
\defaultfontfeatures{Mapping=tex-text}
%\setmonofont[Scale=MatchLowercase]{Andale Mono}

% Colors for links, text and headings
\usepackage{hyperref}
\definecolor{linkcolor}{HTML}{506266} % Blue-gray color for links
\definecolor{shade}{HTML}{F5DD9D} % Peach color for the contact information box
\definecolor{text1}{HTML}{2b2b2b} % Main document font color, off-black
\definecolor{headings}{HTML}{701112} % Dark red color for headings
% Other color palettes: shade=B9D7D9 and linkcolor=A40000; shade=D4D7FE and linkcolor=FF0080

\hypersetup{colorlinks,breaklinks, urlcolor=linkcolor, linkcolor=linkcolor} % Set up links and colors

\usepackage{fancyhdr}
\usepackage{amssymb}
\pagestyle{fancy}
\fancyhf{}
% Headers and footers can be added with the \lhead{} \rhead{} \lfoot{} \rfoot{} commands
% Example footer:
%\rfoot{\color{headings} {\sffamily Last update: \today}. Typeset with Xe\LaTeX}

\renewcommand{\headrulewidth}{0pt} % Get rid of the default rule in the header

\usepackage{titlesec} % Allows creating custom \section's

% Format of the section titles
\titleformat{\section}{\color{headings}
\scshape\Large\raggedright}{}{0em}{}[\color{black}\titlerule]

\title{Advanced Calc Homework 6}
\author{Elliott Capek}
\titlespacing{\section}{0pt}{0pt}{5pt} % Spacing around titles

\begin{document}

\maketitle{}

\section{Problem 2}
\textbf{Prove that $\lim_{n\rightarrow\infty}a_n = a$ if and only if the sequence $a_1, a, a_2, a, a_3, a,...$ is convergent.}

To prove this, we will need to show \textbf{A.)} that $\lim_{n\rightarrow\infty}a_n = a$ implies $a_1, a, a_2, a, a_3, a,...$ is convergent and \textbf{B.)} that $a_1, a, a_2, a, a_3, a,...$ is convergent implies $\lim_{n\rightarrow\infty}a_n = a$.\\

\textbf{A.)}\\
Let $\lim_{n\rightarrow\infty}a_n = a$. Then $\forall \epsilon>0$ $\exists N$ such that $\forall n \geq N$, $|a_n-a|<\epsilon$. Consider sequence $A = \{a_1, a, a_2, a, a_3, ...\}$. Note that $|a-a|<\epsilon$. Consider $J$ such that $a_N = A_J$, which necessarily must exist since $A \subseteq$. Note that $\forall j>J$, either $A_j = a$, in which case $|A_j-a|<\epsilon$, or $A_j \in \{a_n\}$ such that $A_j = a_n$ where $n > N$, thus $|A_j-a|<\epsilon$. Thus $A$ is convergent.\\

\textbf{B.) $a_1, a, a_2, a, a_3, a,...$ is convergent implies $\lim_{n\rightarrow\infty}a_n = a$}\\
Assume $A = \{a_1, a, a_2, a, a_3, ...\}$ converges. We want to show $\lim_{n\rightarrow\infty}A_n = a$. In search of a contradiction, let $A$ converge to $a + b$ for some $b \neq 0$. Then for $\epsilon > b$, $\exists N$ such that $\forall n > N$, $|A_N-a-b|<\epsilon$. However, note that $\exists n > N$ such that $a = A_n$, ths $|a - a - b| = |b| < \epsilon$. This is a contradiction. Thus, $\lim_{n\rightarrow\infty} A = a$.\\

Now $\forall \epsilon>0$, $\exists J$ such that $\forall j > J$, $|A_J - a|<\epsilon$. Note that by construction, $\exists M$ such that $a_M = A_J$ or $a_M = A_{J+1}$. Then $\forall m > M$, $|a_M - a|<\epsilon$. Thus $\lim_{n\rightarrow\infty}\{a_n\} = a$.\\

\section{Problem 3}
We want to show

\[
f(x) =
\begin{cases}
  0 & x <    0 \\
  x & x \geq 0 \\
\end{cases}
\]

is continuous at $x_0 = 0$. This entails showing that $\forall \epsilon>0$, $\exists \delta>0$ such that $|x-x_0|<\delta$ implies $|f(x)-f(x_0)|<\epsilon$. Consider the following manipulation, where $x_0 = 0$:

\begin{align*}
  |f(x) - f(x_0)| &= |f(x)| \leq |x| = |x-x_0|
\end{align*}

Thus $|x-x_0|<\epsilon$ implies $|f(x)-f(x_0)|<\epsilon$. Thus $f(x)$ is continuous.\\

\section{Problem 5} \textbf{Prove that if p(x) is a polynomial of degree $n$, then there are most $n$ solutions to the equation $p(x) = 0$.}\\
We will do this by induction. First, we will show the $k=1$ case where $p^{n-1}(x) = 0$ has at most one solution. Then we will show the inductive step $k(i)$ implies $k(i+1)$, that $p^{i}(x) = 0$ has $n$ solutions implies $p^{i-1}(x) = 0$ has at most $n+1$ solutions.\\

\textbf{Base case}\\
Consider degree-n polynomial $p(x)$. Consider the $n-1$th derivative, $p^{n-1}(x)=n\!x+c$. $$p^{n-1}(x)=0$ has at least one solution; consider $x=\frac{-c}{n}$. We want to show this is a unique solution. In search of a contradiction, consider another solution $x_0 \neq \frac{-c}{n}$ such that $p^{n-1}(x_0) = 0$. $p^{n-1}(x)$ is a polynomial, and thus is everywhere differentiable and continuous. Then by Rolle's Theorem $\exists x'$ such that, if $x_0>\frac{-c}{n}$ $x'\in\(\frac{-c}{n},x_0\)$ or if $x_0<\frac{-c}{n}$ $x'\in\(x_0,\frac{-c}{n}\)$ and $p^{n}(x)=0$. However, $p^{n}(x)=n\!$. This is a contradiction. Thus, $p^{n-1}(x)$ has exactly one solution.\\

\textbf{Inductive step}\\


\end{document}

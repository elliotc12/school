\documentclass[10pt]{article} % Font size - 10pt, 11pt or 12pt

\nonstopmode

\usepackage[hmargin=1.25cm, vmargin=1.5cm]{geometry} % Document margins

\usepackage{amsmath}

\usepackage[usenames,dvipsnames]{xcolor} % Allows the definition of hex colors

% Fonts and tweaks for XeLaTeX
\usepackage{fontspec,xltxtra,xunicode}
\defaultfontfeatures{Mapping=tex-text}
%\setmonofont[Scale=MatchLowercase]{Andale Mono}

% Colors for links, text and headings
\usepackage{hyperref}
\definecolor{linkcolor}{HTML}{506266} % Blue-gray color for links
\definecolor{shade}{HTML}{F5DD9D} % Peach color for the contact information box
\definecolor{text1}{HTML}{2b2b2b} % Main document font color, off-black
\definecolor{headings}{HTML}{701112} % Dark red color for headings
% Other color palettes: shade=B9D7D9 and linkcolor=A40000; shade=D4D7FE and linkcolor=FF0080

\hypersetup{colorlinks,breaklinks, urlcolor=linkcolor, linkcolor=linkcolor} % Set up links and colors

\usepackage{fancyhdr}
\usepackage{amssymb}
\pagestyle{fancy}
\fancyhf{}
% Headers and footers can be added with the \lhead{} \rhead{} \lfoot{} \rfoot{} commands
% Example footer:
%\rfoot{\color{headings} {\sffamily Last update: \today}. Typeset with Xe\LaTeX}

\renewcommand{\headrulewidth}{0pt} % Get rid of the default rule in the header

\usepackage{titlesec} % Allows creating custom \section's

% Format of the section titles
\titleformat{\section}{\color{headings}
\scshape\Large\raggedright}{}{0em}{}[\color{black}\titlerule]

\title{Advanced Calc Homework 6}
\author{Elliott Capek}
\titlespacing{\section}{0pt}{0pt}{5pt} % Spacing around titles

\begin{document}

\maketitle{}

\section{Problem 1}
Because $b>1$, by the Archimedean Property, $\exists n \in \mathbb{N}$ such that $b > 1 - \frac{1}{n}$. Now consider:\\

\begin{align*}
  b^n &\geq \left(1 + \frac{1}{n}\right)^n\\
  &= \sum_{i=0}^{n}{n \choose i}\frac{1}{n^i}\\
  &= \sum_{i=2}^{n}{n \choose i}\frac{1}{n^i} + 2
\end{align*}

Thus, since all components of $\sum_{i=2}^{n}{n \choose i}\frac{1}{n^i}$ are positive, $b^n \geq 2$. Thus for any $x>2$, $\exists m \in \mathbb{N}$ such that $m > log_2(x)$ where $b^{n+m} \in \left(b, b^2, b^3, ...\right)$ and $b^{n+m}>x$. Thus, $\left(b, b^2, b^3, ...\right)$ is not bounded from above.\\

\section{Problem 2}
\textbf{Prove that $\lim_{n\rightarrow\infty}a_n = a$ if and only if the sequence $a_1, a, a_2, a, a_3, a,...$ is convergent.}

To prove this, we will need to show \textbf{A.)} that $\lim_{n\rightarrow\infty}a_n = a$ implies $a_1, a, a_2, a, a_3, a,...$ is convergent and \textbf{B.)} that $a_1, a, a_2, a, a_3, a,...$ is convergent implies $\lim_{n\rightarrow\infty}a_n = a$.\\

\textbf{A.)}\\
Let $\lim_{n\rightarrow\infty}a_n = a$. Then $\forall \epsilon>0$ $\exists N$ such that $\forall n \geq N$, $|a_n-a|<\epsilon$. Consider sequence $A = \{a_1, a, a_2, a, a_3, ...\}$. Note that $|a-a|<\epsilon$. Consider $J$ such that $a_N = A_J$, which necessarily must exist since $A \subseteq$. Note that $\forall j>J$, either $A_j = a$, in which case $|A_j-a|<\epsilon$, or $A_j \in \{a_n\}$ such that $A_j = a_n$ where $n > N$, thus $|A_j-a|<\epsilon$. Thus $A$ is convergent.\\

\textbf{B.) $a_1, a, a_2, a, a_3, a,...$ is convergent implies $\lim_{n\rightarrow\infty}a_n = a$}\\
Assume $A = \{a_1, a, a_2, a, a_3, ...\}$ converges. We want to show $\lim_{n\rightarrow\infty}A_n = a$. In search of a contradiction, let $A$ converge to $a + b$ for some $b \neq 0$. Then for $\epsilon > b$, $\exists N$ such that $\forall n > N$, $|A_N-a-b|<\epsilon$. However, note that $\exists n > N$ such that $a = A_n$, ths $|a - a - b| = |b| < \epsilon$. This is a contradiction. Thus, $\lim_{n\rightarrow\infty} A = a$.\\

Now $\forall \epsilon>0$, $\exists J$ such that $\forall j > J$, $|A_J - a|<\epsilon$. Note that by construction, $\exists M$ such that $a_M = A_J$ or $a_M = A_{J+1}$. Then $\forall m > M$, $|a_M - a|<\epsilon$. Thus $\lim_{n\rightarrow\infty}\{a_n\} = a$.\\

\section{Problem 3}
We want to show

\[
f(x) =
\begin{cases}
  0 & x <    0 \\
  x & x \geq 0 \\
\end{cases}
\]

is continuous at $x_0 = 0$. This entails showing that $\forall \epsilon>0$, $\exists \delta>0$ such that $|x-x_0|<\delta$ implies $|f(x)-f(x_0)|<\epsilon$. Consider the following manipulation, where $x_0 = 0$. Note that $|f(x)|\leq x$ for all $x$.\\

\begin{align*}
  |f(x) - f(x_0)| &= |f(x)| \leq |x| = |x-x_0|
\end{align*}

Thus $|x-x_0|<\epsilon$ implies $|f(x)-f(x_0)|<\epsilon$. Thus $f(x)$ is continuous.\\

\section{Problem 4}
Consider the following algebraic manipulation:

\begin{align*}
  \lim_{x\rightarrow0}\frac{f(x)}{g(x)} &= \lim_{x\rightarrow0}\frac{f(x)-f(0)}{g(x)-g(0)}\\
  &= \lim_{x\rightarrow0}\frac{\frac{f(x)-f(0)}{x}}{\frac{g(x)-g(0)}{x}}\\
  &= \frac{f'(0)}{g'(0)}\\
\end{align*}

Note that because $f'$ and $g'$ are continuous at zero, their limits can be taken:

\begin{align*}
  \lim_{x\rightarrow0}\frac{f(x)}{g(x)} &= \lim_{x\rightarrow0}\frac{f'(x)}{g'(x)}\\
\end{align*}

\section{Problem 5} \textbf{Prove that if p(x) is a polynomial of degree $n$, then there are most $n$ solutions to the equation $p(x) = 0$.}\\
We will do this by induction. First, we will show the $k=1$ case where $p^{n-1}(x) = 0$ has at most one solution. Then we will show the inductive step $k(i)$ implies $k(i+1)$, that $p^{i}(x) = 0$ has $j$ solutions implies $p^{i-1}(x) = 0$ has at most $j+1$ solutions.\\

\textbf{Base case}\\
Consider degree-n polynomial $p(x)$. Consider the $n-1$th derivative, $p^{n-1}(x)=n!x+c$. $p^{n-1}(x)=0$ has at least one solution; consider $x=\frac{-c}{n}$. We want to show this is a unique solution. In search of a contradiction, consider another solution $x_0 \neq \frac{-c}{n}$ such that $p^{n-1}(x_0) = 0$. $p^{n-1}(x)$ is a polynomial, and thus is everywhere differentiable and continuous. Then by Rolle's Theorem $\exists x'$ such that, if $x_0>\frac{-c}{n}$ $x'\in(\frac{-c}{n},x_0)$ or if $x_0<\frac{-c}{n}$ $x'\in(x_0,\frac{-c}{n})$ and $p^{n}(x)=0$. However, $p^{n}(x)=n!$. This is a contradiction. Thus, $p^{n-1}(x)$ has exactly one solution.\\

\textbf{Inductive step}\\
We now want to show that if $p^{i}(x)=0$ has $n$ unique solutions, then $p^{i-1}(x)=0$ has at most $n+1$ unique solutions. Let $\{a_1, a_2, a_3, ..., a_n\}$ be the set of unique roots of $p^i(x)=0$. Consider interval $(a_m, a_{m+1})$. Note that, since there are no roots between $a_m$ and $a_{m+1}$, either $\forall x \in (a_m, a_{m+1})$, thus $p^i(x)>0$, or $\forall x \in (a_m, a_{m+1})$, thus $p^i(x)<0$. In the first case, $\lim_{x_0\rightarrow x} \frac{p^{i-1}(x_0)-p^{i-1}(x)}{x_0-x} > 0$, so $x_0>x$ implies $p^{i-1}(x_0)>p^{i-1}(x)$, and $x_0<x$ implies $p^{i-1}(x_0)<p^{i-1}(x)$. Thus if $\exists x' \in (a_m, a_{m+1})$ such that $p^{i-1}(x')=0$, then $\forall x \in (a_m, a_{m+1})$, $x>x'$ implies $p^{i-1}(x)>0$, or if $x<x'$ then $p^{i-1}(x)<0$. In the second case, similarly if $\exists x' \in (a_m, a_{m+1})$ such that $p^{i-1}(x')=0$, then $\forall x \in (a_m, a_{m+1})$, $x>x'$ implies $p^{i-1}(x)<0$, or if $x<x'$ then $p^{i-1}(x)>0$. Thus there can occur only one root in $(a_m,a_{m+1})$.\\

Now consider interval $[a_m,a_{m+1}]$. Note that if either $f(a_m)=0$ or $f(a_{m+1})$, then $\forall x \in (a_m,a_{m+1})$ either $p^{i-1}(x)>0$ or $p^{i-1}(x)<0$ by the strict monotonicity of the closed interval inside the open one. Thus each open interval $[a_m,a_{m+1}]$ has at most one root inside it. There are $n-1$ such open intervals, and thus $n-1$ such roots.\\

Also consider the side intervals $(-\infty,a_1]$ and $[a_n,\infty)$. By a similar reasoning to that above, if any element $x \in (-\infty,a_1]$ or $x \in [a_n,\infty)$ meets criteria $p^{i-1}(x)=0$, then for all other elements $x'\neq x$ in the interval, either $p^{i-1}(x)>0$ or $p^{i-1}(x)<0$. Thus there can be only one root in each of these intervals. Thus, there can be a maximum of $n-1+2=n+1$ roots for $p^{i-1}$.\\

Thus, by induction, $p(x)$ can have at most n roots.\\

\section{Problem Six}
Consider continuous function $f$ and antiderivative $F$ defined as $F(x) = \int_0^{x}f(t)dt$. By the Second Fundamental Theorem of Calculus:\\

\begin{align*}
  f(0) &= F'(0) = \lim_{x_0\rightarrow0} \frac{F(x_0)-F(0)}{x_0}\\
  &= \lim_{x_0\rightarrow0} \frac{\int_0^{x_0}f(t)dt-\int_0^0f(t)dt}{x_0}\\
  &= \lim_{x_0\rightarrow0} \frac{\int_0^{x_0}f(t)dt}{x_0}
\end{align*}

We also know that $f(0) = \lim_{x\rightarrow0}f(0)=c$, thus:

\begin{align*}
  c &= \lim_{x_0\rightarrow0} \frac{\int_0^{x_0}f(t)dt}{x_0}
\end{align*}

\end{document}

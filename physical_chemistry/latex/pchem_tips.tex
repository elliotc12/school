\documentclass[10pt]{article} % Font size - 10pt, 11pt or 12pt

\usepackage{amsmath}
\usepackage{gensymb}
\usepackage[hmargin=1.25cm, vmargin=1.5cm]{geometry} % Document margins

\usepackage[usenames,dvipsnames]{xcolor} % Allows the definition of hex colors

% Fonts and tweaks for XeLaTeX
\usepackage{fontspec,xltxtra,xunicode}
\defaultfontfeatures{Mapping=tex-text}
%\setmonofont[Scale=MatchLowercase]{Andale Mono}

% Colors for links, text and headings
\usepackage{hyperref}
\definecolor{linkcolor}{HTML}{506266} % Blue-gray color for links
\definecolor{shade}{HTML}{F5DD9D} % Peach color for the contact information box
\definecolor{text1}{HTML}{2b2b2b} % Main document font color, off-black
\definecolor{headings}{HTML}{701112} % Dark red color for headings
% Other color palettes: shade=B9D7D9 and linkcolor=A40000; shade=D4D7FE and linkcolor=FF0080

\hypersetup{colorlinks,breaklinks, urlcolor=linkcolor, linkcolor=linkcolor} % Set up links and colors

\usepackage{fancyhdr}
\pagestyle{fancy}
\fancyhf{}
% Headers and footers can be added with the \lhead{} \rhead{} \lfoot{} \rfoot{} commands
% Example footer:
%\rfoot{\color{headings} {\sffamily Last update: \today}. Typeset with Xe\LaTeX}

\renewcommand{\headrulewidth}{0pt} % Get rid of the default rule in the header

\usepackage{titlesec} % Allows creating custom \section's

% Format of the section titles
\titleformat{\section}{\color{headings}
\scshape\Large\raggedright}{}{0em}{}[\color{black}\titlerule]

\title{Physical Chemistry Tips}
\begin{document}

\maketitle{}

\section{Equations}
\begin{align*}
  \left(\frac{dU}{dV}\right)_T &= T\left(\frac{dP}{dT}\right)_V - P\\
  \left(\frac{dH}{dP}\right)_T &= -T\left(\frac{dV}{dT}\right)_P + V\\
  &\mbox{Arrived at via first law, expanding}\\
  &\mbox{a differential and a Maxwell relation}\\
\end{align*}

\begin{align*}
  \frac{d\left(\frac{G}{T}\right)}{dT} &= -\frac{H}{T^2}\\
\end{align*}

\begin{align*}
  H &= -T^2\left(\frac{\partial\left(G/T\right)}{\partial T}\right)_P
\end{align*}

\begin{align*}
  \kappa &= -\frac{1}{V}\left(\frac{\partial V}{\partial P}\right)_T \hspace{1cm} \mbox{Cubic expansion}\\
\alpha &= \frac{1}{V}\left(\frac{\partial V}{\partial T}\right)_P \hspace{1cm} \mbox{Cubic expansion}\\
\end{align*}

\begin{align*}
  \left(\frac{\partial U}{\partial V}\right)_T &= \left(\frac{\partial H}{\partial P}\right)_T = 0\\
  \left(\frac{\partial U}{\partial T}\right)_V &= C_V \hspace{1cm} \mbox{For an ideal gas}\\
\end{align*}

\begin{align*}
  \frac{dQ}{T} &= dS\\
  TdS &= (dH)_p = C_pdT\\
  \frac{dS}{dT} &=\frac{C_p}{T}\\
\end{align*}

\begin{align*}
  \frac{\partial G}{\partial P}_T &= V\\
  \Delta \overline{G}_{ideal} &= \Delta G_{ideal}\degree + RT\ln\left(\frac{P}{P\degree}\right)\\
  \Delta \overline{G} &= \Delta G\degree + RT\ln\left(\frac{f}{P\degree}\right)\\
  \frac{f}{P} &= e^{\frac{1}{RT}\int_0^P(\bar{V}-\bar{V}^{id})dP} = e^{\int_0^P\frac{Z-1}{P}dP}\\
  \mu_i &= \mu_i\degree + RT\ln(a_i)\\
  \mu_i &= \mu_i\degree + RT\ln\left(\frac{f_i}{P\degree}\right)\\
\end{align*}

\begin{align*}
  K &= \Pi a_{i,eq}^{\nu_i}\\
  Q &= \Pi a_{i}^{\nu_i}\\
  K &= e^{\frac{-\Delta_r G\degree}{RT}}\\
  K_{gas} &= \Pi\left(\frac{f_{i,eq}}{P\degree}\right)^{\nu_i}\\
  K_{gas,ideal} &= \Pi\left(\frac{P_{i,eq,ideal}}{P\degree}\right)^{\nu_i}\\
\end{align*}
Equilibrium constant. The product of all the concentrations of the reaction. At standard state, the Gibbs free energy is $\sum \mu_i\degree\nu_i$.

\begin{align*}
  P_i &= \frac{n_iRT}{V} = C_iRT\\
  K_p &= \Pi \left(\frac{C_iRT}{P\degree}\right)^{\nu_i} = \left(\frac{C_i\degree RT}{P\degree}\right)^{\nu_i} \Pi \left(\frac{C_i}{C_i\degree}\right)^{\nu_i}\\
  K_c &= \Pi\left(\frac{c_i}{c\degree}\right)^{\nu_i}\\
  k &= \frac{c}{K}\\
\end{align*}

\begin{align*}
  V &= \frac{mRT}{MP}\\
  &\mbox{For dissociation A -> 2B}\\
  K &= \frac{4\xi^2}{1-\xi^2}\\
  \xi &= \frac{M_1-M_2}{M_2}\\
\end{align*}

\begin{align*}
  k &= e^{\frac{-\Delta G}{RT}}\\
\end{align*}

\begin{align*}
  \Delta_f G_{isomer} &= -RT\ln\left(\sum e^{\frac{-\Delta_f G_i}{RT}}\right)\\
  \Delta_f H_{isomer}\degree &= \sum r_i \Delta_f H_i\degree\\
  r_i &= e^{\frac{\Delta_f G_{isomer} - \Delta_f G_i}{RT}}\\
\end{align*}

\begin{align*}
  \mbox{Ideal gas expansion - isothermal?}&\hspace{2cm}\Delta S = R\ln(\frac{V_2}{V_1}) = R\ln(\frac{P_1}{P_2})\\
  \mbox{Ideal gas mixing}&\hspace{2cm}\Delta S = -R\sum y_i\ln(y_i)\\
  \mbox{Nonideal phase transition}&\hspace{2cm} \Delta S = \frac{\Delta H}{T}\\
  \mbox{Constant V heating}&\hspace{2cm} \Delta S = \int_{T1}^{T2} \frac{C_V}{T}dT\\
  \mbox{Constant P heating}&\hspace{2cm} \Delta S = \int_{T1}^{T2} \frac{C_P}{T}dT\\
  \mbox{Phase change constant T,P}&\hspace{2cm} \Delta S = \frac{\Delta H}{T}\\
\end{align*}

\begin{align*}
  \mbox{Boltzmann Hypothesis}&\hspace{2cm} S = k\ln(\Omega)\\
\end{align*}

\begin{align*}
  \mbox{Second law of thermodynamics}&\hspace{2cm} dS \geq \frac{dQ_{rev}}{T}\\
  \mbox{Third law of thermodynamics}&\hspace{2cm}\mbox{Crystals have zero entropy}\\
\end{align*}

\begin{align*}
  \mbox{Carnot cycle}&\hspace{2cm}\eta = \frac{w}{q_1} = \frac{\Delta T}{T_1}\\
\end{align*}

\begin{align*}
  \mu_i &= \mu_i\degree + RT\ln\left(\frac{P_i}{P\degree}\right)\\
  \overline{G}_i &= \overline{G}_i\degree + RT\ln\left(\frac{P_i}{P\degree}\right)\\
  \Delta_r G &= \sum \nu_i\mu_i = \sum \nu_i \Delta_f G_i\\
\end{align*}

\begin{align*}
  \ln\left(\frac{K_2}{K_1}\right) &= \frac{\Delta_r H\degree(T_2-T_1)}{RT_1T_2}\\
\end{align*}

\section{Things to know}
\begin{align*}
  &\mbox{Prove entropy is exact for both ideal and nonideal gasses}\\
  &\mbox{Derive first relations in document, the ones with dU/dV}\\
\end{align*}

\end{document}

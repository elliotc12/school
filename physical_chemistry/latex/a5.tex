\documentclass[10pt]{article} % Font size - 10pt, 11pt or 12pt

\usepackage{amsmath}
\usepackage[hmargin=1.25cm, vmargin=1.5cm]{geometry} % Document margins

\usepackage[usenames,dvipsnames]{xcolor} % Allows the definition of hex colors

% Fonts and tweaks for XeLaTeX
\usepackage{fontspec,xltxtra,xunicode}
\defaultfontfeatures{Mapping=tex-text}
%\setmonofont[Scale=MatchLowercase]{Andale Mono}

% Colors for links, text and headings
\usepackage{hyperref}
\definecolor{linkcolor}{HTML}{506266} % Blue-gray color for links
\definecolor{shade}{HTML}{F5DD9D} % Peach color for the contact information box
\definecolor{text1}{HTML}{2b2b2b} % Main document font color, off-black
\definecolor{headings}{HTML}{701112} % Dark red color for headings
% Other color palettes: shade=B9D7D9 and linkcolor=A40000; shade=D4D7FE and linkcolor=FF0080

\hypersetup{colorlinks,breaklinks, urlcolor=linkcolor, linkcolor=linkcolor} % Set up links and colors

\usepackage{fancyhdr}
\pagestyle{fancy}
\fancyhf{}
% Headers and footers can be added with the \lhead{} \rhead{} \lfoot{} \rfoot{} commands
% Example footer:
%\rfoot{\color{headings} {\sffamily Last update: \today}. Typeset with Xe\LaTeX}

\renewcommand{\headrulewidth}{0pt} % Get rid of the default rule in the header

\usepackage{titlesec} % Allows creating custom \section's

% Format of the section titles
\titleformat{\section}{\color{headings}
\scshape\Large\raggedright}{}{0em}{}[\color{black}\titlerule]

\title{Physical Chemistry Assignment Five: 5.2, 5.9, 5.17, 5.27, 5.36, 5.37, 5.53, 5.55, 5.62, 5.74}
\author{Elliott Capek}
\begin{document}

\maketitle{}

\section{Problem 5.2}

We know the total pressure and perfect product at equilibrium, so we can find all the partial pressures at equilbrium.

\begin{align*}
  P_{NH_3} &= (0.0204)(10.13 bar) = 0.207 bar\\
  P_{H_2} &= 2.48 bar\\
  P_{N_2} &= 7.442 bar\\
  K &= \frac{P_{NH_3}/P'}{(P_{H_2}/P')^\frac{3}{2}(P_{N_2}/P')^{\frac{1}{2}}}\\
  &= \frac{0.207}{2.48^\frac{3}{2}7.442^\frac{1}{2}} = 6.5 * 10^{-3}\\
\end{align*}

\section{Problem 5.9}
We use the fact that $k = \frac{4\xi^2}{1-\xi^2}$ for diatomic gas dissociation, so $\xi = \left(\frac{k}{k+4}\right)^\frac{1}{2}$\\

\begin{align*}
  \Delta_rG' &= -RT\ln(K)\\
  &= -(8.314)(2000K)\ln(K)\\
  K &= e^\frac{\Delta_rG'}{-(8.314)(2000K)}\\
  \vspace{1cm}\\
  \Delta_rG' &= 2*\Delta_fG'_{H} - \Delta_fG'_{H_2}\\
  &= 2*106760 J/mol = 213520 J/mol\\
  k &= e^\frac{213520}{-(8.314)(2000K)} = 2.64*10^{-6}\\
  \xi &= \left(\frac{2.64*10^{-6}}{2.64*10^{-6}+4}\right)^\frac{1}{2} = 0.000814\\
  \%_{H} &= 0.0814\%\\
  \vspace{1cm}\\
  \Delta_rG' &= 2*\Delta_fG'_{O} - \Delta_fG'_{O_2}\\
  &= 2*121552 J/mol = 243104 J/mol\\
  k &= e^\frac{243104}{-(8.314)(2000K)} = 4.47*10^{-7}\\
  \xi &= \left(\frac{4.47*10^{-7}}{4.47*10^{-7}+4}\right)^\frac{1}{2} = 0.000334\\
  \%_{O} &= 0.0334\%\\
  \vspace{1cm}\\
  \Delta_rG' &= 2*\Delta_fG'_{I} - \Delta_fG'_{I_2}\\
  &= 2*(-29000) J/mol = -58000 J/mol\\
  k &= e^\frac{-58000}{-(8.314)(2000K)} = 32.7\\
  \xi &= \left(\frac{32.7}{32.7+4}\right)^\frac{1}{2} = 0.94\\
  \%_{I} &= 94\%\\
\end{align*}

\section{Problem 5.17}
\begin{align*}
  \mbox{Initial conditions: } n_{CO} = 1, n_{H_2} = 2, n_{meth} = 0\\
  \mbox{Equilibrium conditions: } n_{CO} = 0.1, n_{H_2} = 0.2, n_{meth} = 0.9\\
  P_{CO} &= \frac{0.1}{1.2}P = 0.083P\\
  P_{H_2} &= \frac{0.2}{1.2}P = 0.166P\\
  P_{meth} &= \frac{0.1}{1.2}P = 0.753P\\
  K &= 6.09*10^{-3} = \frac{P_{meth}/P}{P_{CO}/P(P_{H_2}/P)^2}\\
  P &= 231 bar\\
\end{align*}

\section{Problem 5.27}
\begin{align*}
  K_c &= \frac{[A_2]}{[A]^2} = \frac{n_{A_2}}{n_A}V\\
  n_0 &= n_A + 2\frac{n_A^2K_c}{V}\\
  n_A &= n_0\left(1-2K_cn_0/V\right)
  \hspace{1cm} \mbox{Taylor Series approx}\\
  \frac{PV}{RT} &= n_A + n_{A_2} = n_A\left(1+K_cn_A/V\right)\\
  \frac{PV}{RT} &= n_0\left(1-2K_cn_0/V\right)\left(1+K_cn_A/V\right)\\
  \frac{PV}{RT} &= 1 - \frac{K_c}{V}
  \hspace{1cm}\mbox{Ignore small cross terms}\\
\end{align*}

\section{Problem 5.36}
\begin{align*}
  \Delta_rH' &= \frac{8.314*523*1273*\ln(\frac{0.0721}{100})}{750} = -53.4kJ/mol\\
  \ln\left(\frac{K_1}{K_2}\right) &= \frac{53400*120}{8.314*1393*1273}\\
\end{align*}

Finish!!!!!!!!!!!!!!!!!!!!!!!!

\section{Problem 5.37}
\begin{align*}
  hi
\end{align*}

Finish!!!!!!!!!!!!!!!!!!!!!!!!

\section{Problem 5.53}
\begin{align*}
  &\mbox{Initial conditions: } P_{NH_2} = P'\\
  &\mbox{Equilibrium conditions: } P_{NH_2} = P'(1-\alpha), P_{N_2} = \frac{1}{2}\alpha P', P_{H_2} = \frac{3}{2}\alpha P'\\
  \vspace{1cm}\\
  P_t &= P'(1-\alpha) + \frac{1}{2}\alpha P' + \frac{3}{2}\alpha P'\\
  P' &= \frac{P_t}{1+\alpha}\\
  \vspace{1cm}\\
  K_c &= \frac{\left(\frac{1}{2}\alpha P'\right)^{1/2}\left(\frac{3}{2}\alpha P'\right)^{3/2}}{P'\left(1-\alpha\right)}\\
  &= \frac{((\frac{3}{2})^\frac{3}{2}(\frac{1}{2})^\frac{1}{2})\alpha^2 P'}{P'(1-\alpha)}\\
  &= \frac{(\frac{3}{2})^\frac{3}{2}(\frac{1}{2})^\frac{1}{2}\alpha^2 P'}{P'(1-\alpha)}\\
  K_c &= \frac{((\frac{3}{2})^\frac{3}{2}(\frac{1}{2})^\frac{1}{2})\alpha^2 P_t}{1-\alpha^2}\\
  K_c(1-\alpha^2) &= ((\frac{3}{2})^\frac{3}{2}(\frac{1}{2})^\frac{1}{2})\alpha^2P_t\\
  1-\alpha^2 &= \frac{((\frac{3}{2})^\frac{3}{2}(\frac{1}{2})^\frac{1}{2})\alpha^2P_t}{K_c}\\
  \alpha^2 &= \frac{1}{1+\frac{((\frac{3}{2})^\frac{3}{2}(\frac{1}{2})^\frac{1}{2})P_t}{K_c}}\\
  \alpha^2 &= \frac{1}{1+P_tk}\\
  \alpha &= \sqrt{\frac{1}{1+P_tk}}\\
\end{align*}

\end{document}

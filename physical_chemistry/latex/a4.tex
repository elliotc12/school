\documentclass[10pt]{article} % Font size - 10pt, 11pt or 12pt

\usepackage{amsmath}
\usepackage[hmargin=1.25cm, vmargin=1.5cm]{geometry} % Document margins

\usepackage[usenames,dvipsnames]{xcolor} % Allows the definition of hex colors

% Fonts and tweaks for XeLaTeX
\usepackage{fontspec,xltxtra,xunicode}
\defaultfontfeatures{Mapping=tex-text}
%\setmonofont[Scale=MatchLowercase]{Andale Mono}

% Colors for links, text and headings
\usepackage{hyperref}
\definecolor{linkcolor}{HTML}{506266} % Blue-gray color for links
\definecolor{shade}{HTML}{F5DD9D} % Peach color for the contact information box
\definecolor{text1}{HTML}{2b2b2b} % Main document font color, off-black
\definecolor{headings}{HTML}{701112} % Dark red color for headings
% Other color palettes: shade=B9D7D9 and linkcolor=A40000; shade=D4D7FE and linkcolor=FF0080

\hypersetup{colorlinks,breaklinks, urlcolor=linkcolor, linkcolor=linkcolor} % Set up links and colors

\usepackage{fancyhdr}
\pagestyle{fancy}
\fancyhf{}
% Headers and footers can be added with the \lhead{} \rhead{} \lfoot{} \rfoot{} commands
% Example footer:
%\rfoot{\color{headings} {\sffamily Last update: \today}. Typeset with Xe\LaTeX}

\renewcommand{\headrulewidth}{0pt} % Get rid of the default rule in the header

\usepackage{titlesec} % Allows creating custom \section's

% Format of the section titles
\titleformat{\section}{\color{headings}
\scshape\Large\raggedright}{}{0em}{}[\color{black}\titlerule]

\title{Physical Chemistry Assignment Four}
\author{Elliott Capek}
\begin{document}

\maketitle{}

\section{4.10}

a.)
\begin{align*}
  \Delta H &= -T^2\frac{\partial (\Delta G/T)}{\partial T}\\
  -\frac{\Delta H}{T^2} &= \frac{\partial (\Delta G/T)}{\partial T}\\
  \int_{T_1}^{T_2} -\frac{\Delta H}{T^2}dT &= \frac{\Delta G}{T}\Bigg|_{\frac{\Delta G_1}{T_1}}^{\frac{\Delta G_2}{T_2}}\\
  \Delta H\left(\frac{1}{T_2} - \frac{1}{T_1}\right) &= \frac{\Delta G_2}{T_2} - \frac{\Delta G_1}{T_1}\\
  \Delta G_2 &= \Delta H\left(1 - \frac{T_2}{T_1}\right) + \Delta G_1 \frac{T_2}{T_1}\\
\end{align*}

b.)
\begin{align*}
  \Delta H &= -T^2\frac{\partial (\Delta G/T)}{\partial T}\\
  -\frac{\Delta H}{T^2} &= \frac{\partial (\Delta G/T)}{\partial T}\\
  \int_{T_1}^{T_2} &\left(-\frac{\Delta H_1 + (T-T_1)\Delta C_P}{T^2}\right)dT = \frac{\Delta G}{T}\Bigg|_{\frac{\Delta G_1}{T_1}}^{\frac{\Delta G_2}{T_2}}\\
  &\Delta H_1\left(\frac{1}{T_2}-\frac{1}{T_1}\right) + \ln\left(\frac{T_1}{T_2}\right)\Delta C_P - T_1\Delta C_P\left(\frac{1}{T_2}-\frac{1}{T_1}\right) = \frac{\Delta G_2}{T_2} - \frac{\Delta G_1}{T_1}\\
  \Delta G_2 &= \left(\Delta H_1 - T_1\Delta C_P\right)\left(1 - \frac{T_2}{T_1}\right) + T_2\ln\left(\frac{T_1}{T_2}\right)\Delta C_P + \frac{T_2\Delta G_1}{T_1}\\
\end{align*}

\section{4.11}
\begin{align*}
  \bar{V} &= 1 \left(\frac{1 m}{10^2 g}\right)^3 \frac{18.01 g}{1 mol} = 1.801 * 10^{-5} \frac{m^3}{mol}\\
  \Delta\bar{G} &= \left(1.801 * 10^{-5} \frac{m^3}{mol}\right)\left(10^3 bar * 10^5 \frac{Pa}{bar}\right) = 1.8 \frac{kJ}{mol}\\
\end{align*}

\section{4.12}
\begin{align*}
  \Delta \bar{G} &= -\bar{S}dT + \bar{V}dP\\
  \Delta \bar{G} &= RT\ln\left(\frac{P_2}{P_1}\right)\\
  &= (8.314 \frac{J}{K mol})(298K)\ln(0.1) = -5.7 \frac{kJ}{mol}\\
\end{align*}

G is a state function, so all systems with the same initial and final states have the same $\Delta G$.

\section{4.15}
\begin{align*}
  \Delta H_{vap} &= 33.2 \frac{kJ}{mol}\\
  w &= -P_{ext}\Delta V = -(10^5 Pa)\frac{(1 mol)(8.314 \frac{J}{K mol} * 383 K)}{10^5 Pa}\\
  &= -3.184 \frac{kJ}{mol}\\
  q &= dH_P = 33.2 \frac{kJ}{mol}\\
  \Delta \bar{H} &= 33.2 \frac{kJ}{mol}\\
  \Delta \bar{U} &= -3.184 \frac{kJ}{mol} + 33.2 \frac{kJ}{mol} = 30.02 \frac{kJ}{mol}\\
  \Delta \bar{G} &= 0\\
  \Delta \bar{S} &= \frac{q}{T} = 86.6 \frac{J}{mol K}\\
\end{align*}

\section{4.16}
\begin{align*}
  G &= aT + b + \frac{c}{T}\\
  S &= -\left(\frac{\partial G}{\partial T}\right)_P = \frac{c}{T^2} - a\\
  H &= G + TS = b + \frac{2c}{T}\\
\end{align*}

\section{4.17}
First we take the water from atmospheric pressure to its vapor pressure, so we can do the gas transition reversibly. This transition is $\Delta G_1$. We then convert it to a gas, this is $\Delta G_2$. We then bring the gas to the vapor pressure of the solid, this is $\Delta G_3$. We then sublimate the gas, this is $\Delta G_4$. Finally for $\Delta G_5$ we take the solid back up to the starting pressure.
\begin{align*}
  V_l &= 1.8*10^{-5} m^3\\
  V_s &= 1.96*10^{-5} m^3\\
  \Delta G_1 &= \int VdP = 1.8*10^{-5} m^3 * \left(489Pa - 10^5 Pa\right) = -1.79J\\
  \Delta G_2 &= 0 \hspace{1cm}\mbox{State transitions @ vapor pressure isenthalpic}\\
  \Delta G_3 &= \int VdP = \int \frac{RT}{P}dP = (8.314)(298)\ln\left(\frac{475}{489}\right) = -65.2J\\
  \Delta G_4 &= 0 \hspace{1cm}\mbox{State transitions @ vapor pressure isenthalpic}\\
  \Delta G_5 &= \int VdP = 1.96*10^{-5} m^3 * \left(10^5Pa - 475Pa\right) = 1.95J\\
  \Delta G &= -1.79J + -65.2J + 1.95J = -65.04\frac{J}{mol}\\
\end{align*}

\section{4.24}
a.)
\begin{align*}
  f &= Pe^{\int_0^p \frac{Z-1}{p}dp}\\
  &= Pe^{\int_0^p \frac{B}{RT}dp}\\
  &= Pe^{\frac{Bp}{RT}}\\
  &= Pe^{Z-1}\\
\end{align*}

b.)
\begin{align*}
  f_{H_2} &= PZ = P + \frac{1}{RT}\left(b - \frac{a}{RT}\right)P^2\\
  &= 50 \mbox{bar} + \frac{1}{\left(8.314*10^{-5} \frac{m^3 \mbox{bar}}{\mbox{K mol}}\right)298K}\left(0.026  - \frac{0.247}{\left(8.314*10^{-5} \frac{m^3 \mbox{bar}}{\mbox{K mol}}\right)298K}\right)\left(2500 \mbox{bar}^2\right)\\
  &= 52 bar\\
\end{align*}

\section{4.27}
To separate, take each gas and separate it. This is a zero-energy process. Then take each separated gas back to 1 mol. Use the ideal gas approximation:

\begin{align*}
  w &= -2\int pdV = -2nRT \int \frac{1}{V}dV = -2nRT\ln\left(\frac{V_2}{V_1}\right)\\
  &= 2(8.314)(298)\ln(2) = 3.4kJ\\
\end{align*}

\section{4.50}
\begin{align*}
  Z &= 1 + \left(b - \frac{a}{RT}\right)\frac{P}{RT}\\
  V &= \frac{RT}{P} + b - \frac{a}{RT}\\
  \Delta G &= \int_{P_1}^{P_2} dG = \int_{P_1}^{P_2} \frac{\partial G}{\partial P} dP\\
  &= \int_{P_1}^{P_2} VdP - \int_{P_1}^{P_2} \left(\frac{RT}{P} + b - \frac{a}{RT}\right) dP\\
  &= RT\ln\left(\frac{P_2}{P_1}\right) + \left(b-\frac{a}{RT}\right)\left(P_2-P_1\right)\\
\end{align*}

\section{4.54}
\begin{align*}
  \Delta S &= -R\sum n_i\ln y_i\\
  &= -\left(8.314\right)\left(\ln\left(\frac{1}{3}\right) + 2\ln\left(\frac{2}{3}\right)\right) = 15.87 \frac{J}{K}\\
  \Delta G &= -TS = -298*18.87 = -4.7kJ\\
\end{align*}

\end{document}

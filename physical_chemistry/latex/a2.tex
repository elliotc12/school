\documentclass[10pt]{article} % Font size - 10pt, 11pt or 12pt

\usepackage{amsmath}
\usepackage[hmargin=1.25cm, vmargin=1.5cm]{geometry} % Document margins

\usepackage[usenames,dvipsnames]{xcolor} % Allows the definition of hex colors

% Fonts and tweaks for XeLaTeX
\usepackage{fontspec,xltxtra,xunicode}
\defaultfontfeatures{Mapping=tex-text}
%\setmonofont[Scale=MatchLowercase]{Andale Mono}

% Colors for links, text and headings
\usepackage{hyperref}
\definecolor{linkcolor}{HTML}{506266} % Blue-gray color for links
\definecolor{shade}{HTML}{F5DD9D} % Peach color for the contact information box
\definecolor{text1}{HTML}{2b2b2b} % Main document font color, off-black
\definecolor{headings}{HTML}{701112} % Dark red color for headings
% Other color palettes: shade=B9D7D9 and linkcolor=A40000; shade=D4D7FE and linkcolor=FF0080

\hypersetup{colorlinks,breaklinks, urlcolor=linkcolor, linkcolor=linkcolor} % Set up links and colors

\usepackage{fancyhdr}
\pagestyle{fancy}
\fancyhf{}
% Headers and footers can be added with the \lhead{} \rhead{} \lfoot{} \rfoot{} commands
% Example footer:
%\rfoot{\color{headings} {\sffamily Last update: \today}. Typeset with Xe\LaTeX}

\renewcommand{\headrulewidth}{0pt} % Get rid of the default rule in the header

\usepackage{titlesec} % Allows creating custom \section's

% Format of the section titles
\titleformat{\section}{\color{headings}
\scshape\Large\raggedright}{}{0em}{}[\color{black}\titlerule]

\title{Physical Chemistry Assignment Two:\\2.2 2.4 ...}
\author{Elliott Capek}

\begin{document}

\maketitle{}

\section{2.2}

\section{2.4}

The differential $df = dx - \frac{x}{y}dy$ is inexact:
\begin{align}
  df &= dx - \frac{x}{y}dy\\
  df &= M(x,y)dx + N(x,y)dy\\
  M(x,y) &= 1\\
  N(x,y) &= -\frac{x}{y}\\
  \frac{\partial M}{\partial y}_x &= 0\\
  \frac{\partial N}{\partial x}_y &= \frac{x}{y^2}\\
  \frac{\partial M}{\partial y}_x &\neq \frac{\partial N}{\partial x}_y\\
\end{align}

The differential $dg = \frac{1}{y}df$ is exact:
\begin{align}
  dg &= \frac{1}{y}df\\
     &= \frac{1}{y}dx - \frac{x}{y^2}dy\\
  dg &= M(x,y)dx + N(x,y)dy\\
  M(x,y) &= \frac{1}{y}\\
  N(x,y) &= -\frac{x}{y^2}\\
  \frac{\partial M}{\partial y}_x &= -\frac{1}{y^2}\\
  \frac{\partial N}{\partial x}_y &= -\frac{1}{y^2}\\
  \frac{\partial M}{\partial y}_x &= \frac{\partial N}{\partial x}_y\\
\end{align}


\section{2.11}

For a Van der Waal's Gas, isothermal reversible work is given by:

\begin{align}
  W &= \int_{v1}^{v2} \frac{\partial U}{\partial T}_V dT - PdV\\
  W &= \int_{v1}^{v2} -PdV\\
  W &= -\int_{v1}^{v2} \frac{nRT}{\bar{V} - b}dV - \frac{a}{\bar{V}^2} dV\\
  W &= -\int_{v1}^{v2} \frac{nRT}{\bar{V} - b}dV + \int_{v1}^{v2} \frac{a}{\bar{V}^2}dV\\
  W &= -\int_{v1-b}^{v2-b} \frac{nRT}{u}du + \int_{v1}^{v2} \frac{a}{\bar{V}^2}dV\\
  W &= -nRT\log(\frac{V_2-b}{V_1-b}) + a(\frac{1}{\bar{V}_1} - \frac{1}{\bar{V}_2})\\      
\end{align}

For an ideal gas, isothermal reversible work is given by:\\

\begin{align}
  P &= \frac{nRT}{V}\\
  W &= -\int_{v1}^{v2} PdV\\
  W &= -\int_{v1}^{v2} \frac{nRT}{V}dV\\
  W &= -nRT\log(\frac{V_2}{V_1})\\  
\end{align}

For one mole of $CH_4$ at 25C, the ideal gas work is $-(1)(8.314)(298)\log(\frac{50}{1}) = 9692J$. The Van der Waal's work is $-(1)(8.314)(298)\log(\frac{50-0.04278}{1-0.04278}) + (2.283)(\frac{1}{1} - \frac{1}{50}) = 4257J$

\end{document}

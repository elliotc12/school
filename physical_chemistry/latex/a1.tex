\documentclass[10pt]{article} % Font size - 10pt, 11pt or 12pt

\usepackage{amsmath}
\usepackage[hmargin=1.25cm, vmargin=1.5cm]{geometry} % Document margins

\usepackage{marvosym} % Required for symbols in the colored box
\usepackage{ifsym} % Required for symbols in the colored box

\usepackage[usenames,dvipsnames]{xcolor} % Allows the definition of hex colors

% Fonts and tweaks for XeLaTeX
\usepackage{fontspec,xltxtra,xunicode}
\defaultfontfeatures{Mapping=tex-text}
%\setmonofont[Scale=MatchLowercase]{Andale Mono}

% Colors for links, text and headings
\usepackage{hyperref}
\definecolor{linkcolor}{HTML}{506266} % Blue-gray color for links
\definecolor{shade}{HTML}{F5DD9D} % Peach color for the contact information box
\definecolor{text1}{HTML}{2b2b2b} % Main document font color, off-black
\definecolor{headings}{HTML}{701112} % Dark red color for headings
% Other color palettes: shade=B9D7D9 and linkcolor=A40000; shade=D4D7FE and linkcolor=FF0080

\hypersetup{colorlinks,breaklinks, urlcolor=linkcolor, linkcolor=linkcolor} % Set up links and colors

\usepackage{fancyhdr}
\pagestyle{fancy}
\fancyhf{}
% Headers and footers can be added with the \lhead{} \rhead{} \lfoot{} \rfoot{} commands
% Example footer:
%\rfoot{\color{headings} {\sffamily Last update: \today}. Typeset with Xe\LaTeX}

\renewcommand{\headrulewidth}{0pt} % Get rid of the default rule in the header

\usepackage{titlesec} % Allows creating custom \section's

% Format of the section titles
\titleformat{\section}{\color{headings}
\scshape\Large\raggedright}{}{0em}{}[\color{black}\titlerule]

\title{Physical Chemistry Assignment One:\\1.4, 1.5, 1.14, 1.17, 1.18, 1.20, 1.26, 1.31, 1.33, 1.34}
\author{Elliott Capek}

\begin{document}

\maketitle{}

\section{1.4}
Mass: 1.588g\\
$n_t = \frac{1.588g}{92.08 \frac{g}{mol N_2O_4}} = 0.0172$ $ mol N_2O_4$\\
P: 1.0133 bar $* \frac{10^5 Pa}{bar} = 1.0133*10^5Pa$\\
T: 298K\\
$V_{tot}$: 500 $cm^3 * \frac{m}{100 cm}^3 = 5*10^{-4}m^3$\\

Goal: find $n_1$ and $n_2$, the mols of $N_2O_4$ and $NO_2$, respectively.\\

$n_1 = n_t-x$\\
$n_2 = 2x$\\

$PV_1 = (n_t-x)RT$\\
$PV_2 = 2xRT$\\

We add these equations and see that:\\

$P(V_1 + V_2) = (n_t+x)RT$\\
$(1.0133*10^5 \frac{N}{m^2})(5*10^{-4}m^3) = (0.0172 mol + x)(8.314 \frac{J}{mol*K})(298K)$\\

From this we can see that $x = 0.00325$\\

Therefore we end up with 0.01395 mol $N_2O_4$ and 0.00325 mol $NO_2$.\\

Mole fractions: 0.81 and 0.19.\\

Percent dissociated: 19\%.

\section{1.5}
$Z = $
$\frac{dZ}{dP} = $

\section{1.14}

\section{1.17}

\begin{align}
\alpha &= \frac{1}{V}(\frac{\partial V}{\partial T})_P\\
&= \frac{1}{V}(\frac{\partial \frac{nRT}{P}}{\partial T})_P\\
&= \frac{nR}{VP}
\end{align}

\begin{align}
\kappa &= \frac{-1}{V}(\frac{\partial \frac{nRT}{P}}{\partial P})_T\\
&= \frac{nRT}{VP^2}
\end{align}

\end{document}

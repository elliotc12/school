\documentclass[10pt]{article} % Font size - 10pt, 11pt or 12pt

\usepackage{amsmath}
\usepackage[hmargin=1.25cm, vmargin=1.5cm]{geometry} % Document margins

\usepackage{marvosym} % Required for symbols in the colored box
\usepackage{ifsym} % Required for symbols in the colored box

\usepackage[usenames,dvipsnames]{xcolor} % Allows the definition of hex colors

% Fonts and tweaks for XeLaTeX
\usepackage{fontspec,xltxtra,xunicode}
\defaultfontfeatures{Mapping=tex-text}
%\setmonofont[Scale=MatchLowercase]{Andale Mono}

% Colors for links, text and headings
\usepackage{hyperref}
\definecolor{linkcolor}{HTML}{506266} % Blue-gray color for links
\definecolor{shade}{HTML}{F5DD9D} % Peach color for the contact information box
\definecolor{text1}{HTML}{2b2b2b} % Main document font color, off-black
\definecolor{headings}{HTML}{701112} % Dark red color for headings
% Other color palettes: shade=B9D7D9 and linkcolor=A40000; shade=D4D7FE and linkcolor=FF0080

\hypersetup{colorlinks,breaklinks, urlcolor=linkcolor, linkcolor=linkcolor} % Set up links and colors

\usepackage{fancyhdr}
\pagestyle{fancy}
\fancyhf{}
% Headers and footers can be added with the \lhead{} \rhead{} \lfoot{} \rfoot{} commands
% Example footer:
%\rfoot{\color{headings} {\sffamily Last update: \today}. Typeset with Xe\LaTeX}

\renewcommand{\headrulewidth}{0pt} % Get rid of the default rule in the header

\usepackage{titlesec} % Allows creating custom \section's

% Format of the section titles
\titleformat{\section}{\color{headings}
\scshape\Large\raggedright}{}{0em}{}[\color{black}\titlerule]

\title{Physical Chemistry Assignment One:\\1.4, 1.5, 1.14, 1.17, 1.18, 1.20, 1.26, 1.31, 1.33, 1.34}
\author{Elliott Capek}

\begin{document}

\maketitle{}

\section{1.4}
Mass: 1.588g\\
$n_t = \frac{1.588g}{92.08 \frac{g}{mol N_2O_4}} = 0.0172$ $ mol N_2O_4$\\
P: 1.0133 bar $* \frac{10^5 Pa}{bar} = 1.0133*10^5Pa$\\
T: 298K\\
$V_{tot}$: 500 $cm^3 * \frac{m}{100 cm}^3 = 5*10^{-4}m^3$\\

Goal: find $n_1$ and $n_2$, the mols of $N_2O_4$ and $NO_2$, respectively.\\

$n_1 = n_t-x$\\
$n_2 = 2x$\\

$PV_1 = (n_t-x)RT$\\
$PV_2 = 2xRT$\\

We add these equations and see that:\\

$P(V_1 + V_2) = (n_t+x)RT$\\
$(1.0133*10^5 \frac{N}{m^2})(5*10^{-4}m^3) = (0.0172 mol + x)(8.314 \frac{J}{mol*K})(298K)$\\

From this we can see that $x = 0.00325$\\

Therefore we end up with 0.01395 mol $N_2O_4$ and 0.0065 mol $NO_2$, for a total of $0.02045$.\\

Mole fractions: 0.68 and 0.32.\\

Percent dissociated: 19\%.

\section{1.5}
\begin{align}
  Z = 1 + B'P + C'P^2 + ... &= 1 + \frac{B}{RT}P + \frac{C-B^2}{(RT)^2)}\\
  \frac{\partial }{\partial P} \Big(1 + \frac{B}{RT}P + \frac{C-B^2}{(RT)^2}P^2 \Big) &= \frac{B}{RT} + \frac{2C-2B^2}{(RT)^2}P\\
  \lim_{P\to0} \frac{B}{RT} + \frac{2C-2B^2}{(RT)^2}P &= \frac{B}{RT} = B'\\
\end{align}

\clearpage

\section{1.14}

\begin{align}
  \kappa &= -V^{-1}\frac{dV}{dP}_T\\
  nRT &= (P+\frac{an^2}{V^2})(V-nb)\\
  nRT &= PV - Pnb + \frac{an^2}{V} - \frac{abn^3}{V^2}\\
  \frac{\partial}{\partial P}nRT &= \frac{\partial}{\partial P} \Big(PV - Pnb + \frac{an^2}{V} - \frac{abn^3}{V^2} \Big)\\
  0 &= P\frac{\partial V}{\partial P}_T + V - nb + \frac{-an^2}{V^2}\frac{\partial V}{\partial P}_T - \frac{-2abn^3}{V^3}\frac{\partial V}{\partial P}_T\\
  nb - V &= (P + \frac{-an^2}{V^2} - \frac{-2abn^3}{V^3}) \frac{\partial V}{\partial P}_T\\
  \frac{nb - V}{P + \frac{-an^2}{V^2} - \frac{-2abn^3}{V^3}} &= \frac{\partial V}{\partial P}_T\\
  \kappa = \frac{-1}{V}\frac{\partial V}{\partial P}_T &= \frac{-1}{V}\frac{nb - V}{P + \frac{-an^2}{V^2} - \frac{-2abn^3}{V^3}}\\
\end{align}

We then look at the limiting case when volume goes to infinity, using L'Hospital's Rule: \\

\begin{align}
  \lim_{V\to \infty} \kappa &= \lim_{V\to \infty} \frac{-1}{V}\frac{nb - V}{P + \frac{-an^2}{V^2} - \frac{-2abn^3}{V^3}}\\
  &= \frac{\frac{\partial}{\partial V}}{\frac{\partial}{\partial V}} \frac{nb - V}{PV + \frac{-an^2}{V} - \frac{-2abn^3}{V^2}}\\
  &= \frac{-1}{P + \frac{an^2}{V^2} + \frac{-4abn^3}{V^3}}\\
  &= \frac{-1}{P}\\
\end{align}
This is identical to the answer we found in problem 1.17.

\section{1.17}

\begin{align}
  \alpha &= \frac{1}{V}(\frac{\partial V}{\partial T})_P\\
  &= \frac{1}{\frac{nRT}{P}}(\frac{\partial \frac{nRT}{P}}{\partial T})_P\\
  &= \frac{P}{nRT} \frac{nR}{P}\\
  &= \frac{1}{T}\\
\end{align}

\begin{align}
  \kappa &= \frac{-1}{V}(\frac{\partial V}{\partial P})_T\\  
  &= \frac{-1}{\frac{nRT}{P}}(\frac{\partial \frac{nRT}{P}}{\partial P})_T\\
  &= \frac{-P}{nRT}\frac{-nRT}{P^2}\\
  &= \frac{1}{P}\\
\end{align}

\section{1.18}

\begin{align}
  \alpha = \big(\frac{\partial V}{\partial T}\big)_P\frac{1}{V} &= c_P\\
  \big(\frac{\partial V}{\partial T}\big)_P\frac{1}{V} &= c_P\\
  \int \frac{1}{V} dV  &= \int c_P dT\\
  ln(V) = c_PT + C\\
  e^{c_PT+C} = V\\
  V = Ce^{\alpha T}\\
\end{align}
This is true at constant P. \\

\begin{align}
  \kappa = \big(\frac{\partial V}{\partial P}\big)_T \frac{-1}{V} &= c_T\\
  \big(\frac{\partial V}{\partial P}\big)_T \frac{1}{V} &= -c_T\\
  \int frac{1}{V} dV &= \int -c_T dP\\
  ln(V) &= -c_TP + C\\
  V &= e^{-c_TP+C}\\
  V &= Ce^{-c_TP} = Ce^{-\kappa P}\\
\end{align}
This is only true at constant T.\\

We can then see by combining these two derivations for the general, non-constant case that $V = Ce^{\alpha T -\kappa P}$.

\section{1.20}

\begin{align}
  \frac{\partial P}{\partial V}_T &= \big(\frac{\partial}{\partial V}_T\big) \frac{nRT}{V-nb}\\
  &= \frac{-nRT}{(V-nb)^2}\\
\end{align}

and

\begin{align}
  \frac{\partial P}{\partial T}_T &= \big(\frac{\partial}{\partial T}\big)_V \frac{nRT}{V-nb}\\
  &= \frac{nR}{V-nb}\\
\end{align}

Then we can see that:

\begin{align}
  \frac{\partial^2P}{\partial V\partial T} = \frac{\partial}{\partial T} \frac{-nRT}{(V-nb)^2}
  &= \frac{-nr}{(V-nb)^2}\\
\end{align}

and

\begin{align}
  \frac{\partial^2P}{\partial T\partial V} = \frac{\partial}{\partial V} \frac{nR}{V-nb}
  &= \frac{-nr}{(V-nb)^2}\\
\end{align}

and so

\begin{align}
  \frac{\partial^2P}{\partial T\partial V} = \frac{\partial^2P}{\partial V\partial T}
\end{align}

\section{1.26}
\begin{align}
  B = \sum_{i=1}^2\sum_{j=1}^2y_iy_jB_{ij} &= y_1y_1B_{11}+y_1y_2B_{12}+y_2y_1B_{12}+y_2y_2B_{22}\\
  &= y_1^2B_{11}+y_2^2B_{22}+2y_1y_2B_{12}\\
\end{align}

\section{1.31}
(a)\\
\begin{align}
  PV&=nRT\\
  P(0.5L) &= (1 mol)R(600K)\\
  P(0.5L * \frac{m^3}{1000L}) &= (1 mol)(8.314 \frac{N*m}{mol*K})(600K)\\
  P &= \frac{8.314*600}{0.0005} \frac{N}{m^2} = 9.976e6 Pa\\
\end{align}

(b)\\
From Wikipedia (en.wikipedia.org/wiki/Van\_der\_Waals\_constants\_(data\_page)):
$$a = 24.71 \frac{L^2bar}{mol^2} = 24.71 \frac{L^2bar}{mol^2} * \frac{m^3}{10^3 L}^2 * \frac{10^5 Pa}{bar} = 2.471 \frac{m^6Pa}{mol^2}$$
$$b = 0.1735\frac{L}{mol} = 0.1735\frac{L}{mol} * \frac{m^3}{1000 L} = 1.6*10^{-4} \frac{m^3}{mol}$$

\begin{align}
  (P+\frac{a}{\bar{V}^2})(\bar{V}-b) &= RT\\
  (P+\frac{2.471}{\bar{V}^2})(\bar{V} - 1.6*10^{-4}) &= RT\\
  (P+\frac{2.471}{0.0005^2})(0.0005 - 1.6*10^{-4}) &= (8.314)(600)\\
  (P+9884000) &= \frac{(8.314)(600)}{(0.00034)}\\
  P &= \frac{(8.314)(600)}{(0.00034)} - 9884000\\
  P = 4.787764 * 10^6 Pa\\
\end{align}

\section{1.33}

\begin{align}
  \kappa = -V^{-1}(\frac{\partial V}{\partial P}_)T\\
  -\kappa V = \frac{\partial V}{\partial P}_T\\
  \int -\kappa V \partial P_T = \int \partial V\\
  \int -\kappa \partial P_T = \int \frac{1}{V}\partial V\\
  -P\kappa + C= ln(V)\\
  V = e^{-\kappa P}e^{C}\\
  V = C_0e^{-\kappa P}\\
\end{align}

\section{1.34}

\begin{align}
  P(\bar{V}-b) &= RT\\
  \bar{V}-b &=\frac{RT}{P}\\
  \bar{V} &= \frac{RT}{P}+b\\
  V &= \frac{nRT}{P}+nb\\
\end{align}

\begin{align}
  \alpha &= \frac{1}{V}\frac{\partial V}{\partial T}_P\\
  &= \frac{1}{V}\frac{\partial \frac{nRT}{P}+nb}{\partial T}_P\\
  &= \frac{1}{V} \frac{nR}{P}\\
  &= \frac{nR}{PV}\\
\end{align}

\begin{align}
  \kappa &= \frac{-1}{V}\frac{\partial V}{\partial P}_T\\
  &= \frac{-1}{V}(\frac{\partial \frac{nRT}{P}+nb}{\partial P})_T\\
  &= \frac{-1}{V}\frac{-nRT}{P^2}\\
  &= \frac{nRT}{VP^2}\\
\end{align}

\end{document}


  dV(P) = (dv/dP)_T * dP
  dV(P) - -V\kappa dP
  dV(P) / V = -kdP
  integrate both sides indefinitely
  ln(V) = -KP
  V = Ce^(-kP)  

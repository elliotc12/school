\documentclass[10pt]{article} % Font size - 10pt, 11pt or 12pt

\usepackage{amsmath}
\usepackage[hmargin=1.25cm, vmargin=1.5cm]{geometry} % Document margins

\usepackage[usenames,dvipsnames]{xcolor} % Allows the definition of hex colors

% Fonts and tweaks for XeLaTeX
\usepackage{fontspec,xltxtra,xunicode}
\defaultfontfeatures{Mapping=tex-text}
%\setmonofont[Scale=MatchLowercase]{Andale Mono}

% Colors for links, text and headings
\usepackage{hyperref}
\definecolor{linkcolor}{HTML}{506266} % Blue-gray color for links
\definecolor{shade}{HTML}{F5DD9D} % Peach color for the contact information box
\definecolor{text1}{HTML}{2b2b2b} % Main document font color, off-black
\definecolor{headings}{HTML}{701112} % Dark red color for headings
% Other color palettes: shade=B9D7D9 and linkcolor=A40000; shade=D4D7FE and linkcolor=FF0080

\hypersetup{colorlinks,breaklinks, urlcolor=linkcolor, linkcolor=linkcolor} % Set up links and colors

\usepackage{fancyhdr}
\pagestyle{fancy}
\fancyhf{}
% Headers and footers can be added with the \lhead{} \rhead{} \lfoot{} \rfoot{} commands
% Example footer:
%\rfoot{\color{headings} {\sffamily Last update: \today}. Typeset with Xe\LaTeX}

\renewcommand{\headrulewidth}{0pt} % Get rid of the default rule in the header

\usepackage{titlesec} % Allows creating custom \section's

% Format of the section titles
\titleformat{\section}{\color{headings}
\scshape\Large\raggedright}{}{0em}{}[\color{black}\titlerule]

\title{Physical Chemistry Assignment Two}
\author{Elliott Capek}

\begin{document}
\maketitle{}

\section{Problem 3.4}

\begin{align*}
  \Delta S_1 &= \int \frac{dW}{T} = \int_{V_1}^{V_2} \frac{R}{V}dV
  \hspace{2cm} \mbox{Isothermal} \rightarrow dU = 0\\
  &= R\ln\left(\frac{V_2}{V_1}\right)\\
  \vspace{1cm}\\
  \Delta S_2 &= \int \frac{dU}{T} = \int_{T_2}^{T_1} \frac{C_v}{T}dT
  \hspace{2cm} \mbox{Adiabatic Path: } \Delta S = 0\\
  &= C_v\ln\left(\frac{T_1}{T_2}\right)\\
  &= C_v\ln\left(\frac{V_2^{R/C_v}}{V_1^{R/C_v}}\right)\\
  &= R\ln\left(\frac{V_2}{V_1}\right)\\
\end{align*}

\section{Problem 3.5}
\textbf{a.}
\begin{align*}
  \Delta S &= \frac{\Delta H}{T} = \frac{40.69 \frac{kJ}{mol}}{373.15K} = 109 \frac{J}{K mol}\\
\end{align*}
\textbf{b.}
Universal entropy changes for reversible processes are zero.\\

\section{Problem 3.6}
\begin{align*}
  \Delta H &= \int_{T_1}^{T_2} C_p dT = \int_{T_1}^{T_2} 26.684 + 42.264*10^{-3}T - 142.4*10^{-7}T^2 dT = 33.36 \frac{kJ}{mol}\\
  \Delta S &= \int_{T_1}^{T_2} \frac{C_p}{T} dT = \int_{T_1}^{T_2} \frac{26.684}{T} + 42.264*10^{-3} - 142.4*10^{-7}T dT = 55.4 \frac{J}{K mol}\\
\end{align*}

\section{Problem 3.9}
\begin{align*}
  \Delta S_1 &= \frac{\Delta H}{T} = \int_{T_1}^{\frac{T_1+T_2}{2}} \frac{C_p}{T}dT\\
    &= C_p\ln\left(\frac{T_1+T_2}{2T_1}\right)\\
    \Delta S_2 &= C_p\ln\left(\frac{T_1+T_2}{2T_2}\right)\\
    \Delta S &= C_p\ln\left(\frac{\left(T_1+T_2\right)^2}{4T_1T_2}\right)\\
\end{align*}

\section{Problem 3.12}
\textbf{a.}
\begin{align*}
  \Delta U &= 0
  \hspace{5cm} \mbox{Isothermal process}\\
  w &= -nRT\ln\left(\frac{V_2}{V_2}\right) = -(1 mol)(8.314 \frac{J}{mol K})(300K)\ln\left(\frac{300}{90}\right) = -3kJ\\
  q &= -w = 3kJ\\
  \Delta S &= \frac{dQ}{T} = 10 \frac{J}{K}\\
\end{align*}

\textbf{b.}
\begin{align*}
  \Delta U &= 0\\
  w &=  -1kJ\\
  q &= = 1kJ\\
  \Delta S &= \frac{dQ}{T} = 3.3 \frac{J}{K}\\
\end{align*}

\textbf{c.}
\begin{align*}
  &\mbox{Must be the same, state function}\\
  \Delta U &= 0\\
  w &=  -1kJ\\
  q &= 1kJ\\
  \Delta S &= \frac{dQ}{T} = 3.3 \frac{J}{K}\\
\end{align*}

\section{Problem 3.13}
\textbf{a.}
\begin{align*}
  \Delta S &= nR\ln\left(\frac{V_2}{V_1}\right)\\
  &= nR\ln\left(\frac{P_1}{P_2}\right) = (1 mol)(8.314)\ln(10) = 19.14 \frac{J}{K}\\
  &\mbox{Entropy is a state function, both cases have same entropy change}\\
\end{align*}

\textbf{b.}
Universal entropy change for a reversible process is zero. This is true for both cases, since entropy is a state function.\\

\section{Problem 3.20}
\begin{align*}
  \Delta S_1 &= \int_{V_1}^{V_1+V_2} \frac{dQ}{T} = \int_{V_1}^{V_1+V_2} \frac{dW}{T}\\
  &= n_1R\ln\left(\frac{V_1+V_2}{V_1}\right)\\
  &= n_1R\ln\left(\frac{\frac{n_1}{P_1} + \frac{n_2}{P_2}}{\frac{n_1}{P_1}}\right)\\
  &= n_1R\ln\left(\frac{3/2}{1}\right) = 0.4R\\
  \vspace{1cm}\\
  \Delta S_2 &= n_2R\ln\left(\frac{\frac{n_1}{P_1} + \frac{n_2}{P_2}}{\frac{n_2}{P_2}}\right)\\
  &= n_1R\ln\left(\frac{3/2}{1/2}\right) = R\\
  \vspace{1cm}
  \Delta S &= 1.4R = 11.6 \frac{J}{K}\\
\end{align*}

\end{document}
